%----------------------------------------------------------------------------------------
%  metropolis template (refactored)
%----------------------------------------------------------------------------------------
\documentclass[handout,aspectratio=169]{beamer}

% \documentclass の直後で hyperref のオプションを渡す(metropolisでも安全)
\PassOptionsToPackage{unicode=true,colorlinks=true,linkcolor=blue,urlcolor=blue}{hyperref}

\usetheme{metropolis}
\metroset{block=fill, sectionpage=progressbar, progressbar=foot}

% 背景色(tech のとき白など): Pythonで差し込み
\setbeamercolor{background canvas}{bg=white}

%--------------------------
% 日本語
%--------------------------
\usepackage{luatexja}
\usepackage{luatexja-fontspec}
\usepackage{luatexja-ruby}
\setsansjfont{Hiragino Sans}[BoldFont={Hiragino Sans W6}]

%--------------------------
% 基本パッケージ(重複なし)
%--------------------------
\usepackage[table]{xcolor}
\usepackage{graphicx}
\usepackage[abs]{overpic}
\usepackage{tikz}
\usepackage{array}
\usepackage{tabularx}
\usepackage{booktabs}
\usepackage{makecell}
\usepackage{mathtools}
\usepackage{longtable}
\usepackage{pdfpages}
\usepackage{etoolbox} % AtBeginEnvironment 等

% minted(※ -shell-escape 必須)
\usepackage{minted}
\setminted{
  frame=single,
  framesep=2mm,
  fontsize=\footnotesize,
  breaklines=true
}

% (必要なときだけ)tcolorbox
\usepackage[most]{tcolorbox}

% hyperref は最後
\usepackage{hyperref}

%--------------------------
% パス
%--------------------------
\newcommand{\assetpath}{/Volumes/NBPlan/TTC/授業資料/2025年度/}
\graphicspath{{images/}{\assetpath/1020201.アルゴリズム2/06/images/}{../project_assets/images/}{../project_assets/emoji/emoji_pngs/}}

%--------------------------
% フッター
%--------------------------
\newcommand{\myfootertext}{1020201.アルゴリズム2/06}
\setbeamertemplate{footline}{%
  \leavevmode
  \hbox to \paperwidth{%
    \hspace*{0.2cm}
    \scriptsize\color{gray!50} \myfootertext
    \hfill
    \scriptsize\color{gray} \insertframenumber{} / \inserttotalframenumber
    \hspace*{0.4cm}
  }%
  \vspace{1pt}
}

%--------------------------
% TeacherFrame(外部)
%--------------------------
\usepackage{../teacherframe}

%--------------------------
% フレームタイトル:番号. タイトル
% ※ ここで出すだけ。insertframetitle を再定義しない(安全)
%--------------------------
\setbeamertemplate{frametitle}{%
  \vspace{0.6ex}%
  \begin{beamercolorbox}[wd=\paperwidth,sep=0.5ex,leftskip=0.9em,rightskip=0.5em]{frametitle}%
    \usebeamerfont{frametitle}%
    \insertframenumber.\,\insertframetitle%
  \end{beamercolorbox}%
}

%--------------------------
% 表用:列型
%--------------------------
\newcolumntype{C}[1]{>{\centering\arraybackslash}p{#1}}
\newcolumntype{M}[1]{>{\raggedright\arraybackslash}m{#1}}

%--------------------------
% ブロック(必要なら)
%--------------------------
\definecolor{myblue}{HTML}{7488FF}
\definecolor{mylightblue}{HTML}{E3EEFF}
\setbeamertemplate{blocks}[rounded]
\setbeamercolor{block title}{bg=myblue, fg=white}
\setbeamercolor{block body}{bg=mylightblue, fg=black}

%========================================================
% exampleblock(examplebox相当)だけの調整
%  - タイトル文字:白
%  - 背景色:現行のまま(bgは指定しない)
%  - 本文 itemize:文字も●も黒(exampleblock内だけ)
%========================================================

% タイトル文字だけ白(背景は触らない)
\setbeamercolor{block title example}{fg=white}

% 本文の通常文字色は「現行のまま」を基本にする(必要なら黒にしてもよい)
% ここは bg を触らないのが目的なので fg だけ調整可能
\setbeamercolor{block body example}{fg=black}

% exampleblock の中だけ itemize の色(●と文字)を黒に
\AtBeginEnvironment{exampleblock}{%
  \setbeamercolor{itemize item}{fg=black}
  \setbeamercolor{itemize subitem}{fg=black}
  \setbeamercolor{itemize subsubitem}{fg=black}
  \setbeamercolor{item}{fg=black} % 念のため
}

% exampleblock を抜けたらテーマ標準に戻す(色が残る事故防止)
\AtEndEnvironment{exampleblock}{%
  \setbeamercolor{itemize item}{fg=normal text.fg}
  \setbeamercolor{itemize subitem}{fg=normal text.fg}
  \setbeamercolor{itemize subsubitem}{fg=normal text.fg}
  \setbeamercolor{item}{fg=normal text.fg}
}

%--------------------------
% 奇数ページのスライドのを表示する
%(教示用だけでそれ以外はこの処理は動かない)
%--------------------------
% --- 教師用だけ、スライドを奇数開始に強制するトグル ---
\newif\ifoddslideenforce
\oddslideenforcefalse   % デフォルトOFF(pr/hoはOFF)

% --- 再帰防止ガード ---
\newif\ifoddslideguard
\oddslideguardfalse

% --- 偶数ページなら空白スライドを1枚入れて奇数に戻す ---
\newcommand{\ensureoddslide}{%
  \ifoddslideguard\relax\else
    \oddslideguardtrue
    \ifodd\value{page}\relax
      % 何もしない(次が奇数)
    \else
      \begin{frame}[plain,noframenumbering]
        \note{}% notes出力時に2枚消費させる保険
      \end{frame}
    \fi
    \oddslideguardfalse
  \fi
}

% --- frameが始まる直前に自動挿入(教師用だけ)---
\BeforeBeginEnvironment{frame}{%
  \ifoddslideenforce
    \ensureoddslide
  \fi
}

%--------------------------
% note / noteT の「常時安全化」
%  - tech 以外:\noteT は無視(エラーにならない)
%  - tech:notesmode_tech で上書き定義
%--------------------------
\providecommand{\notetitletext}{}      % 既にあっても衝突しない
\providecommand{\noteT}[2]{}           % デフォルトは何もしない

% frame開始ごとにタイトル変数をクリア(前の noteT が残らないように)
\AtBeginEnvironment{frame}{\gdef\notetitletext{}}

%--------------------------
% 切替(Pythonから差し込み)
%--------------------------
\mypausemodetrue
\teachermodetrue
\setbeameroption{show notes}
%-------------

% --- tech のときだけ noteT を有効化(テンプレートの \providecommand を上書き) ---
\makeatletter
\renewcommand{\noteT}[2]{%
 \gdef\notetitletext{#1}%
 \note{#2}%
}
% タイトル未指定のときのために初期化
\renewcommand{\notetitletext}{}%

\setbeamertemplate{note page}{%
 \begin{minipage}{\linewidth}
 \vspace{1.2ex} % タイトルを少し下げる(必要に応じて調整)
 {\Large\bfseries
 \ifx\notetitletext\@empty
 \insertframetitle
 \else
 \notetitletext
 \fi
 }\par
 \vspace{-1.2ex}
 \rule{\linewidth}{0.8pt}\par
 \vspace{0.8ex}
 {\scriptsize \insertnote}
 \end{minipage}
}
\makeatother

%教師用のPDFは奇数ページからスライドを出力
\oddslideenforcetrue


%----------------------------------------------------------------------------------------
% タイトル
%----------------------------------------------------------------------------------------
\title{ 06 確率分布①(離散分布の考え方) }
\date{}
\newcommand{\codedir}{\assetpath/1020201.アルゴリズム2/06}

\begin{document}

\begin{frame}[plain,noframenumbering]
  \titlepage
  \bigskip
  \begin{center}
    \ifteachermode 教師用 \fi
  \end{center}
\end{frame}

% セクションページ(必要なら)
\setbeamertemplate{section page}{
  \begin{centering}
    \vfill
    \rule{\linewidth}{2pt}\par
    \vspace{1ex}
    {\usebeamerfont{section title}\Huge\bfseries \insertsection}\par
    \vspace{1ex}
    \rule{\linewidth}{2pt}\par
    \vfill
  \end{centering}
}
\setbeamerfont{section title}{size=\LARGE,series=\bfseries}

\AtBeginSection[]{
  \begin{frame}[plain,noframenumbering]
    \sectionpage
  \end{frame}
}

% 本編開始でフレーム番号を0から(必要なら)
\setcounter{framenumber}{0}

\input{emoji_macros}

% @@@--(metropolis)--@@@

% ----------------------------------------------------------------------------------------
%   Slide 01: 本日のテーマ
% ----------------------------------------------------------------------------------------
\begin{frame}{確率分布①(離散分布の考え方)}
本日は、これまでに学んだ「確率」をグラフにする方法を学びます。

\begin{itemize}
  \item \textbf{実験:} コインを10回なげると、表は何回出る?
  \item \textbf{ルール:} 何度もくり返すと見えてくる「形」。
  \item \textbf{名前:} その「結果の分かれ方」を「\ruby{確率分布}{かくりつぶんぷ}」と呼びます。
\end{itemize}

公式を覚える前に、まずは「実験の結果がどう分かれるか」をイメージしましょう。

\noteT{講義の狙い}{
身近なコイン投げを例に、1回の結果ではなく、全体の「分かれ方」に意識を向けさせます。
}
\end{frame}

% ----------------------------------------------------------------------------------------
%   Slide 02: 実験:コインを10回なげてみよう
% ----------------------------------------------------------------------------------------
\begin{frame}{コインを10回なげる実験を考えます}
この「10回なげる実験」を1セットとして、同じ条件で何セットもくり返します。

すると、何回も繰り返したセットの実験ごとに「表が出た回数」は、
\begin{center}
  \textbf{0回、1回、2回、……、10回}
\end{center}
のように変わります。

このように、実験をくり返したときに現れる\par
\textbf{「結果の分かれ方」}をまとめて表したものを、\par
\textbf{\ruby{確率分布}{かくりつぶんぷ}}といいます。

\noteT{導入}{
学生が頭の中でコインを投げている様子をイメージさせ、専門用語と結びつけます。
}
\end{frame}

% ----------------------------------------------------------------------------------------
%   Slide 03: 確率分布を「形」にしてみる
% ----------------------------------------------------------------------------------------
\begin{frame}{分かれ方をグラフにすると「形」が見える}
さきほどの「表が出た回数」をグラフに並べてみると、どうなるでしょうか。

\begin{itemize}
  \item ちょうど真ん中の「5回」が一番出やすい。
  \item 「0回」や「10回」は、めったに出ない。
\end{itemize}

このように、確率を並べると「山のような形」ができあがります。
これが、確率の\textbf{分布(ちらばり具合)}です。

\vspace{1em}
\textbf{ポイント:}\par
1回ごとの現象はいろいろ変化しますが、多くくり返したときは、この「確率分布(ルール)」に従います。

\noteT{視覚化}{
「分布 = 形」であることを視覚的に印象づけます。詳細は次ページへ。
}
\end{frame}

% ----------------------------------------------------------------------------------------
%   Slide 04: コイン10回投げの「結果の分かれ方」
% ----------------------------------------------------------------------------------------
\begin{frame}{コイン10回投げの「結果の分かれ方」}
数学的な計算(ルール)にもとづいて、表が出る回数の確率を表と図にまとめました。

\begin{columns}
  \begin{column}{0.4\textwidth}
    \begin{table}[]
      \centering
      \small
      \begin{tabular}{c|c}
        \textbf{表の回数} & \textbf{確率} \\ \hline
        0 & 0.001 \\
        1 & 0.010 \\
        2 & 0.044 \\
        3 & 0.117 \\
        4 & 0.205 \\
        5 & \textbf{0.246} \\
        6 & 0.205 \\
        7 & 0.117 \\
        8 & 0.044 \\
        9 & 0.010 \\
        10 & 0.001
      \end{tabular}
    \end{table}
  \end{column}
  
  \begin{column}{0.6\textwidth}
    % ここにアップロードいただいた「二項分布グラフ.png」を挿入
    \includegraphics[width=\textwidth]{二項分布グラフ.png}
    
    \vspace{0.5em}
    \textbf{図からわかること:}
    \begin{itemize}
      \item \textbf{5回}が出る確率が最も高い。
      \item 5回から離れるほど、確率は低くなっている。
    \end{itemize}
  \end{column}
\end{columns}

\noteT{解説}{
表の数値が「5」を中心に左右対称になっていること、グラフがきれいな「山の形」であることを確認させます。
}
\end{frame}

% ----------------------------------------------------------------------------------------
%   Slide 05: ヒストグラムと 確率分布:似ているが同じではない
% ----------------------------------------------------------------------------------------
\begin{frame}{ヒストグラムと 確率分布:似ているが同じではない}
見た目は似ていますが、\textbf{意味と役割がちがいます}。

\begin{itemize}
  \item \textbf{ヒストグラム:}\par
        集めたデータを、\textbf{数えてまとめた結果}(過去)
  \item \textbf{確率分布:}\par
        データが \textbf{どのように出るかというルール}(理論)
\end{itemize}

\vspace{0.5em}
\textbf{なぜ形が似る?}\par
同じ条件でデータをたくさん集め、\textbf{割合(相対度数)}で見ると、\par
ヒストグラムの形は、\textbf{確率分布の形に近づきます}。

\noteT{位置づけ}{
「事実(ヒストグラム)」と「理想のルール(確率分布)」を明確に区別させます。
}
\end{frame}

% ----------------------------------------------------------------------------------------
%   Slide 06: なぜ「ルール(確率分布)」を知る必要があるの?
% ----------------------------------------------------------------------------------------
\begin{frame}{なぜ「ルール(確率分布)」を知る必要があるの?}
計算で「未来のルール」がわかると、実際に実験をしなくても予測ができるからです。

\begin{itemize}
  \item \textbf{実験:} 1万回コインを投げるのは大変!(時間も体力も必要)
  \item \textbf{計算:} 確率分布のルールを使えば、一瞬で予測ができる。
\end{itemize}

\textbf{ビジネスでの活用例:}\par
「1000個の製品を作ったとき、不良品が1個以下に収まる確率は?」\par
これも、実際に作って壊さなくても、二項分布という「ルール」で計算できます。

\noteT{必要性}{
実習に入る前に、計算(理論)を学ぶメリットを伝えて意欲を高めます。
}
\end{frame}

% ----------------------------------------------------------------------------------------
%   Slide 07: Excelを使って「確率の山」を作ろう
% ----------------------------------------------------------------------------------------
\begin{frame}{Excelを使って「確率の山」を作ろう}
それでは、スライド04で見た「きれいな山の形」を、自分たちでExcelを使って再現してみましょう。

\begin{itemize}
  \item \textbf{目標:} スライド04と同じ表とグラフを自力で作る。
  \item \textbf{道具:} BINOM.DIST(バイノム・ディスト)関数。
\end{itemize}

この関数は、二項分布(コイン投げのような2択のくり返し)の確率を計算してくれる、統計学の強力な武器です。

\noteT{区切り}{
ここで座学を一度区切り、PC操作(実習)への切り替えを促します。
}
\end{frame}

% ----------------------------------------------------------------------------------------
%   Slide 08: 【実習1】データの準備と数式の入力
% ----------------------------------------------------------------------------------------
\begin{frame}{実習1:データの準備と数式の入力}
まずは、確率を計算するための「表」をExcelで作ります。

\textbf{手順:}
\begin{enumerate}
  \item \textbf{見出し:} A1セルに「表の回数」、B1セルに「確率」と入力。
  \item \textbf{数字:} A2~A12セルに \textbf{0 から 10} までの数字を入力。
  \item \textbf{数式:} B2セルに以下の式を半角で入力してください。
\end{enumerate}

\begin{block}{B2セルに入力する式}
  \texttt{=BINOM.DIST(A2, 10, 0.5, FALSE)}
\end{block}

入力できたら、B2セルの右下をダブルクリックして、10回まで計算(オートフィル)しましょう。

\noteT{操作のコツ}{
BINOM.DISTの引数「A2(回数), 10(全部で), 0.5(確率), FALSE(ぴったり)」の意味を口頭で補足します。
}
\end{frame}

% ----------------------------------------------------------------------------------------
%   Slide 09: 【実習1】数字をグラフにする
% ----------------------------------------------------------------------------------------
\begin{frame}{実習1:数字をグラフにする}
計算した数値から、確率の「形」を可視化しましょう。

\textbf{手順:}
\begin{enumerate}
  \item \textbf{範囲選択:} A1からB12まで、マウスでドラッグして選択します。
  \item \textbf{挿入:} 「挿入」タブ $\rightarrow$ 「おすすめグラフ」をクリック。
  \item \textbf{種類:} \textbf{「散布図(平滑線とマーカー)」}を選びます。
\end{enumerate}

\begin{itemize}
  \item \textbf{チェック:} スライド04で見せてもらった折れ線グラフと同じ「山の形」になりましたか?
\end{itemize}

\noteT{グラフの選択}{
棒グラフでも間違いではありませんが、アップロードされた画像(折れ線)に合わせることで、理論的な「形」としての美しさを強調します。
}
\end{frame}

% ----------------------------------------------------------------------------------------
%   Slide 10: 【実習1】もし確率(p)が変わったら?
% ----------------------------------------------------------------------------------------
\begin{frame}{実習1:もし確率(p)が変わったら?}
今のグラフは「50\%の確率(0.5)」の山です。これを変えてみましょう。

\textbf{やってみよう:}
\begin{itemize}
  \item B2セルの式を \texttt{=BINOM.DIST(A2, 10, \textbf{0.1}, FALSE)} に書き換えて、下までコピーし直してください。
\end{itemize}

\textbf{観察:}
\begin{itemize}
  \item 山の頂上(一番高いところ)は、どこに移動しましたか?
  \item 山の形はどう変わりましたか?
\end{itemize}

\textbf{結論:}
成功確率(p)が変わると、分布の「形」と「中心の位置」が動きます。

\noteT{変化の体験}{
期待値(n × p)という言葉を使わずに、確率が変われば「最も起きやすい場所」が変わることを視覚的に体験させます。\\
\\
分布が動く体験(発展)
}
\end{frame}

% ----------------------------------------------------------------------------------------
%   Slide 11: 「ちょうど」と「以下」の違い
% ----------------------------------------------------------------------------------------
\begin{frame}{「ちょうど5回」と「5回以下」の違い}
これまでは「ぴったりその回数」の確率を見てきましたが、実社会では別の数え方もよく使います。

\begin{itemize}
  \item \textbf{点(てん):} 表が「ちょうど2回」出る確率は?
  \item \textbf{範囲(はんい):} 表が「2回以下(0回、1回、2回の合計)」出る確率は?
\end{itemize}

この「~回以下」という、そこまでの合計を \ruby{累積}{るいせき} 確率といいます。
Excelでは、関数の最後を \textbf{TRUE} に変えるだけで、この合計を計算できます。

\noteT{概念の導入}{
「ぴったり」と「積み上げ」の違いを意識させます。これが実務(合格点以上、不良品数以下など)で役立つことを示唆します。\\
\\
累積(簡単に触れる位置)
}
\end{frame}

% ----------------------------------------------------------------------------------------
%   Slide 12: 【実習2】「積み上げ」の計算をしてみよう
% ----------------------------------------------------------------------------------------
\begin{frame}{実習2:累積(るいせき)確率の計算}
先ほどの表の隣に、新しい列を作って「積み上げ」の計算をしてみましょう。

\textbf{手順:}
\begin{enumerate}
  \item C1セルに「累積確率」と入力。
  \item C2セルに以下の式を入力し、C12までコピーします。
\end{enumerate}

\begin{block}{C2セルに入力する式(最後が TRUE に変わります)}
  \texttt{=BINOM.DIST(A2, 10, 0.5, \textbf{TRUE})}
\end{block}

\begin{itemize}
  \item \textbf{確認:} C2セルはB2と同じですが、下の行へ行くほど数値が増えていきますか?
\end{itemize}

\noteT{操作}{
FALSE(その点だけ)と TRUE(そこまでの合計)の数値の変化を直接目で見て比較させます。\\
\\
累積(簡単に触れる位置)
}
\end{frame}

% ----------------------------------------------------------------------------------------
%   Slide 13: 累積確率は「階段」のような形
% ----------------------------------------------------------------------------------------
\begin{frame}{累積確率は「階段」のような形}
新しく作った C列(累積確率)をグラフにしてみましょう。

\textbf{手順:}
\begin{enumerate}
  \item A1~A12セルと、\textbf{C1~C12セル}を選択してグラフを作ります。
  \item (Ctrlキーを押しながらマウスで選ぶと、離れた列を選択できます)
\end{enumerate}

\textbf{観察:}
\begin{itemize}
  \item 右肩上がりの「階段」のような形になりましたか?
  \item 最後の「10枚」のところの確率は、ちょうど \textbf{1.0 (100\%)} になりましたか?
\end{itemize}

\noteT{視覚化}{
合計が1になるという確率の基本原則と、累積の「増えていくイメージ」をグラフで定着させます。\\
\\
累積(簡単に触れる位置)
}
\end{frame}

% ----------------------------------------------------------------------------------------
%   Slide 14: 累積確率で「めったにないこと」を見つける
% ----------------------------------------------------------------------------------------
\begin{frame}{累積確率で「めったにないこと」を見つける}
Excelで計算した累積確率は、\textbf{「異常やチャンスの判断」}に使えます。

\begin{itemize}
  \item \textbf{問い:} コインを10回投げて表が9枚出た。これは「よくあること」?
\end{itemize}

\textbf{累積確率(TRUE)で確認すると:}
\begin{itemize}
  \item 8枚以下の累積確率:\textbf{0.989} (98.9\%)
  \item 9枚以上出る確率:1 - 0.989 = \textbf{0.011 (1.1\%)}
\end{itemize}

\textbf{見方のポイント:}
「9枚以上出る確率は、わずか1.1\%しかない」=「めったに起きないことが起きた(このコインは怪しい?)」と判断する根拠になります。

\noteT{判断の基準}{
「点」の確率ではなく、その値「以上/以下」のまとまり(累積)で見ることで、客観的な判断材料になることを教えます。\\
\\
累積(簡単に触れる位置)
}
\end{frame}

% ----------------------------------------------------------------------------------------
%   Slide 15: 【実務例】合格ラインと不備の予測
% ----------------------------------------------------------------------------------------
\begin{frame}{【実務例】合格ラインと不備の予測}
ビジネスや日常生活では、以下のように累積確率(TRUE)を使います。

\begin{enumerate}
  \item \textbf{品質管理:} 
    「100個の製品の中に、不良品が2個\textbf{以下}に収まる確率は?」
  \item \textbf{サービス設計:}
    「レジに5人\textbf{以上}並んでしまう確率は?(=店員を増やすべきか?)」
  \item \textbf{試験:}
    「80点\textbf{以上}取れる人は、全体の上位何%か?」
\end{enumerate}

\vspace{1em}
\textbf{Excelでのコツ:}
「~回以上」を求めたいときは、\textbf{「1 - (その手前までの累積確率)」}で計算できます。

\noteT{活用の広がり}{
Excelの操作が、将来の品質管理やマーケティングの分析に直結していることを示し、学習の納得感を高めます。\\
\\
累積(簡単に触れる位置)
}
\end{frame}
% ----------------------------------------------------------------------------------------
%   Slide 16: 今回学んだルールの名前:二項分布
% ----------------------------------------------------------------------------------------
\begin{frame}{今回学んだルールの名前:二項分布}
実習で作ったこの「確率分布」には、特別な名前があります。

\begin{itemize}
  \item \textbf{名前:} \textbf{\ruby{二項分布}{にこうぶんぷ}}(Binomial Distribution)
  \item \textbf{特徴:}
    \begin{enumerate}
      \item 「表か裏か」「合格か不合格か」のように、\textbf{結果が2つ}しかない。
      \item その試行を、同じ条件で \textbf{$n$ 回 くり返す}。
    \end{enumerate}
\end{itemize}

世の中の「2択」を扱うデータの多くは、このルールで説明が可能です。

\noteT{整理}{
実習後の用語解説で定着を図ります。
}
\end{frame}

% ----------------------------------------------------------------------------------------
%   Slide 17: 「とびとびの値」:離散型(りさんがた)
% ----------------------------------------------------------------------------------------
\begin{frame}{「とびとびの値」:離散型(りさんがた)}
二項分布のグラフ(スライド04)を思い出してください。

\begin{itemize}
  \item 回数は「1回、2回……」と数えられます。
  \item \textbf{1.5回} や \textbf{2.7回} という結果はありません。
\end{itemize}

このように、整数のようにつながっておらず「とびとびの値」をとるものを
\textbf{\ruby{離散型}{りさんがた} 確率分布} といいます。

\vspace{0.5em}


\noteT{分類}{
次回学ぶ「連続型(正規分布)」との違いを意識させます。
}
\end{frame}

% ----------------------------------------------------------------------------------------
%   Slide 18: 分布の形を決める「2つの要素」
% ----------------------------------------------------------------------------------------
\begin{frame}{分布の形を決める「2つの要素」}
二項分布の山の形や位置は、次の2つの数字(パラメータ)だけで決まります。

\begin{enumerate}
  \item \textbf{試行回数 ($n$)}:全部で何回やるか?
    \begin{itemize}
      \item $n$ を増やすと、山は右に移動し、形は「左右対称」に近づきます。
    \end{itemize}
  \item \textbf{成功確率 ($p$)}:1回あたりの確率は?
    \begin{itemize}
      \item $p$ が小さいと山は左へ、大きいと右へ寄ります。
    \end{itemize}
\end{enumerate}

\noteT{法則性}{
次スライドの図でこれを確認します。\\
\\
分布が動く体験(発展)
}
\end{frame}

% ----------------------------------------------------------------------------------------
%   Slide 19: 【比較】条件(nとp)で変わる山の形
% ----------------------------------------------------------------------------------------
\begin{frame}{【比較】条件(nとp)で変わる山の形}
Excelの数値を書き換えて作った比較図です。

\begin{center}
  \includegraphics[width=0.85\textwidth]{二項分布比較図.png}
\end{center}


\begin{itemize}
  \item \textbf{回数を増やす:} 頂上が右へ動き、山が滑らかになる。
  \item \textbf{確率を変える:} 山全体が左右にスライドする。
\end{itemize}

\noteT{視覚化}{
パラメータによる「分布のコントロール」を理解させます。\\
\\
分布が動く体験(発展)
}
\end{frame}
% ----------------------------------------------------------------------------------------
%   Slide 19-1: 比較図からわかること(まとめ)
% ----------------------------------------------------------------------------------------
\begin{frame}{比較図からわかること(まとめ)}
先ほどの4つのグラフを比べると、次のルールが見えてきます。

\begin{itemize}
  \item \textbf{試行回数 ($n$) の影響:}
    \begin{itemize}
      \item $n$ が大きいほど、グラフはトゲトゲしさがなくなり、左右対称の滑らかな「山の形」に近づいていきます。
    \end{itemize}
  \item \textbf{成功確率 ($p$) の影響:}
    \begin{itemize}
      \item $p=0.5$ のときは、山がちょうど真ん中にきます。
      \item $p$ が $0.5$ より小さいと左へ、$0.5$ より大きいと右へ、山全体がスライドします。
    \end{itemize}
\end{itemize}

このように、回数と確率という「前提条件」が決まれば、未来に起こる結果の分布(ルール)は、数式で一通りに決まります。

\noteT{補足説明}{
グラフの視覚的な変化を、統計学的な言葉($n$や$p$の影響)として再定義し、学生の理解を固めます。\\
\\
分布が動く体験(発展)
}
\end{frame}
% ----------------------------------------------------------------------------------------
%   Slide 20: 離散分布のまとめ:判断の武器にする
% % ----------------------------------------------------------------------------------------
% \begin{frame}{離散分布のまとめ:判断の武器にする}
% 今日学んだ「確率分布」は、単なるグラフではありません。

% \begin{itemize}
%   \item \textbf{異常の発見:} 累積確率で、めったに起きない1\%以下の事態を見つける。
%   \item \textbf{未来の予知:} 実験しなくても、Excelで結果の分かれ方を予測できる。
% \end{itemize}

% 「1回ごとの運」に一喜一憂せず、「全体のルール(分布)」で判断しましょう。

% \noteT{意義}{
% 統計学的な考え方の本質を伝えます。
% }
% \end{frame}

% ----------------------------------------------------------------------------------------
%   Slide 21: 第5回の総まとめ
% ----------------------------------------------------------------------------------------
\begin{frame}{本日の総まとめ}
\begin{enumerate}
  \item \textbf{確率分布}:データがどう分かれるかを示す「ルール」。
  \item \textbf{二項分布}:2択を繰り返すときの代表的な分布。
  \item \textbf{離散型}:数えられる「とびとびの値」を持つ分布。
  \item \textbf{BINOM.DIST}:Excelで「点(FALSE)」と「累積(TRUE)」を使い分ける。
\end{enumerate}

\vspace{1em}
お疲れ様でした。次は「学習カルテ(小テスト)」に取り組みましょう。

\noteT{終了}{
授業を終了し、評価へ移ります。
}
\end{frame}
\end{document}
