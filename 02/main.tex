%----------------------------------------------------------------------------------------
%  metropolis template (refactored)
%----------------------------------------------------------------------------------------
\documentclass[aspectratio=169]{beamer}

% \documentclass の直後で hyperref のオプションを渡す(metropolisでも安全)
\PassOptionsToPackage{unicode=true,colorlinks=true,linkcolor=blue,urlcolor=blue}{hyperref}

\usetheme{metropolis}
\metroset{block=fill, sectionpage=progressbar, progressbar=foot}

% 背景色(tech のとき白など): Pythonで差し込み


%--------------------------
% 日本語
%--------------------------
\usepackage{luatexja}
\usepackage{luatexja-fontspec}
\usepackage{luatexja-ruby}
\setsansjfont{Hiragino Sans}[BoldFont={Hiragino Sans W6}]

%--------------------------
% 基本パッケージ(重複なし)
%--------------------------
\usepackage[table]{xcolor}
\usepackage{graphicx}
\usepackage[abs]{overpic}
\usepackage{tikz}
\usepackage{array}
\usepackage{tabularx}
\usepackage{booktabs}
\usepackage{makecell}
\usepackage{mathtools}
\usepackage{longtable}
\usepackage{pdfpages}
\usepackage{etoolbox} % AtBeginEnvironment 等

% minted(※ -shell-escape 必須)
\usepackage{minted}
\setminted{
  frame=single,
  framesep=2mm,
  fontsize=\footnotesize,
  breaklines=true
}

% (必要なときだけ)tcolorbox
\usepackage[most]{tcolorbox}

% hyperref は最後
\usepackage{hyperref}

%--------------------------
% パス
%--------------------------
\newcommand{\assetpath}{/Volumes/NBPlan/TTC/授業資料/2025年度/}
\graphicspath{{images/}{\assetpath/1020201.アルゴリズム2/02/images/}{../project_assets/images/}{../project_assets/emoji/emoji_pngs/}}

%--------------------------
% フッター
%--------------------------
\newcommand{\myfootertext}{1020201.アルゴリズム2/02}
\setbeamertemplate{footline}{%
  \leavevmode
  \hbox to \paperwidth{%
    \hspace*{0.2cm}
    \scriptsize\color{gray!50} \myfootertext
    \hfill
    \scriptsize\color{gray} \insertframenumber{} / \inserttotalframenumber
    \hspace*{0.4cm}
  }%
  \vspace{1pt}
}

%--------------------------
% TeacherFrame(外部)
%--------------------------
\usepackage{../teacherframe}

%--------------------------
% フレームタイトル:番号. タイトル
% ※ ここで出すだけ。insertframetitle を再定義しない(安全)
%--------------------------
\setbeamertemplate{frametitle}{%
  \vspace{0.6ex}%
  \begin{beamercolorbox}[wd=\paperwidth,sep=0.5ex,leftskip=0.9em,rightskip=0.5em]{frametitle}%
    \usebeamerfont{frametitle}%
    \insertframenumber.\,\insertframetitle%
  \end{beamercolorbox}%
}

%--------------------------
% 表用:列型
%--------------------------
\newcolumntype{C}[1]{>{\centering\arraybackslash}p{#1}}
\newcolumntype{M}[1]{>{\raggedright\arraybackslash}m{#1}}

%--------------------------
% ブロック(必要なら)
%--------------------------
\definecolor{myblue}{HTML}{7488FF}
\definecolor{mylightblue}{HTML}{E3EEFF}
\setbeamertemplate{blocks}[rounded]
\setbeamercolor{block title}{bg=myblue, fg=white}
\setbeamercolor{block body}{bg=mylightblue, fg=black}

%========================================================
% exampleblock(examplebox相当)だけの調整
%  - タイトル文字:白
%  - 背景色:現行のまま(bgは指定しない)
%  - 本文 itemize:文字も●も黒(exampleblock内だけ)
%========================================================

% タイトル文字だけ白(背景は触らない)
\setbeamercolor{block title example}{fg=white}

% 本文の通常文字色は「現行のまま」を基本にする(必要なら黒にしてもよい)
% ここは bg を触らないのが目的なので fg だけ調整可能
\setbeamercolor{block body example}{fg=black}

% exampleblock の中だけ itemize の色(●と文字)を黒に
\AtBeginEnvironment{exampleblock}{%
  \setbeamercolor{itemize item}{fg=black}
  \setbeamercolor{itemize subitem}{fg=black}
  \setbeamercolor{itemize subsubitem}{fg=black}
  \setbeamercolor{item}{fg=black} % 念のため
}

% exampleblock を抜けたらテーマ標準に戻す(色が残る事故防止)
\AtEndEnvironment{exampleblock}{%
  \setbeamercolor{itemize item}{fg=normal text.fg}
  \setbeamercolor{itemize subitem}{fg=normal text.fg}
  \setbeamercolor{itemize subsubitem}{fg=normal text.fg}
  \setbeamercolor{item}{fg=normal text.fg}
}

%--------------------------
% 奇数ページのスライドのを表示する
%(教示用だけでそれ以外はこの処理は動かない)
%--------------------------
% --- 教師用だけ、スライドを奇数開始に強制するトグル ---
\newif\ifoddslideenforce
\oddslideenforcefalse   % デフォルトOFF(pr/hoはOFF)

% --- 再帰防止ガード ---
\newif\ifoddslideguard
\oddslideguardfalse

% --- 偶数ページなら空白スライドを1枚入れて奇数に戻す ---
\newcommand{\ensureoddslide}{%
  \ifoddslideguard\relax\else
    \oddslideguardtrue
    \ifodd\value{page}\relax
      % 何もしない(次が奇数)
    \else
      \begin{frame}[plain,noframenumbering]
        \note{}% notes出力時に2枚消費させる保険
      \end{frame}
    \fi
    \oddslideguardfalse
  \fi
}

% --- frameが始まる直前に自動挿入(教師用だけ)---
\BeforeBeginEnvironment{frame}{%
  \ifoddslideenforce
    \ensureoddslide
  \fi
}

%--------------------------
% note / noteT の「常時安全化」
%  - tech 以外:\noteT は無視(エラーにならない)
%  - tech:notesmode_tech で上書き定義
%--------------------------
\providecommand{\notetitletext}{}      % 既にあっても衝突しない
\providecommand{\noteT}[2]{}           % デフォルトは何もしない

% frame開始ごとにタイトル変数をクリア(前の noteT が残らないように)
\AtBeginEnvironment{frame}{\gdef\notetitletext{}}

%--------------------------
% 切替(Pythonから差し込み)
%--------------------------
\mypausemodetrue
\teachermodefalse

%-------------

\providecommand{\noteT}[2]{} % noteT を無視


%----------------------------------------------------------------------------------------
% タイトル
%----------------------------------------------------------------------------------------
\title{ 02 統計で何ができるか }
\date{}
\newcommand{\codedir}{\assetpath/1020201.アルゴリズム2/02}

\begin{document}

\begin{frame}[plain,noframenumbering]
  \titlepage
  \bigskip
  \begin{center}
    \ifteachermode 教師用 \fi
  \end{center}
\end{frame}

% セクションページ(必要なら)
\setbeamertemplate{section page}{
  \begin{centering}
    \vfill
    \rule{\linewidth}{2pt}\par
    \vspace{1ex}
    {\usebeamerfont{section title}\Huge\bfseries \insertsection}\par
    \vspace{1ex}
    \rule{\linewidth}{2pt}\par
    \vfill
  \end{centering}
}
\setbeamerfont{section title}{size=\LARGE,series=\bfseries}

\AtBeginSection[]{
  \begin{frame}[plain,noframenumbering]
    \sectionpage
  \end{frame}
}

% 本編開始でフレーム番号を0から(必要なら)
\setcounter{framenumber}{0}

\input{emoji_macros}

% @@@--(metropolis)--@@@

% ----------------------------------------------------------------------------------------
%   Slide 01: 今日のゴール:統計を「使う」ための第一歩
% ----------------------------------------------------------------------------------------
\begin{frame}{今日のゴール:統計を「使う」ための第一歩}
今日から、\textbf{統計}の学習を始めます。

統計とは、\textbf{身の回りで起きていることを、数字を使って整理し、考えるための道具}です。
この授業では、いきなり難しい計算を学ぶのではなく、
「統計を使うとはどういうことか」を体験しながら理解していきます。

\vspace{0.8em}
\textbf{本日の到達目標}
\begin{enumerate}
  \item \textbf{統計の役割:} なぜ「数字」で説明するのかを知る
  \item \textbf{データの形:} コンピュータが扱いやすい「正しい表」を理解する
  \item \textbf{第一歩:} Excelでデータの全体像をつかむ
\end{enumerate}

\centering
\textbf{統計は、計算ではなく「判断を助ける道具」です。}

\noteT{導入}{
「今日は統計の最初の日です」と明確に宣言する。
数学が苦手でも大丈夫だと安心させてから目標を示す。
}
\end{frame}

% ----------------------------------------------------------------------------------------
%   Slide 02: 統計学の役割:バラバラな記録を「情報」に変える
% ----------------------------------------------------------------------------------------
\begin{frame}{統計学の役割:バラバラな記録を「情報」に変える}
統計学とは、大量にある「事実の記録(データ)」を、意味のある「情報」に変換する技術です。

\begin{itemize}
  \item \textbf{要約:} たくさんのデータを、平均値などの「1つの数字」で表現する。
  \item \textbf{判断:} 新しい施策に効果があったのか、偶然なのかを見極める。
  \item \textbf{予測:} 過去の傾向から、将来のリスクや可能性を予見する。
\end{itemize}

経験や勘(「たぶんこうだろう」)を、客観的な根拠(「データがこう示している」)に置き換えることが統計学の目的です。

\noteT{統計の役割}{「勘 vs 根拠」の対比を明確にし、ビジネスや研究での必要性を伝えます。}
\end{frame}

% ----------------------------------------------------------------------------------------
%   Slide 03: データとは何か
% ----------------------------------------------------------------------------------------
\begin{frame}{データとは「事実の記録」である}
「データ」とは、目の前で起きた出来事を記録したものです。数値だけでなく、言葉や記号も大切なデータになります。

\begin{block}{データの具体例}
  \begin{itemize}
    \item \textbf{数値:} 売上金額、気温、テストの点数、ポテトの「重さ」
    \item \textbf{文字・記号:} 商品名、性別、合否、味の「印象」(薄い/濃い)
  \end{itemize}
\end{block}

数値だけでなく、アンケートの回答や「はい/いいえ」といった記号も、集めて整理すれば立派な分析対象になります。

\noteT{データの定義}{「数字以外もデータである」ことを認識させ、分析の幅を広げます。}
\end{frame}

% ----------------------------------------------------------------------------------------
%   Slide 04: データの種類:質的/量的(1枚に統合してテンポUP)
% ----------------------------------------------------------------------------------------
\begin{frame}{データの種類:質的データと量的データ}
データは大きく \textbf{2種類} に分かれます。「計算してよいか?」がポイントです。

\vspace{0.5em}
\begin{columns}
  \begin{column}{0.48\textwidth}
    \textbf{質的データ(カテゴリ)}
    \begin{itemize}
      \item 分類・区別のためのデータ
      \item 例:曜日、味の印象
      \item \textbf{基本は数える}(件数・割合)
    \end{itemize}
  \end{column}
  \begin{column}{0.48\textwidth}
    \textbf{量的データ(数値)}
    \begin{itemize}
      \item 大きさ・量を表すデータ
      \item 例:重さ\_g、身長、売上
      \item \textbf{計算して要約}(平均・最大など)
    \end{itemize}
  \end{column}
\end{columns}

\vspace{0.5em}
\centering
\textbf{同じ「列」でも、何をしたいかで扱いが変わる。}

\noteT{尺度水準}{
難しい用語は出さず、「計算して意味があるか?」で判断させる。
}
\end{frame}

% ----------------------------------------------------------------------------------------
%   Slide 05: 検証の舞台(±5gは第1回では出さない)
% ----------------------------------------------------------------------------------------
\begin{frame}{検証の舞台:駅前のハンバーガーショップ}
「私」は駅前のハンバーガーショップをよく利用しています。特にお気に入りはポテトMサイズです。

\begin{itemize}
  \item \textbf{公表値:} ポテトMサイズは \textbf{135g} とされている
  \item \textbf{現場:} すべてを毎回計量するのは難しく、\textbf{目分量}になりやすい
  \item \textbf{疑問:} 本当に135gと言えるのだろうか?
\end{itemize}
\center{
  \includegraphics[scale=0.4]{burger.png}
}
\noteT{舞台設定}{
「公表値135g」だけを確定情報として提示する。
±の許容はこの時点では出さず、後の議論(検定)に回す。
}
\end{frame}

% ----------------------------------------------------------------------------------------
%   Slide 05:【事件発生】SNSの噂と自分の直感
% ----------------------------------------------------------------------------------------
\begin{frame}{【事件発生】SNSの噂と自分の直感}
ある日、「私」はSNSで気になる投稿を見かけました。

\vspace{1em}

\begin{columns}
    \begin{column}{0.2\textwidth}
        \includegraphics[scale=0.4]{sns.png}
    \end{column}
    \begin{column}{0.78\textwidth}
        ① 駅前店のポテト、135gより少ない気がする」\\
        ② 「味が日によってバラバラ」\\
        ③ 「店員によって重さが違う、味も...」
    \end{column}
\end{columns}

\begin{itemize}
    \item \textbf{自分の実感:}「確かに、昨日買ったときもスカスカだったような…」
\end{itemize}

\vspace{10pt}
「本当のところはどうなんだろう?」という素朴な疑問が、すべての始まりでした。

\noteT{導入のポイント}{
学生に「これ、皆さんも経験ありませんか?」と問いかけ、当事者意識を持たせてください。
ネットの書き込みをただ眺める客から、真実を確かめる調査員への視点切り替えを行います。
}
\end{frame}

% ----------------------------------------------------------------------------------------
%   Slide 06:【行動】1週間の徹底調査
% ----------------------------------------------------------------------------------------
\begin{frame}{【行動】1週間の徹底調査(S00の記録)}
「私」はうわさを確かめるため、実際に1週間毎日ポテトを購入して測ってみることにしました。

\begin{columns}
    \begin{column}{0.6\textwidth}
        \includegraphics[width=\textwidth]{秤に乗ったポテトの写真.png}
    \end{column}
    \begin{column}{0.38\textwidth}
        \small
        \textbf{【調査ルール】}\\
        ① 毎日Mサイズを1袋買う\\
        ② キッチンスケールで重さを測る\\
        ③ 味や調理状態も記録する
    \end{column}
\end{columns}

\begin{itemize}
    \item \textbf{集まった証拠:}月曜日から日曜日まで、合計7袋分の生データ。
    \item \textbf{記録の内容:}「重さ」だけでなく、「揚げ色」や「塩加減」も細かくメモ。
\end{itemize}

\noteT{データの信頼性}{
「感情的に文句を言うのではなく、まず記録(データ)を取る」という、統計的思考の第一歩を強調します。
この7日分のデータが、後の実習で使用する「S00」のデータそのものであることを伝えます。
}
\end{frame}

% ----------------------------------------------------------------------------------------
%   Slide 07:【発見と葛藤】2gの不足は「罪」か「誤差」か?
% ----------------------------------------------------------------------------------------
\begin{frame}{【結果】2gの差は判断できるのか?}
1週間分のデータを集計すると、次のようになりました。

\begin{center}
    \begin{tabular}{|c|c|c|c|c|c|c|c||c|}
    \hline
    曜日 & 月 & 火 & 水 & 木 & 金 & 土 & 日 & \textbf{平均} \\ \hline
    重さ & 130g & 138g & 125g & 140g & 132g & 129g & 137g & \textbf{133g} \\ \hline
    \end{tabular}
\end{center}

\begin{itemize}
    \item 135gを超える日もあるが、平均は \textbf{133g}
    \item 公式値より \textbf{2g少ない}
\end{itemize}

\vspace{5pt}
しかし——
\begin{itemize}
    \item この \textbf{2g} は問題と言えるのか?
    \item それとも、\textbf{たまたま}起きた誤差なのか?
\end{itemize}

\vspace{5pt}
\centering
\textbf{\large この時点では、まだ判断できない}

\noteT{第9回への種まき}{
この「2gの差をどう解釈するか」がこの授業のメインテーマであることを告げます。
この時点では「2g=誤差」と思う学生と「2g=事件」と思う学生に分かれるはずです。
その直感を、12回かけて「論理」に変えていくことを約束してください。
}
\end{frame}
% ----------------------------------------------------------------------------------------
%   Slide A: 実習パートへの切り替え
% ----------------------------------------------------------------------------------------
\begin{frame}{ここからは「『私』の調査結果を検証する」}
これまではSNSの評判や個人の感想といった「主観的な情報」を扱ってきました。ここからは、「私」が実際に店舗へ足を運んで集めた「データ」を使って、客観的に事実を確かめていきます。

\begin{itemize}
  \item \textbf{統計の役割:} 「私」の感覚を信じるのではなく、目の前の「数字」を使って事実を確かめるための道具です。
  \item \textbf{実習の目的:} 「私」が記録したデータから、Excelを使ってお店の実態を論理的に導き出す手法を学びます。
\end{itemize}

「統計は、『私』の調査結果を数字で裏付けるための道具」という意識を持って取り組みましょう。

\noteT{実習への切り替え}{ここからPC操作が始まります。\n自分のデータファイル(poteto1.xlsx)を開くよう指示し、授業のモードを切り替えてください。}
\end{frame}

% ----------------------------------------------------------------------------------------
%   Slide B: 今日使うデータの確認(poteto1.xlsx)
% ----------------------------------------------------------------------------------------
\begin{frame}{今日使うデータ:「私」が調査した「1週間の記録」}
実習では、「私」がこの1週間、店舗で計測して入力した Excelファイル(poteto1.xlsx)を使用します。

\begin{itemize}
  \item \textbf{1行 = 1日分:} 「私」が計測した7日間の生データです。
  \item \textbf{項目(列)の確認:}
  \begin{itemize}
    \item \textbf{曜日:} 計測した日(月~日)。
    \item \textbf{重さ\_g:} 自分で測ったポテトの重量。※数値データ
    \item \textbf{味の印象・メモ:} その時の感想。※文字データ
  \end{itemize}
\end{itemize}

\textbf{本日のポイント:} 今日は、計算が可能な「重さ\_g」の数値データのみを扱います。味の印象やメモは、今は使いません。

\noteT{データの前提}{学生にファイルを開かせ、特に「重さ\_g」の列が計算対象であることを認識させてください。}
\end{frame}

% ----------------------------------------------------------------------------------------
%   Slide C: 実習(1) 計算する前にデータを「眺める」
% ----------------------------------------------------------------------------------------
\begin{frame}{実習(1):計算する前にデータを「眺める」}
Excelで関数を入力する前に、まずは「私」のデータをじっくり眺めてみましょう。数字を直接見ることで、多くの気づきが得られます。

\begin{itemize}
  \item \textbf{「私」のデータを探してみよう}
  \begin{enumerate}
    \item 7日間の中で、一番「重かった日」と「軽かった日」はいつですか?
    \item 目標の $135$ g を超えている日は、「私」のデータの中に何日ありましたか?
    \item 「私」の7日間の結果だけを見て、お店に対して何を感じますか?
  \end{enumerate}
\end{itemize}

計算機(Excel)に頼り切る前に、「私」の目で「アタリ」や「ハズレ」を実感することが大切です。

\noteT{直感の言語化}{いきなり関数を打たせず、まずは自分のデータを眺めて「木曜日は多いな」「水曜日は少ないな」といった特徴を探す時間を設けてください。}
\end{frame}

% ----------------------------------------------------------------------------------------
%   Slide D-1: 平均の概念
% ----------------------------------------------------------------------------------------
\begin{frame}{平均とは「全体の平らな姿」を見ること}
「平均」は、バラバラな重さのデータを「もし全部同じだったら?」と平らにならしたときの値です。

\begin{itemize}
  \item \textbf{数式の意味:}
  \[ \text{平均} = \frac{\sum_{i=1}^{n} x_i}{n} \]
  ※「『私』の7日間の合計重量を、測った日数($n=7$)で割る」ことで、1日あたりの標準的な量を算出します。
  \item \textbf{平均で見えるもの:} 「私」の調査における「全体的な重さの目安」。
  \item \textbf{見えなくなるもの:} 「すごく重かった日」や「軽すぎた日」の個別の体験。
\end{itemize}

平均は便利ですが、これだけでは「日ごとの差(ばらつき)」は見えなくなります。

\noteT{意味理解}{「足して割る」という操作だけでなく、「平らにならす」というイメージを強調してください。}
\end{frame}

% ----------------------------------------------------------------------------------------
%   Slide D-2: 実習(2) Excelで「平均」を計算する
% ----------------------------------------------------------------------------------------
\begin{frame}{実習(2):Excelで「平均」を計算する}
「私」のデータの「平均重量」を、Excelの関数を使って計算しましょう。

\begin{itemize}
  \item \textbf{使用する関数:} \texttt{AVERAGE}(アベレージ)
  \item \textbf{操作手順:}
  \begin{enumerate}
    \item 答えを表示したいセルを選択する。
    \item \texttt{=AVERAGE(} と入力する。
    \item 「私」のデータの「重さ\_g(7日分)」をマウスで選択する。
    \item Enterキーを押す。
  \end{enumerate}
\end{itemize}

\textbf{今回の「私」の結果:}
計算すると **$133.0$ g** になりましたか?目標の $135$ g に近いでしょうか。

\noteT{操作成功体験}{ poteto1.xlsx のデータ通りであれば 133 になります。学生が正しく範囲選択できているか確認してください。}
\end{frame}

% ----------------------------------------------------------------------------------------
%   Slide D-3: 最大・最小の概念
% ----------------------------------------------------------------------------------------
\begin{frame}{最大・最小とは「体験の幅」を見ること}
平均(中心)だけでなく、「データの端(はし)」を見ることも非常に重要です。

\begin{itemize}
  \item \textbf{最大値(MAX):} 「私」がこの1週間で体験した「最高のアタリ」。
  \item \textbf{最小値(MIN):} 「私」がこの1週間で体験した「最悪のハズレ」。
\end{itemize}

\textbf{キーメッセージ:}
お客さんは「平均」を食べているのではありません。その日の「1袋」がすべてです。平均が $133$ g あっても、最小値が $125$ g ならば、「私」は不満を感じるかもしれません。

\noteT{ビジネス視点}{「平均さえ良ければいい」という考え方の落とし穴を説明します。}
\end{frame}

% ----------------------------------------------------------------------------------------
%   Slide D-4: 実習(3) Excelで「最大・最小」を計算する
% ----------------------------------------------------------------------------------------
\begin{frame}{実習(3):Excelで「最大・最小」を計算する}
平均の時と同じ手順で、「私」のデータの「端っこ」を数値化しましょう。

\begin{itemize}
  \item \textbf{使用する関数:}
  \begin{itemize}
    \item 最大値:\texttt{MAX}(マックス)
    \item 最小値:\texttt{MIN}(ミニマム)
  \end{itemize}
  \item \textbf{手順:} 平均の時と同じ「7日分の範囲」を選択して計算します。
\end{itemize}

\textbf{今回の「私」の結果:}
最大は **$140$ g**、最小は **$125$ g** になりましたか?

\noteT{操作の定着}{関数名が変わるだけで手順は共通であることを伝え、操作に慣れさせます。}
\end{frame}

% ----------------------------------------------------------------------------------------
%   Slide D-5: 中央値の概念
% ----------------------------------------------------------------------------------------
\begin{frame}{中央値とは「並べた時の真ん中」を見ること}
平均とは別の角度から「真ん中」を探すのが、中央値(メディアン)です。

\begin{itemize}
  \item \textbf{定義:} データを小さい順に並べたとき、ちょうど真ん中にくる値です。
  \item \textbf{位置の計算:}
  \[ \text{位置} = \frac{n + 1}{2} = \frac{7 + 1}{2} = 4 \text{番目} \]
  ※「私」のデータを軽い順に並べて、$4$ 番目にくる重さが中央値です。
  \item \textbf{メリット:} 1日だけ「山盛りのサービス(外れ値)」があっても、その影響を受けにくい指標です。
\end{itemize}

「平均と中央値は役割が違う」ことを意識しましょう。

\noteT{概念の対比}{平均は合計に引きずられ、中央値は順番を重視するという違いを強調します。}
\end{frame}

% ----------------------------------------------------------------------------------------
%   Slide D-6: 実習(4) Excelで「中央値」を計算する
% ----------------------------------------------------------------------------------------
\begin{frame}{実習(4):Excelで「中央値」を計算する}
Excelでは、自分で並べ替え作業をする必要はありません。

\begin{itemize}
  \item \textbf{使用する関数:} \texttt{MEDIAN}(メディアン)
  \item \textbf{操作:} 範囲を選択するだけで、Excelが裏側で自動的に並べ替えを行い、真ん中の値を抽出します。
\end{itemize}

\textbf{今回の「私」の結果:}
中央値は **$132.0$ g** になりましたか? 平均値($133.0$)と少しズレがあることがわかります。

\noteT{操作の拡張}{MEDIANの綴りを間違えないよう、スライドを示して確認させてください。}
\end{frame}

% ----------------------------------------------------------------------------------------
%   Slide E: 数値の整理
% ----------------------------------------------------------------------------------------
\begin{frame}{「私」の計算結果を並べて比較してみよう}
Excelで算出した4つの指標(代表値)を一つの表にまとめます。

\begin{itemize}
  \item \textbf{比較する「私」の数字:} 
  \begin{itemize}
    \item 平均:$133.0$ g
    \item 中央値:$132.0$ g
    \item 最大:$140$ g / 最小:$125$ g
  \end{itemize}
  \item \textbf{問い:}
  \begin{itemize}
    \item どれが「一番普通(いつもの量)」を表していると感じますか?
    \item 目標の $135$ g と比べて、「私」の調査結果はどう評価できますか?
  \end{itemize}
\end{itemize}

自分の計算結果を多角的に見ることで、お店に対する「『私』の主張」の根拠が生まれます。

\noteT{整理}{自分のデータを客観的な数値で整理させ、「言えそうなこと」を再考させます。}
\end{frame}

% ----------------------------------------------------------------------------------------
%   Slide F: 限界の認識
% ----------------------------------------------------------------------------------------
\begin{frame}{この時点で、何が分からない?}
計算は正しくできました。しかし、この結果だけで「この店のポテトは足りない」と断定して良いでしょうか。

\begin{itemize}
  \item \textbf{「私」のデータの限界:}
  \begin{itemize}
    \item データは「『私』一人」が測った、$7$ 日分だけです。
    \item たまたま「私」が買った1週間が特殊(ハズレが多い週)だった可能性はありませんか?
  \end{itemize}
\end{itemize}

\textbf{結論:}
数字は出ましたが、まだ「お店全体を判断」するには証拠が足りません。客観的な結論を出すには、もっと多くの調査結果が必要です。

\noteT{到達点の明確化}{「自分のデータ(n=7)だけでは判断できない」という気づきを促します。}
\end{frame}

% ----------------------------------------------------------------------------------------
%   Slide G: 次回への導線
% ----------------------------------------------------------------------------------------
\begin{frame}{では、どうすれば判断できる?}
「私」の7個のデータでは偶然かもしれません。判断を確かなものにするには、どうすればいいでしょうか。

\begin{itemize}
  \item \textbf{解決策:} データの「数(サンプルサイズ)」を増やす。
  \item \textbf{次回の予定:} クラス全員のデータを合体させてみよう。
\end{itemize}

「私」のデータは小さくても、クラス全員分集めれば、$100$ 個以上の巨大なデータになります。

\textbf{「みんなのデータを合わせれば、真実が見えてくる」}

\noteT{次回の導線}{各自の調査結果が、次回の全体分析に繋がることを示唆して締めくくります。}
\end{frame}
\end{document}
