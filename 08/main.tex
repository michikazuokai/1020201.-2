%----------------------------------------------------------------------------------------
%  metropolis template (refactored)
%----------------------------------------------------------------------------------------
\documentclass[handout,aspectratio=169]{beamer}

% \documentclass の直後で hyperref のオプションを渡す(metropolisでも安全)
\PassOptionsToPackage{unicode=true,colorlinks=true,linkcolor=blue,urlcolor=blue}{hyperref}

\usetheme{metropolis}
\metroset{block=fill, sectionpage=progressbar, progressbar=foot}

% 背景色(tech のとき白など): Pythonで差し込み
\setbeamercolor{background canvas}{bg=white}

%--------------------------
% 日本語
%--------------------------
\usepackage{luatexja}
\usepackage{luatexja-fontspec}
\usepackage{luatexja-ruby}
\setsansjfont{Hiragino Sans}[BoldFont={Hiragino Sans W6}]

%--------------------------
% 基本パッケージ(重複なし)
%--------------------------
\usepackage[table]{xcolor}
\usepackage{graphicx}
\usepackage[abs]{overpic}
\usepackage{tikz}
\usetikzlibrary{positioning, shapes.geometric} % ← shapes.geometric を追加
\usepackage{array}
\usepackage{tabularx}
\usepackage{booktabs}
\usepackage{makecell}
\usepackage{mathtools}
\usepackage{longtable}
\usepackage{pdfpages}
\usepackage{etoolbox} % AtBeginEnvironment 等
\usepackage[normalem]{ulem}
\usetikzlibrary{calc}
\usetikzlibrary{backgrounds}

\usepackage{pgf}

% マーカー(ハイライター)の定義
\newcommand{\markline}[2][yellow]{%
  \tikz[baseline=(X.base)]{%
    \node[inner sep=0pt,outer sep=0pt] (X) {#2};
    \begin{scope}[on background layer]
      \fill[#1, opacity=0.35, rounded corners=0.8pt]
        ([xshift=-0.15em,yshift=0.00ex]X.south west) rectangle
        ([xshift= 0.15em,yshift=2.15ex]X.south east);
    \end{scope}
  }%
}



% minted(※ -shell-escape 必須)
\usepackage{minted}
\setminted{
  frame=single,
  framesep=2mm,
  fontsize=\footnotesize,
  breaklines=true
}

% (必要なときだけ)tcolorbox
\usepackage[most]{tcolorbox}
\tcbuselibrary{skins}        % 高度なデザイン機能
\tcbuselibrary{raster}

% hyperref は最後
\usepackage{hyperref}

% --- 1. カラーパレットの定義 ---
\definecolor{CanvaGreen}{HTML}{2E7D32} % メインの濃い緑(タイトル・枠線)
\definecolor{PaleGreen}{HTML}{F1F8E9}  % 背景の薄い緑
\definecolor{MyDarkGreen}{HTML}{587a7f}
\definecolor{DeepText}{HTML}{1C1C1C}   % 本文の文字色
\definecolor{MyWhiteBlue}{HTML}{F2FAFB} % 追加:あなたが指定した色
\definecolor{BananaColor}{HTML}{FFFD78}


% --- 2. tcolorbox のデフォルト設定(全ボックスに適用) ---
%\tcbset{
%    enhanced,                          % 高度な装飾を有効化
%    colback=white,                 % 本文背景色
%    colframe=MyDarkGreen,               % 枠線の色
%    coltitle=white,                    % タイトル文字色
%    fonttitle=\bfseries\sffamily,      % タイトルを太字・ゴシック
%    boxrule=1pt,                       % 枠線の太さ
%    arc=2mm,                           % 角の丸み
%    left=3mm, right=3mm,               % 左右の余白
%    top=0.5mm, bottom=0.5mm,               % 本文の上下余白
%    toptitle=0.8mm, bottomtitle=0.5mm, % タイトル内の上下余白
%    before skip=0.8em, after skip=0.2em,   % ボックス前後の行間
%    shadow={0mm}{0mm}{0mm}{black!0}    % 影を完全に消してフラットに
%}
% --- 2. tcolorbox のデフォルト設定 ---
\tcbset{
    enhanced,
    colback=MyWhiteBlue,            % 本文背景を F2FAFB に変更
    colframe=MyDarkGreen,
    coltitle=white,                 % タイトル文字は白(背景が濃い色の場合)
    % coltitle=DeepText,            % もしタイトル背景も薄くするならこちら
    fonttitle=\bfseries\sffamily\small, % タイトルを少し小さくしてスリム化
    boxrule=0.5pt,                  % 1pt から 0.5pt へ細分化
    arc=1mm,                        % 2mm から 1mm へ変更
    sharp corners=south,            % 下側を直角に固定
    % --- 余白の調整 ---
    left=3mm, right=3mm,
    top=0.5mm, bottom=0.5mm,
    toptitle=0mm,                   % タイトル上の余白をゼロに
    bottomtitle=0mm,                % タイトル下の余白をゼロに
    % ------------------
    before skip=0.8em, after skip=0.2em,
    shadow={0mm}{0mm}{0mm}{black!0}
}

% --- カスタムボックスの定義 ---
\newtcolorbox{myListbox}[1]{
  enhanced,
  detach title,              % 1. 標準のタイトル位置を解除
  % 2. 本文が始まる直前(before upper)でタイトルを直接描画する
  before upper={{\bfseries\large #1}\par\medskip},
  % タイトルの後に改行を入れる設定
  after title={\par\medskip},
  colbacktitle=white, 
  colframe=gray!50,
  colback=white,
  titlerule=0pt,
  boxrule=1pt,
  fonttitle=\bfseries\color{black},
  title=#1,
  after skip=1.5ex,
  % --- 箇条書きの余白を強制的にゼロにする設定 ---
  before upper={
    \setbeamertemplate{itemize ispan}{0pt} % 項目間の余白
    \setbeamertemplate{itemize items}[default]
    \setlength{\leftmargini}{1.5em}
    % Beamerの内部変数を直接操作して行間を詰める
    \addtobeamertemplate{itemize/enumerate body begin}{}{\setlength{\itemsep}{0pt}\setlength{\parskip}{0pt}}
  }
}

%--------------------------
% パス
%--------------------------
\newcommand{\assetpath}{/Volumes/NBPlan/TTC/授業資料/2025年度/}
\graphicspath{{images/}{\assetpath/1020201.アルゴリズム2/08/images/}{../project_assets/images/}{../project_assets/emoji/emoji_pngs/}}
% PGFがある特定の階層を定義
\newcommand{\pgfpath}{\assetpath/1020201.アルゴリズム2/08/images/}

%--------------------------
% フッター
%--------------------------
\newcommand{\myfootertext}{1020201.アルゴリズム2/08}
\setbeamertemplate{footline}{%
  \leavevmode
  \hbox to \paperwidth{%
    \hspace*{0.2cm}
    \scriptsize\color{gray!50} \myfootertext
    \hfill
    \scriptsize\color{gray} \insertframenumber{} / \inserttotalframenumber
    \hspace*{0.4cm}
  }%
  \vspace{1pt}
}

%--------------------------
% TeacherFrame(外部)
%--------------------------
\usepackage{../teacherframe}

%--------------------------
% フレームタイトル:番号. タイトル
% ※ ここで出すだけ。insertframetitle を再定義しない(安全)
%--------------------------
\setbeamertemplate{frametitle}{%
  \vspace{0.6ex}%
  \begin{beamercolorbox}[wd=\paperwidth,sep=0.5ex,leftskip=0.9em,rightskip=0.5em]{frametitle}%
    \usebeamerfont{frametitle}%
    \insertframenumber.\,\insertframetitle%
  \end{beamercolorbox}%
}

%--------------------------
% 表用:列型
%--------------------------
\newcolumntype{C}[1]{>{\centering\arraybackslash}p{#1}}
\newcolumntype{M}[1]{>{\raggedright\arraybackslash}m{#1}}

%--------------------------
% ブロック(必要なら)
%--------------------------
\definecolor{myblue}{HTML}{7488FF}
\definecolor{mylightblue}{HTML}{E3EEFF}
\setbeamertemplate{blocks}[rounded]
\setbeamercolor{block title}{bg=myblue, fg=white}
\setbeamercolor{block body}{bg=mylightblue, fg=black}

%========================================================
% exampleblock(examplebox相当)だけの調整
%  - タイトル文字:白
%  - 背景色:現行のまま(bgは指定しない)
%  - 本文 itemize:文字も●も黒(exampleblock内だけ)
%========================================================

% タイトル文字だけ白(背景は触らない)
\setbeamercolor{block title example}{fg=white}

% 本文の通常文字色は「現行のまま」を基本にする(必要なら黒にしてもよい)
% ここは bg を触らないのが目的なので fg だけ調整可能
\setbeamercolor{block body example}{fg=black}

% exampleblock の中だけ itemize の色(●と文字)を黒に
\AtBeginEnvironment{exampleblock}{%
  \setbeamercolor{itemize item}{fg=black}
  \setbeamercolor{itemize subitem}{fg=black}
  \setbeamercolor{itemize subsubitem}{fg=black}
  \setbeamercolor{item}{fg=black} % 念のため
}

% exampleblock を抜けたらテーマ標準に戻す(色が残る事故防止)
\AtEndEnvironment{exampleblock}{%
  \setbeamercolor{itemize item}{fg=normal text.fg}
  \setbeamercolor{itemize subitem}{fg=normal text.fg}
  \setbeamercolor{itemize subsubitem}{fg=normal text.fg}
  \setbeamercolor{item}{fg=normal text.fg}
}

%--------------------------
% 奇数ページのスライドのを表示する
%(教示用だけでそれ以外はこの処理は動かない)
%--------------------------
% --- 教師用だけ、スライドを奇数開始に強制するトグル ---
\newif\ifoddslideenforce
\oddslideenforcefalse   % デフォルトOFF(pr/hoはOFF)

% --- 再帰防止ガード ---
\newif\ifoddslideguard
\oddslideguardfalse

% --- 偶数ページなら空白スライドを1枚入れて奇数に戻す ---
\newcommand{\ensureoddslide}{%
  \ifoddslideguard\relax\else
    \oddslideguardtrue
    \ifodd\value{page}\relax
      % 何もしない(次が奇数)
    \else
      \begin{frame}[plain,noframenumbering]
        \note{}% notes出力時に2枚消費させる保険
      \end{frame}
    \fi
    \oddslideguardfalse
  \fi
}

% --- frameが始まる直前に自動挿入(教師用だけ)---
\BeforeBeginEnvironment{frame}{%
  \ifoddslideenforce
    \ensureoddslide
  \fi
}

%--------------------------
% note / noteT の「常時安全化」
%  - tech 以外:\noteT は無視(エラーにならない)
%  - tech:notesmode_tech で上書き定義
%--------------------------
\providecommand{\notetitletext}{}      % 既にあっても衝突しない
\providecommand{\noteT}[2]{}           % デフォルトは何もしない

% frame開始ごとにタイトル変数をクリア(前の noteT が残らないように)
\AtBeginEnvironment{frame}{\gdef\notetitletext{}}

%--------------------------
% 切替(Pythonから差し込み)
%--------------------------
\mypausemodetrue
\teachermodetrue
\setbeameroption{show notes}
%-------------

% --- tech のときだけ noteT を有効化(テンプレートの \providecommand を上書き) ---
\makeatletter
\renewcommand{\noteT}[2]{%
 \gdef\notetitletext{#1}%
 \note{#2}%
}
% タイトル未指定のときのために初期化
\renewcommand{\notetitletext}{}%

\setbeamertemplate{note page}{%
 \begin{minipage}{\linewidth}
 \vspace{1.2ex} % タイトルを少し下げる(必要に応じて調整)
 {\Large\bfseries
 \ifx\notetitletext\@empty
 \insertframetitle
 \else
 \notetitletext
 \fi
 }\par
 \vspace{-1.2ex}
 \rule{\linewidth}{0.8pt}\par
 \vspace{0.8ex}
 {\scriptsize \insertnote}
 \end{minipage}
}
\makeatother

%教師用のPDFは奇数ページからスライドを出力
\oddslideenforcetrue


%--------------------------
% 方眼紙(グリッド)をスライドに重ね
%--------------------------
\input{grid_debug}

%-------------------------------------------------------------------------
% 「ラインマーカー」そのものです。線の太さ・色・透明度を自由にできます。
%
% #1 色(省略可)
% #2 下端 yshift
% #3 上端 yshift
% #4 文字
% \marklineA{0.35ex}{1.55ex}{通常サイズ}
% {\Large \marklineA{0.45ex}{2.10ex}{大きい文字}}
%-------------------------------------------------------------------------
\newcommand{\marklineA}[4][yellow]{%
  \tikz[baseline=(X.base)]{%
    \node[inner sep=0pt,outer sep=0pt] (X) {#4};
    \begin{scope}[on background layer]
      \fill[#1, opacity=0.35, rounded corners=0.8pt]
        ([xshift=-0.15em,yshift=#2]X.south west) rectangle
        ([xshift= 0.15em,yshift=#3]X.south east);
    \end{scope}
  }%
}

%----------------------------------------------------------------------------------------
% タイトル
%----------------------------------------------------------------------------------------
\title{ 08 大数の法則と中心極限定理 }
\date{}
\newcommand{\codedir}{\assetpath/1020201.アルゴリズム2/08}

\begin{document}

\begin{frame}[plain,noframenumbering]
  \titlepage
  \bigskip
  \begin{center}
    \ifteachermode 教師用 \fi
  \end{center}
\end{frame}

% セクションページ(必要なら)
\setbeamertemplate{section page}{
  \begin{centering}
    \vfill
    \rule{\linewidth}{2pt}\par
    \vspace{1ex}
    {\usebeamerfont{section title}\Huge\bfseries \insertsection}\par
    \vspace{1ex}
    \rule{\linewidth}{2pt}\par
    \vfill
  \end{centering}
}
\setbeamerfont{section title}{size=\LARGE,series=\bfseries}

\AtBeginSection[]{
  \begin{frame}[plain,noframenumbering]
    \sectionpage
  \end{frame}
}

% 本編開始でフレーム番号を0から(必要なら)
\setcounter{framenumber}{0}

\input{emoji_macros}

% @@@--(metropolis)--@@@
% ----------------------------------------------------------------------------------------
%   Slide 01: 今日のテーマ(第7回)
% ----------------------------------------------------------------------------------------

%@@PAGEBAND@@
% ----------------------------------------------------------------------------------------
%   page 01
% ----------------------------------------------------------------------------------------
\begin{frame}{平均は本当に安定するのか?}
今日のテーマは、とても素朴な疑問です。

\vspace{0.8em}
\begin{center}
\Large
\textbf{「平均をとれば、もう安心」\\
それは本当でしょうか?}
\end{center}

\vspace{1.0em}
これまでの授業では、
\begin{itemize}
  \item データを集める
  \item 平均を計算する
\end{itemize}
ところまでは、当たり前のように行ってきました。

\noteT{狙い}{
平均は「安定した代表値」という思い込みを、
あえて言葉にして表に出す。
ここではまだ否定しない。
}
\end{frame}
% ----------------------------------------------------------------------------------------
%   Slide 02: ポテト調査を思い出そう
% ----------------------------------------------------------------------------------------

%@@PAGEBAND@@
% ----------------------------------------------------------------------------------------
%   page 02
% ----------------------------------------------------------------------------------------
\begin{frame}{ポテト30人分の調査を思い出そう}
ハンバーガーショップのポテト調査では──

\vspace{0.6em}
\begin{itemize}
  \item 1日につき \textbf{30人分} の重さを測った
  \item その30人分をまとめて \textbf{平均} を計算した
\end{itemize}

\vspace{0.8em}
ここで自然に出てくる考えは:

\vspace{0.6em}
\begin{center}
\Large
\textbf{「30人も集めれば、\\
平均はほぼ同じになるのでは?」}
\end{center}

\noteT{狙い}{
学生が「そう思っていた」と共感できる
自然な予想を言語化する。
}
\end{frame}

% ----------------------------------------------------------------------------------------
%   Slide 02.5: 平均に対する直感
% ----------------------------------------------------------------------------------------

%@@PAGEBAND@@
% ----------------------------------------------------------------------------------------
%   page 03
% ----------------------------------------------------------------------------------------
\begin{frame}{多くの人が、こう考えています}
平均について、私たちはついこう考えがちです。

\vspace{0.8em}
\begin{itemize}
  \item 人数が少ないと、平均はブレやすい
  \item 人数が多くなると、平均は安定する
  \item 30人も集めれば、もう十分では?
\end{itemize}

\vspace{1.0em}
\begin{center}
\Large
\textbf{この考えは、感覚的にはとても自然}
\end{center}

\vspace{0.6em}
しかし──

\noteT{狙い}{
学生が「自分もそう思っていた」と
安心してうなずけるスライド。
ここでは否定しない。
}
\end{frame}

% ----------------------------------------------------------------------------------------
%   Slide 03: 今日の問い
% ----------------------------------------------------------------------------------------

%@@PAGEBAND@@
% ----------------------------------------------------------------------------------------
%   page 04
% ----------------------------------------------------------------------------------------
\begin{frame}{今日、確かめたいこと}
そこで、今日は次のことを確かめます。

\vspace{0.8em}
\textbf{問い①}
\begin{itemize}
  \item 30人分の平均は、毎回ほぼ同じになるのか?
  \item それとも、まだ少しずつ違うのか?
\end{itemize}

\vspace{0.8em}
\textbf{問い②}
\begin{itemize}
  \item 平均を \textbf{何回も} 作ったら、どうなるのか?
  \item その集まりには、何か特徴が現れるのか?
\end{itemize}

\vspace{0.8em}
\begin{center}
\textbf{今日は、計算ではなく「実験」で確かめます}
\end{center}

\noteT{狙い}{
この後の実習①・②が、
単なるExcel操作ではなく
「問いに答える実験」だと理解させる。
}
\end{frame}

% ----------------------------------------------------------------------------------------
%   Slide X: では、実験してみよう
% ----------------------------------------------------------------------------------------

%@@PAGEBAND@@
% ----------------------------------------------------------------------------------------
%   page 05
% ----------------------------------------------------------------------------------------
\begin{frame}{では、実験してみよう}
ここまでの話は、すべて \textbf{予想} です。

\vspace{0.8em}
\begin{itemize}
  \item 30人も平均すれば安定する?
  \item 日が違っても、平均は同じ?
\end{itemize}

\vspace{1.0em}
\begin{center}
\Large
\textbf{実際のデータで、確かめてみます}
\end{center}

\vspace{0.8em}
今日の実習では、
\begin{itemize}
  \item 1日分(30人)の平均を作り
  \item それを日ごとに比べます
\end{itemize}

\noteT{狙い}{
「実習=作業」ではなく
「予想を検証する実験」だと位置づける。
}
\end{frame}

% ----------------------------------------------------------------------------------------
%   実習①-1:日別平均を調べる
% ----------------------------------------------------------------------------------------

%@@PAGEBAND@@
% ----------------------------------------------------------------------------------------
%   page 06
% ----------------------------------------------------------------------------------------
\begin{frame}{実習①:日別平均は本当に同じか?}
\textbf{今日の実習の目的}

\vspace{0.6em}
\begin{itemize}
  \item ポテト30人分の重さデータがある
  \item 1日分(30人)をまとめて \textbf{平均} を計算する
  \item 日ごとの平均を比べてみる
\end{itemize}

\vspace{0.8em}
\textbf{確認したいこと}

\begin{itemize}
  \item 30人も平均すれば、毎回ほぼ同じになる?
  \item それとも、まだ \textbf{揺れ} が残る?
\end{itemize}

\noteT{狙い}{
「平均=安定しそう」という直感を一度信じさせてから、
実際のデータで揺れを確認させる。
}
\end{frame}
% ----------------------------------------------------------------------------------------
%   実習①-2:日別平均を計算する
% ----------------------------------------------------------------------------------------

%@@PAGEBAND@@
% ----------------------------------------------------------------------------------------
%   page 07
% ----------------------------------------------------------------------------------------
\begin{frame}[fragile]{実習①:日別平均を計算する}
\textbf{手順①:各日の平均を出す}

\vspace{-0.8em}
\begin{enumerate}
  \item poteto30.xlsx を開く
  \item G4〜M4に月〜日の値を入れる
  \item G5〜M5に式をセット
  \item 曜日別の平均を計算する
\end{enumerate}

\vspace{0.2em}
\textbf{例(月曜日 の場合)}
\vspace{-0.4em}
%\begin{itemize}
%  \item セルに次の式を入力:
%\end{itemize}
%\vspace{-0.8em}
\hspace*{2em}%
\begin{tcolorbox}[title=セルに次の式を入力]
\texttt{=AVERAGEIF(B:B,G4,C:C)}
)
\end{tcolorbox}
%%% \markline{\texttt{=AVERAGEIF(B:B,G4,C:C)}}

%\begin{minipage}{0.90\linewidth}
%\begin{block}{}
%\ttfamily
%=AVERAGEIF(B:B,G4,C:C)
%\end{block}
%\end{minipage}
%\markline{\texttt{=AVERAGEIF(B:B,G4,C:C)}}
% \begin{center}
% \texttt{=AVERAGE(B2:B31)}
% \end{center}

\vspace{0.8em}
\textbf{月〜日 すべてについて平均を求める}

\noteT{操作}{
「30人分=1日分の調査」という考えを強調。
まだサンプル・母集団という言葉は使わない。
}
\end{frame}
% ----------------------------------------------------------------------------------------
%   実習①-3:日別平均を比べる
% ----------------------------------------------------------------------------------------

%@@PAGEBAND@@
% ----------------------------------------------------------------------------------------
%   page 08
% ----------------------------------------------------------------------------------------
\begin{frame}{実習①:日別平均を比べてみる}
\textbf{手順②:平均を並べて確認}

\vspace{0.6em}
\begin{itemize}
  \item 月曜日〜日曜日 の平均値を横に並べる
  \item 数値をそのまま見比べる
\end{itemize}

\vspace{0.8em}
\textbf{余裕があれば}

\begin{itemize}
  \item 日別平均の折れ線グラフを作成する
  \item 縦軸:平均値/横軸:日
\end{itemize}

\vspace{0.8em}
\textbf{確認}

\begin{itemize}
  \item 平均は毎回まったく同じ?
  \item 少しずつ上下に動いていない?
\end{itemize}

\noteT{狙い}{
「平均をとっても揺れる」という事実を、
理屈なしで体験させる。
次の実習への違和感を残す。
}
\end{frame}

% ----------------------------------------------------------------------------------------
%   実習②-1:平均をたくさん作る
% ----------------------------------------------------------------------------------------

%@@PAGEBAND@@
% ----------------------------------------------------------------------------------------
%   page 09
% ----------------------------------------------------------------------------------------
\begin{frame}{実習②:平均をたくさん作る}
\textbf{ここからの実習の目的}

\vspace{0.6em}
\begin{itemize}
  \item 実習①では、日ごとの平均が \textbf{揺れる} ことを確認した
  \item では、平均を \textbf{何回も} 作るとどうなるか?
\end{itemize}

\vspace{0.8em}
\textbf{今日やること}

\begin{itemize}
  \item 30人分のポテトデータから
  \item 「1回分の平均」を何度も作る
  \item その平均たちを集めて眺める
\end{itemize}

\noteT{狙い}{
「平均は1つの値ではなく、集めると分布になる」
という感覚を体験で作る。
}
\end{frame}

% ----------------------------------------------------------------------------------------
%   実習②-2:平均を1回作る(関数方式)
% ----------------------------------------------------------------------------------------

%@@PAGEBAND@@
% ----------------------------------------------------------------------------------------
%   page 10
% ----------------------------------------------------------------------------------------
\begin{frame}[fragile]{実習②:平均を1回作る(関数で一発)}
\textbf{手順①:抽出件数を決める}

\vspace{0.3em}
\begin{itemize}
  \item \textbf{B1セル}に「取り出す件数」を入力(例:30)
\end{itemize}

\vspace{0.6em}
\textbf{手順②:ランダムに抜き出した平均を1つ作る}

\vspace{0.2em}
\begin{itemize}
  \item \textbf{B2セル}に次の式を入力(A2:A211が210件のデータ)
\end{itemize}

\vspace{-0.2em}
%	\begin{minipage}{0.95\linewidth}
%	\begin{block}{B2に入力する式}
%	\ttfamily
%	=AVERAGE(\\
%	\ \ TAKE(\\
%	\ \ \ \ SORTBY(\$A\$2:\$A\$211, RANDARRAY(ROWS(\$A\$2:\$A\$211))),\\
%	\ \ \ \ \$B\$1\\
%	\ \ )\\
%	)
%	\end{block}
%	\end{minipage}

\begin{tcolorbox}[title=B2に入力する式]
=AVERAGE(\\
\hspace{20pt} TAKE(\\
\hspace{40pt} SORTBY(\$A\$2:\$A\$211, RANDARRAY(ROWS(\$A\$2:\$A\$211))),\\
\hspace{40pt} \$B\$1\\
\hspace{20pt} )\\
)
\end{tcolorbox}

\vspace{0.4ex}
\textbf{確認:} B2に「1回分の平均」が表示される

\noteT{狙い}{
並べ替え作業を\textbf{関数に隠す}ことで、実習を「作業」から「観察」へ寄せる。
B1=30は「30個取り出した平均」を意味する(用語はまだ出さない)。
}
\end{frame}

% ----------------------------------------------------------------------------------------
%   実習②-3:平均を何回も作る(コピーで量産)
% ----------------------------------------------------------------------------------------

%@@PAGEBAND@@
% ----------------------------------------------------------------------------------------
%   page 11
% ----------------------------------------------------------------------------------------
\begin{frame}{実習②:平均を何回も作る(コピーで量産)}
\textbf{手順③:平均をたくさん作る}

\vspace{0.6em}
\begin{itemize}
  \item B2の式を \textbf{下方向にコピー}して平均を量産する
  \item 例:B2〜B51(\textbf{50回分の平均})
\end{itemize}

\vspace{0.8em}
\textbf{ポイント}

\begin{itemize}
  \item コピーした各セルは、毎回ランダムに取り出すので値が少し変わる
  \item これらの平均値1つ1つを「結果」として集める
\end{itemize}

\noteT{操作}{
値が動いて見えたらOK。
動きが気になる場合は「コピー→値貼り付け」で固定してもよい(任意)。
}
\end{frame}

% ----------------------------------------------------------------------------------------
%   実習②-4:平均の集まりを見る(まずは数字で観察)
% ----------------------------------------------------------------------------------------

%@@PAGEBAND@@
% ----------------------------------------------------------------------------------------
%   page 12
% ----------------------------------------------------------------------------------------
\begin{frame}{実習②:平均の集まりを観察する}
\textbf{手順④:平均の列を眺める}

\vspace{0.6em}
\begin{itemize}
  \item B2〜B51 に平均値が並んだ(例:50個)
  \item まずは「どのあたりに集まっているか」を数字で確認する
\end{itemize}

\vspace{0.8em}
\textbf{確認}

\begin{itemize}
  \item 平均は150g付近に集まっている?
  \item 実習①(日別平均)より揺れは小さく見える?
\end{itemize}

\noteT{狙い}{
「平均も集めると分布になる」を体験する直前段階。
ここではまだヒストグラムを作らず、数値の集まりを観察する。
}
\end{frame}

% ----------------------------------------------------------------------------------------
%   実習②-5:平均のヒストグラム(関数方式に合わせた版)
% ----------------------------------------------------------------------------------------

%@@PAGEBAND@@
% ----------------------------------------------------------------------------------------
%   page 13
% ----------------------------------------------------------------------------------------
\begin{frame}{実習②:平均のヒストグラムを作る}
\textbf{手順⑤:平均値(B列)だけでヒストグラム}

\vspace{0.3ex}
\begin{itemize}
  \item B2 から下にコピーして作った「平均値の列」を使う
  \item 例:\textbf{B2:B51(50回分の平均)}
\end{itemize}

\vspace{0.4ex}
\textbf{操作(Excel)}
\begin{enumerate}
  \item 範囲 \textbf{B2:B51} を選択
  \item \textbf{挿入} → \textbf{統計グラフ} → \textbf{ヒストグラム}
  \item (必要なら)\textbf{軸の書式設定}で\textbf{ビン幅}や\textbf{ビン数}を調整
\end{enumerate}

\vspace{0.4ex}
\textbf{観察ポイント}
\begin{itemize}
  \item 平均値は \textbf{どのあたり} に集まっている?(150g付近?)
  \item \textbf{広がり(ばらつき)} は大きい?小さい?
  \item 元データのヒストグラム(ポテトの重さ)と比べて、\textbf{形}はどう違う?
\end{itemize}

\noteT{狙い}{
「平均を集めると分布になる」を視覚で体験させる。
ここでは用語(中心極限定理)は言わず、
「平均のヒストグラムは、元データより細くまとまる」感覚を作る。
}
\end{frame}

% ----------------------------------------------------------------------------------------
%   Slide J: 実習で何が起きたか(事実の整理)
% ----------------------------------------------------------------------------------------

%@@PAGEBAND@@
% ----------------------------------------------------------------------------------------
%   page 14
% ----------------------------------------------------------------------------------------
\begin{frame}{実習で何が起きたか}
ここまでの実習で、次のことを確認しました。

\vspace{0.6em}
\begin{itemize}
  \item 30人分の平均でも、値は \textbf{毎回少しずつ違った}
  \item 平均を1回だけ見ると、\textbf{まだ揺れが残る}
  \item 平均をたくさん集めると、\textbf{中心に集まった}
\end{itemize}

\vspace{0.8em}
\textbf{重要:}
\begin{center}
%%%\begin{tcolorbox}[hbox, sharp corners, arc=0mm]
\begin{tcolorbox}[width=7cm,  colback=BananaColor,sharp corners, arc=0mm]
\centering
平均は「固定の値」ではなく、\\
\textbf{揺れを持った結果のひとつ}
\end{tcolorbox}
\end{center}

\noteT{狙い}{
結果の評価を入れず、「事実」だけを整理する。
次の「なぜ?」に自然につなぐ。
}
\end{frame}

% ----------------------------------------------------------------------------------------
%   Slide K: 平均もデータである
% ----------------------------------------------------------------------------------------

%@@PAGEBAND@@
% ----------------------------------------------------------------------------------------
%   page 15
% ----------------------------------------------------------------------------------------
\begin{frame}{平均も「1つのデータ」}
ここで見方を少し変えます。

\vspace{0.6em}
\begin{itemize}
  \item ポテトの重さ → データ
  \item 30人分の平均 → \textbf{これも1つのデータ}
\end{itemize}

\vspace{0.8em}
\textbf{実習②でやったこと}

\begin{itemize}
  \item 平均を \textbf{何回も} 作った
  \item その平均たちを \textbf{集めた}
\end{itemize}

\vspace{0.8em}
%	\begin{center}
%	\textbf{平均の集まりも、分布になる}
%	\end{center}

\begin{center}
\begin{tcolorbox}[width=7cm,  colback=BananaColor,sharp corners, arc=0mm]
\centering
\textbf{平均の集まりも、分布になる}
\end{tcolorbox}
\end{center}


\noteT{狙い}{
「平均=答え」から
「平均=結果のひとつ」へ視点をずらす。
}
\end{frame}

% ----------------------------------------------------------------------------------------
%   Slide L: 平均の分布の特徴
% ----------------------------------------------------------------------------------------

%@@PAGEBAND@@
% ----------------------------------------------------------------------------------------
%   page 16
% ----------------------------------------------------------------------------------------
\begin{frame}{平均の分布で見えた特徴}
平均だけを集めたヒストグラムを見ると──

\vspace{0.6em}
\begin{itemize}
  \item 真ん中あたりに集まっている
  \item 元データより \textbf{広がりが小さい}
  \item なめらかな \textbf{山の形} になっている
\end{itemize}

\vspace{0.8em}
\textbf{比較すると}

\begin{itemize}
  \item 元の重さデータ:ばらつきが大きい
  \item 平均のデータ:ばらつきが小さい
\end{itemize}

\noteT{狙い}{
「平均をとると安定する」
という現象を、視覚の記憶として固定する。
}
\end{frame}

% ----------------------------------------------------------------------------------------
%   Slide M: ここで生まれる疑問
% ----------------------------------------------------------------------------------------

%@@PAGEBAND@@
% ----------------------------------------------------------------------------------------
%   page 17
% ----------------------------------------------------------------------------------------
\begin{frame}{ここで生まれる疑問}
実習結果を見て、次の疑問が自然に出てきます。

\vspace{0.6em}
\begin{itemize}
  \item なぜ平均を集めると、中心に集まるのか?
  \item なぜ回数が増えるほど、揺れが小さくなるのか?
  \item なぜ山の形が、だんだん整ってくるのか?
\end{itemize}

\vspace{0.8em}
\begin{center}
\begin{tcolorbox}[width=7cm,  colback=BananaColor,sharp corners, arc=0mm]
\centering
\textbf{これは偶然ではない}
\end{tcolorbox}
\end{center}


\noteT{狙い}{
「不思議だ」で止めず、
理論が必要になる地点まで導く。
}
\end{frame}

% ----------------------------------------------------------------------------------------
%   Slide N: この現象には名前がある
% ----------------------------------------------------------------------------------------

%@@PAGEBAND@@
% ----------------------------------------------------------------------------------------
%   page 18
% ----------------------------------------------------------------------------------------
\begin{frame}{この現象には、名前がある}
今見てきた現象は、  
統計の世界では \textbf{よく知られた性質} です。

\vspace{0.8em}
\begin{itemize}
  \item 平均をとると安定する
  \item 回数を増やすほど、その傾向は強くなる
\end{itemize}

\vspace{0.8em}
この考え方を  
\begin{center}
\begin{tcolorbox}[width=7cm,  colback=BananaColor,sharp corners, arc=0mm]
\centering
\textbf{大数の法則}
\end{tcolorbox}
\end{center}
と呼びます。

\noteT{狙い}{
ここで初めて理論名を出す。
実験→現象→命名、の順を守る。
}
\end{frame}

% ----------------------------------------------------------------------------------------
%   Slide O: 平均の分布が整う理由(中心極限定理への橋)
% ----------------------------------------------------------------------------------------

%@@PAGEBAND@@
% ----------------------------------------------------------------------------------------
%   page 19
% ----------------------------------------------------------------------------------------
\begin{frame}{なぜ、平均の分布は整っていくのか}
ここで、平均の作られ方を思い出します。

\vspace{0.6em}
\begin{itemize}
  \item 1回の平均は、\textbf{30人分の重さ}を使っている
  \item つまり、\textbf{30個の値を足して割ったもの}
\end{itemize}

\vspace{0.8em}
\textbf{言い換えると:}

\begin{itemize}
  \item 平均は「たくさんの小さなズレ」をまとめた結果
  \item ズレが正にも負にもあると、\textbf{打ち消し合う}
\end{itemize}

\vspace{0.8em}
\begin{center}
\begin{tcolorbox}[width=7cm,  colback=BananaColor,sharp corners, arc=0mm]
\centering
\textbf{その結果、極端な値が出にくくなる}
\end{tcolorbox}
\end{center}

\noteT{狙い}{
「平均=足し算の集まり」であることを明確にし、
偶然ではないことを直感的に理解させる。
}
\end{frame}

% ----------------------------------------------------------------------------------------
%   Slide O': 中心極限定理とは何か(直感版)
% ----------------------------------------------------------------------------------------

%@@PAGEBAND@@
% ----------------------------------------------------------------------------------------
%   page 20
% ----------------------------------------------------------------------------------------
\begin{frame}{中心極限定理(直感的な説明)}
このように、

\vspace{0.6em}
\begin{itemize}
  \item たくさんの値を使って平均を作ると
  \item 元のデータの形に関係なく
  \item 平均の分布は \textbf{似た形} になっていく
\end{itemize}

\vspace{0.8em}
この性質を
\vspace{-0.4ex}
\begin{center}
\begin{tcolorbox}[width=7cm,  colback=BananaColor!70,sharp corners, arc=0mm]
\centering
\textbf{中心極限定理}
\end{tcolorbox}
\end{center}
\vspace{0.2ex}
と呼びます。

\vspace{0.6em}
※ 今日は「なぜ起きるか」を体験で理解する回  
  数式での説明は次回以降に扱います

\noteT{狙い}{
CLTを「公式」ではなく
「平均を集めたときに必ず起きる性質」として定義する。
}
\end{frame}

% ----------------------------------------------------------------------------------------
%   Slide P: 今日の整理
% ----------------------------------------------------------------------------------------

%@@PAGEBAND@@
% ----------------------------------------------------------------------------------------
%   page 21
% ----------------------------------------------------------------------------------------
\begin{frame}{今日の整理}
今日の実習と整理から、次のことが言えます。

\vspace{0.6em}
\begin{itemize}
  \item 平均は1回では揺れる
  \item たくさん集めると安定する
  \item 平均の分布には決まった性質がある
\end{itemize}

\vspace{0.8em}
\begin{center}
\begin{tcolorbox}[width=7cm,  colback=BananaColor!70,sharp corners, arc=0mm]
\centering
\textbf{だから平均は、判断に使える}
\end{tcolorbox}
\end{center}

\noteT{狙い}{
「平均が使える理由」を
体験ベースで納得させる。
}
\end{frame}

% ----------------------------------------------------------------------------------------
% ===== 第7回:実習後まとめ(Slide Q〜S)=====
%   Slide Q: 今日わかったこと(整理)
% ----------------------------------------------------------------------------------------

%@@PAGEBAND@@
% ----------------------------------------------------------------------------------------
%   page 22
% ----------------------------------------------------------------------------------------
\begin{frame}{今日わかったこと(実習の整理)}
\textbf{実習①:日別平均を見た}

\vspace{0.4em}
\begin{itemize}
  \item 30人分の平均でも、日によって \textbf{少し揺れる}
  \item 平均は「固定の値」ではなく、\textbf{毎回少し変わる}ものだった
\end{itemize}

\vspace{0.7em}
\textbf{実習②:平均をたくさん作って集めた}

\vspace{0.4em}
\begin{itemize}
  \item 平均を繰り返し作ると、平均値にも \textbf{分布(山の形)}ができた
  \item 元データよりも、\textbf{平均の分布は広がりが小さく}なった
\end{itemize}

\vspace{0.8em}
\begin{center}
\textbf{まとめ:平均は「揺れる」けれど、集めると「安定のしかた」が見える}
\end{center}

\noteT{狙い}{
実習①・②を「何が起きたか」だけで整理する。
ここではまだ理由(理屈)に踏み込まず、現象の確定を優先する。
}
\end{frame}

% ----------------------------------------------------------------------------------------
%   Slide R: 大数の法則と中心極限定理(区別)
% ----------------------------------------------------------------------------------------

%@@PAGEBAND@@
% ----------------------------------------------------------------------------------------
%   page 23
% ----------------------------------------------------------------------------------------
\begin{frame}{大数の法則と中心極限定理:何が違う?}
今日見た現象には、\textbf{2つの名前}が関係します。

\vspace{0.7em}
\begin{columns}[T,onlytextwidth]
  \begin{column}{0.50\textwidth}
    \textbf{大数の法則(平均が安定)}
    \vspace{0.3em}
    \begin{itemize}
      \item 1回の平均を考える
      \item 人数(件数)を増やすほど
      \item 平均は \textbf{真ん中へ寄って安定}する
    \end{itemize}

    \vspace{0.4em}
    \textbf{キーワード:}
    \begin{itemize}
      \item 「平均との差が小さくなる」
      \item 「平均がブレにくくなる」
    \end{itemize}
  \end{column}

  \begin{column}{0.50\textwidth}
    \textbf{中心極限定理(平均の分布が山になる)}
    \vspace{0.3em}
    \begin{itemize}
      \item 平均を \textbf{何回も} 作る
      \item その平均たちを集めると
      \item 分布の形が \textbf{山に似てくる}
    \end{itemize}

    \vspace{0.4em}
    \textbf{キーワード:}
    \begin{itemize}
      \item 「平均にも分布がある」
      \item 「平均の分布は整ってくる」
    \end{itemize}
  \end{column}
\end{columns}

\vspace{0.7em}
\begin{center}
\begin{tcolorbox}[width=10cm,  colback=BananaColor!60,sharp corners, arc=0mm]
\centering
\textbf{大数の法則=「1つの平均が安定」}\\
\textbf{中心極限定理=「平均の集まりが山になる」}
\end{tcolorbox}
\end{center}


\noteT{狙い}{
両者をごちゃ混ぜにしないための最小の整理。
数式なしで、見た現象と対応づけるだけに徹する。
}
\end{frame}

% ----------------------------------------------------------------------------------------
%   Slide S: 次につながる意味(推測統計・検定への接続)
% ----------------------------------------------------------------------------------------

%@@PAGEBAND@@
% ----------------------------------------------------------------------------------------
%   page 24
% ----------------------------------------------------------------------------------------
\begin{frame}{次につながる意味:なぜこれが大事?}
今日の結論は、これです。

\vspace{0.6em}
\begin{itemize}
  \item 平均は「揺れる」\quad →\quad だから \textbf{1回の結果だけでは判断できない}
  \item しかし、平均の揺れ方には \textbf{規則性} がある
  \item 平均を集めると \textbf{山の形(分布)} が見えてくる
\end{itemize}

\vspace{0.8em}
\textbf{ここから先でやりたいこと}

\vspace{0.4em}
\begin{itemize}
  \item 「この平均は普通?それとも外れている?」
  \item 「どれくらい外れたら、外れていると言える?」
  \item 「偶然の揺れ」か「本当に違う」かを区別したい
\end{itemize}

\vspace{0.8em}
\begin{center}
\textbf{次回以降:標本と母集団 → 推測 → 仮説検定}\\
(今日の現象が、その土台になる)
\end{center}

\noteT{狙い}{
推測統計・検定の必然性を「平均は揺れる」から導く。
ここでは専門語(母集団など)を出す場合は一言に留め、次回以降に回す。
}
\end{frame}
\end{document}
