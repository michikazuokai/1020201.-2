% @@@--(metropolis)--@@@

% ============================================================
% 中盤・塊1:標本と母集団(危うさを作る)
% ============================================================
% ------------------------------------------------------------
% Slide 1:母集団と標本 ― まず言葉を決める
% ------------------------------------------------------------
\begin{frame}{母集団と標本:まず言葉を決める}
統計の話では、まず次の2つを区別します。

\vspace{0.8em}
\begin{itemize}
  \item \textbf{母集団}:本当は知りたい「全部」
  \item \textbf{標本}:実際に手に入る「一部」
\end{itemize}

\vspace{0.8em}
ポテトの重さの例で言うと──
\begin{itemize}
  \item 母集団:この店で提供される \textbf{すべてのポテト}
  \item 標本:その中から \textbf{実際に量った一部}
\end{itemize}

\vspace{0.8em}
\begin{center}
統計は、\textbf{一部(標本)から全部(母集団)を考える}ための考え方
\end{center}

\begin{tikzpicture}[remember picture,overlay]
  \node[anchor=north west]
    at ([xshift=8.7cm,yshift=-1.5cm]current page.north west)
    {\includegraphics[width=7cm]{母集団標本.png}};
\end{tikzpicture}



\noteT{狙い}{
ここで用語を固定する。
厳密な定義よりも「全部/一部」という直感を優先する。
以降、この2語を迷わず使える状態にする。
}
% 図の提案:
% ・大きな円(母集団)+中の点
% ・その中から数点だけを取り出して右側に並べる(標本)
\end{frame}

% \begin{frame}{母集団と標本:まず言葉を決める}
% 統計の話では、まず次の2つを区別します。
% \vspace{-0.4em}
% \begin{itemize}
%   \item \textbf{母集団}:本当は知りたい「全部」
%   \item \textbf{標本}:実際に手に入る「一部」
% \end{itemize}

% \vspace{-0.4em}
% ポテトの重さの例で言うと──
% \begin{itemize}
%   \item 母集団:この店で提供される \textbf{すべてのポテト}
%   \item 標本:その中から \textbf{実際に量った一部}
% \end{itemize}

% \vspace{-0.8em}
% \begin{center}
% 統計は、\textbf{一部(標本)から全部(母集団)を考える}ための考え方
% \end{center}


% \begin{center}
%     \includegraphics[scale=0.4]{母集団標本.png}
% \end{center}

% \noteT{狙い}{
% ここで用語を固定する。
% 厳密な定義よりも「全部/一部」という直感を優先する。
% 以降、この2語を迷わず使える状態にする。
% }

% \end{frame}

% ------------------------------------------------------------
% Slide 2:標本は偏る ― 同じ母集団でも結果が変わる
% ------------------------------------------------------------
\begin{frame}{標本は偏る:同じ母集団でも結果が変わる}
標本は「一部」である以上、\textbf{偏り}が生じます。

\vspace{0.8em}
\begin{itemize}
  \item たまたま軽い人が多い日
  \item たまたま重い人が多い日
\end{itemize}

\vspace{0.8em}
同じ母集団から取った標本でも、
\begin{itemize}
  \item 標本の中身は毎回同じとは限らない
  \item その結果、\textbf{計算した平均も変わる}
\end{itemize}

\vspace{0.8em}
\begin{center}
\textbf{標本だけを見て、母集団を断定するのは危険}
\end{center}

\begin{tikzpicture}[remember picture,overlay]
  \node[anchor=north west]
    at ([xshift=8.7cm,yshift=-1.5cm]current page.north west)
    {\includegraphics[width=7cm]{標本大小.png}};
\end{tikzpicture}


\noteT{狙い}{
「標本=答えではない」ことをはっきり言い切る。
ここで一度、推測の危うさをしっかり意識させる。
}

\end{frame}

% ------------------------------------------------------------
% Slide 3:ここで困る ― 全部は測れないのに判断が必要
% ------------------------------------------------------------
\begin{frame}{ここで困る:全部は測れないのに判断が必要}
%  \showgrid
理想を言えば──

\vspace{0.6em}
\begin{itemize}
  \item 母集団すべてを量れれば
  \item 迷わず正確に判断できる
\end{itemize}

\vspace{0.4em}
しかし現実には、
\begin{itemize}
  \item 時間・コストの制約がある
  \item 全数調査はほぼ不可能
\end{itemize}

\vspace{0.4em}
それでも、
\begin{itemize}
  \item 店の品質は判断したい
  \item 「だいたい問題ないか」を知りたい
\end{itemize}

\vspace{0.4em}
\begin{center}
\textbf{一部しか見られないのに、判断しなければならない}
\end{center}


\begin{tikzpicture}[remember picture,overlay]
  \node[anchor=north west]
    at ([xshift=8.7cm,yshift=-1.5cm]current page.north west)
    {\includegraphics[width=7cm]{数の比較.png}};
\end{tikzpicture}

\noteT{狙い}{
次に「では、何を頼りに判断するのか?」という
必然的な問いを生むためのスライド。
ここで止めず、次の塊(平均・CLT)へつなぐ。
}
\end{frame}

% ============================================================
% 中盤・塊2:だから平均を見る(第7回の体験と接続)
% ============================================================

% ------------------------------------------------------------
% Slide 4:なぜ「平均」に注目するのか
% ------------------------------------------------------------
\begin{frame}{なぜ「平均」に注目するのか} 
標本は偏る可能性があり、
\textbf{そのままでは判断に使いにくい}ことを見ました。

\vspace{0.8em}
そこで、次のように考えます。

\vspace{0.6em}
\begin{itemize}
  \item 1人分の重さはバラつく
  \item しかし、\textbf{何人分かをまとめる}とどうなるか?
\end{itemize}

\vspace{0.8em}
この「まとめた値」として使われるのが、
\begin{center}
%\markline{\Large\textbf{平均}}
{\marklineA{-1.0ex}{3.10ex}{{\Large\textbf{平均}}}}
\end{center}

\vspace{0.6em}
平均は、
\begin{itemize}
  \item 個々のバラつきをならし
  \item 全体の傾向を表そうとする量
\end{itemize}

\noteT{狙い}{
「平均は当たり前に使うもの」ではなく、
「なぜ使うのか」を言葉として再定義する。
}
% 図の提案:
% ・バラついた点(個人データ)→ 1つの代表値(平均)
\end{frame}

% ------------------------------------------------------------
% Slide 5:平均も1回では揺れる
% ------------------------------------------------------------
\begin{frame}{平均も1回では揺れる}
平均は「まとめた値」ですが、
\textbf{1回の計算で完全に安定するわけではありません}。

\vspace{0.8em}
\begin{itemize}
  \item どの人を含めるかによって
  \item 平均の値は \textbf{少しずつ変わる}
\end{itemize}

\vspace{0.8em}
つまり、
\begin{itemize}
  \item 平均も「1つの結果」
  \item 毎回まったく同じにはならない
\end{itemize}

\vspace{0.8em}
\begin{center}
\textbf{平均を1回見ただけでは、まだ不安}
\end{center}

\noteT{狙い}{
第7回・実習①で体験した
「平均も揺れる」という事実を、
ここで言葉として再接続する。
}
% 図の提案:
% ・日ごとに少しずつ違う平均値の並び
\end{frame}

% ------------------------------------------------------------
% Slide 6(差し替え):平均を集めると、様子が変わる(LLN/CLTの命名)
% ------------------------------------------------------------
\begin{frame}{平均を集めると、様子が変わる}
平均を \textbf{何回も} 作って集めると、
次のようなことが起きます。

\vspace{0.7em}
\begin{itemize}
  \item 平均値は、ある範囲に \textbf{集まっていく}
  \item 極端に大きい・小さい値は \textbf{出にくくなる}
  \item 集まり方は、\textbf{山のような形} に見えてくる
\end{itemize}

\vspace{0.8em}
この現象には、統計でよく使う \textbf{2つの名前} があります。

\vspace{0.4em}
\begin{itemize}
  \item \textbf{大数の定理}:人数(件数)を増やすほど、\textbf{平均が真ん中へ寄って安定する}
  \item \textbf{中心極限定理}:平均を集めると、分布の形が \textbf{山(正規分布に近い形)} になっていく
\end{itemize}

\noteT{狙い}{
現象(平均の集まりの特徴)を言葉で再確認した直後に、
「この現象には名前がある」としてLLN/CLTを提示する。
ここでは数式や条件の厳密さは深追いせず、
次の塊で「なぜ推測に使えるのか」へつなぐ。
}
% 図の提案(必要ならここだけで十分):
% ・平均のヒストグラム(中央に集まる山)を1枚
\end{frame}


% ------------------------------------------------------------
% Slide 7:平均の分布は、なぜ使えそうか
% ------------------------------------------------------------
\begin{frame}{平均の分布は、なぜ使えそうか}
平均を集めた分布を見ると、
次の点に気づきます。

\vspace{0.8em}
\begin{itemize}
  \item どのあたりの値が出やすいか
  \item どのくらいのズレなら「よくある」か
\end{itemize}

\vspace{0.8em}
つまり、
\begin{itemize}
  \item 平均の分布を知っていれば
  \item 1回の平均が \textbf{普通かどうか} を考えられる
\end{itemize}

\vspace{0.8em}
\begin{center}
\textbf{平均の分布は、判断の基準になりそう}
\end{center}

\noteT{狙い}{
「平均を集める意味」を
推測統計につながる言葉で表現する。
次の塊でCLTの役割を明確化する準備。
}
% 図の提案:
% ・平均の分布に1点(今回の平均)を重ねる図
\end{frame}

% ============================================================
% 中盤・塊3:CLTが推測統計の根拠になる
% ============================================================

% ------------------------------------------------------------
% Slide 8:平均の分布が分かる、ということ
% ------------------------------------------------------------
\begin{frame}{平均の分布が分かる、ということ}
中心極限定理によって分かるのは、
「平均がどう集まるか」です。

\vspace{0.8em}
それは、
\begin{itemize}
  \item どのあたりの平均が \textbf{よく出る}か
  \item どのくらいのズレなら \textbf{珍しくない}か
\end{itemize}

\vspace{0.8em}
言い換えると、
\begin{itemize}
  \item 平均が \textbf{どの範囲に出やすいか}
  \item その \textbf{確率の分布} が分かる
\end{itemize}

\vspace{0.8em}
\begin{center}
\textbf{「平均の出やすさ」が分かるようになる}
\end{center}

\noteT{狙い}{
CLTは「平均が正規分布になる」という暗記ではなく、
「平均の出やすさが分かるようになる」という意味だと整理する。
}
% 図の提案(この塊で1枚あれば十分):
% ・正規分布の山+中央は出やすい、端は出にくい、という注釈
\end{frame}

% ------------------------------------------------------------
% Slide 9:ここで初めて「確率」が使える
% ------------------------------------------------------------
\begin{frame}{ここで初めて「確率」が使える}
平均の分布が分かると、
次のことが考えられるようになります。

\vspace{0.8em}
\begin{itemize}
  \item この平均は、\textbf{よくある}範囲か?
  \item それとも、\textbf{めったに起きない}範囲か?
\end{itemize}

\vspace{0.8em}
これは、
\begin{itemize}
  \item 偶然のばらつきの範囲か
  \item 偶然では説明しにくいズレか
\end{itemize}

を区別する、ということです。

\vspace{0.8em}
\begin{center}
\textbf{平均を「確率」で評価できる}
\end{center}

\noteT{狙い}{
「確率」が突然出てくるのではなく、
CLTによって使えるようになった道具だと位置づける。
}
\end{frame}

% ------------------------------------------------------------
% Slide 10:これが「推測統計」の考え方
% ------------------------------------------------------------
\begin{frame}{これが「推測統計」の考え方}
ここまでの流れをまとめると、次のようになります。

\vspace{0.6em}
\begin{itemize}
  \item 母集団は直接見えない
  \item 標本しか手に入らない
  \item しかし、\textbf{標本平均の分布}は分かる
\end{itemize}

\vspace{0.8em}
そこで、
\begin{itemize}
  \item 標本から得た平均が
  \item その分布の中で \textbf{どの位置にあるか} を考える
\end{itemize}

\vspace{0.8em}
\begin{center}
\textbf{これが、推測統計の基本的な考え方}
\end{center}

\noteT{狙い}{
推測統計を「新しい分野」ではなく、
今まで積み上げた考えの自然な帰結として定義する。
}
\end{frame}

% ------------------------------------------------------------
% Slide 11:次回につながる問い
% ------------------------------------------------------------
\begin{frame}{次回につながる問い}
平均を確率で評価できるようになると、
次の問いが生まれます。

\vspace{0.8em}
\begin{itemize}
  \item どのくらいズレたら「外れている」と言えるのか?
  \item その判断基準は、どう決めるのか?
\end{itemize}

\vspace{0.8em}
\begin{center}
\textbf{この問いに答えるのが、仮説検定}
\end{center}

\vspace{0.6em}
次回は、
\begin{itemize}
  \item 仮説を立てる
  \item 確率で評価する
\end{itemize}
という考え方を整理します。

\noteT{狙い}{
CLT → 推測統計 → 仮説検定、という一本道を明示する。
次回内容が「飛躍」ではないことを強調する。
}
\end{frame}
