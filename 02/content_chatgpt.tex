% @@@--(metropolis)--@@@
%========================================
% 第1回(授業では2回目)
% ガイダンス/統計で何ができるか
% 準拠:アルゴリズム2 Beamer共通仕様 v2025
% 教師メモ:noteT{title}{body}
%========================================

\begin{frame}{今日の位置づけ}
\begin{itemize}
  \item アルゴリズム2(Excel統計)
  \item 前回:ガイダンス・環境確認
  \item 今回:データ分析の入口
  \item 確率・分布・推測への準備
  \item Excelで考える授業
\end{itemize}
\noteT{この回の位置づけ}{
今日は本格的な計算に入る前の「入口の回」である。\par
統計とは何をする学問なのか、その全体像をまず共有する。\par
確率や分布は後で扱うが、今日は土台作りであることを強調する。\par
}
\end{frame}
%----------------------------------------------------------------------------------------
%   page
%----------------------------------------------------------------------------------------
\begin{frame}{統計で何ができるか}
\begin{itemize}
  \item データを要約する
  \item 違いや傾向を見つける
  \item 判断や予測につなげる
  \item 勘ではなく根拠を示す
\end{itemize}
\noteT{統計の役割}{
統計は計算のためではなく、説明や判断のための道具である。\par
「数字を使って説明できるようになる」ことをゴールとして示す。\par
}
\end{frame}
%----------------------------------------------------------------------------------------
%   page
%----------------------------------------------------------------------------------------
\begin{frame}{統計はどこで使われているか}
  \begin{columns}[T,onlytextwidth]
     \begin{column}{0.48\textwidth}
\begin{itemize}
  \item 売上・在庫・来客数
  \item アンケート結果
  \item 成績・評価
  \item 品質管理(不良率)
\end{itemize}
  \end{column}
  \begin{column}{0.52\textwidth}
    \centering
    \includegraphics[width=1\linewidth]{統計学}
  \end{column}
\end{columns}
\noteT{身近な利用例}{
学生の身近な例を挙げて、統計が特別な世界の話ではないと伝える。\par
この後の演習で扱う○/×データにつながる例を意識して話す。\par
}
\end{frame}
%----------------------------------------------------------------------------------------
%   page
%----------------------------------------------------------------------------------------
\begin{frame}{データとは何か}
\begin{itemize}
  \item データ=事実を記録したもの
  \item 数値だけでなく文字や記号も含む
  \item 1件ずつの記録が基本
  \item 集めることで意味が見える
\end{itemize}
\noteT{データの定義}{
「データ=数字」という思い込みをここで外す。\par
○や×、当たり・はずれも立派なデータであると明言する。\par
}
\end{frame}
%----------------------------------------------------------------------------------------
%   page
%----------------------------------------------------------------------------------------
\begin{frame}{測定尺度(尺度水準)}
データの分類方法で、どのような尺度によって測定(観測)されるのかとういう点で分類した考え方です。

\vspace{2em}
\includegraphics[width=1\linewidth]{測定尺度}

\noteT{測定尺度}{
データの分類方法\par
どのような尺度によって測定(観測)\par
}
\end{frame}
%----------------------------------------------------------------------------------------
%   page
%----------------------------------------------------------------------------------------
\begin{frame}{Excel表の基本構造}
\begin{itemize}
  \item 1行=1件(1人・1回)
  \item 1列=同じ種類の情報
  \item 空白やズレはNG
  \item 見た目より構造
\end{itemize}
\noteT{分析できる表}{
確率や平均の前に「数えられる表」を作る必要がある。\par
1行1件という考え方が、この後すべての分析の前提になる。\par
}
\end{frame}
%----------------------------------------------------------------------------------------
%   page
%----------------------------------------------------------------------------------------
\begin{frame}{表を「作る」と「整理する」}
\begin{itemize}
  \item 罫線=見た目の表
  \item 整理=分析できる形
  \item 1件が複数行だと数えられない
  \item まず整理
\end{itemize}
\noteT{整理の重要性}{
統計は「きれいな表」ではなく「数えられる表」が重要。\par
この後の演習で、整理されていない表は扱えないことを体験させる。\par
}
\end{frame}
%----------------------------------------------------------------------------------------
%   page
%----------------------------------------------------------------------------------------
\begin{frame}{データの種類(導入)}
\begin{itemize}
  \item 数値データ(点数・金額)
  \item カテゴリデータ(○/×)
  \item 今日はカテゴリ中心
  \item 次回以降で数値も扱う
\end{itemize}
\noteT{今日扱うデータ}{
この回ではカテゴリデータを使って話を進める。\par
確率や割合は、この種類のデータと相性が良いことを示す。\par
}
\end{frame}
%----------------------------------------------------------------------------------------
%   page
%----------------------------------------------------------------------------------------
\begin{frame}{なぜExcelを使うのか}
\begin{itemize}
  \item 件数が多くても数えられる
  \item やり直しが簡単
  \item 条件を変えて試せる
  \item 思考の実験場
\end{itemize}
\noteT{Excelの役割}{
Excel操作の授業ではなく、考えるための道具であることを強調。\par
この後の演習で、手作業との差を実感させる。\par
}
\end{frame}
%----------------------------------------------------------------------------------------
%   page
%----------------------------------------------------------------------------------------
\begin{frame}{今日の流れ}
\vspace{-1.5ex}
\includegraphics[width=0.22\linewidth]{流れ}
\noteT{見通し提示}{
確率は突然出てこない。\par
「数える→割合」という流れの先にある話題だと宣言しておく。\par
}
\end{frame}

% \begin{frame}{今日の流れ}
% \begin{itemize}
%   \item データを確認
%   \item 回数を数える
%   \item 割合を出す
%   \item 次につながる話
% \end{itemize}
% \noteT{見通し提示}{
% 確率は突然出てこない。\par
% 「数える→割合」という流れの先にある話題だと宣言しておく。\par
% }
% \end{frame}
%----------------------------------------------------------------------------------------
%   page
%----------------------------------------------------------------------------------------
%========================================
% 実習パート差し替え(第1回用)
% 使用ファイル:02_実習.xlsx
% 使用シート:chap2-1
%========================================
%----------------------------------------------------------------------------------------
%   page
%----------------------------------------------------------------------------------------
\begin{frame}{実習:使用するデータ}
\begin{itemize}
  \item 実習用ファイル:\textbf{02\_実習.xlsx}
  \item 使用シート:\textbf{chap2-1}
  \item A列:商品ID(文字)
  \item B列:売上個数(数値)
\end{itemize}
\noteT{実習データの確認}{
全員が同じファイル・同じシートを開いているかをまず確認する。\par
列の意味(文字/数値)をここで明確にしておくと後が楽になる。\par
}
\end{frame}
%----------------------------------------------------------------------------------------
%   page
%----------------------------------------------------------------------------------------
\begin{frame}{実習①:商品の件数を調べる(COUNTA)}
\begin{itemize}
  \item 商品は何種類あるか?
  \item A列(商品ID)に注目する
  \item 空白でないセルの数を数える
  \item 使用関数:\texttt{COUNTA}
\end{itemize}
\noteT{COUNTAの目的}{
ここでは「商品が何件あるか」を知りたい。\par
文字データなので COUNT ではなく COUNTA を使う点を強調する。\par
}
\end{frame}
%----------------------------------------------------------------------------------------
%   page
%----------------------------------------------------------------------------------------
\begin{frame}{実習①:COUNTA の入力例}
\begin{itemize}
  \item 例:\texttt{=COUNTA(A2:A9)}
  \item 結果:商品IDの件数
  \item これは「全体件数」にあたる
  \item 次の計算の基準になる
\end{itemize}
\noteT{全体件数という考え方}{
統計では「まず全体を知る」ことが重要。\par
この件数が、後の比較や判断の土台になる。\par
}
\end{frame}
%----------------------------------------------------------------------------------------
%   page
%----------------------------------------------------------------------------------------
\begin{frame}{実習②:売上個数データの件数(COUNT)}
\begin{itemize}
  \item 売上個数のデータは何件あるか?
  \item B列(売上個数)に注目
  \item 数値だけを数える
  \item 使用関数:\texttt{COUNT}
\end{itemize}
\noteT{COUNTとの違い}{
COUNT は数値だけを数える関数。\par
A列では使えないが、B列では使えることを確認させる。\par
}
\end{frame}
%----------------------------------------------------------------------------------------
%   page
%----------------------------------------------------------------------------------------
\begin{frame}{実習②:COUNT の入力例}
\begin{itemize}
  \item 例:\texttt{=COUNT(B2:B9)}
  \item 結果:売上個数データの件数
  \item COUNTA と結果が同じでも意味は違う
\end{itemize}
\noteT{同じ数でも意味が違う}{
件数が同じでも「何を数えたか」が違う。\par
統計ではこの違いを意識することが大切。\par
}
\end{frame}
%----------------------------------------------------------------------------------------
%   page
%----------------------------------------------------------------------------------------
\begin{frame}{実習③:売上個数の合計を求める}
\begin{itemize}
  \item 全部で何個売れているか?
  \item B列の売上個数を合計
  \item 使用関数:\texttt{SUM}
\end{itemize}
\noteT{合計の意味}{
合計はデータ全体の規模を表す。\par
「全部でどれくらいか」を知る基本的な指標。\par
}
\end{frame}
%----------------------------------------------------------------------------------------
%   page
%----------------------------------------------------------------------------------------
\begin{frame}{実習③:SUM の入力例}
\begin{itemize}
  \item 例:\texttt{=SUM(B2:B9)}
  \item 結果:総売上個数
  \item 数字の大きさを確認する
\end{itemize}
\noteT{結果の確認}{
値の大小に意味はまだ付けない。\par
「こういうことが分かる」という体験を重視する。\par
}
\end{frame}
%----------------------------------------------------------------------------------------
%   page
%----------------------------------------------------------------------------------------
\begin{frame}{実習④:最大値と最小値}
\begin{itemize}
  \item 一番多く売れた商品は?
  \item 一番少ない商品は?
  \item 使用関数:\texttt{MAX}, \texttt{MIN}
\end{itemize}
\noteT{極端な値を見る}{
最大値・最小値は「目立つデータ」を見つける指標。\par
平均との差やばらつきの話につながる伏線になる。\par
}
\end{frame}
%----------------------------------------------------------------------------------------
%   page
%----------------------------------------------------------------------------------------
\begin{frame}{実習④:MAX / MIN の入力例}
\begin{itemize}
  \item 例:\texttt{=MAX(B2:B9)}
  \item 例:\texttt{=MIN(B2:B9)}
  \item 結果の違いを確認する
\end{itemize}
\noteT{結果を眺める}{
なぜ違うかは考えなくてよい。\par
「違う値が出る」ことだけ確認させる。\par
}
\end{frame}

%----------------------------------------
\begin{frame}{実習⑤:平均値と中央値}
\begin{itemize}
  \item 売上個数の代表的な値を見る
  \item 使用関数:\texttt{AVERAGE}, \texttt{MEDIAN}
  \item 同じ列から異なる値が出る
\end{itemize}
\noteT{代表値の導入}{
今日は意味の説明はしない。\par
「代表の仕方が複数ある」ことだけ体験させる。\par
}
\end{frame}
%----------------------------------------------------------------------------------------
%   page
%----------------------------------------------------------------------------------------
\begin{frame}{実習⑤:AVERAGE / MEDIAN の入力例}
\begin{itemize}
  \item 例:\texttt{=AVERAGE(B2:B9)}
  \item 例:\texttt{=MEDIAN(B2:B9)}
  \item 数字が違うことを確認
\end{itemize}
\noteT{次回への橋渡し}{
なぜ違うかは次の回で扱う。\par
今日は「統計で何が分かるか」を体感できれば十分。\par
}
\end{frame}
%----------------------------------------------------------------------------------------
%   page
%----------------------------------------------------------------------------------------
\begin{frame}{実習のまとめ}
\begin{itemize}
  \item 件数・合計・最大・最小が分かった
  \item 同じデータでも見方が変わる
  \item 統計はデータを要約する道具
  \item 次回は意味を深く考える
\end{itemize}
\noteT{実習の着地点}{
統計=計算ではなく「見方を増やす」ことだとまとめる。\par
次回はその意味を丁寧に説明する。\par
}
\end{frame}

%----------------------------------------------------------------------------------------
%   page
%----------------------------------------------------------------------------------------
\begin{frame}{なぜ確率につながるか}
    \begin{columns}[T,onlytextwidth]
  \begin{column}{0.6\textwidth}
\begin{itemize}
  \item データから「どれくらい?」が知りたくなる
  \item どれくらい=割合
  \item 起こりやすさとして見る
\end{itemize}
  \end{column}
  \begin{column}{0.4\textwidth}
    \centering
    \includegraphics[width=.7\linewidth]{サイコロ}
  \end{column}
\end{columns}
\noteT{確率への橋渡し}{
確率は新しい話題ではない。\par
割合を「起こりやすさ」として見たときの名前だと説明する。\par
}
\end{frame}
%----------------------------------------------------------------------------------------
%   page
%----------------------------------------------------------------------------------------
\begin{frame}{「割合」から「確率」へ呼び方を変える}
\begin{center}
  \Large
  これまでのデータ(過去)の話:\textbf{「割合」} \\
  \vspace{1em}
  $\downarrow$ 視点を変えると $\downarrow$ \\
  \vspace{1em}
  これからの予想(未来)の話:\textbf{「確率」}
\end{center}

\begin{itemize}
  \item 過去のデータで「当たり」が3割あった
  \item $\rightarrow$ 次に引くときも、3割の\textbf{「起こりやすさ」}がありそうだ
  \item この「起こりやすさ」のことを\textbf{確率}と呼ぶ
\end{itemize}

\noteT{視点の切り替え}{
「割合」と「確率」は計算方法は同じですが、見ている「時間軸」が違うだけだと伝えます。\par
過去の事実を「割合」と呼び、未来の予想を「確率」と呼ぶ、という整理です。\par
「昨日のシュート練習で10本中7本入った(割合:70\%)。 
だから、今日の試合で次に打つシュートも70\%の確率で入りそうだ、と予想する。 
ほら、計算は何も変わっていないけれど、言葉だけが『割合』から『確率』に進化しましたよね」
}
\end{frame}
%----------------------------------------------------------------------------------------
%   page
%----------------------------------------------------------------------------------------
\begin{frame}{今日のまとめ}
\begin{itemize}
  \item データは1件ずつ
  \item まず数える
  \item 次に割合
  \item 次回は確率の理論
\end{itemize}
\noteT{着地点}{
この回は入口。\par
次回に向けて「なぜ確率を学ぶのか」が分かれば成功。\par
}
\end{frame}


