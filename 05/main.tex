%----------------------------------------------------------------------------------------
%  metropolis template (refactored)
%----------------------------------------------------------------------------------------
\documentclass[handout,aspectratio=169]{beamer}

% \documentclass の直後で hyperref のオプションを渡す(metropolisでも安全)
\PassOptionsToPackage{unicode=true,colorlinks=true,linkcolor=blue,urlcolor=blue}{hyperref}

\usetheme{metropolis}
\metroset{block=fill, sectionpage=progressbar, progressbar=foot}

% 背景色(tech のとき白など): Pythonで差し込み
\setbeamercolor{background canvas}{bg=white}

%--------------------------
% 日本語
%--------------------------
\usepackage{luatexja}
\usepackage{luatexja-fontspec}
\usepackage{luatexja-ruby}
\setsansjfont{Hiragino Sans}[BoldFont={Hiragino Sans W6}]

%--------------------------
% 基本パッケージ(重複なし)
%--------------------------
\usepackage[table]{xcolor}
\usepackage{graphicx}
\usepackage[abs]{overpic}
\usepackage{tikz}
\usepackage{array}
\usepackage{tabularx}
\usepackage{booktabs}
\usepackage{makecell}
\usepackage{mathtools}
\usepackage{longtable}
\usepackage{pdfpages}
\usepackage{etoolbox} % AtBeginEnvironment 等

% minted(※ -shell-escape 必須)
\usepackage{minted}
\setminted{
  frame=single,
  framesep=2mm,
  fontsize=\footnotesize,
  breaklines=true
}

% (必要なときだけ)tcolorbox
\usepackage[most]{tcolorbox}

% hyperref は最後
\usepackage{hyperref}

%--------------------------
% パス
%--------------------------
\newcommand{\assetpath}{/Volumes/NBPlan/TTC/授業資料/2025年度/}
\graphicspath{{images/}{\assetpath/1020201.アルゴリズム2/05/images/}{../project_assets/images/}{../project_assets/emoji/emoji_pngs/}}

%--------------------------
% フッター
%--------------------------
\newcommand{\myfootertext}{1020201.アルゴリズム2/05}
\setbeamertemplate{footline}{%
  \leavevmode
  \hbox to \paperwidth{%
    \hspace*{0.2cm}
    \scriptsize\color{gray!50} \myfootertext
    \hfill
    \scriptsize\color{gray} \insertframenumber{} / \inserttotalframenumber
    \hspace*{0.4cm}
  }%
  \vspace{1pt}
}

%--------------------------
% TeacherFrame(外部)
%--------------------------
\usepackage{../teacherframe}

%--------------------------
% フレームタイトル:番号. タイトル
% ※ ここで出すだけ。insertframetitle を再定義しない(安全)
%--------------------------
\setbeamertemplate{frametitle}{%
  \vspace{0.6ex}%
  \begin{beamercolorbox}[wd=\paperwidth,sep=0.5ex,leftskip=0.9em,rightskip=0.5em]{frametitle}%
    \usebeamerfont{frametitle}%
    \insertframenumber.\,\insertframetitle%
  \end{beamercolorbox}%
}

%--------------------------
% 表用:列型
%--------------------------
\newcolumntype{C}[1]{>{\centering\arraybackslash}p{#1}}
\newcolumntype{M}[1]{>{\raggedright\arraybackslash}m{#1}}

%--------------------------
% ブロック(必要なら)
%--------------------------
\definecolor{myblue}{HTML}{7488FF}
\definecolor{mylightblue}{HTML}{E3EEFF}
\setbeamertemplate{blocks}[rounded]
\setbeamercolor{block title}{bg=myblue, fg=white}
\setbeamercolor{block body}{bg=mylightblue, fg=black}

%--------------------------
% 奇数ページのスライドのを表示する
%(教示用だけでそれ以外はこの処理は動かない)
%--------------------------
% --- 教師用だけ、スライドを奇数開始に強制するトグル ---
\newif\ifoddslideenforce
\oddslideenforcefalse   % デフォルトOFF(pr/hoはOFF)

% --- 再帰防止ガード ---
\newif\ifoddslideguard
\oddslideguardfalse

% --- 偶数ページなら空白スライドを1枚入れて奇数に戻す ---
\newcommand{\ensureoddslide}{%
  \ifoddslideguard\relax\else
    \oddslideguardtrue
    \ifodd\value{page}\relax
      % 何もしない(次が奇数)
    \else
      \begin{frame}[plain,noframenumbering]
        \note{}% notes出力時に2枚消費させる保険
      \end{frame}
    \fi
    \oddslideguardfalse
  \fi
}

% --- frameが始まる直前に自動挿入(教師用だけ)---
\BeforeBeginEnvironment{frame}{%
  \ifoddslideenforce
    \ensureoddslide
  \fi
}

%--------------------------
% note / noteT の「常時安全化」
%  - tech 以外:\noteT は無視(エラーにならない)
%  - tech:notesmode_tech で上書き定義
%--------------------------
\providecommand{\notetitletext}{}      % 既にあっても衝突しない
\providecommand{\noteT}[2]{}           % デフォルトは何もしない

% frame開始ごとにタイトル変数をクリア(前の noteT が残らないように)
\AtBeginEnvironment{frame}{\gdef\notetitletext{}}

%--------------------------
% 切替(Pythonから差し込み)
%--------------------------
\mypausemodetrue
\teachermodetrue
\setbeameroption{show notes}
%-------------

% --- tech のときだけ noteT を有効化(テンプレートの \providecommand を上書き) ---
\makeatletter
\renewcommand{\noteT}[2]{%
 \gdef\notetitletext{#1}%
 \note{#2}%
}
% タイトル未指定のときのために初期化
\renewcommand{\notetitletext}{}%

\setbeamertemplate{note page}{%
 \begin{minipage}{\linewidth}
 \vspace{1.2ex} % タイトルを少し下げる(必要に応じて調整)
 {\Large\bfseries
 \ifx\notetitletext\@empty
 \insertframetitle
 \else
 \notetitletext
 \fi
 }\par
 \vspace{-1.2ex}
 \rule{\linewidth}{0.8pt}\par
 \vspace{0.8ex}
 {\scriptsize \insertnote}
 \end{minipage}
}
\makeatother

%教師用のPDFは奇数ページからスライドを出力
\oddslideenforcetrue


%----------------------------------------------------------------------------------------
% タイトル
%----------------------------------------------------------------------------------------
\title{ 05 確率の基礎②(独立・期待値) }
\date{}
\newcommand{\codedir}{\assetpath/1020201.アルゴリズム2/05}

\begin{document}

\begin{frame}[plain,noframenumbering]
  \titlepage
  \bigskip
  \begin{center}
    \ifteachermode 教師用 \fi
  \end{center}
\end{frame}

% セクションページ(必要なら)
\setbeamertemplate{section page}{
  \begin{centering}
    \vfill
    \rule{\linewidth}{2pt}\par
    \vspace{1ex}
    {\usebeamerfont{section title}\Huge\bfseries \insertsection}\par
    \vspace{1ex}
    \rule{\linewidth}{2pt}\par
    \vfill
  \end{centering}
}
\setbeamerfont{section title}{size=\LARGE,series=\bfseries}

\AtBeginSection[]{
  \begin{frame}[plain,noframenumbering]
    \sectionpage
  \end{frame}
}

% 本編開始でフレーム番号を0から(必要なら)
\setcounter{framenumber}{0}

\input{emoji_macros}

% @@@--(metropolis)--@@@

% ----------------------------------------------------------------------------------------
%   Slide 01: 本日のテーマ
% ----------------------------------------------------------------------------------------
\begin{frame}{独立・期待値}
今日は、確率を使って「未来の予想」をする方法を学びます。

\begin{itemize}
  \item \textbf{\ruby{独立}{どくりつ}:} 前の結果が、次に「関係(かんけい)しない」こと。
  \item \textbf{\ruby{期待値}{きたいち}:} 1回あたり「だいたいこれくらい」という見込み。
  \item \textbf{実習:} 自分の手とExcelを使って、期待値を計算します。
\end{itemize}

「たった1回の結果」で\ruby{一喜一憂}{いっきいちゆう}せず、長い目でデータを見る力をつけましょう。

\noteT{講義の狙い}{
独立を知ることで勘違いを防ぎ、期待値で損得を数字で判断できるようにします。
}
\end{frame}

% ----------------------------------------------------------------------------------------
%   Slide 02: 独立とは:関係がないこと
% ----------------------------------------------------------------------------------------
\begin{frame}{独立:前の結果は、次に関係しない}
2つの出来事があり、1回目の結果が2回目の確率を変えないとき、これを\textbf{独立}と言います。

\begin{itemize}
  \item \textbf{意味:} 前に何が起きても、次の確率は「リセット」されます。
  \item \textbf{例:コイン投げ}
\end{itemize}

1回目に「表」が出たからといって、2回目に「裏」が出やすくなることはありません。コインは前のことを「覚えていない」からです。

\noteT{独立}{
独立 = まわりの影響を受けない、というイメージを伝えます。
}
\end{frame}

% ----------------------------------------------------------------------------------------
%   Slide 03: 【問い】「運(うん)」や「流れ」はある?
% ----------------------------------------------------------------------------------------
\begin{frame}{【問い】「そろそろ当たるはず」は本当?}
あるゲームのガチャで、当たる確率が 10\% だとします。
9回連続で外れました。

\textbf{問い:10回目に当たる確率は、10\% より高くなっていますか?}

\begin{enumerate}
  \item 高くなっている(次こそは当たるはず!)
  \item 低くなっている(今日は運が悪いから当たらない)
  \item \textbf{変わらず 10\% である}
\end{enumerate}

\vspace{1em}
(答えは次のスライドで説明します)

\noteT{直感のミス}{
人間が陥りやすい思い込みを指摘します。
}
\end{frame}

% ----------------------------------------------------------------------------------------
%   Slide 04: 【答え】確率はいつも同じ
% ----------------------------------------------------------------------------------------
\begin{frame}{【答え】独立なら、確率はいつも同じ}
正解は \textbf{「3. 変わらず 10\% である」} です。

\begin{itemize}
  \item \textbf{理由:} ガチャが独立な仕組みなら、1回目も10回目も、当たる確率は同じです。
  \item \textbf{注意:} 「10回やれば1回は当たるだろう」というのは、やる前の予想です。\textbf{「目の前の1回」}の確率は、前の結果に関係なく 10\% のままです。
\end{itemize}

統計では、気持ちや運ではなく、この「独立」というルールで冷静に考えます。

\noteT{解説}{
単発の確率とセットの確率を分けて説明します。
}
\end{frame}

% ----------------------------------------------------------------------------------------
%   Slide 05: 期待値とは何か
% ----------------------------------------------------------------------------------------
\begin{frame}{期待値:1回あたりの「平均的な見込み」}
\textbf{期待値(Expected Value)}とは、1回やると「平均してどれくらいの結果になるか」を計算した数字です。

\begin{itemize}
  \item \textbf{イメージ:} もらえる金額や点数の「平均的な予想」。
  \item \textbf{使い方:} 「その勝負(しょうぶ)をするべきか、やめるべきか」を\textbf{\ruby{判断}{はんだん}}する材料になります。
\end{itemize}

期待値を知ることで、損か得かを数字で見ることができます。

\noteT{期待値}{
期待値 = 将来もらえる平均的な値、と定義します。
}
\end{frame}

% ----------------------------------------------------------------------------------------
%   Slide 06: 期待値はどうやって計算する?
% ----------------------------------------------------------------------------------------
\begin{frame}{期待値の計算の手順}
期待値は、次の「3ステップ」で計算します。

\begin{block}{期待値の計算ルール}
  \[ \text{期待値} = \sum ( \text{もらえる値} \times \text{その確率} ) \]
\end{block}

\textbf{計算の手順:}
\begin{enumerate}
  \item どんな「結果(値)」があるか全部書き出す。
  \item それぞれの「確率」を調べる。
  \item \textbf{「値 $\times$ 確率」}を計算して、全部足し算する。
\end{enumerate}

\noteT{手順}{
掛け算をしてから足す、という流れを強調します。
}
\end{frame}

% ----------------------------------------------------------------------------------------
%   Slide 07: 【例】サイコロを振ったときの期待値
% ----------------------------------------------------------------------------------------
\begin{frame}{期待値の計算例:サイコロの目}
普通のサイコロを1回振るとき、出る目の期待値を求めます。

\begin{table}[]
\begin{tabular}{|l|c|c|c|c|c|c|}
\hline
出る目 ($x$) & 1 & 2 & 3 & 4 & 5 & 6 \\ \hline
確率 ($p$) & 1/6 & 1/6 & 1/6 & 1/6 & 1/6 & 1/6 \\ \hline
\end{tabular}
\end{table}

\textbf{期待値の計算:}
\[ E = (1 \times \frac{1}{6}) + (2 \times \frac{1}{6}) + (3 \times \frac{1}{6}) + (4 \times \frac{1}{6}) + (5 \times \frac{1}{6}) + (6 \times \frac{1}{6}) \]
\[ E = \frac{1+2+3+4+5+6}{6} = \frac{21}{6} = \mathbf{3.5} \]

\textbf{解釈:} サイコロを何回も振ると、1回あたりの平均は 3.5 に近づきます。

\noteT{サイコロの例}{
3.5 という目は実際には出ませんが、これが「平均的な見込み」であることを強調します。
}
\end{frame}

% ----------------------------------------------------------------------------------------
%   Slide 08: 期待値と「平均」はどう違う?
% ----------------------------------------------------------------------------------------
\begin{frame}{平均と期待値}
どちらも「データの真ん中」のことですが、時間の向きが違います。

\begin{itemize}
  \item \textbf{平均:} すでに起きた「過去」のデータの真ん中。
  \item \textbf{期待値(今日):} これから起きる「未来」の予想の真ん中。
\end{itemize}

\textbf{つながり:}
過去と同じ割合で何かが起きるなら、\textbf{「過去の平均」は「未来の期待値」}と同じになります。


\noteT{平均との差}{
集計(過去)と予測(未来)の違いを伝えます。
}
\end{frame}

% ----------------------------------------------------------------------------------------
%   Slide 09: 【実習1】期待値を計算してみよう
% ----------------------------------------------------------------------------------------
\begin{frame}{実習1:賞金の「見込み」を計算しよう}
以下のくじがあります。1回あたりの期待値はいくらですか?

\begin{itemize}
  \item 1等:10,000円(確率 1\% = 0.01)
  \item 2等:500円(確率 10\% = 0.10)
  \item ハズレ:0円(確率 89\% = 0.89)
\end{itemize}

\textbf{ヒント:}
$10,000 \times 0.01 = 100$ などを計算して、最後に全部足し算してください。

\noteT{実習1}{
自分の手で計算させます。答えは150円。
}
\end{frame}

% ----------------------------------------------------------------------------------------
%   Slide 10: 【答え】期待値は 150円
% ----------------------------------------------------------------------------------------
\begin{frame}{【答え】期待値の活用方法}
計算の答えは \textbf{150円} です。これがこのくじの「価値(かち)」です。

\begin{itemize}
  \item \textbf{判断のポイント:} 
  \item もしこのくじが \textbf{200円} で売っていたら?
  \item $\rightarrow$ 期待値(150円)より高いので、やり続けると損をします。
\end{itemize}

期待値を使えば、このように「損か得か」を冷静に比べることができます。

\noteT{解説}{
期待値が公正な価格を見極める道具になることを話します。
}
\end{frame}

% ----------------------------------------------------------------------------------------
%   Slide 11: ビジネスでの使い方(プランの比較)
% ----------------------------------------------------------------------------------------
\begin{frame}{期待値で「仕事」を選ぶ}
期待値を使えば、違う種類のチャンスを同じように比べられます。

\textbf{どちらのプロジェクトが「得」か?}
\begin{itemize}
  \item \textbf{プランA:} 20\%で500万円儲かるが、80\%はゼロ。
  \item \textbf{プランB:} 50\%で200万円儲かるが、50\%はゼロ。
\end{itemize}

\textbf{計算の結果:}
どちらも期待値は \textbf{100万円} です。
同じ土俵で比べることができました。

\noteT{ビジネス}{
期待値で選択肢を整理するメリットを伝えます。
}
\end{frame}

% ----------------------------------------------------------------------------------------
%   Slide 12: 長い目で見ると、結果は安定する
% ----------------------------------------------------------------------------------------
\begin{frame}{繰り返すと期待値に近づく}
期待値が「長い目の平均」と言われる理由です。

\begin{itemize}
  \item \textbf{性質:} 1回ごとの結果はバラバラでも、何百回と繰り返すと、実際の平均は「期待値」にそっくりになります。
  \item \textbf{例:コイン投げ} 10回投げると偏ることもあるが、1万回投げれば表の割合はほぼ \textbf{50\%} になります。
\end{itemize}

(この詳しいルールは別の授業で学びます)



\noteT{補足}{
短期的な荒れと長期的な安定について教えます。
}
\end{frame}

% ----------------------------------------------------------------------------------------
%   Slide 13: 【実習2】Excelで「独立」を見る
% ----------------------------------------------------------------------------------------
\begin{frame}{実習2:Excelでコイン投げをシミュレーション}
Excelの乱数(ランダムな数字)を使って、コイン投げをします。

\textbf{操作:}
\begin{enumerate}
  \item A1セルに \texttt{=RANDBETWEEN(0, 1)} と入力(0=裏、1=表)。
  \item A100までコピーします。
  \item B1セルに \texttt{=AVERAGE(A1:A100)} と入力します。
\end{enumerate}

F9キーを押すと数字が変わります。A列(個別の結果)は激しく変わりますが、B1(平均)は \textbf{0.5} の近くで安定しませんか?

\noteT{実習2}{
F9キーで再計算させ、独立性と安定性を体感させます。
}
\end{frame}

% ----------------------------------------------------------------------------------------
%   Slide 14: 実習2の解説:独立と平均の姿
% ----------------------------------------------------------------------------------------
\begin{frame}{実習2のまとめ:ミクロはカオス、マクロは秩序}
今のExcel操作で分かったことは次の通りです。

\begin{itemize}
  \item \textbf{個々のセル(独立):} 0か1か全く予想できません。
  \item \textbf{全体の平均(期待値):} 100個集まると、ほぼ 0.5 になります。
\end{itemize}

統計学は、この「バラバラな出来事(独立事象)」をたくさん集めて、「確かな予測(期待値)」に変える学問です。

\noteT{実習2解説}{
個別の不確かさが集団の確かさに変わる面白さを伝えます。
}
\end{frame}

% ----------------------------------------------------------------------------------------
%   Slide 15: 【実習3】Excelで「期待値」を算出しよう
% ----------------------------------------------------------------------------------------
\begin{frame}{実習3:Excelを使って期待値を計算する}
先ほどの「実習1」のくじ引きを、Excelの表で作ってみましょう。

\begin{enumerate}
  \item A2:A4に「賞金(10000, 500, 0)」を入力。
  \item B2:B4に「確率(0.01, 0.1, 0.89)」を入力。
  \item C2に \texttt{=A2*B2} と入力し、C4までコピー。
  \item C5に \texttt{=SUM(C2:C4)} で合計を出す。
\end{enumerate}



\noteT{実習3}{
計算プロセスをExcelのセルに対応させることで、実務への繋がりを作ります。
}
\end{frame}

% ----------------------------------------------------------------------------------------
%   Slide 16: 実習3のメリット:データの変更に強い
% ----------------------------------------------------------------------------------------
\begin{frame}{Excelで計算するメリット}
なぜ手計算ではなくExcelを使うのでしょうか?

\begin{itemize}
  \item \textbf{理由:} 確率や金額が変わっても、数字を打ち替えるだけで期待値がすぐに再計算されるからです。
  \item \textbf{応用:} 「もしハズレの確率を少し減らしたら、期待値はいくら上がるか?」といったシミュレーションが瞬時に行えます。
\end{itemize}

期待値はビジネスの「シミュレーション」に欠かせないツールです。

\noteT{実習3解説}{
静的な計算から動的な分析への意識変容を促します。
}
\end{frame}

% ----------------------------------------------------------------------------------------
%   Slide 17: 確率でよくあるミス:分母の勘違い
% ----------------------------------------------------------------------------------------
\begin{frame}{注意! 確率の落とし穴}
ミスを防ぐためのチェックポイントです。

\begin{itemize}
  \item \textbf{分母を決めない:} 必ず「全パターン」を先に数えましょう。
  \item \textbf{主観を入れる:} 「昨日も雨だったから、今日は晴れるだろう」といった主観を捨て、客観的な比率で考えます。
  \item \textbf{独立を忘れる:} 前の結果に引っ張られず、現在の確率を確認しましょう。
\end{itemize}

\noteT{ミス防止}{
典型的なミスを言語化し、演習時の混乱を防ぎます。
}
\end{frame}

% ----------------------------------------------------------------------------------------
%   Slide 18: 記述統計と確率の関係(再確認)
% ----------------------------------------------------------------------------------------
\begin{frame}{まとめ:なぜ記述統計から始めたのか}
これまでの学習を繋げましょう。

\begin{itemize}
  \item \textbf{記述統計:} 過去のデータを集計して「分布(形)」を見ました。
  \item \textbf{確率:} その「分布」から、次に起こる「期待値」を考えました。
\end{itemize}

データから「形」を見つけ、そこから「将来」を見通す。これが統計学の流れです。

\noteT{全体の繋がり}{
第1回からの一貫性を確認します。
}
\end{frame}

% ----------------------------------------------------------------------------------------
%   Slide 19: 本日のまとめ
% ----------------------------------------------------------------------------------------
\begin{frame}{今日のまとめ}
\begin{itemize}
  \item \textbf{独立:} 前の結果は次に影響しない。
  \item \textbf{期待値:} $\text{期待値} = \sum (\text{値} \times \text{確率})$。長い目で見たときの平均。
  \item \textbf{予測:} 過去のデータ(平均)から、未来の確率(期待値)を考えることができる。
\end{itemize}

期待値を使えば、不確実な未来も数字で考えることができます。

\noteT{まとめ}{
最重要ポイントを再度復唱します。
}
\end{frame}

% ----------------------------------------------------------------------------------------
%   Slide 20: 次回の予告
% ----------------------------------------------------------------------------------------
\begin{frame}{次は「データの形」を学びます}
今日は「1回あたりの平均(期待値)」という「中心」の数字を学びました。

\begin{itemize}
  \item \textbf{次回(第5回):\ruby{確率分布}{かくりつぶんぷ}}
  \item \textbf{内容:} 期待値のまわりに、どのようにデータが広がっているか、その「形」をグラフで見ていきます。
\end{itemize}

お疲れ様でした。

\noteT{次回予告}{
期待値が次回の「グラフの中心」になることを予告します。
}
\end{frame}
\end{document}
