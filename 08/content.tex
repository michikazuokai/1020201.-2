% @@@--(metropolis)--@@@
% ----------------------------------------------------------------------------------------
%   Slide 01: 今日のテーマ(第7回)
% ----------------------------------------------------------------------------------------

%@@PAGEBAND@@
% ----------------------------------------------------------------------------------------
%   page 01
% ----------------------------------------------------------------------------------------
\begin{frame}{平均は本当に安定するのか?}
今日のテーマは、とても素朴な疑問です。

\vspace{0.8em}
\begin{center}
\Large
\textbf{「平均をとれば、もう安心」\\
それは本当でしょうか?}
\end{center}

\vspace{1.0em}
これまでの授業では、
\begin{itemize}
  \item データを集める
  \item 平均を計算する
\end{itemize}
ところまでは、当たり前のように行ってきました。

\noteT{狙い}{
平均は「安定した代表値」という思い込みを、
あえて言葉にして表に出す。
ここではまだ否定しない。
}
\end{frame}
% ----------------------------------------------------------------------------------------
%   Slide 02: ポテト調査を思い出そう
% ----------------------------------------------------------------------------------------

%@@PAGEBAND@@
% ----------------------------------------------------------------------------------------
%   page 02
% ----------------------------------------------------------------------------------------
\begin{frame}{ポテト30人分の調査を思い出そう}
ハンバーガーショップのポテト調査では──

\vspace{0.6em}
\begin{itemize}
  \item 1日につき \textbf{30人分} の重さを測った
  \item その30人分をまとめて \textbf{平均} を計算した
\end{itemize}

\vspace{0.8em}
ここで自然に出てくる考えは:

\vspace{0.6em}
\begin{center}
\Large
\textbf{「30人も集めれば、\\
平均はほぼ同じになるのでは?」}
\end{center}

\noteT{狙い}{
学生が「そう思っていた」と共感できる
自然な予想を言語化する。
}
\end{frame}

% ----------------------------------------------------------------------------------------
%   Slide 02.5: 平均に対する直感
% ----------------------------------------------------------------------------------------

%@@PAGEBAND@@
% ----------------------------------------------------------------------------------------
%   page 03
% ----------------------------------------------------------------------------------------
\begin{frame}{多くの人が、こう考えています}
平均について、私たちはついこう考えがちです。

\vspace{0.8em}
\begin{itemize}
  \item 人数が少ないと、平均はブレやすい
  \item 人数が多くなると、平均は安定する
  \item 30人も集めれば、もう十分では?
\end{itemize}

\vspace{1.0em}
\begin{center}
\Large
\textbf{この考えは、感覚的にはとても自然}
\end{center}

\vspace{0.6em}
しかし──

\noteT{狙い}{
学生が「自分もそう思っていた」と
安心してうなずけるスライド。
ここでは否定しない。
}
\end{frame}

% ----------------------------------------------------------------------------------------
%   Slide 03: 今日の問い
% ----------------------------------------------------------------------------------------

%@@PAGEBAND@@
% ----------------------------------------------------------------------------------------
%   page 04
% ----------------------------------------------------------------------------------------
\begin{frame}{今日、確かめたいこと}
そこで、今日は次のことを確かめます。

\vspace{0.8em}
\textbf{問い①}
\begin{itemize}
  \item 30人分の平均は、毎回ほぼ同じになるのか?
  \item それとも、まだ少しずつ違うのか?
\end{itemize}

\vspace{0.8em}
\textbf{問い②}
\begin{itemize}
  \item 平均を \textbf{何回も} 作ったら、どうなるのか?
  \item その集まりには、何か特徴が現れるのか?
\end{itemize}

\vspace{0.8em}
\begin{center}
\textbf{今日は、計算ではなく「実験」で確かめます}
\end{center}

\noteT{狙い}{
この後の実習①・②が、
単なるExcel操作ではなく
「問いに答える実験」だと理解させる。
}
\end{frame}

% ----------------------------------------------------------------------------------------
%   Slide X: では、実験してみよう
% ----------------------------------------------------------------------------------------

%@@PAGEBAND@@
% ----------------------------------------------------------------------------------------
%   page 05
% ----------------------------------------------------------------------------------------
\begin{frame}{では、実験してみよう}
ここまでの話は、すべて \textbf{予想} です。

\vspace{0.8em}
\begin{itemize}
  \item 30人も平均すれば安定する?
  \item 日が違っても、平均は同じ?
\end{itemize}

\vspace{1.0em}
\begin{center}
\Large
\textbf{実際のデータで、確かめてみます}
\end{center}

\vspace{0.8em}
今日の実習では、
\begin{itemize}
  \item 1日分(30人)の平均を作り
  \item それを日ごとに比べます
\end{itemize}

\noteT{狙い}{
「実習=作業」ではなく
「予想を検証する実験」だと位置づける。
}
\end{frame}

% ----------------------------------------------------------------------------------------
%   実習①-1:日別平均を調べる
% ----------------------------------------------------------------------------------------

%@@PAGEBAND@@
% ----------------------------------------------------------------------------------------
%   page 06
% ----------------------------------------------------------------------------------------
\begin{frame}{実習①:日別平均は本当に同じか?}
\textbf{今日の実習の目的}

\vspace{0.6em}
\begin{itemize}
  \item ポテト30人分の重さデータがある
  \item 1日分(30人)をまとめて \textbf{平均} を計算する
  \item 日ごとの平均を比べてみる
\end{itemize}

\vspace{0.8em}
\textbf{確認したいこと}

\begin{itemize}
  \item 30人も平均すれば、毎回ほぼ同じになる?
  \item それとも、まだ \textbf{揺れ} が残る?
\end{itemize}

\noteT{狙い}{
「平均=安定しそう」という直感を一度信じさせてから、
実際のデータで揺れを確認させる。
}
\end{frame}
% ----------------------------------------------------------------------------------------
%   実習①-2:日別平均を計算する
% ----------------------------------------------------------------------------------------

%@@PAGEBAND@@
% ----------------------------------------------------------------------------------------
%   page 07
% ----------------------------------------------------------------------------------------
\begin{frame}[fragile]{実習①:日別平均を計算する}
\textbf{手順①:各日の平均を出す}

\vspace{-0.8em}
\begin{enumerate}
  \item poteto30.xlsx を開く
  \item G4〜M4に月〜日の値を入れる
  \item G5〜M5に式をセット
  \item 曜日別の平均を計算する
\end{enumerate}

\vspace{0.2em}
\textbf{例(月曜日 の場合)}
\vspace{-0.4em}
%\begin{itemize}
%  \item セルに次の式を入力:
%\end{itemize}
%\vspace{-0.8em}
\hspace*{2em}%
\begin{tcolorbox}[title=セルに次の式を入力]
\texttt{=AVERAGEIF(B:B,G4,C:C)}
)
\end{tcolorbox}
%%% \markline{\texttt{=AVERAGEIF(B:B,G4,C:C)}}

%\begin{minipage}{0.90\linewidth}
%\begin{block}{}
%\ttfamily
%=AVERAGEIF(B:B,G4,C:C)
%\end{block}
%\end{minipage}
%\markline{\texttt{=AVERAGEIF(B:B,G4,C:C)}}
% \begin{center}
% \texttt{=AVERAGE(B2:B31)}
% \end{center}

\vspace{0.8em}
\textbf{月〜日 すべてについて平均を求める}

\noteT{操作}{
「30人分=1日分の調査」という考えを強調。
まだサンプル・母集団という言葉は使わない。
}
\end{frame}
% ----------------------------------------------------------------------------------------
%   実習①-3:日別平均を比べる
% ----------------------------------------------------------------------------------------

%@@PAGEBAND@@
% ----------------------------------------------------------------------------------------
%   page 08
% ----------------------------------------------------------------------------------------
\begin{frame}{実習①:日別平均を比べてみる}
\textbf{手順②:平均を並べて確認}

\vspace{0.6em}
\begin{itemize}
  \item 月曜日〜日曜日 の平均値を横に並べる
  \item 数値をそのまま見比べる
\end{itemize}

\vspace{0.8em}
\textbf{余裕があれば}

\begin{itemize}
  \item 日別平均の折れ線グラフを作成する
  \item 縦軸:平均値/横軸:日
\end{itemize}

\vspace{0.8em}
\textbf{確認}

\begin{itemize}
  \item 平均は毎回まったく同じ?
  \item 少しずつ上下に動いていない?
\end{itemize}

\noteT{狙い}{
「平均をとっても揺れる」という事実を、
理屈なしで体験させる。
次の実習への違和感を残す。
}
\end{frame}

% ----------------------------------------------------------------------------------------
%   実習②-1:平均をたくさん作る
% ----------------------------------------------------------------------------------------

%@@PAGEBAND@@
% ----------------------------------------------------------------------------------------
%   page 09
% ----------------------------------------------------------------------------------------
\begin{frame}{実習②:平均をたくさん作る}
\textbf{ここからの実習の目的}

\vspace{0.6em}
\begin{itemize}
  \item 実習①では、日ごとの平均が \textbf{揺れる} ことを確認した
  \item では、平均を \textbf{何回も} 作るとどうなるか?
\end{itemize}

\vspace{0.8em}
\textbf{今日やること}

\begin{itemize}
  \item 30人分のポテトデータから
  \item 「1回分の平均」を何度も作る
  \item その平均たちを集めて眺める
\end{itemize}

\noteT{狙い}{
「平均は1つの値ではなく、集めると分布になる」
という感覚を体験で作る。
}
\end{frame}

% ----------------------------------------------------------------------------------------
%   実習②-2:平均を1回作る(関数方式)
% ----------------------------------------------------------------------------------------

%@@PAGEBAND@@
% ----------------------------------------------------------------------------------------
%   page 10
% ----------------------------------------------------------------------------------------
\begin{frame}[fragile]{実習②:平均を1回作る(関数で一発)}
\textbf{手順①:抽出件数を決める}

\vspace{0.3em}
\begin{itemize}
  \item \textbf{B1セル}に「取り出す件数」を入力(例:30)
\end{itemize}

\vspace{0.6em}
\textbf{手順②:ランダムに抜き出した平均を1つ作る}

\vspace{0.2em}
\begin{itemize}
  \item \textbf{B2セル}に次の式を入力(A2:A211が210件のデータ)
\end{itemize}

\vspace{-0.2em}
%	\begin{minipage}{0.95\linewidth}
%	\begin{block}{B2に入力する式}
%	\ttfamily
%	=AVERAGE(\\
%	\ \ TAKE(\\
%	\ \ \ \ SORTBY(\$A\$2:\$A\$211, RANDARRAY(ROWS(\$A\$2:\$A\$211))),\\
%	\ \ \ \ \$B\$1\\
%	\ \ )\\
%	)
%	\end{block}
%	\end{minipage}

\begin{tcolorbox}[title=B2に入力する式]
=AVERAGE(\\
\hspace{20pt} TAKE(\\
\hspace{40pt} SORTBY(\$A\$2:\$A\$211, RANDARRAY(ROWS(\$A\$2:\$A\$211))),\\
\hspace{40pt} \$B\$1\\
\hspace{20pt} )\\
)
\end{tcolorbox}

\vspace{0.4ex}
\textbf{確認:} B2に「1回分の平均」が表示される

\noteT{狙い}{
並べ替え作業を\textbf{関数に隠す}ことで、実習を「作業」から「観察」へ寄せる。
B1=30は「30個取り出した平均」を意味する(用語はまだ出さない)。
}
\end{frame}

% ----------------------------------------------------------------------------------------
%   実習②-3:平均を何回も作る(コピーで量産)
% ----------------------------------------------------------------------------------------

%@@PAGEBAND@@
% ----------------------------------------------------------------------------------------
%   page 11
% ----------------------------------------------------------------------------------------
\begin{frame}{実習②:平均を何回も作る(コピーで量産)}
\textbf{手順③:平均をたくさん作る}

\vspace{0.6em}
\begin{itemize}
  \item B2の式を \textbf{下方向にコピー}して平均を量産する
  \item 例:B2〜B51(\textbf{50回分の平均})
\end{itemize}

\vspace{0.8em}
\textbf{ポイント}

\begin{itemize}
  \item コピーした各セルは、毎回ランダムに取り出すので値が少し変わる
  \item これらの平均値1つ1つを「結果」として集める
\end{itemize}

\noteT{操作}{
値が動いて見えたらOK。
動きが気になる場合は「コピー→値貼り付け」で固定してもよい(任意)。
}
\end{frame}

% ----------------------------------------------------------------------------------------
%   実習②-4:平均の集まりを見る(まずは数字で観察)
% ----------------------------------------------------------------------------------------

%@@PAGEBAND@@
% ----------------------------------------------------------------------------------------
%   page 12
% ----------------------------------------------------------------------------------------
\begin{frame}{実習②:平均の集まりを観察する}
\textbf{手順④:平均の列を眺める}

\vspace{0.6em}
\begin{itemize}
  \item B2〜B51 に平均値が並んだ(例:50個)
  \item まずは「どのあたりに集まっているか」を数字で確認する
\end{itemize}

\vspace{0.8em}
\textbf{確認}

\begin{itemize}
  \item 平均は150g付近に集まっている?
  \item 実習①(日別平均)より揺れは小さく見える?
\end{itemize}

\noteT{狙い}{
「平均も集めると分布になる」を体験する直前段階。
ここではまだヒストグラムを作らず、数値の集まりを観察する。
}
\end{frame}

% ----------------------------------------------------------------------------------------
%   実習②-5:平均のヒストグラム(関数方式に合わせた版)
% ----------------------------------------------------------------------------------------

%@@PAGEBAND@@
% ----------------------------------------------------------------------------------------
%   page 13
% ----------------------------------------------------------------------------------------
\begin{frame}{実習②:平均のヒストグラムを作る}
\textbf{手順⑤:平均値(B列)だけでヒストグラム}

\vspace{0.3ex}
\begin{itemize}
  \item B2 から下にコピーして作った「平均値の列」を使う
  \item 例:\textbf{B2:B51(50回分の平均)}
\end{itemize}

\vspace{0.4ex}
\textbf{操作(Excel)}
\begin{enumerate}
  \item 範囲 \textbf{B2:B51} を選択
  \item \textbf{挿入} → \textbf{統計グラフ} → \textbf{ヒストグラム}
  \item (必要なら)\textbf{軸の書式設定}で\textbf{ビン幅}や\textbf{ビン数}を調整
\end{enumerate}

\vspace{0.4ex}
\textbf{観察ポイント}
\begin{itemize}
  \item 平均値は \textbf{どのあたり} に集まっている?(150g付近?)
  \item \textbf{広がり(ばらつき)} は大きい?小さい?
  \item 元データのヒストグラム(ポテトの重さ)と比べて、\textbf{形}はどう違う?
\end{itemize}

\noteT{狙い}{
「平均を集めると分布になる」を視覚で体験させる。
ここでは用語(中心極限定理)は言わず、
「平均のヒストグラムは、元データより細くまとまる」感覚を作る。
}
\end{frame}

% ----------------------------------------------------------------------------------------
%   Slide J: 実習で何が起きたか(事実の整理)
% ----------------------------------------------------------------------------------------

%@@PAGEBAND@@
% ----------------------------------------------------------------------------------------
%   page 14
% ----------------------------------------------------------------------------------------
\begin{frame}{実習で何が起きたか}
ここまでの実習で、次のことを確認しました。

\vspace{0.6em}
\begin{itemize}
  \item 30人分の平均でも、値は \textbf{毎回少しずつ違った}
  \item 平均を1回だけ見ると、\textbf{まだ揺れが残る}
  \item 平均をたくさん集めると、\textbf{中心に集まった}
\end{itemize}

\vspace{0.8em}
\textbf{重要:}
\begin{center}
%%%\begin{tcolorbox}[hbox, sharp corners, arc=0mm]
\begin{tcolorbox}[width=7cm,  colback=BananaColor,sharp corners, arc=0mm]
\centering
平均は「固定の値」ではなく、\\
\textbf{揺れを持った結果のひとつ}
\end{tcolorbox}
\end{center}

\noteT{狙い}{
結果の評価を入れず、「事実」だけを整理する。
次の「なぜ?」に自然につなぐ。
}
\end{frame}

% ----------------------------------------------------------------------------------------
%   Slide K: 平均もデータである
% ----------------------------------------------------------------------------------------

%@@PAGEBAND@@
% ----------------------------------------------------------------------------------------
%   page 15
% ----------------------------------------------------------------------------------------
\begin{frame}{平均も「1つのデータ」}
ここで見方を少し変えます。

\vspace{0.6em}
\begin{itemize}
  \item ポテトの重さ → データ
  \item 30人分の平均 → \textbf{これも1つのデータ}
\end{itemize}

\vspace{0.8em}
\textbf{実習②でやったこと}

\begin{itemize}
  \item 平均を \textbf{何回も} 作った
  \item その平均たちを \textbf{集めた}
\end{itemize}

\vspace{0.8em}
%	\begin{center}
%	\textbf{平均の集まりも、分布になる}
%	\end{center}

\begin{center}
\begin{tcolorbox}[width=7cm,  colback=BananaColor,sharp corners, arc=0mm]
\centering
\textbf{平均の集まりも、分布になる}
\end{tcolorbox}
\end{center}


\noteT{狙い}{
「平均=答え」から
「平均=結果のひとつ」へ視点をずらす。
}
\end{frame}

% ----------------------------------------------------------------------------------------
%   Slide L: 平均の分布の特徴
% ----------------------------------------------------------------------------------------

%@@PAGEBAND@@
% ----------------------------------------------------------------------------------------
%   page 16
% ----------------------------------------------------------------------------------------
\begin{frame}{平均の分布で見えた特徴}
平均だけを集めたヒストグラムを見ると──

\vspace{0.6em}
\begin{itemize}
  \item 真ん中あたりに集まっている
  \item 元データより \textbf{広がりが小さい}
  \item なめらかな \textbf{山の形} になっている
\end{itemize}

\vspace{0.8em}
\textbf{比較すると}

\begin{itemize}
  \item 元の重さデータ:ばらつきが大きい
  \item 平均のデータ:ばらつきが小さい
\end{itemize}

\noteT{狙い}{
「平均をとると安定する」
という現象を、視覚の記憶として固定する。
}
\end{frame}

% ----------------------------------------------------------------------------------------
%   Slide M: ここで生まれる疑問
% ----------------------------------------------------------------------------------------

%@@PAGEBAND@@
% ----------------------------------------------------------------------------------------
%   page 17
% ----------------------------------------------------------------------------------------
\begin{frame}{ここで生まれる疑問}
実習結果を見て、次の疑問が自然に出てきます。

\vspace{0.6em}
\begin{itemize}
  \item なぜ平均を集めると、中心に集まるのか?
  \item なぜ回数が増えるほど、揺れが小さくなるのか?
  \item なぜ山の形が、だんだん整ってくるのか?
\end{itemize}

\vspace{0.8em}
\begin{center}
\begin{tcolorbox}[width=7cm,  colback=BananaColor,sharp corners, arc=0mm]
\centering
\textbf{これは偶然ではない}
\end{tcolorbox}
\end{center}


\noteT{狙い}{
「不思議だ」で止めず、
理論が必要になる地点まで導く。
}
\end{frame}

% ----------------------------------------------------------------------------------------
%   Slide N: この現象には名前がある
% ----------------------------------------------------------------------------------------

%@@PAGEBAND@@
% ----------------------------------------------------------------------------------------
%   page 18
% ----------------------------------------------------------------------------------------
\begin{frame}{この現象には、名前がある}
今見てきた現象は、  
統計の世界では \textbf{よく知られた性質} です。

\vspace{0.8em}
\begin{itemize}
  \item 平均をとると安定する
  \item 回数を増やすほど、その傾向は強くなる
\end{itemize}

\vspace{0.8em}
この考え方を  
\begin{center}
\begin{tcolorbox}[width=7cm,  colback=BananaColor,sharp corners, arc=0mm]
\centering
\textbf{大数の法則}
\end{tcolorbox}
\end{center}
と呼びます。

\noteT{狙い}{
ここで初めて理論名を出す。
実験→現象→命名、の順を守る。
}
\end{frame}

% ----------------------------------------------------------------------------------------
%   Slide O: 平均の分布が整う理由(中心極限定理への橋)
% ----------------------------------------------------------------------------------------

%@@PAGEBAND@@
% ----------------------------------------------------------------------------------------
%   page 19
% ----------------------------------------------------------------------------------------
\begin{frame}{なぜ、平均の分布は整っていくのか}
ここで、平均の作られ方を思い出します。

\vspace{0.6em}
\begin{itemize}
  \item 1回の平均は、\textbf{30人分の重さ}を使っている
  \item つまり、\textbf{30個の値を足して割ったもの}
\end{itemize}

\vspace{0.8em}
\textbf{言い換えると:}

\begin{itemize}
  \item 平均は「たくさんの小さなズレ」をまとめた結果
  \item ズレが正にも負にもあると、\textbf{打ち消し合う}
\end{itemize}

\vspace{0.8em}
\begin{center}
\begin{tcolorbox}[width=7cm,  colback=BananaColor,sharp corners, arc=0mm]
\centering
\textbf{その結果、極端な値が出にくくなる}
\end{tcolorbox}
\end{center}

\noteT{狙い}{
「平均=足し算の集まり」であることを明確にし、
偶然ではないことを直感的に理解させる。
}
\end{frame}

% ----------------------------------------------------------------------------------------
%   Slide O': 中心極限定理とは何か(直感版)
% ----------------------------------------------------------------------------------------

%@@PAGEBAND@@
% ----------------------------------------------------------------------------------------
%   page 20
% ----------------------------------------------------------------------------------------
\begin{frame}{中心極限定理(直感的な説明)}
このように、

\vspace{0.6em}
\begin{itemize}
  \item たくさんの値を使って平均を作ると
  \item 元のデータの形に関係なく
  \item 平均の分布は \textbf{似た形} になっていく
\end{itemize}

\vspace{0.8em}
この性質を
\vspace{-0.4ex}
\begin{center}
\begin{tcolorbox}[width=7cm,  colback=BananaColor!70,sharp corners, arc=0mm]
\centering
\textbf{中心極限定理}
\end{tcolorbox}
\end{center}
\vspace{0.2ex}
と呼びます。

\vspace{0.6em}
※ 今日は「なぜ起きるか」を体験で理解する回  
  数式での説明は次回以降に扱います

\noteT{狙い}{
CLTを「公式」ではなく
「平均を集めたときに必ず起きる性質」として定義する。
}
\end{frame}

% ----------------------------------------------------------------------------------------
%   Slide P: 今日の整理
% ----------------------------------------------------------------------------------------

%@@PAGEBAND@@
% ----------------------------------------------------------------------------------------
%   page 21
% ----------------------------------------------------------------------------------------
\begin{frame}{今日の整理}
今日の実習と整理から、次のことが言えます。

\vspace{0.6em}
\begin{itemize}
  \item 平均は1回では揺れる
  \item たくさん集めると安定する
  \item 平均の分布には決まった性質がある
\end{itemize}

\vspace{0.8em}
\begin{center}
\begin{tcolorbox}[width=7cm,  colback=BananaColor!70,sharp corners, arc=0mm]
\centering
\textbf{だから平均は、判断に使える}
\end{tcolorbox}
\end{center}

\noteT{狙い}{
「平均が使える理由」を
体験ベースで納得させる。
}
\end{frame}

% ----------------------------------------------------------------------------------------
% ===== 第7回:実習後まとめ(Slide Q〜S)=====
%   Slide Q: 今日わかったこと(整理)
% ----------------------------------------------------------------------------------------

%@@PAGEBAND@@
% ----------------------------------------------------------------------------------------
%   page 22
% ----------------------------------------------------------------------------------------
\begin{frame}{今日わかったこと(実習の整理)}
\textbf{実習①:日別平均を見た}

\vspace{0.4em}
\begin{itemize}
  \item 30人分の平均でも、日によって \textbf{少し揺れる}
  \item 平均は「固定の値」ではなく、\textbf{毎回少し変わる}ものだった
\end{itemize}

\vspace{0.7em}
\textbf{実習②:平均をたくさん作って集めた}

\vspace{0.4em}
\begin{itemize}
  \item 平均を繰り返し作ると、平均値にも \textbf{分布(山の形)}ができた
  \item 元データよりも、\textbf{平均の分布は広がりが小さく}なった
\end{itemize}

\vspace{0.8em}
\begin{center}
\textbf{まとめ:平均は「揺れる」けれど、集めると「安定のしかた」が見える}
\end{center}

\noteT{狙い}{
実習①・②を「何が起きたか」だけで整理する。
ここではまだ理由(理屈)に踏み込まず、現象の確定を優先する。
}
\end{frame}

% ----------------------------------------------------------------------------------------
%   Slide R: 大数の法則と中心極限定理(区別)
% ----------------------------------------------------------------------------------------

%@@PAGEBAND@@
% ----------------------------------------------------------------------------------------
%   page 23
% ----------------------------------------------------------------------------------------
\begin{frame}{大数の法則と中心極限定理:何が違う?}
今日見た現象には、\textbf{2つの名前}が関係します。

\vspace{0.7em}
\begin{columns}[T,onlytextwidth]
  \begin{column}{0.50\textwidth}
    \textbf{大数の法則(平均が安定)}
    \vspace{0.3em}
    \begin{itemize}
      \item 1回の平均を考える
      \item 人数(件数)を増やすほど
      \item 平均は \textbf{真ん中へ寄って安定}する
    \end{itemize}

    \vspace{0.4em}
    \textbf{キーワード:}
    \begin{itemize}
      \item 「平均との差が小さくなる」
      \item 「平均がブレにくくなる」
    \end{itemize}
  \end{column}

  \begin{column}{0.50\textwidth}
    \textbf{中心極限定理(平均の分布が山になる)}
    \vspace{0.3em}
    \begin{itemize}
      \item 平均を \textbf{何回も} 作る
      \item その平均たちを集めると
      \item 分布の形が \textbf{山に似てくる}
    \end{itemize}

    \vspace{0.4em}
    \textbf{キーワード:}
    \begin{itemize}
      \item 「平均にも分布がある」
      \item 「平均の分布は整ってくる」
    \end{itemize}
  \end{column}
\end{columns}

\vspace{0.7em}
\begin{center}
\begin{tcolorbox}[width=10cm,  colback=BananaColor!60,sharp corners, arc=0mm]
\centering
\textbf{大数の法則=「1つの平均が安定」}\\
\textbf{中心極限定理=「平均の集まりが山になる」}
\end{tcolorbox}
\end{center}


\noteT{狙い}{
両者をごちゃ混ぜにしないための最小の整理。
数式なしで、見た現象と対応づけるだけに徹する。
}
\end{frame}

% ----------------------------------------------------------------------------------------
%   Slide S: 次につながる意味(推測統計・検定への接続)
% ----------------------------------------------------------------------------------------

%@@PAGEBAND@@
% ----------------------------------------------------------------------------------------
%   page 24
% ----------------------------------------------------------------------------------------
\begin{frame}{次につながる意味:なぜこれが大事?}
今日の結論は、これです。

\vspace{0.6em}
\begin{itemize}
  \item 平均は「揺れる」\quad →\quad だから \textbf{1回の結果だけでは判断できない}
  \item しかし、平均の揺れ方には \textbf{規則性} がある
  \item 平均を集めると \textbf{山の形(分布)} が見えてくる
\end{itemize}

\vspace{0.8em}
\textbf{ここから先でやりたいこと}

\vspace{0.4em}
\begin{itemize}
  \item 「この平均は普通?それとも外れている?」
  \item 「どれくらい外れたら、外れていると言える?」
  \item 「偶然の揺れ」か「本当に違う」かを区別したい
\end{itemize}

\vspace{0.8em}
\begin{center}
\textbf{次回以降:標本と母集団 → 推測 → 仮説検定}\\
(今日の現象が、その土台になる)
\end{center}

\noteT{狙い}{
推測統計・検定の必然性を「平均は揺れる」から導く。
ここでは専門語(母集団など)を出す場合は一言に留め、次回以降に回す。
}
\end{frame}
