%----------------------------------------------------------------------------------------
%  metropolis template (refactored)
%----------------------------------------------------------------------------------------
\documentclass[handout,aspectratio=169]{beamer}

% \documentclass の直後で hyperref のオプションを渡す(metropolisでも安全)
\PassOptionsToPackage{unicode=true,colorlinks=true,linkcolor=blue,urlcolor=blue}{hyperref}

\usetheme{metropolis}
\metroset{block=fill, sectionpage=progressbar, progressbar=foot}

% 背景色(tech のとき白など): Pythonで差し込み
\setbeamercolor{background canvas}{bg=white}

%--------------------------
% 日本語
%--------------------------
\usepackage{luatexja}
\usepackage{luatexja-fontspec}
\usepackage{luatexja-ruby}
\setsansjfont{Hiragino Sans}[BoldFont={Hiragino Sans W6}]

%--------------------------
% 基本パッケージ(重複なし)
%--------------------------
\usepackage[table]{xcolor}
\usepackage{graphicx}
\usepackage[abs]{overpic}
\usepackage{tikz}
\usetikzlibrary{positioning, shapes.geometric} % ← shapes.geometric を追加
\usepackage{array}
\usepackage{tabularx}
\usepackage{booktabs}
\usepackage{makecell}
\usepackage{mathtools}
\usepackage{longtable}
\usepackage{pdfpages}
\usepackage{etoolbox} % AtBeginEnvironment 等
\usepackage[normalem]{ulem}
\usetikzlibrary{calc}
\usetikzlibrary{backgrounds}

\usepackage{pgf}

% マーカー(ハイライター)の定義
\newcommand{\markline}[2][yellow]{%
  \tikz[baseline=(X.base)]{%
    \node[inner sep=0pt,outer sep=0pt] (X) {#2};
    \begin{scope}[on background layer]
      \fill[#1, opacity=0.35, rounded corners=0.8pt]
        ([xshift=-0.15em,yshift=0.00ex]X.south west) rectangle
        ([xshift= 0.15em,yshift=2.15ex]X.south east);
    \end{scope}
  }%
}



% minted(※ -shell-escape 必須)
\usepackage{minted}
\setminted{
  frame=single,
  framesep=2mm,
  fontsize=\footnotesize,
  breaklines=true
}

% (必要なときだけ)tcolorbox
\usepackage[most]{tcolorbox}
\tcbuselibrary{skins}        % 高度なデザイン機能
\tcbuselibrary{raster}

% hyperref は最後
\usepackage{hyperref}

% --- 1. カラーパレットの定義 ---
\definecolor{CanvaGreen}{HTML}{2E7D32} % メインの濃い緑(タイトル・枠線)
\definecolor{PaleGreen}{HTML}{F1F8E9}  % 背景の薄い緑
\definecolor{MyDarkGreen}{HTML}{587a7f}
\definecolor{DeepText}{HTML}{1C1C1C}   % 本文の文字色
\definecolor{MyWhiteBlue}{HTML}{F2FAFB} % 追加:あなたが指定した色
\definecolor{BananaColor}{HTML}{FFFD78}


% --- 2. tcolorbox のデフォルト設定(全ボックスに適用) ---
%\tcbset{
%    enhanced,                          % 高度な装飾を有効化
%    colback=white,                 % 本文背景色
%    colframe=MyDarkGreen,               % 枠線の色
%    coltitle=white,                    % タイトル文字色
%    fonttitle=\bfseries\sffamily,      % タイトルを太字・ゴシック
%    boxrule=1pt,                       % 枠線の太さ
%    arc=2mm,                           % 角の丸み
%    left=3mm, right=3mm,               % 左右の余白
%    top=0.5mm, bottom=0.5mm,               % 本文の上下余白
%    toptitle=0.8mm, bottomtitle=0.5mm, % タイトル内の上下余白
%    before skip=0.8em, after skip=0.2em,   % ボックス前後の行間
%    shadow={0mm}{0mm}{0mm}{black!0}    % 影を完全に消してフラットに
%}
% --- 2. tcolorbox のデフォルト設定 ---
\tcbset{
    enhanced,
    colback=MyWhiteBlue,            % 本文背景を F2FAFB に変更
    colframe=MyDarkGreen,
    coltitle=white,                 % タイトル文字は白(背景が濃い色の場合)
    % coltitle=DeepText,            % もしタイトル背景も薄くするならこちら
    fonttitle=\bfseries\sffamily\small, % タイトルを少し小さくしてスリム化
    boxrule=0.5pt,                  % 1pt から 0.5pt へ細分化
    arc=1mm,                        % 2mm から 1mm へ変更
    sharp corners=south,            % 下側を直角に固定
    % --- 余白の調整 ---
    left=3mm, right=3mm,
    top=0.5mm, bottom=0.5mm,
    toptitle=0mm,                   % タイトル上の余白をゼロに
    bottomtitle=0mm,                % タイトル下の余白をゼロに
    % ------------------
    before skip=0.8em, after skip=0.2em,
    shadow={0mm}{0mm}{0mm}{black!0}
}

% --- カスタムボックスの定義 ---
\newtcolorbox{myListbox}[1]{
  enhanced,
  detach title,              % 1. 標準のタイトル位置を解除
  % 2. 本文が始まる直前(before upper)でタイトルを直接描画する
  before upper={{\bfseries\large #1}\par\medskip},
  % タイトルの後に改行を入れる設定
  after title={\par\medskip},
  colbacktitle=white, 
  colframe=gray!50,
  colback=white,
  titlerule=0pt,
  boxrule=1pt,
  fonttitle=\bfseries\color{black},
  title=#1,
  after skip=1.5ex,
  % --- 箇条書きの余白を強制的にゼロにする設定 ---
  before upper={
    \setbeamertemplate{itemize ispan}{0pt} % 項目間の余白
    \setbeamertemplate{itemize items}[default]
    \setlength{\leftmargini}{1.5em}
    % Beamerの内部変数を直接操作して行間を詰める
    \addtobeamertemplate{itemize/enumerate body begin}{}{\setlength{\itemsep}{0pt}\setlength{\parskip}{0pt}}
  }
}

%--------------------------
% パス
%--------------------------
\newcommand{\assetpath}{/Volumes/NBPlan/TTC/授業資料/2025年度/}
\graphicspath{{images/}{\assetpath/1020201.アルゴリズム2/10/images/}{../project_assets/images/}{../project_assets/emoji/emoji_pngs/}}
% PGFがある特定の階層を定義
\newcommand{\pgfpath}{\assetpath/1020201.アルゴリズム2/10/images/}

%--------------------------
% フッター
%--------------------------
\newcommand{\myfootertext}{1020201.アルゴリズム2/10}
\setbeamertemplate{footline}{%
  \leavevmode
  \hbox to \paperwidth{%
    \hspace*{0.2cm}
    \scriptsize\color{gray!50} \myfootertext
    \hfill
    \scriptsize\color{gray} \insertframenumber{} / \inserttotalframenumber
    \hspace*{0.4cm}
  }%
  \vspace{1pt}
}

%--------------------------
% TeacherFrame(外部)
%--------------------------
\usepackage{../teacherframe}

%--------------------------
% フレームタイトル:番号. タイトル
% ※ ここで出すだけ。insertframetitle を再定義しない(安全)
%--------------------------
\setbeamertemplate{frametitle}{%
  \vspace{0.6ex}%
  \begin{beamercolorbox}[wd=\paperwidth,sep=0.5ex,leftskip=0.9em,rightskip=0.5em]{frametitle}%
    \usebeamerfont{frametitle}%
    \insertframenumber.\,\insertframetitle%
  \end{beamercolorbox}%
}

%--------------------------
% 表用:列型
%--------------------------
\newcolumntype{C}[1]{>{\centering\arraybackslash}p{#1}}
\newcolumntype{M}[1]{>{\raggedright\arraybackslash}m{#1}}

%--------------------------
% ブロック(必要なら)
%--------------------------
\definecolor{myblue}{HTML}{7488FF}
\definecolor{mylightblue}{HTML}{E3EEFF}
\setbeamertemplate{blocks}[rounded]
\setbeamercolor{block title}{bg=myblue, fg=white}
\setbeamercolor{block body}{bg=mylightblue, fg=black}

%========================================================
% exampleblock(examplebox相当)だけの調整
%  - タイトル文字:白
%  - 背景色:現行のまま(bgは指定しない)
%  - 本文 itemize:文字も●も黒(exampleblock内だけ)
%========================================================

% タイトル文字だけ白(背景は触らない)
\setbeamercolor{block title example}{fg=white}

% 本文の通常文字色は「現行のまま」を基本にする(必要なら黒にしてもよい)
% ここは bg を触らないのが目的なので fg だけ調整可能
\setbeamercolor{block body example}{fg=black}

% exampleblock の中だけ itemize の色(●と文字)を黒に
\AtBeginEnvironment{exampleblock}{%
  \setbeamercolor{itemize item}{fg=black}
  \setbeamercolor{itemize subitem}{fg=black}
  \setbeamercolor{itemize subsubitem}{fg=black}
  \setbeamercolor{item}{fg=black} % 念のため
}

% exampleblock を抜けたらテーマ標準に戻す(色が残る事故防止)
\AtEndEnvironment{exampleblock}{%
  \setbeamercolor{itemize item}{fg=normal text.fg}
  \setbeamercolor{itemize subitem}{fg=normal text.fg}
  \setbeamercolor{itemize subsubitem}{fg=normal text.fg}
  \setbeamercolor{item}{fg=normal text.fg}
}

%--------------------------
% 奇数ページのスライドのを表示する
%(教示用だけでそれ以外はこの処理は動かない)
%--------------------------
% --- 教師用だけ、スライドを奇数開始に強制するトグル ---
\newif\ifoddslideenforce
\oddslideenforcefalse   % デフォルトOFF(pr/hoはOFF)

% --- 再帰防止ガード ---
\newif\ifoddslideguard
\oddslideguardfalse

% --- 偶数ページなら空白スライドを1枚入れて奇数に戻す ---
\newcommand{\ensureoddslide}{%
  \ifoddslideguard\relax\else
    \oddslideguardtrue
    \ifodd\value{page}\relax
      % 何もしない(次が奇数)
    \else
      \begin{frame}[plain,noframenumbering]
        \note{}% notes出力時に2枚消費させる保険
      \end{frame}
    \fi
    \oddslideguardfalse
  \fi
}

% --- frameが始まる直前に自動挿入(教師用だけ)---
\BeforeBeginEnvironment{frame}{%
  \ifoddslideenforce
    \ensureoddslide
  \fi
}

%--------------------------
% note / noteT の「常時安全化」
%  - tech 以外:\noteT は無視(エラーにならない)
%  - tech:notesmode_tech で上書き定義
%--------------------------
\providecommand{\notetitletext}{}      % 既にあっても衝突しない
\providecommand{\noteT}[2]{}           % デフォルトは何もしない

% frame開始ごとにタイトル変数をクリア(前の noteT が残らないように)
\AtBeginEnvironment{frame}{\gdef\notetitletext{}}

%--------------------------
% 切替(Pythonから差し込み)
%--------------------------
\mypausemodetrue
\teachermodetrue
\setbeameroption{show notes}
%-------------

% --- tech のときだけ noteT を有効化(テンプレートの \providecommand を上書き) ---
\makeatletter
\renewcommand{\noteT}[2]{%
 \gdef\notetitletext{#1}%
 \note{#2}%
}
% タイトル未指定のときのために初期化
\renewcommand{\notetitletext}{}%

\setbeamertemplate{note page}{%
 \begin{minipage}{\linewidth}
 \vspace{1.2ex} % タイトルを少し下げる(必要に応じて調整)
 {\Large\bfseries
 \ifx\notetitletext\@empty
 \insertframetitle
 \else
 \notetitletext
 \fi
 }\par
 \vspace{-1.2ex}
 \rule{\linewidth}{0.8pt}\par
 \vspace{0.8ex}
 {\scriptsize \insertnote}
 \end{minipage}
}
\makeatother

%教師用のPDFは奇数ページからスライドを出力
\oddslideenforcetrue


%--------------------------
% 方眼紙(グリッド)をスライドに重ね
%--------------------------
\input{grid_debug}

%-------------------------------------------------------------------------
% 「ラインマーカー」そのものです。線の太さ・色・透明度を自由にできます。
%
% #1 色(省略可)
% #2 下端 yshift
% #3 上端 yshift
% #4 文字
% \marklineA{0.35ex}{1.55ex}{通常サイズ}
% {\Large \marklineA{0.45ex}{2.10ex}{大きい文字}}
%-------------------------------------------------------------------------
\newcommand{\marklineA}[4][yellow]{%
  \tikz[baseline=(X.base)]{%
    \node[inner sep=0pt,outer sep=0pt] (X) {#4};
    \begin{scope}[on background layer]
      \fill[#1, opacity=0.35, rounded corners=0.8pt]
        ([xshift=-0.15em,yshift=#2]X.south west) rectangle
        ([xshift= 0.15em,yshift=#3]X.south east);
    \end{scope}
  }%
}

%----------------------------------------------------------------------------------------
% タイトル
%----------------------------------------------------------------------------------------
\title{ 10 仮説検定①(考え方とp値) }
\date{}
\newcommand{\codedir}{\assetpath/1020201.アルゴリズム2/10}

\begin{document}

\begin{frame}[plain,noframenumbering]
  \titlepage
  \bigskip
  \begin{center}
    \ifteachermode 教師用 \fi
  \end{center}
\end{frame}

% セクションページ(必要なら)
\setbeamertemplate{section page}{
  \begin{centering}
    \vfill
    \rule{\linewidth}{2pt}\par
    \vspace{1ex}
    {\usebeamerfont{section title}\Huge\bfseries \insertsection}\par
    \vspace{1ex}
    \rule{\linewidth}{2pt}\par
    \vfill
  \end{centering}
}
\setbeamerfont{section title}{size=\LARGE,series=\bfseries}

\AtBeginSection[]{
  \begin{frame}[plain,noframenumbering]
    \sectionpage
  \end{frame}
}

% 本編開始でフレーム番号を0から(必要なら)
\setcounter{framenumber}{0}

\input{emoji_macros}

% @@@--(metropolis)--@@@
% ===== 第9回:導入(推測統計 → 仮説検定)ポテトのストーリー =====
% ※「アルゴリズム2 Beamer共通仕様 v2025」準拠(frame群のみ)
% ※teacherframeは使わない(noteTのみ)

%----------------------------------------------------------------------------------------
% Slide 01: 導入ストーリー(問い)
%----------------------------------------------------------------------------------------

%@@PAGEBAND@@
% ----------------------------------------------------------------------------------------
%   page 01
% ----------------------------------------------------------------------------------------
\begin{frame}{導入:本当に「平均との差」がある?(ポテトの例)}
\small
あるハンバーガーショップでは、ポテトの量は「いつも同じくらい」に見えます。\\
しかし実際には、毎回すこしずつ量が違います(ばらつきがある)。

\vspace{0.6em}
今日もらったポテトは、いつもより\textbf{少ない}ように見えました。\\
この店では、Mサイズのポテトの重さは135gと公表されています。

\vspace{0.8em}
\textbf{問い:}\\
これは\textbf{たまたま少なかった}だけでしょうか?\\
それとも\textbf{本当に量が減った}と言ってよいでしょうか?

\begin{center}
  \includegraphics[scale=0.22]{ポテト比較.png}
\end{center}
% ---- 図・イラスト提案(ここに挿入) -----------------------------------------------
% 図案A(おすすめ:模式図)
%   ・ポテト箱のイラストを2つ並べる(左:いつも、右:今日)
%   ・右だけ量が少ない見た目にする
%   ・写真よりイラストの方が留学生に伝わりやすいことが多い
%
% 図案B(写真)
%   ・「ポテト(通常)」「ポテト(少なめ)」の写真を左右に配置
%   ・著作権の問題が出やすいので、自作画像 or フリー素材推奨
%
% 挿入例(ファイル名は仮):
% \begin{center}
%   \includegraphics[width=0.82\linewidth]{img/potato_compare.png}
% \end{center}
% -----------------------------------------------------------------------------

\noteT{導入:ポテトの例(目的と運用)}{
【目的】推測統計から仮説検定へ「判断モード」に切り替える。\\
【口頭補足】「平均と違う」だけでは判断できない。ポイントは「ばらつきの中で珍しいか」。\\
【運用】学生にまず直感で答えさせる(挙手:たまたま派/減った派)。次スライドで回収する。}
\end{frame}

%----------------------------------------------------------------------------------------
% 推測統計から仮説検定への流れ(統一図)
%----------------------------------------------------------------------------------------

%@@PAGEBAND@@
% ----------------------------------------------------------------------------------------
%   page 02
% ----------------------------------------------------------------------------------------
\begin{frame}{推測から判断へ:統計の役割の違い}

\begin{center}
  \includegraphics[scale=0.55]{推測仮説.png}
\end{center}


推測統計は「準備」、仮説検定は「判断」

\noteT{推測統計と仮説検定の関係}{
【目的】両者を別物ではなく「同じ流れの途中と最後」として理解させる。\\
【口頭補足】今までは一番上まで。今日は下まで進む。\\
【運用】用語(帰無仮説・p値)は、この流れの後で名前をつける。}
\end{frame}


%----------------------------------------------------------------------------------------
% Slide 03: 推測統計では判断できない
%----------------------------------------------------------------------------------------

%@@PAGEBAND@@
% ----------------------------------------------------------------------------------------
%   page 03
% ----------------------------------------------------------------------------------------
\begin{frame}{推測統計では「判断」まではできない}
\small
これまで学んだ推測統計でできることは次です。

\vspace{0.4em}
\begin{itemize}
  \item 平均(だいたいの中心)を\textbf{見積もる}
  \item ばらつき(標準偏差など)を\textbf{把握する}
  \item 「だいたいこのくらい」と\textbf{説明する}
\end{itemize}

\vspace{0.6em}
標本平均が 135g と違っていても、それが「偶然のばらつき」かどうかは、\textbf{推測だけでは}判断できません。


\begin{tcolorbox}[title=今日の問い(判断)]
\textbf{「本当に量が減ったと言ってよいか?」}\\
=「偶然」で片づけてよいかを\textbf{決めたい}
\end{tcolorbox}

\noteT{推測の限界を明示(混乱予防)}{
【目的】「平均との差がある=結論が出る」という誤解をここで止める。\\
【口頭補足】推測は「量を知る」。検定は「言ってよいか判断する」。\\
【運用】「差がある」ことと「本当に変化したと言える」ことは別、と繰り返す。}
\end{frame}

%----------------------------------------------------------------------------------------
% Slide 04: 判断のための統計=仮説検定
%----------------------------------------------------------------------------------------

%@@PAGEBAND@@
% ----------------------------------------------------------------------------------------
%   page 04
% ----------------------------------------------------------------------------------------
\begin{frame}{判断するための統計:仮説検定}
\small
「本当に量が減ったと言ってよいか?」に答えるために、次の考え方を使います。

\vspace{0.4em}
\begin{enumerate}
  \item まず\textbf{いつも通り}だと仮定する(基準を置く)
  \item その仮定のもとで、今日の結果が\textbf{どれくらい珍しいか}を見る
  \item とても珍しければ、\textbf{いつも通り}は怪しいと判断する
\end{enumerate}

\vspace{0.5em}
この「\textbf{珍しさで判断する}」方法を\textbf{仮説検定}と呼びます。

\noteT{仮説検定の導入(用語は後付け)}{
【目的】検定を「難語」ではなく「判断の手順」として理解させる。\\
【口頭補足】この後で、いつも通り=帰無仮説、珍しさ=p値、線引き=有意水準、と名前を付ける。\\
【運用】ここでは用語を増やさない(流れだけ固定)。}
\end{frame}

%----------------------------------------------------------------------------------------
% Slide 05: 帰無仮説(役割で固定)
%----------------------------------------------------------------------------------------

%@@PAGEBAND@@
% ----------------------------------------------------------------------------------------
%   page 05
% ----------------------------------------------------------------------------------------
\begin{frame}{帰無仮説:\textbf{「いつも通りだ」と仮定する考え}}
\small
仮説検定では、まず\textbf{基準となる考え}を置きます。

\vspace{0.4em}
\begin{tcolorbox}[title={\ruby{帰無仮説}{きむかせつ}($H_0$)}]
フライドポテトの平均重量は、\textbf{公表値どおり 135g である(いつも通り)}
\end{tcolorbox}

\vspace{0.4em}
ポテトの例なら:
\begin{itemize}
  \item[] 「ポテトの量は、いつも通り(変わっていない)」
\end{itemize}

\vspace{0.4em}
ここがポイント:
\begin{itemize}
  \item \textbf{先に}「いつも通り」を仮定してから、データで疑う
\end{itemize}

\noteT{帰無仮説の固定(なぜ単純化するか)}{
【目的】帰無仮説を「式」ではなく「役割(基準)」として理解させる。\\
【口頭補足】帰無仮説の形はいろいろあるが共通点は「特別な変化はない」。今日はこの役割を固定する。\\
【運用】このスライドでは対立仮説をまだ出さない(混乱防止)。}
\end{frame}

%----------------------------------------------------------------------------------------
% Slide 06: 対立仮説(対になる考え)
%----------------------------------------------------------------------------------------

%@@PAGEBAND@@
% ----------------------------------------------------------------------------------------
%   page 06
% ----------------------------------------------------------------------------------------
\begin{frame}{対立仮説:\textbf{「変わった」と考える側}}
\small
\ruby{帰無仮説}{きむかせつ}(いつも通り)が怪しいときに選ばれる「もう一つの考え」を置きます。

\vspace{0.4em}
\begin{tcolorbox}[title={\ruby{対立仮説}{たいりつかせつ}($H_1$)}]
フライドポテトの平均重量は、\textbf{135g ではない(重い場合・軽い場合の両方を含む)}
\end{tcolorbox}

\vspace{0.4em}
ポテトの例なら:
\begin{itemize}
  \item[] 「ポテトの量は減った(変化がある)」
\end{itemize}

\vspace{0.4em}
\textbf{注意:} 検定は「$H_1$ を証明する」ではなく、\\
\textbf{$H_0$ が成り立ちにくい}と言えるかを判断します。

\noteT{対立仮説の扱い(誤解予防)}{
【目的】「対立仮説を証明する」と誤解させない。\\
【口頭補足】やることは「$H_0$が怪しいか」を見る。結果として$H_1$側を採る。\\
【運用】言葉は「正しい/間違い」ではなく「棄却/棄却できない」を使う準備。}
\end{frame}

%----------------------------------------------------------------------------------------
% Slide 07: 珍しさ=分布の上で考える
%----------------------------------------------------------------------------------------

%@@PAGEBAND@@
% ----------------------------------------------------------------------------------------
%   page 07
% ----------------------------------------------------------------------------------------
\begin{frame}{「珍しさ」は分布の上で考える}
\small
データには\textbf{ばらつき}があるため、結果は毎回同じになりません。\\
だから「平均との差がある」だけでは判断できません。

\vspace{0.6em}
\begin{tcolorbox}[title={発想}]
\textbf{結果を「分布の上のどこに出たか」で考える}
\end{tcolorbox}

\vspace{0.4em}
\begin{itemize}
  \item 中心に近い:よくある(珍しくない)
  \item 端に近い:珍しい(めったに起きない)
\end{itemize}

\noteT{分布に乗せる意味}{
【目的】p値計算の「根拠」を先に見せ、Excel計算のブラックボックス感を減らす。\\
【口頭補足】今回は正規分布(ベルカーブ)だけを使う。他の分布は次回以降。\\
【運用】グラフの細部は不要。「中心/端」の直感だけ固定する。}
\end{frame}

%----------------------------------------------------------------------------------------
% Slide 08: 方法が変わると分布も変わる(触れるだけ)
%----------------------------------------------------------------------------------------

%@@PAGEBAND@@
% ----------------------------------------------------------------------------------------
%   page 08
% ----------------------------------------------------------------------------------------
\begin{frame}{統計の方法が変わると「使う分布」も変わる}
\small
同じ「珍しさで判断」でも、\textbf{何を判断するか}によって使う分布が変わります。

\vspace{0.5em}
\begin{itemize}
  \item 平均の判断(今回):\textbf{母集団は正規分布}
  \item z値を使って位置を測る
  \item ほかの判断:別の分布を使うことがある(例:\textbf{t分布} など)
\end{itemize}

\vspace{0.5em}
\begin{tcolorbox}[title={今日の立ち位置}]
今日は「\textbf{分布の上で珍しさを見る}」を理解する回。\\
分布の種類を増やすのは次の段階。
\end{tcolorbox}

\noteT{分布が変わる話(最小限)}{
【目的】p値やt値が「分布の上の位置」だと理解させ、次回(平均の検定)につなぐ。\\
【口頭補足】今日は正規分布で進める。t分布は「次回、平均の検定で登場」と予告だけ。\\
【運用】ここで深入りしない(名称の羅列にしない)。}
\end{frame}

%----------------------------------------------------------------------------------------
%   検定統計量 Z:平均との差を「標準化」する
%----------------------------------------------------------------------------------------

%@@PAGEBAND@@
% ----------------------------------------------------------------------------------------
%   page 09
% ----------------------------------------------------------------------------------------
\begin{frame}{検定統計量 $Z$:平均との差を「標準化」する}

\begin{columns}
  %-------------------------------
  % 左:式+言葉
  %-------------------------------
  \begin{column}{0.60\linewidth}

  \vspace{-0.3em}
  \begin{center}
  {\Large
  $Z=\dfrac{\bar{x}-\mu_{0}}{\sigma/\sqrt{n}}$
  }
  \end{center}

  \vspace{0.4em}
  \begin{itemize}
    \item $Z$ は、\textbf{標準正規分布の上の「位置」}を表す数
    \item 分子 $\bar{x}-\mu_{0}$ は、\textbf{平均との差(ずれ)}
    \item 分母 $\sigma/\sqrt{n}$ は、\textbf{標本平均のばらつき(標準誤差)}
    \item つまり $Z$ は、\textbf{「ずれ」が標準誤差の何倍か}
  \end{itemize}

  \vspace{0.4em}
  \emjpin \textbf{$|Z|$ が大きいほど「珍しい」位置になる(端に寄る)}
  \end{column}

  %-------------------------------
  % 右:図(模式図)
  %-------------------------------
  \begin{column}{0.38\linewidth}
  \centering
  \begin{tikzpicture}[x=1.0cm,y=1.0cm]
    % 軸
    \draw[->] (-2.3,0) -- (2.3,0);
    \node[below right] at (2.25,0) {\scriptsize $Z$};

    % ベルカーブ(簡易)
    \draw[smooth, thick, domain=-2.1:2.1, samples=60]
      plot (\x, {1.6*exp(-(\x*\x))});

    % 中心
    \draw[dashed] (0,0) -- (0,1.6);
    \node[below] at (0,0) {\scriptsize 0};

    % 位置(例として左側)
    \draw[thick, ->] (-1.4,0.15) -- (-1.4,0);
    \node[below] at (-1.4,0) {\scriptsize $Z$};

    % 注釈
    \node[align=left] at (0.0,1.85) {\scriptsize 分布の上で\\\scriptsize どこに来た?};
  \end{tikzpicture}

  \vspace{0.6em}
  \scriptsize
  \begin{itemize}
    \item 中心(0)から遠いほど珍しい
    \item 次に $p$ 値で「珍しさ」を確率にする
  \end{itemize}
  \end{column}
\end{columns}

\noteT{Zの式の説明(口頭補足)}{
このスライドの目的は「計算手順」ではなく、
Zの式が何を測っているか(物差し)を理解させること。

分子は「平均との差」。
分母は「標本平均がどれくらい揺れるか」(標準誤差)。
よって Z は「平均との差が、揺れの何倍か」。

ここで p値や有意水準の話はしない。
次のスライドで「Z→p値(珍しさの確率化)」へ進む。
}
\end{frame}
%----------------------------------------------------------------------------------------
%   Z値が決まると、p値が決まる
%----------------------------------------------------------------------------------------

%@@PAGEBAND@@
% ----------------------------------------------------------------------------------------
%   page 10
% ----------------------------------------------------------------------------------------
\begin{frame}{Z値が決まると、p値が決まる}

\begin{columns}
  %-------------------------------
  % 左:言葉による整理
  %-------------------------------
  \begin{column}{0.58\linewidth}

  \begin{itemize}
    \item $Z$ 値は  
          \textbf{正規分布の上での「位置」}を表す
    \item 中心($Z=0$)から  
          \textbf{どれだけ離れているか}を見る指標
    \item その位置が  
          \textbf{どれくらい珍しいか}を  
          \textbf{確率で表したもの}が $p$ 値
  \end{itemize}

  \vspace{0.6em}
  \emjpin
  \textbf{$Z$ は「位置」、$p$ は「確率」}

  \vspace{0.6em}
  \begin{itemize}
    \item $|Z|$ が大きい  
          $\Rightarrow$ 分布の端に近い
    \item 分布の端に近い  
          $\Rightarrow$ $p$ 値は小さくなる
  \end{itemize}

  \end{column}

  %-------------------------------
  % 右:模式図
  %-------------------------------
  \begin{column}{0.38\linewidth}
  \centering
  \begin{tikzpicture}[x=1.0cm,y=1.0cm]
    % 軸
    \draw[->] (-2.4,0) -- (2.4,0);
    \node[below right] at (2.35,0) {\scriptsize $Z$};

    % ベルカーブ
    \draw[smooth, thick, domain=-2.2:2.2, samples=60]
      plot (\x, {1.6*exp(-(\x*\x))});

    % 中心
    \draw[dashed] (0,0) -- (0,1.6);
    \node[below] at (0,0) {\scriptsize 0};

    % Z位置
    \draw[thick, ->] (-1.8,0.15) -- (-1.8,0);
    \node[below] at (-1.8,0) {\scriptsize $Z$};

    % p値領域(外側)
    \fill[yellow, opacity=0.4, domain=-2.4:-1.8]
      plot (\x, {1.6*exp(-(\x*\x))}) -- (-1.8,0) -- (-2.4,0) -- cycle;

    \node[align=left] at (0.7,1.5) {\scriptsize 外側の面積\\\scriptsize $\Rightarrow p$ 値};
  \end{tikzpicture}

  \vspace{0.6em}
  \scriptsize
  ※ この図では、左側の外側の面積を示している  
  \end{column}
\end{columns}

\noteT{Z値とp値の関係}{
このスライドの目的は、
Zとpを「比較」させないこと。

Zは分布上の位置、
pはその位置より外側に出る確率。
Zが決まればpは自動的に決まる。

判断は次のスライドで
「p値と有意水準を比べる」として行う。
}
\end{frame}

%----------------------------------------------------------------------------------------
% Slide 09: p値=珍しさ(定義)
%----------------------------------------------------------------------------------------

%@@PAGEBAND@@
% ----------------------------------------------------------------------------------------
%   page 11
% ----------------------------------------------------------------------------------------
\begin{frame}{p値:\textbf{「どれくらい珍しいか」を数値にしたもの}}
\small
p値は、\ruby{帰無仮説}{きむかせつ}(いつも通り)が正しいと\textbf{仮定したとき}の「珍しさ」です。

\vspace{0.6em}
\begin{tcolorbox}[title={p値(文章での定義)}]
\textbf{$H_0$ が正しいと仮定したとき、}\\
\textbf{z値が示す位置よりも外側に出る確率}
\end{tcolorbox}

\vspace{0.5em}
読み替え(今日の授業ではこれでOK):
\begin{itemize}
  \item p値が小さい = \textbf{とても珍しい} = $H_0$ は怪しい
  \item p値が大きい = \textbf{珍しくない} = $H_0$ を疑う理由が弱い
\end{itemize}

\noteT{p値の導入(言い換え固定)}{
【目的】p値を「計算結果」ではなく「珍しさ」として固定する。\\
【口頭補足】必ず「$H_0$が正しいと仮定したとき」を口に出す。\\
【運用】この後に誤解スライドを入れて意味を補強する。}
\end{frame}

%----------------------------------------------------------------------------------------
% Slide 10: p値の誤解をつぶす
%----------------------------------------------------------------------------------------

%@@PAGEBAND@@
% ----------------------------------------------------------------------------------------
%   page 12
% ----------------------------------------------------------------------------------------
\begin{frame}{よくある誤解:p値は「$H_0$が正しい確率」ではない}
\small
p値は、次の意味\textbf{ではありません}。

\vspace{0.4em}
\begin{itemize}
  \item p値=「\ruby{帰無仮説}{きむかせつ}が正しい確率」\hfill(\textbf{×})
  \item p値=「効果の大きさ」\hfill(\textbf{×})
\end{itemize}

\vspace{0.6em}
\begin{tcolorbox}[title={ここだけ覚える}]
p値は\textbf{「$H_0$のもとでの珍しさ」}を表す数値
\end{tcolorbox}

\noteT{誤解対策(毎年のつまずきポイント)}{
【目的】用語の誤解を早期に止める(留学生ほどここで崩れやすい)。\\
【口頭補足】「p値が小さい=$H_0$が間違いと確定」ではない。判断には次の「線引き」が必要。\\
【運用】短く、強く。長い説明にしない。}
\end{frame}

%----------------------------------------------------------------------------------------
% Slide 11: 有意水準=線引き(ルール)
%----------------------------------------------------------------------------------------

%@@PAGEBAND@@
% ----------------------------------------------------------------------------------------
%   page 13
% ----------------------------------------------------------------------------------------
\begin{frame}{有意水準:\textbf{どこからを「珍しい」と決めるかの線}}
\small
p値は「珍しさ」ですが、\textbf{珍しいと決める基準}も必要です。\\
その基準(線引きのルール)が\textbf{\ruby{有意水準}{ゆういすいじゅん}}です。

\vspace{0.6em}
\begin{tcolorbox}[title={\ruby{有意水準}{ゆういすいじゅん}(例)}]
\textbf{5\%}(0.05)など:\\
「これより珍しければ、偶然とは言いにくい」と決める線
\end{tcolorbox}

\vspace{0.4em}
\begin{itemize}
  \item p値は\textbf{データから出る}
  \item \ruby{有意水準}{ゆういすいじゅん}は\textbf{人が先に決める約束}
\end{itemize}

\noteT{有意水準の意味(p値との区別)}{
【目的】p値と有意水準の役割分担をはっきり分ける。\\
【口頭補足】「p値が小さい/大きい」だけで結論を言わない。必ず「有意水準と比べる」。\\
【運用】ここで初めて「判断のルール」が完成する。}
\end{frame}
%----------------------------------------------------------------------------------------
%   正規分布と棄却域(有意水準5%の見える化)
%----------------------------------------------------------------------------------------

%@@PAGEBAND@@
% ----------------------------------------------------------------------------------------
%   page 14
% ----------------------------------------------------------------------------------------
\begin{frame}{正規分布と棄却域(有意水準5\%)}

\begin{columns}
  %-------------------------------
  % 左:図
  %-------------------------------
  \begin{column}{0.58\linewidth}
    \centering
    \includegraphics[width=\linewidth]{normal_rejection_5percent}
    
    \vspace{0.5em}
    \scriptsize
    ※ 正規分布上で、有意水準5\%の棄却域を示した図
  \end{column}

  %-------------------------------
  % 右:説明
  %-------------------------------
  \begin{column}{0.40\linewidth}
    \begin{itemize}
      \item 有意水準 5\% とは  
            \textbf{「正しいと仮定しても、5\%以下しか起こらない範囲」}
      \item 両側検定では  
            \textbf{左右それぞれ 2.5\%} が棄却域
      \item 境界となる値が  
            \textbf{$z = \pm 1.96$}
      \item 計算した $z$ 値が  
            \textbf{この黄色の範囲に入るか}を確認する
    \end{itemize}
  \end{column}
\end{columns}

\noteT{この図の使い方(有意水準と判断の準備)}{
ここでは p値の計算はしない。
目的は、
「有意水準5%が分布のどこにあるか」を
視覚で理解させること。

z値は
「分布のどこに位置するか」
有意水準は
「どこから先を珍しいと決める線」
であることを強調する。

次のスライドで、
p値と有意水準を比べる
「判断のルール」に進む。
}
\end{frame}

%----------------------------------------------------------------------------------------
% Slide 12: 判断(棄却/棄却できない)
%----------------------------------------------------------------------------------------

%@@PAGEBAND@@
% ----------------------------------------------------------------------------------------
%   page 15
% ----------------------------------------------------------------------------------------
\begin{frame}{判断のルール:p値と有意水準を比べる}
\small
判断は、\textbf{p値}と\textbf{\ruby{有意水準}{ゆういすいじゅん}}を比べるだけです。

\vspace{0.6em}
\begin{tcolorbox}[title={判断}]
\begin{itemize}
  \item \textbf{p値 $<$ \ruby{有意水準}{ゆういすいじゅん}}:$H_0$ を\textbf{\ruby{棄却}{ききゃく}}する(いつも通りは怪しい)
  \item \textbf{p値 $\ge$ \ruby{有意水準}{ゆういすいじゅん}}:$H_0$ を\textbf{\ruby{棄却}{ききゃく}できない}(怪しいと言えない)
\end{itemize}
\end{tcolorbox}

\vspace{0.4em}
\textbf{注意:} 「\ruby{棄却}{ききゃく}できない」=「$H_0$が正しいと確定」ではありません。

\noteT{判断語の統一(留学生対応)}{
【目的】結論を「正しい/間違い」で言わせず、検定の表現に慣れさせる。\\
【口頭補足】$H_0$は“採択”ではなく“棄却できない”。言葉の違いに注意。\\
【運用】このスライドを使って、短い練習問題(口頭)を1回入れてよい。}
\end{frame}
%----------------------------------------------------------------------------------------
%   p値は棄却域に入っているか(図で確認)
%----------------------------------------------------------------------------------------

%@@PAGEBAND@@
% ----------------------------------------------------------------------------------------
%   page 16
% ----------------------------------------------------------------------------------------
\begin{frame}{p値は棄却域に入っているか}

\begin{columns}
  %-------------------------------
  % 左:図
  %-------------------------------
  \begin{column}{0.58\linewidth}
    \centering
    \includegraphics[width=\linewidth]{pvalue_in_rejection}
    
    \vspace{0.5em}
    \scriptsize
    ※ p値(黄色部分)が棄却域に含まれている例
  \end{column}

  %-------------------------------
  % 右:説明
  %-------------------------------
  \begin{column}{0.40\linewidth}
    \begin{itemize}
      \item p値は  
            \textbf{「これくらい極端な結果が出る確率」}
      \item 有意水準は  
            \textbf{「珍しいと判断する境界」}
      \item p値が棄却域に入る  
            $\Rightarrow$ \textbf{有意水準より小さい}
      \item 今回は  
            \textbf{帰無仮説を棄却する}
    \end{itemize}
  \end{column}
\end{columns}

\noteT{この図の位置づけ(判断の確定)}{
前のスライドで示した
「p値と有意水準を比べる」
という判断ルールを、
正規分布の上で確認するスライド。

ここで新しい概念は出さない。
計算→ルール→図で確認、
という流れを完成させる役割。

次はこの考え方を
実習(Excel)で再現させる。
}
\end{frame}

%----------------------------------------------------------------------------------------
% Slide 13: 実習の全体手順(30件)
%----------------------------------------------------------------------------------------

%@@PAGEBAND@@
% ----------------------------------------------------------------------------------------
%   page 17
% ----------------------------------------------------------------------------------------
\begin{frame}{実習:30件データで「珍しさ」を計算して判断する}
\small
\textbf{目的:} Excelでp値を出し、\textbf{意味(珍しさ)}と\textbf{判断}を結びつける。

\vspace{0.4em}
\begin{enumerate}
  \item データ30件から\textbf{平均}と\textbf{ばらつき}を出す
  \item 「いつも通り」の基準($\mu_0$)を置く(\ruby{帰無仮説}{きむかせつ})
  \item 平均の差を\textbf{標準誤差}で割って、検定統計量($z$)を作る
  \item その位置より外側の確率(\textbf{p値})を出す
  \item \ruby{有意水準}{ゆういすいじゅん}と比べて\textbf{結論文}を書く
\end{enumerate}

\noteT{実習仕様(具体)}{
【データ】前回までと同じデータから「ランダム抽出30件」を使用。\\
【基準】$\mu_0$(いつも通りの平均)は「教材側で指定」または「過去平均を指定」。\\
【計算(Excelの例)】\\
・平均:\texttt{AVERAGE(range)}\\
・標準偏差(標本):\texttt{STDEV.S(range)}\\
・標本数:\texttt{COUNT(range)}(=30)\\
・標準誤差:\texttt{=s/SQRT(n)}\\
・位置(z相当):\texttt{=(xbar-mu0)/SE}\\
・p値(両側):\texttt{=2*(1-NORM.S.DIST(ABS(z),TRUE))}\\
【成果物(提出/確認)】(1)$H_0$の文章 (2)p値 (3)有意水準 (4)結論文(棄却/棄却できない)。\\
【口頭補足】このzは「平均との差を“分布の単位”で測った位置」。次回はこれがt値になる。}
\end{frame}

%----------------------------------------------------------------------------------------
% Slide 14: 実習で書く結論文テンプレ
%----------------------------------------------------------------------------------------

%@@PAGEBAND@@
% ----------------------------------------------------------------------------------------
%   page 18
% ----------------------------------------------------------------------------------------
\begin{frame}{実習:結論文の書き方(テンプレ)}
\small
Excelで数値が出ても、\textbf{最後は文章で判断}します。\\
次の形で書けばOKです(丸暗記でよい)。

\vspace{0.6em}
\begin{tcolorbox}[title={結論文テンプレ(例)}]
\ruby{有意水準}{ゆういすいじゅん}を0.05とすると、p値は\underline{\hspace{2.5em}}であり、\\
\underline{\hspace{3.0em}}(小さい/大きい)ので、\ruby{帰無仮説}{きむかせつ}は\\
\underline{\hspace{4.0em}}(\ruby{棄却}{ききゃく}する/\ruby{棄却}{ききゃく}できない)。
\end{tcolorbox}

\vspace{0.4em}
\textbf{注意:} 「正しい/間違い」とは書かない。

\noteT{文章化の狙い}{
【目的】第10回(平均の検定)で「結論を文章で書く」をスムーズにする。\\
【口頭補足】p値と\ruby{有意水準}{ゆういすいじゅん}の比較が文章の中心。\\
【運用】早く終わった学生には「結論の意味を自分の言葉で言い換える」を追加。}
\end{frame}

%----------------------------------------------------------------------------------------
% Slide 15: まとめ(今日できるようになったこと)
%----------------------------------------------------------------------------------------

%@@PAGEBAND@@
% ----------------------------------------------------------------------------------------
%   page 19
% ----------------------------------------------------------------------------------------
\begin{frame}{まとめ}
\small
今日のゴールは「検定の計算」ではなく、\textbf{判断の考え方}でした。

\vspace{0.6em}
\begin{itemize}
  \item \textbf{推測}:平均・ばらつきを知る(準備)
  \item \textbf{検定}:珍しさ(p値)で判断する
  \item \ruby{帰無仮説}{きむかせつ}=\textbf{いつも通りだと仮定する考え}
  \item p値=\textbf{$H_0$のもとでの珍しさ}
  \item \ruby{有意水準}{ゆういすいじゅん}=\textbf{線引きのルール}
\end{itemize}

\noteT{まとめ(短く強く)}{
【目的】用語を再度「役割」に戻して固定する。\\
【口頭補足】次回は「同じ流れ」を平均の検定として完成させる。\\
【運用】学生に一言でまとめさせる:「p値は何?」→「珍しさ」。}
\end{frame}

%----------------------------------------------------------------------------------------
% Slide 16: 次回予告(第10回)
%----------------------------------------------------------------------------------------

%@@PAGEBAND@@
% ----------------------------------------------------------------------------------------
%   page 20
% ----------------------------------------------------------------------------------------
\begin{frame}{次回予告:平均の検定}
\small
次回は、今日の流れを\textbf{平均の判断}としてもう一段具体化します。

\vspace{0.6em}
\begin{itemize}
  \item 今日の流れ:\\
  「いつも通り」$\rightarrow$「珍しさ(p値)」$\rightarrow$「判断」
  \item 次回は、平均の差を測る数値として\textbf{t値}が登場する
  \item ただし考え方は同じ:\textbf{分布の上で位置を見る}
\end{itemize}

\noteT{第10回への橋渡し}{
【目的】第9回で終わらず「同じ考え方が続く」ことを明確にする。\\
【口頭補足】t値は新しい“計算”ではなく「平均との差の位置情報」。\\
【運用】第10回冒頭でポテト例に戻し、「平均との差を判断する」を再提示すると接続が良い。}
\end{frame}
\end{document}
