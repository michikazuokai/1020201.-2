% @@@--(metropolis)--@@@
% ----------------------------------------------------------------------------------------
%   Slide 01: タイトル
% ----------------------------------------------------------------------------------------
\begin{frame}{第6回:確率分布②(正規分布)}
\begin{center}
  \textbf{自然界に潜む最も美しい山の形を理解する}
\end{center}

本日は、データが「つながった値」をとる場合のルール、
\textbf{「\ruby{正規分布}{せいきぶんぷ}」}について学びます。

\begin{itemize}
  \item なぜ多くの現象がこの形になるのか?
  \item 山の形を変える「2つの数字」の役割。
  \item 実務で「異常」を見つける方法。
\end{itemize}

身の回りにある様々なデータが、この正規分布のルールに従っていることを確認しましょう。

\noteT{講義の狙い}{
正規分布の重要性と、Excelでの可視化を通じて実務に応用できることを強調し、学習意欲を高めます。
}
\end{frame}

% ----------------------------------------------------------------------------------------
%   Slide 02: 前回の復習:回数を増やした二項分布はどうなったか?
% ----------------------------------------------------------------------------------------
\begin{frame}{前回の復習:回数を増やした二項分布はどうなったか?}
前回、コイン投げ(二項分布)の試行回数を増やしたグラフを覚えていますか?

\begin{itemize}
  \item コインを10回投げる $\rightarrow$ 山がギザギザ
  \item コインを100回投げる $\rightarrow$ 山が滑らか
  \item コインを1000回投げる $\rightarrow$ さらに滑らかで、左右対称の美しい形に近づく
\end{itemize}

この「試行回数をたくさん増やすと、どんどんきれいな山になる」という性質が、
今日のテーマである\textbf{「正規分布」}に深く関係しています。

\begin{center}
  \includegraphics[width=0.7\textwidth]{二項分布のn増加比較図.png}
\end{center}

\noteT{前回との接続}{
二項分布の経験から正規分布へのスムーズな導入を図ります。図は前回作成したものの一部を使用するか、再作成を促します。
}
\end{frame}

% ----------------------------------------------------------------------------------------
%   Slide 03: 正規分布(Normal Distribution)とは何か
% ----------------------------------------------------------------------------------------
\begin{frame}{正規分布(Normal Distribution)とは何か}
\textbf{正規分布}とは、統計学で最もよく使われる、ある特定の形の確率分布のことです。

\begin{itemize}
  \item \textbf{形:} 中央が一番高く、左右対称の「\ruby{釣り鐘型}{つりがねがた}(ベルカーブ)」の形をしています。
  \item \textbf{特徴:} 平均値に近い値が出やすく、平均値から離れるほど出にくくなります。
\end{itemize}

この形は、私たちの身の回りにある多くの自然現象や社会現象で現れることが知られています。
そのため、「標準的な分布」という意味で\textbf{「正規」}という名前がついています。

\vspace{-1.4em}
\begin{center}
  \includegraphics[width=0.5\textwidth]{正規分布の基本グラフ.png}
\end{center}

\noteT{定義}{
正規分布の基本的な形と特徴を言葉と図で説明し、次スライド以降の具体的な内容へと繋げます。
}
\end{frame}

% ----------------------------------------------------------------------------------------
%   Slide 04: 「とびとび(離散型)」から「つながった(連続型)」へ
% ----------------------------------------------------------------------------------------
\begin{frame}{「とびとび(離散型)」から「つながった(連続型)」へ}
前回学んだ二項分布と、今回の正規分布の最大の違いは「値の性質」にあります。

\begin{itemize}
    \item \textbf{\ruby{離散型}{りさんがた}(前回):} 
    コインの表の回数のように「1回、2回」と数えられる「とびとびの値」です。
    \item \textbf{\ruby{連続型}{れんぞくがた}(今回):} 
    身長や重さ、時間のように、小数点でどこまでも細かく分けられる「つながった値」です。
\end{itemize}

\textbf{なぜ重要か:}
回数を無限に増やしていくと、とびとびの点だったデータは、
最終的に一本の滑らかな曲線(正規分布)として扱えるようになります。

\noteT{分類の整理}{
「数えられるか、測るものか」という視点で、離散型と連続型の違いを明確に意識させます。
}
\end{frame}

% ----------------------------------------------------------------------------------------
%   Slide 05: 身の回りにある正規分布
% ----------------------------------------------------------------------------------------
\begin{frame}{身の回りにある正規分布}
正規分布は、特別な計算の中だけでなく、私たちの身近な至るところに現れます。

\begin{itemize}
    \item \textbf{身体のデータ:} 成人の身長、体重、新生児の出生体重など。
    \item \textbf{教育・心理データ:} 大勢が受けたテストの点数、IQ(知能指数)など。
    \item \textbf{工業製品の誤差:} 機械で切断した部品の長さのわずかなズレ、充填される飲料の量など。
\end{itemize}

これらのデータは、どれも「平均値の近く」に最も多くの人が集まり、
極端に大きい(あるいは小さい)人は左右に少しずつ存在する、という共通のルールに従います。

\noteT{具体例}{
抽象的な概念を、学生が想像しやすい具体的なデータと結びつけます。
}
\end{frame}

% ----------------------------------------------------------------------------------------
%   Slide 06:正規分布をデザインする「2つの数字」
% ----------------------------------------------------------------------------------------
\begin{frame}{正規分布をデザインする「2つの数字」}
正規分布の山の形は、たった2つの「担当者(数字)」が決めています。

\begin{itemize}
    \item \textbf{平均($\mu$:ミュー):} 
    山の「左右の位置」を決める担当。集団のレベルや基準を表します。
    \item \textbf{広がり($\sigma$:シグマ):} 
    山の「太さ・形」を決める担当。集団の安定感やリスクを表します。
\end{itemize}

この2つが決まれば、どんなデータも一つの美しい山で表現できます。
イメージを掴むために、「位置(平均)が違う例」と「広がり(幅)が違う例」を順番に見ていきましょう。

\noteT{デザイナーの紹介}{
平均と広がりの役割を提示した上で、これからそれぞれの具体例を提示することを予告し、学生の関心を引きつけます。
}
\end{frame}

% ----------------------------------------------------------------------------------------
%   Slide 07:【平均の例】山の位置が示す「基準」と「実力」
% ----------------------------------------------------------------------------------------
\begin{frame}{【平均の例】山の位置が示す「基準」と「実力」}
正規分布の真ん中にある「平均($\mu$)」は、その集団の「実力」や「基準」を意味します。

\begin{itemize}
    \item \textbf{【スポーツ:プロ vs アマチュア】}
    プロ選手とアマチュア選手では、球速や得点力の「平均(中心)」が全く違います。平均が高いほど、集団全体のレベルが一段上にあることを示します。
    \item \textbf{【生活:地域による物価】}
    都市部と地方では、家賃の正規分布の「山の位置」が大きくズレます。平均は、その場所における「当たり前(標準)」の基準を表しています。
\end{itemize}



このように、平均が変わると山は「そのままの形で左右にスライド」します。
しかし、実社会では「位置」以上に重要なのが「山の形(広がり)」なのです。

\noteT{位置の理解}{
平均=スライド(基準の変化)であることを納得させ、次の「広がり」の議論へ繋げます。
}
\end{frame}

% ----------------------------------------------------------------------------------------
%   Slide 08:【広がりの例1】工業製品の信頼(高級ブランド vs 普及品)
% ----------------------------------------------------------------------------------------
\begin{frame}{【広がりの例1】工業製品の信頼(高級ブランド vs 普及品)}
どちらも「1日の誤差の平均は0秒」というデータがあるとします。

\begin{itemize}
    \item \textbf{広がりが小さい(高級ブランド):}
    どの個体を買っても、ほぼ狂いません。この「当たり外れのなさ」が、世界中で認められる「品質」の正体です。
    \item \textbf{広がりが大きい(安価な普及品):}
    平均は0秒でも、1分進むものや1分遅れるものが混ざっています。これでは一つひとつの製品を信頼して使うことができません。
\end{itemize}

\textbf{結論:}\par
平均が同じでも、広がりが大きい製品は「期待外れ」のリスクが高くなります。実務では、平均よりも「広がりの小ささ」が信頼の基準になります。

\noteT{品質の定義}{
「品質が良い=平均が高い」ではなく、「ばらつき(広がり)が小さい」ことであると強調します。
}
\end{frame}

% ----------------------------------------------------------------------------------------
%   Slide 09:【広がりの例2】教育の現場(理解度のバラつき)
% ----------------------------------------------------------------------------------------
\begin{frame}{【広がりの例2】教育の現場(理解度のバラつき)}
2つのクラスのテスト平均点が、どちらも「60点」だとします。

\begin{itemize}
    \item \textbf{広がりが小さい:}
    全員が55点〜65点。学生の理解度が近いため、全員に合わせたきめ細やかな説明や、一斉の授業がしやすくなります。
    \item \textbf{広がりが大きい:}
    「10点」の人と「100点」の人が混在。平均は60点でも、理解の差が激しすぎて、全員を同時に満足させる授業が困難になります。
\end{itemize}

\textbf{結論:}\par
平均は「クラス全体の中心地」を、広がりは「配慮すべき個性の幅」を示します。広がりが大きすぎると、一つのやり方で全体を支えることが難しくなります。

\noteT{現場の難しさ}{
「平均点は同じでも、広がりによって対応策が180度変わる」ことを実感させます。
}
\end{frame}

% ----------------------------------------------------------------------------------------
%   Slide 10:【広がりの例3】飲食店のこだわり(職人の店 vs 不安定な店)
% ----------------------------------------------------------------------------------------
\begin{frame}{【広がりの例3】飲食店のこだわり(職人の店 vs 不安定な店)}
10回食べた時の「スープの塩分濃度の平均」が、どちらも適正だとします。

\begin{itemize}
    \item \textbf{広がりが小さい(職人の店):}
    いつ行っても、誰が作っても同じ味。客は「あの味」を求めて安心して通うことができます。
    \item \textbf{広がりが大きい(調理が不安定な店):}
    平均は正しくても「今日は少し濃い、昨日は薄い」とブレが激しい店。これでは客に「安心感」を与えられず、足が遠のいてしまいます。
\end{itemize}

\textbf{結論:}\par
広がりが大きいことは、サービスにおいては「品質の不安定さ」です。プロの仕事には、平均の正しさだけでなく「常に一定である(広がらない)」ことが求められます。

\noteT{安定感=品質}{
味の安定感こそが広がりの小ささであることを伝え、信頼との結びつきを説きます。
}
\end{frame}

% ----------------------------------------------------------------------------------------
%   Slide 11:広がりの正体 =「標準偏差($\sigma$)」
% ----------------------------------------------------------------------------------------
\begin{frame}{広がりの正体 =「標準偏差($\sigma$)」}
これら「データの広がり具合」を数値化した名前を、\textbf{標準偏差($\sigma$:シグマ)}と呼びます。

\begin{itemize}
    \item \textbf{言葉の意味:}
    データが平均から、だいたいどれくらい「ズレ(偏差)」ているかを示す「標準(ものさし)」となる数字です。
    \item \textbf{性質:}
    $\sigma$ が小さい = データが平均に集まっていて「安定している」。\par
    $\sigma$ が大きい = データが平均から散らばっていて「不安定」。
\end{itemize}

では、この「広がり($\sigma$)」は、数学的にどうやって計算しているのでしょうか?

\noteT{単位としての導入}{
難しい式を出す前に、まず「広がりを測るための専用のものさし」であることを定義します。
}
\end{frame}

% ----------------------------------------------------------------------------------------
%   Slide 12:【考え方】ばらつきを「1つの数字」にするには?
% ----------------------------------------------------------------------------------------
\begin{frame}{【考え方】ばらつきを「1つの数字」にするには?}
「データの広がり(ばらつき)」を計算で出すには、どうすればいいでしょうか?

\begin{itemize}
    \item \textbf{アイデア:} 「全員が平均から何点ズレているか」の平均を出せばいいのでは?
    \item \textbf{問題点:} そのまま足すと、「プラスのズレ」と「マイナスのズレ」が打ち消し合って、合計が \textbf{0} になってしまいます。
\end{itemize}

そこで、統計学では「マイナスを消すための工夫」をして計算を進めます。

\noteT{導入}{
「そのまま足すと0になる」という壁を提示し、なぜ工夫が必要なのかを実感させます。
}
\end{frame}

% ----------------------------------------------------------------------------------------
%   Slide 13:【ステップ1】分散($s^2$):ズレを2乗して合計する
% ----------------------------------------------------------------------------------------
\begin{frame}{【ステップ1】分散($s^2$):ズレを2乗して合計する}
マイナスを消すために、平均からのズレをすべて「2乗」します。

\begin{exampleblock}{5人のテスト点数の例(平均60点)}
    \begin{itemize}
        \item Aさん(70点):ズレ +10 $\rightarrow$ 2乗すると \textbf{100}
        \item Bさん(50点):ズレ -10 $\rightarrow$ 2乗すると \textbf{100}
        \item 他3名も同様に計算し、その「2乗したズレの平均」を出します。
    \end{itemize}
\end{exampleblock}

これを \textbf{\ruby{分散}{ぶんさん}} と呼びます。
ただし、2乗しているため、単位が「点」ではなく「点$^2$」という変な数字になっています。

\noteT{分散}{
「2乗するのはマイナスを消すため」という目的を明確にします。
}
\end{frame}

% ----------------------------------------------------------------------------------------
%   Slide 14:【ステップ2】標準偏差($\sigma$):単位を元に戻す
% ----------------------------------------------------------------------------------------
\begin{frame}{【ステップ2】標準偏差($\sigma$):単位を元に戻す}
最後に、2乗して大きくなりすぎた数字に \textbf{ルート($\sqrt{\quad}$)} をかけます。

\begin{itemize}
    \item \textbf{ルートをかける理由:} 
    2乗したことで「面積」のようになってしまった数字を、元の「長さ(点数)」の単位に戻すためです。
    \item \textbf{完成:}
    こうして求められた数字が、今回の主役 \textbf{標準偏差($\sigma$)} です。
\end{itemize}

\textbf{まとめ:}\par
標準偏差とは、計算の過程で「2乗して、平均して、ルートをかけた」もの。
意味としては、\textbf{「平均からの平均的なズレの幅」}のことです。

\noteT{標準偏差}{
なぜルートが必要なのか(単位を戻すため)を説明し、直感的な意味に戻ることを伝えます。
}
\end{frame}

% ----------------------------------------------------------------------------------------
%   Slide 15:Excelなら一瞬:STDEV.P 関数
% ----------------------------------------------------------------------------------------
\begin{frame}{Excelなら一瞬:STDEV.P 関数}
手計算のプロセスを理解したら、あとはExcelに任せましょう。

\begin{center}
    \textbf{=STDEV.P( データの範囲 )}
\end{center}

\begin{itemize}
    \item \textbf{STDEV}:Standard Deviation(標準偏差)の略です。
    \item \textbf{.P}:Population(母集団)。クラス全員など、手元にあるデータすべてのばらつきを出すときに使います。
\end{itemize}

複雑な2乗やルートの計算を、Excelはこの関数一つで代行してくれます。

\noteT{Excelへの接続}{
仕組みを理解した上で、実務では関数を使うという役割分担を明確にします。
}
\end{frame}

% ----------------------------------------------------------------------------------------
%   Slide 16:平均($\mu$)と標準偏差($\sigma$)による「山の変化」
% ----------------------------------------------------------------------------------------
\begin{frame}{平均($\mu$)と標準偏差($\sigma$)による「山の変化」}
この2つの数字を変えると、正規分布の山は自由自在に形を変えます。

\begin{enumerate}
    \item \textbf{平均($\mu$)を変える:}
    山全体の形は変えずに、そのまま \textbf{左右にスライド} します。
    \item \textbf{標準偏差($\sigma$)を変える:}
    山の中心は変えずに、山が \textbf{「細く高く」} なったり \textbf{「太く低く」} なったりします。
\end{enumerate}

\begin{center}
  \includegraphics[width=0.95\textwidth]{正規分布比較図.png}
\end{center}
% [Image. Top: Shifted normal curves (different $\mu$). Bottom: Normal curves with different widths (different $\sigma$).]
\vspace{-1.5em}
この2つの数字を入力するだけで、理想の分布を描くことができます。

\noteT{視覚的統合}{
図を使って、$\mu$と$\sigma$の役割の違いを決定づけます。
}
\end{frame}

% ----------------------------------------------------------------------------------------
%   Slide 17:正規分布の数式が伝えていること
% ----------------------------------------------------------------------------------------
\begin{frame}{正規分布の数式が伝えていること}
正規分布の裏側にはこのような式がありますが、\textbf{覚える必要はありません。}

$$ f(x) = \frac{1}{\sqrt{2\pi\sigma^2}} e^{-\frac{(x-\mu)^2}{2\sigma^2}} $$

この式をじっと見てください。変数は「$x$」以外には、先ほどの \textbf{$\mu$(平均)} と \textbf{$\sigma$(標準偏差)} しか入っていません。

\textbf{ここが重要:}\par
平均と標準偏差の2つさえ分かれば、全てのデータの分布を一通りに予知できる、ということをこの式は証明しているのです。

\noteT{数式の意味}{
数式の威圧感を、「たった2つのパーツで世界を記述できる」という驚きに変えます。
}
\end{frame}

% ----------------------------------------------------------------------------------------
%   Slide 18:【予告】正規分布の「魔法」を使ってみよう
% ----------------------------------------------------------------------------------------
% \begin{frame}{【予告】正規分布の「魔法」を使ってみよう}
% では、これらを使って具体的に何ができるのでしょうか?

% \begin{itemize}
%     \item \textbf{大数定理:} 
%     サンプルを増やすほど、データはより「真の平均」へ近づいていきます。
%     \item \textbf{中心極限定理:} 
%     どんなバラバラなデータも、集めて平均をとれば「正規分布(山)」になります。
% \end{itemize}

% 後半の実習では、Excelを使ってこの驚くべき「自然界の法則」を自分自身の手で体験してもらいます。

% \noteT{実習への接続}{
% 後半の「大数定理」「中心極限定理」への橋渡しを行い、期待感を高めます。
% }
% \end{frame}

%-------------------------------------------------------------------------------------------------------------

% ----------------------------------------------------------------------------------------
%   Slide 19:なぜ、この形が「正規(Normal)」と呼ばれるのか
% ----------------------------------------------------------------------------------------
\begin{frame}{なぜ、この形が「正規(Normal)」と呼ばれるのか}
この山の形は、英語で \textbf{Normal Distribution(普通の分布)} と呼ばれます。

\begin{itemize}
    \item \textbf{理由:}
    身長、試験の点数、製品の重さ、雨粒の大きさ……。自然界や社会のあらゆるデータを集めると、驚くほど共通して「この山の形」にたどり着くからです。
    \item \textbf{学ぶメリット:}
    「世の中のデータは山になる」という前提を知っていれば、平均と標準偏差の2つがわかるだけで、まだ見ぬデータの予測ができるようになります。
\end{itemize}

では、実際にExcelを使って、この「最強の山」を描いてみましょう。

\noteT{重要性}{
深い理論(中心極限定理)は次回の実験に譲り、ここでは「あらゆるところにある便利な形」であることを強調して実習へ誘導します。
}
\end{frame}

% ----------------------------------------------------------------------------------------
%   Slide 20:【実習1】Excelで正規分布を描く(NORM.DIST関数)
% ----------------------------------------------------------------------------------------
\begin{frame}{【実習1】Excelで正規分布を描く(NORM.DIST関数)}
Excelで正規分布の「山の高さ」を計算するには、次の関数を使います。

\begin{center}
    \Large \textbf{=NORM.DIST( x, 平均, 標準偏差, \ruby{関数形式}{FALSE} )}
\end{center}

\begin{itemize}
    \item \textbf{x}:横軸の値(身長170cmなど)。
    \item \textbf{平均 / 標準偏差}:先ほど学んだ「山のデザイナー」の数値。
    \item \textbf{関数形式}:今回は「高さ」を出したいので \textbf{FALSE} と入力します。
\end{itemize}

※「形式」を \textbf{TRUE} にすると、少し意味が変わります(後ほど解説します)。

\noteT{Excel関数の紹介}{
関数の引数と、これまで学んだ「平均・標準偏差」がここで繋がることを示します。
}
\end{frame}

% ----------------------------------------------------------------------------------------
%   Slide 21:【実習1】データの準備:x軸の作成
% ----------------------------------------------------------------------------------------
\begin{frame}{【実習1】データの準備:x軸の作成}
滑らかな曲線を描くために、x(横軸)のデータを細かく作成します。

\begin{enumerate}
    \item \textbf{範囲を決める:}
    例えば平均50なら、20から80くらいまでを範囲にします。
    \item \textbf{刻みを作る:}
    「20, 21, 22...」ではなく、「20.0, 20.1, 20.2...」のように \textbf{0.1刻み} で100行以上データを作ります。
\end{enumerate}



データが細かいほど、カクカクしていない「美しい山」になります。

\noteT{実習の手順}{
「とびとび」ではない「つながった(連続型)」を表現するために、データを細かく作る必要があることを理解させます。
}
\end{frame}

% ----------------------------------------------------------------------------------------
%   Slide 22:【実習1】グラフ化:散布図(平滑線)を使う
% ----------------------------------------------------------------------------------------
\begin{frame}{【実習1】グラフ化:散布図(平滑線)を使う}
正規分布を描くときは、棒グラフではなく \textbf{「散布図(平滑線)」} を使います。

\begin{itemize}
    \item \textbf{なぜ散布図か:}
    横軸が「数」ではなく「値(連続した数値)」だからです。
    \item \textbf{操作:}
    x(横軸の値)と NORM.DIST(高さ)の2列を選択して、「散布図(直線とマーカーなし)」を挿入します。
\end{itemize}

これで、画面上に美しい「正規分布の山」が現れます!

\noteT{可視化のポイント}{
第2回で学んだグラフの使い分けを復習させつつ、連続型データには散布図が適していることを伝えます。
}
\end{frame}

% ----------------------------------------------------------------------------------------
%   Slide 23:【実習2】パラメータ操作:山を動かしてみよう
% ----------------------------------------------------------------------------------------
\begin{frame}{【実習2】パラメータ操作:山を動かしてみよう}
Excelで作成したグラフを使って、正規分布のデザイナー(数字)を書き換えてみましょう。

\begin{itemize}
    \item \textbf{操作1:平均($\mu$)を書き換える}
    50から70、あるいは30に変えてみてください。山が左右にスライドすることを確認しましょう。
    \item \textbf{操作2:標準偏差($\sigma$)を書き換える}
    10から5、あるいは20に変えてみてください。山の「太さ」と「高さ」が同時に変わることに注目しましょう。
\end{itemize}



\textbf{発見:}\par
平均は「場所」を変え、標準偏差は「データの密度」を変えます。標準偏差が小さいほど、データが平均の周りにギュッと凝縮されます。

\noteT{実習2}{
数値を書き換えた瞬間にグラフが動く体験を通じ、$\mu$と$\sigma$が分布の「支配者」であることを実感させます。
}
\end{frame}

% ----------------------------------------------------------------------------------------
%   Slide 24:【理論】「山の高さ」ではなく「面積」を見る
% ----------------------------------------------------------------------------------------
\begin{frame}{【理論】「山の高さ」ではなく「面積」を見る}
正規分布を「確率」として使うとき、最も大切なルールがあります。
それは、山の「高さ」ではなく \textbf{「面積」} に注目することです。

\begin{itemize}
    \item \textbf{面積 = 確率:}
    ある範囲に入る面積が、そのまま「その値が起こる確率」になります。
    \item \textbf{全体の面積は「1」:}
    山全体の面積をすべて合計すると、必ず \textbf{1(100\%)} になるように設計されています。
\end{itemize}


「山のどこに、どれくらいの面積(人・データ)がたまっているか」を考えるのが、統計的な見方です。

\noteT{【ポテトの物差しによる論理構成】}{
導入:ポテトの重さは、盛り付けや水分で毎回必ず「ズレ」が生じます。\\
このズレの標準(5g)を「1」という単位の物差しにして、グラフの「位置」を考えます。\\
\hfill \\
1. 高さは「その重さの密集具合(密度)」\\
・「位置0(150g)」が最も高いのは、そこにデータが密集しているからです。\\
・高さはあくまで「その付近にどれだけポテトが詰まっているか」という勢いを示します。\\
\hfill \\
2. 点(位置)は「幅がないから 0」\\
・「位置+1」というピンポイントの場所には、幅がありません。\\
・どんなに密集していても、幅のない「線」の上からポテトをすくい取ることは不可能です。\\
・「幅 0」なら「量(面積)」も 0。だから「ある重さピッタリの確率」は 0 です。\\
\hfill \\
3. 幅(面積)が「確率」になる\\
・「位置 0 から +1 の間」というように「幅」を決めて初めて、ポテトをすくい取れます。\\
・この「すくい取った量(面積)」が、全体に対する「割合(確率)」になります。\\
\hfill \\
結論:だから「位置(点)」ではなく「累積(面積)」で確率を計算するのです。
}
\end{frame}

% ----------------------------------------------------------------------------------------
%   Slide 25:【応用】$1\sigma, 2\sigma, 3\sigma$ のルール
% ----------------------------------------------------------------------------------------
% \begin{frame}{【応用】$1\sigma, 2\sigma, 3\sigma$ のルール}
% 正規分布には、どんなデータにも共通して当てはまる「面積の黄金比」があります。

% \vspace{-0.5em}

% \begin{itemize}
%     \item \textbf{平均 $\pm 1\sigma$ の範囲:} 全体の約 \textbf{68\%} が入る(ごく普通)
%     \item \textbf{平均 $\pm 2\sigma$ の範囲:} 全体の約 \textbf{95\%} が入る(かなり広い範囲)
%     \item \textbf{平均 $\pm 3\sigma$ の範囲:} 全体の約 \textbf{99.7\%} が入る(ほぼすべて)
% \end{itemize}

% \begin{center}
%   \includegraphics[width=0.65\textwidth]{正規分布3シグマ.png}
% \end{center}

% \vspace{-1.2em}

% このルールを知っていれば、平均から標準偏差2個分($2\sigma$)以上離れたデータは「めったに起きない珍しいこと」だと即座に判断できます。

% \noteT{判定基準}{
% この「68-95-99.7の法則」が、後の仮説検定(「めったに起きないことが起きた」と判断する)の基礎になることを示唆します。
% }
% \end{frame}

% ----------------------------------------------------------------------------------------
%   Slide 26:【具体例】偏差値(SS)の正体
% ----------------------------------------------------------------------------------------
\begin{frame}{【具体例】偏差値(SS)の正体}
皆さんに馴染みのある「偏差値」は、正規分布の代表的な応用例です。

\begin{itemize}
    \item \textbf{偏差値のルール:}
    どんなテストでも、平均を \textbf{50}、標準偏差を \textbf{10} に無理やり固定した「特別な正規分布」に変換した数値です。
    \item \textbf{なぜそんなことをするのか:}
    平均点や難易度が違うテスト同士でも、「全体の中でどのあたりにいるか」を同じものさしで比べるためです。
\end{itemize}

偏差値70とは、「平均から $2\sigma$(20点分)も右側にいる、上位2.5\%の珍しい存在」という意味になります。

\noteT{偏差値}{
偏差値が「平均と標準偏差を固定した正規分布」であることを明かし、統計学が身近な選別基準に使われていることを理解させます。
}
\end{frame}

% ----------------------------------------------------------------------------------------
%   Slide 27:【実習3】累積分布(TRUE)で確率を出す
% ----------------------------------------------------------------------------------------
\begin{frame}{【実習3】累積分布(TRUE)で確率を出す}
では、実際に「ある値以下になる確率(面積)」をExcelで出してみましょう。

\begin{center}
    \Large \textbf{=NORM.DIST( x, 平均, 標準偏差, \ruby{TRUE}{累積} )}
\end{center}

\begin{itemize}
    \item \textbf{最後の引数を TRUE にする:}
    すると、その「x」以下の左側の面積をすべて合計した数字(確率)が出てきます。
    \item \textbf{計算例:}
    偏差値60以下の人は何%いるか? \par
    $\rightarrow$ \texttt{=NORM.DIST(60, 50, 10, TRUE)}
\end{itemize}

[Image showing a normal curve with the left tail up to 'x' shaded, representing the cumulative distribution function (TRUE).]

\noteT{実習3}{
実習1の「高さ(FALSE)」との違いを強調します。実務で使うのは主にこの「面積(TRUE)」の方であることを伝えます。
}
\end{frame}

% ----------------------------------------------------------------------------------------
%   Slide 28:【まとめ】全体の中の位置を知る
% ----------------------------------------------------------------------------------------
\begin{frame}{【まとめ】全体の中の位置を知る}
本日の学びを整理しましょう。

\begin{enumerate}
    \item 正規分布は \textbf{平均(位置)} と \textbf{標準偏差(幅)} で形が決まる。
    \item 標準偏差は「データのばらつき」を示す \textbf{単位(ものさし)} である。
    \item 山の \textbf{面積} を見れば、そのデータが起きる確率がわかる。
\end{enumerate}

正規分布を使えば、バラバラなデータの中から「自分は全体のどこにいるか?」を正確に知ることができます。

\textbf{次回予告:}\par
なぜ、これほどまでに多くのデータがこの「山」の形になるのか。
その魔法の正体(中心極限定理)を、実験で確かめます。

\noteT{まとめ}{
本日の学びを振り返りつつ、次回への「なぜ?」という疑問を投げかけて終了します。
}
\end{frame}