% @@@--(metropolis)--@@@
% ===== 第9回:導入(推測統計 → 仮説検定)ポテトのストーリー =====
% ※「アルゴリズム2 Beamer共通仕様 v2025」準拠(frame群のみ)
% ※teacherframeは使わない(noteTのみ)

%----------------------------------------------------------------------------------------
% Slide 01: 導入ストーリー(問い)
%----------------------------------------------------------------------------------------
\begin{frame}{導入:本当に「平均との差」がある?(ポテトの例)}
\small
あるハンバーガーショップでは、ポテトの量は「いつも同じくらい」に見えます。\\
しかし実際には、毎回すこしずつ量が違います(ばらつきがある)。

\vspace{0.6em}
今日もらったポテトは、いつもより\textbf{少ない}ように見えました。

\vspace{0.8em}
\textbf{問い:}\\
これは\textbf{たまたま少なかった}だけでしょうか?\\
それとも\textbf{本当に量が減った}と言ってよいでしょうか?

\vspace{0.8em}
% ---- 図・イラスト提案(ここに挿入) -----------------------------------------------
% 図案A(おすすめ:模式図)
%   ・ポテト箱のイラストを2つ並べる(左:いつも、右:今日)
%   ・右だけ量が少ない見た目にする
%   ・写真よりイラストの方が留学生に伝わりやすいことが多い
%
% 図案B(写真)
%   ・「ポテト(通常)」「ポテト(少なめ)」の写真を左右に配置
%   ・著作権の問題が出やすいので、自作画像 or フリー素材推奨
%
% 挿入例(ファイル名は仮):
% \begin{center}
%   \includegraphics[width=0.82\linewidth]{img/potato_compare.png}
% \end{center}
% -----------------------------------------------------------------------------

\noteT{導入:ポテトの例(目的と運用)}{
【目的】推測統計から仮説検定へ「判断モード」に切り替える。\\
【口頭補足】「平均と違う」だけでは判断できない。ポイントは「ばらつきの中で珍しいか」。\\
【運用】学生にまず直感で答えさせる(挙手:たまたま派/減った派)。次スライドで回収する。}
\end{frame}

%----------------------------------------------------------------------------------------
% Slide 02: 推測統計ではここまで(限界の明示)
%----------------------------------------------------------------------------------------
\begin{frame}{推測統計では、ここまでしか言えない}
\small
この疑問に対して、これまで学んだ\textbf{推測統計}でできることは次です。

\vspace{0.4em}
\begin{itemize}
  \item 平均(だいたいの中心)を\textbf{見積もる}
  \item ばらつき(標準偏差など)を\textbf{把握する}
  \item 「だいたいこのくらい」と\textbf{説明する}
\end{itemize}

\vspace{0.4em}
しかし推測統計だけでは、次の判断まではできません。

\vspace{0.3em}
\begin{block}{ここが今日のポイント}
\textbf{「本当に量が減ったと言ってよいか?」}\\
=\textbf{判断}(結論を出す)には、別の考え方が必要
\end{block}

% ---- 図・イラスト提案(ここに挿入) -----------------------------------------------
% 図案(おすすめ:2列対比)
%  左:推測統計(数値・区間)/右:判断(Yes/Noに近い)
%  例:
%   ・左に「平均=○○g」「ばらつき=△△」のメモ風アイコン
%   ・右に「減ったと言ってよい?」の吹き出し+?マーク
%
% 挿入例(ファイル名は仮):
% \begin{center}
%   \includegraphics[width=0.88\linewidth]{img/inference_vs_decision.png}
% \end{center}
% -----------------------------------------------------------------------------

\noteT{推測統計の限界(学生の混乱を予防)}{
【目的】「平均を出す=答えが出る」という誤解をここで止める。\\
【口頭補足】推測は「量を知る」、検定は「判断する」。\\
【運用】「平均との差がある」ことと「本当に変わったと言ってよい」ことは別、と繰り返す。}
\end{frame}

%----------------------------------------------------------------------------------------
% Slide 03: だから仮説検定(新しい目的の提示)
%----------------------------------------------------------------------------------------
\begin{frame}{だから「判断するための統計」=仮説検定}
\small
「本当に量が減ったと言ってよいか?」に答えるには、次の考え方を使います。

\vspace{0.4em}
\begin{enumerate}
  \item まず\textbf{いつも通り}だと仮定する(基準を置く)
  \item その仮定のもとで、今日の結果が\textbf{どれくらい珍しいか}を見る
  \item とても珍しければ、\textbf{いつも通り}は怪しいと判断する
\end{enumerate}

\vspace{0.5em}
この「珍しさで判断する」方法を、\textbf{仮説検定}と呼びます。

% ---- 図・イラスト提案(ここに挿入) -----------------------------------------------
% 図案A(おすすめ:3ステップの矢印図)
%   「いつも通り(基準)」→「珍しさ」→「判断」
%   ・各ステップに小さなアイコン(基準=定規、珍しさ=分布、判断=判定印)
%
% 図案B(次につながる布石:正規分布の位置)
%   ・ベルカーブ(正規分布)に点を置き、「端=珍しい」を視覚化
%   ・ここでは細かい式は不要。点の位置だけでOK。
%
% 挿入例(ファイル名は仮):
% \begin{center}
%   \includegraphics[width=0.88\linewidth]{img/hypothesis_test_flow.png}
% \end{center}
% -----------------------------------------------------------------------------

\noteT{仮説検定の導入(次の用語へつなぐ)}{
【目的】検定を「新しい難語」ではなく「判断の手順」として理解させる。\\
【口頭補足】この後で、\textbf{いつも通り=帰無仮説}、\textbf{珍しさ=p値}、\textbf{線引き=有意水準}と名前を後付けする。\\
【運用】このスライドでは用語を増やさない。まず流れだけ固定する。}
\end{frame}
