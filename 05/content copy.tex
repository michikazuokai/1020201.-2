% @@@--(metropolis)--@@@

% ----------------------------------------------------------------------------------------
%   Slide 01: 本日のテーマ
% ----------------------------------------------------------------------------------------
\begin{frame}{確率の基礎②(独立・期待値)}
今回は、確率を「将来の予測」や「意思決定」に役立てるための重要な考え方を学びます。

\begin{itemize}
  \item \textbf{\ruby{独立事象}{どくりつじしょう}:} 前の結果が次の結果に影響しない関係。
  \item \textbf{\ruby{期待値}{きたいち}:} 1回あたり「平均して」どれくらいの結果が見込めるか。
  \item \textbf{統計との繋がり:} 第2回で学んだ「平均」と「期待値」の関係を整理します。
\end{itemize}

「たった1回の結果」に一喜一憂せず、長期的な視点でデータを捉える力を養います。

\noteT{講義の狙い}{
「独立」という概念で直感のミスを防ぎ、「期待値」で不確実な未来を数値化する手法を伝えます。
}
\end{frame}

% ----------------------------------------------------------------------------------------
%   Slide 02: 独立事象(どくりつじしょう)とは
% ----------------------------------------------------------------------------------------
\begin{frame}{独立事象:お互いに影響し合わない関係}
2つの出来事(事象)AとBがあり、Aの結果がBの起こる確率に全く影響を与えないとき、これらを「\ruby{独立事象}{どくりつじしょう}」と呼びます。

\begin{itemize}
  \item \textbf{定義:} 1回目の結果がどうであれ、2回目の確率が変わらないこと。
  \item \textbf{具体例:} コイン投げを2回行う場合。
\end{itemize}

1回目に「表」が出たからといって、2回目に「裏」が出やすくなることはありません。コインには「記憶」がないからです。

\noteT{独立事象}{
「独立」という言葉が「他からの影響を受けない」という意味であることを強調します。
}
\end{frame}

% ----------------------------------------------------------------------------------------
%   Slide 03: 独立事象の計算ルール
% ----------------------------------------------------------------------------------------
\begin{frame}{独立事象の確率:掛け算で求める}
独立な2つの出来事が「両方とも起こる」確率は、それぞれの確率を掛け合わせることで計算できます。

\begin{block}{計算式:積の法則}
  \[ (\text{AとBが両方起こる確率}) = (\text{Aの確率}) \times (\text{Bの確率}) \]
\end{block}

\textbf{例:コイン2枚が両方とも「表」になる確率}
\begin{itemize}
  \item 1枚目が表の確率:$1/2$
  \item 2枚目が表の確率:$1/2$
  \item 両方表:$1/2 \times 1/2 = 1/4$ ($0.25$)
\end{itemize}



\noteT{積の法則}{
第3回で樹形図を使って「4通り中の1通り」と数えた結果と、掛け算の結果が一致することを確認させます。
}
\end{frame}

% ----------------------------------------------------------------------------------------
%   Slide 04: 【問い】「流れ」や「ツキ」は存在するのか?
% ----------------------------------------------------------------------------------------
\begin{frame}{【問い】独立性を直感で間違えていないか?}
あるソーシャルゲームのガチャで、当たる確率が 10\% だとします。
9回連続で外れました。

\textbf{問い:10回目に当たる確率は、10\% より高くなっているでしょうか?}

\begin{enumerate}
  \item 高くなっている(そろそろ当たるはず)
  \item 低くなっている(運が悪い時期だ)
  \item 変わらず 10\% である
\end{enumerate}

\vspace{1em}
(答えは次のスライドに示します)

\noteT{直感の罠}{
ギャンブラーの誤謬(ごびゅう)と呼ばれる、人間が陥りやすい錯覚を指摘します。
}
\end{frame}

% ----------------------------------------------------------------------------------------
%   Slide 05: 【答え】独立試行なら確率は変わらない
% ----------------------------------------------------------------------------------------
\begin{frame}{【答え】独立なら確率は常に一定}
正解は \textbf{「3. 変わらず 10\% である」} です。

\begin{itemize}
  \item \textbf{理由:} ガチャのシステムが「独立」に設計されているなら、過去の「外れ」という結果は、次の抽選に影響を与えません。
  \item \textbf{注意:} 「10回やれば1回は当たるだろう」という期待は、10回全体をまとめて考えた場合の話であり、\textbf{「次の1回」}の確率は常に 10\% のままです。
\end{itemize}

このように、独立事象を正しく理解することは、データに基づいた冷静な判断をするために不可欠です。

\noteT{解説}{
「10回セットの確率」と「単発の確率」を混同しないよう説明します。
}
\end{frame}

% ----------------------------------------------------------------------------------------
%   Slide 06: 期待値(きたいち)とは何か
% ----------------------------------------------------------------------------------------
\begin{frame}{期待値:1回あたりの「平均的な見込み」}
\ruby{期待値}{きたいち}(Expected Value)とは、ある\ruby{試行}{しこう}を行ったとき、平均してどれくらいの結果が得られるかを予測した数値です。

\begin{itemize}
  \item \textbf{意味:} 確率的に決まる「得点」や「金額」などの平均的な見込み。
  \item \textbf{考え方:} 「結果の値」に「その起こりやすさ(確率)」を掛けて、すべて足し合わせます。
\end{itemize}

\ruby{期待値}{きたいち}(を知ることで、「その勝負(または投資)に乗るべきか、避けるべきか」を数値で判断できるようになります。

\noteT{期待値の導入}{
「期待値 = 将来予測される平均値」というイメージを持たせます。
}
\end{frame}

% ----------------------------------------------------------------------------------------
%   Slide 07: 期待値の計算式
% ----------------------------------------------------------------------------------------
\begin{frame}{期待値の計算方法}
期待値は、以下の手順で計算します。

\begin{block}{期待値の公式}
  \[ E = (値_1 \times 確率_1) + (値_2 \times 確率_2) + \dots + (値_n \times 確率_n) \]
  (記号 $\sum$ を使うと: $E = \sum (x_i \cdot p_i)$ )
\end{block}

\textbf{計算のコツ:}
\begin{enumerate}
  \item 起こりうるすべての「結果(値)」を書き出す。
  \item それぞれの「確率」を計算する。
  \item 「値 $\times$ 確率」をすべて足す。
\end{enumerate}

\noteT{公式の意味}{
数学的な難しさよりも、「重み付きの合計」であるというニュアンスを伝えます。
}
\end{frame}

% ----------------------------------------------------------------------------------------
%   Slide 08: 【具体例】サイコロの期待値
% ----------------------------------------------------------------------------------------
\begin{frame}{期待値の計算例:サイコロの目}
普通のサイコロを1回振るとき、出る目の期待値を求めます。

\begin{table}[]
\begin{tabular}{|l|c|c|c|c|c|c|}
\hline
出る目 ($x$) & 1 & 2 & 3 & 4 & 5 & 6 \\ \hline
確率 ($p$) & 1/6 & 1/6 & 1/6 & 1/6 & 1/6 & 1/6 \\ \hline
\end{tabular}
\end{table}

\textbf{期待値の計算:}
\[ E = (1 \times \frac{1}{6}) + (2 \times \frac{1}{6}) + (3 \times \frac{1}{6}) + (4 \times \frac{1}{6}) + (5 \times \frac{1}{6}) + (6 \times \frac{1}{6}) \]
\[ E = \frac{1+2+3+4+5+6}{6} = \frac{21}{6} = \mathbf{3.5} \]

\textbf{解釈:} サイコロを何回も振ると、1回あたりの平均は 3.5 に近づきます。

\noteT{サイコロの例}{
3.5 という目は実際には出ませんが、これが「平均的な見込み」であることを強調します。
}
\end{frame}

% ----------------------------------------------------------------------------------------
%   Slide 09: 期待値と第2回の「平均」の違い
% ----------------------------------------------------------------------------------------
\begin{frame}{期待値と平均値の関係}
どちらも「データの中心」を表しますが、視点が異なります。

\begin{itemize}
  \item \textbf{平均値(記述統計):} すでに起きた「過去のデータ」の重心。
  \item \textbf{期待値(確率論):} これから起きる「未来の結果」の重心。
\end{itemize}

\textbf{繋がり:}
もし、将来も過去と同じ確率(割合)で出来事が起こると仮定するなら、「過去の平均値 = 将来の期待値」と考えることができます。



\noteT{統計との結びつけ}{
第2回の復習を兼ねつつ、確率を学ぶことが予測に繋がることを示します。
}
\end{frame}

% ----------------------------------------------------------------------------------------
%   Slide 10: 【実習】くじ引きの期待値を計算する
% ----------------------------------------------------------------------------------------
\begin{frame}{実習1:賞金の期待値を出す}
以下のくじがあります。1回引くときの賞金の期待値はいくらでしょうか?

\begin{itemize}
  \item 1等:10,000円(確率 1\%)
  \item 2等:500円(確率 10\%)
  \item ハズレ:0円(確率 89\%)
\end{itemize}

\textbf{ヒント:}
\begin{itemize}
  \item $10,000 \times 0.01 = ?$
  \item $500 \times 0.10 = ?$
  \item $0 \times 0.89 = ?$
\end{itemize}
これらを足し合わせてください。

\noteT{実習1}{
実際に手を動かし、金額ベースでの期待値を算出させます。
}
\end{frame}

% ----------------------------------------------------------------------------------------
%   Slide 11: 【答え】くじ引きの期待値
% ----------------------------------------------------------------------------------------
\begin{frame}{【答え】期待値は 150円}
計算式は以下の通りです。

\begin{itemize}
  \item 1等の寄与: $10,000 \times 0.01 = 100$
  \item 2等の寄与: $500 \times 0.10 = 50$
  \item ハズレの寄与: $0 \times 0.89 = 0$
\end{itemize}
\textbf{期待値} $E = 100 + 50 + 0 = \mathbf{150円}$

\textbf{意思決定への応用:}
もしこのくじの販売価格が **200円** だったなら、期待値(150円)よりも高いため、買い続けると損をすることになります。

\noteT{解説}{
期待値が「損得の判断基準」として非常に強力であることを伝えます。
}
\end{frame}

% ----------------------------------------------------------------------------------------
%   Slide 12: 大数の法則(たいすうのほうそく)
% ----------------------------------------------------------------------------------------
\begin{frame}{大数の法則:回数を増やすほど理論に近づく}
期待値が「長期的な平均」と言われる理由は、この法則にあります。

\begin{itemize}
  \item \textbf{法則:} 試行回数を増やせば増やすほど、実際の結果の平均値は、理論上の期待値に限りなく近づいていく。
\end{itemize}

\textbf{例:コイン投げ}
\begin{itemize}
  \item 10回だけ投げると、表が7回(70\%)出ることもあります。
  \item 10,000回投げると、表の割合はほぼ確実に 50\% に近づきます。
\end{itemize}



\noteT{大数の法則}{
「短期的には荒れるが、長期的には安定する」という統計の性質を伝えます。
}
\end{frame}

% ----------------------------------------------------------------------------------------
%   Slide 13: 期待値が教えるビジネスの「リスク」
% ----------------------------------------------------------------------------------------
\begin{frame}{ビジネスにおける期待値の活用}
ビジネスでは「成功する確証」がないことがほとんどですが、期待値を使うことでリスクを評価できます。

\textbf{投資判断の例:}
\begin{itemize}
  \item プランA:60\%の確率で100万円儲かるが、40\%で失う。
  \item プランB:100\%の確率で30万円儲かる。
\end{itemize}

プランAの期待値:$100 \times 0.6 + (-100) \times 0.4 = \mathbf{20万円}$
プランBの期待値:$\mathbf{30万円}$

\textbf{判断:} 期待値で見ると、プランBの方が「平均的には」有利であると言えます。

\noteT{ビジネス活用}{
期待値が「勘」ではなく「計算」で選択肢を比較する道具であることを示します。
}
\end{frame}

% ----------------------------------------------------------------------------------------
%   Slide 14: 【実習】Excelでシミュレーション(独立試行)
% ----------------------------------------------------------------------------------------
\begin{frame}{実習2:Excelで「独立」を確認する}
Excelの乱数(ランダムな数)を使って、独立なコイン投げをシミュレーションします。

\textbf{操作手順:}
\begin{enumerate}
  \item \texttt{=RANDBETWEEN(0, 1)} と入力(0=裏、1=表)。
  \item 下に100行ほどコピーする。
  \item \texttt{=AVERAGE(範囲)} で「表の割合」を計算する。
\end{enumerate}

\textbf{確認ポイント:}
F9キーを押して再計算するたびに、結果はどう変わりますか? 
個々の結果(0か1)はバラバラですが、平均はどのあたりをうろついていますか?

\noteT{実習2}{
Excelで実際に手を動かし、個別の結果は予測不能でも、平均(期待値)は安定することを確認させます。
}
\end{frame}

% ----------------------------------------------------------------------------------------
%   Slide 15: 相対度数と確率の収束
% ----------------------------------------------------------------------------------------
\begin{frame}{第2回からの繋がり:相対度数から確率へ}
第2回では、手元のデータを集計して「相対度数(割合)」を出しました。

\begin{itemize}
  \item \textbf{相対度数:} 実際に起きた結果の割合。
  \item \textbf{確率:} 理論的な「起こりやすさ」。
\end{itemize}

大数の法則により、データの件数(サンプルサイズ)を増やしていくと、**「相対度数」は「確率」に一致**していきます。
だからこそ、過去のデータ(記述統計)を正しく集計することが、未来の予測(確率)に役立つのです。

\noteT{収束の概念}{
記述統計と確率論がバラバラな知識ではなく、密接にリンクしていることをまとめます。
}
\end{frame}

% ----------------------------------------------------------------------------------------
%   Slide 16: まとめ:独立と期待値のポイント
% ----------------------------------------------------------------------------------------
\begin{frame}{本日のまとめ}
\begin{itemize}
  \item \textbf{独立事象:} 前の結果は次の確率に影響しない。
  \item \textbf{期待値:} $E = \sum (\text{値} \times \text{確率})$。長期的な平均見込み。
  \item \textbf{判断基準:} 期待値を使うことで、不確実な選択肢を数値で比較できる。
  \item \textbf{統計的視点:} 1回の結果に惑わされず、大数の法則を信じて全体像(平均)を捉える。
\end{itemize}

期待値は、記述統計で学んだ「平均」の未来バージョンと言えます。

\noteT{まとめ}{
最重要ポイントを再度復唱して定着を図ります。
}
\end{frame}

% ----------------------------------------------------------------------------------------
%   Slide 17: 【予習】次回の内容
% ----------------------------------------------------------------------------------------
\begin{frame}{次回への繋がり:確率分布}
本日は「サイコロの目」や「くじの賞金」など、特定の期待値を計算しました。

\begin{itemize}
  \item \textbf{次回(第5回):} 確率分布(かくりつぶんぷ)
  \item \textbf{内容:} あらゆる結果の確率をグラフにした「データの形」を学びます。
\end{itemize}

本日学んだ期待値は、次回のグラフの「中心」の位置として再び登場します。

\noteT{次回の案内}{
学習の継続性を示し、次への興味を引きます。
}
\end{frame}