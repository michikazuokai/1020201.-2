%----------------------------------------------------------------------------------------
%  metropolis template (refactored)
%----------------------------------------------------------------------------------------
\documentclass[aspectratio=169]{beamer}

% \documentclass の直後で hyperref のオプションを渡す(metropolisでも安全)
\PassOptionsToPackage{unicode=true,colorlinks=true,linkcolor=blue,urlcolor=blue}{hyperref}

\usetheme{metropolis}
\metroset{block=fill, sectionpage=progressbar, progressbar=foot}

% 背景色(tech のとき白など): Pythonで差し込み


%--------------------------
% 日本語
%--------------------------
\usepackage{luatexja}
\usepackage{luatexja-fontspec}
\usepackage{luatexja-ruby}
\setsansjfont{Hiragino Sans}[BoldFont={Hiragino Sans W6}]

%--------------------------
% 基本パッケージ(重複なし)
%--------------------------
\usepackage[table]{xcolor}
\usepackage{graphicx}
\usepackage[abs]{overpic}
\usepackage{tikz}
\usetikzlibrary{positioning, shapes.geometric} % ← shapes.geometric を追加
\usepackage{array}
\usepackage{tabularx}
\usepackage{booktabs}
\usepackage{makecell}
\usepackage{mathtools}
\usepackage{longtable}
\usepackage{pdfpages}
\usepackage{etoolbox} % AtBeginEnvironment 等
\usepackage[normalem]{ulem}
\usetikzlibrary{calc}
\usetikzlibrary{backgrounds}

\usepackage{pgfplots}
\pgfplotsset{compat=1.18} % バージョンに合わせて調整

% マーカー(ハイライター)の定義
\newcommand{\markline}[2][yellow]{%
  \tikz[baseline=(X.base)]{%
    \node[inner sep=0pt,outer sep=0pt] (X) {#2};
    \begin{scope}[on background layer]
      \fill[#1, opacity=0.35, rounded corners=0.8pt]
        ([xshift=-0.15em,yshift=0.00ex]X.south west) rectangle
        ([xshift= 0.15em,yshift=2.15ex]X.south east);
    \end{scope}
  }%
}



% minted(※ -shell-escape 必須)
\usepackage{minted}
\setminted{
  frame=single,
  framesep=2mm,
  fontsize=\footnotesize,
  breaklines=true
}

% (必要なときだけ)tcolorbox
\usepackage[most]{tcolorbox}
\tcbuselibrary{skins}        % 高度なデザイン機能
\tcbuselibrary{raster}

% hyperref は最後
\usepackage{hyperref}

% --- 1. カラーパレットの定義 ---
\definecolor{CanvaGreen}{HTML}{2E7D32} % メインの濃い緑(タイトル・枠線)
\definecolor{PaleGreen}{HTML}{F1F8E9}  % 背景の薄い緑
\definecolor{MyDarkGreen}{HTML}{587a7f}
\definecolor{DeepText}{HTML}{1C1C1C}   % 本文の文字色
\definecolor{MyWhiteBlue}{HTML}{F2FAFB} % 追加:あなたが指定した色

% --- 2. tcolorbox のデフォルト設定(全ボックスに適用) ---
%\tcbset{
%    enhanced,                          % 高度な装飾を有効化
%    colback=white,                 % 本文背景色
%    colframe=MyDarkGreen,               % 枠線の色
%    coltitle=white,                    % タイトル文字色
%    fonttitle=\bfseries\sffamily,      % タイトルを太字・ゴシック
%    boxrule=1pt,                       % 枠線の太さ
%    arc=2mm,                           % 角の丸み
%    left=3mm, right=3mm,               % 左右の余白
%    top=0.5mm, bottom=0.5mm,               % 本文の上下余白
%    toptitle=0.8mm, bottomtitle=0.5mm, % タイトル内の上下余白
%    before skip=0.8em, after skip=0.2em,   % ボックス前後の行間
%    shadow={0mm}{0mm}{0mm}{black!0}    % 影を完全に消してフラットに
%}
% --- 2. tcolorbox のデフォルト設定 ---
\tcbset{
    enhanced,
    colback=MyWhiteBlue,            % 本文背景を F2FAFB に変更
    colframe=MyDarkGreen,
    coltitle=white,                 % タイトル文字は白(背景が濃い色の場合)
    % coltitle=DeepText,            % もしタイトル背景も薄くするならこちら
    fonttitle=\bfseries\sffamily\small, % タイトルを少し小さくしてスリム化
    boxrule=0.5pt,                  % 1pt から 0.5pt へ細分化
    arc=1mm,                        % 2mm から 1mm へ変更
    sharp corners=south,            % 下側を直角に固定
    % --- 余白の調整 ---
    left=3mm, right=3mm,
    top=0.5mm, bottom=0.5mm,
    toptitle=0mm,                   % タイトル上の余白をゼロに
    bottomtitle=0mm,                % タイトル下の余白をゼロに
    % ------------------
    before skip=0.8em, after skip=0.2em,
    shadow={0mm}{0mm}{0mm}{black!0}
}

% --- カスタムボックスの定義 ---
\newtcolorbox{myListbox}[1]{
  enhanced,
  detach title,              % 1. 標準のタイトル位置を解除
  % 2. 本文が始まる直前(before upper)でタイトルを直接描画する
  before upper={{\bfseries\large #1}\par\medskip},
  % タイトルの後に改行を入れる設定
  after title={\par\medskip},
  colbacktitle=white, 
  colframe=gray!50,
  colback=white,
  titlerule=0pt,
  boxrule=1pt,
  fonttitle=\bfseries\color{black},
  title=#1,
  after skip=1.5ex,
  % --- 箇条書きの余白を強制的にゼロにする設定 ---
  before upper={
    \setbeamertemplate{itemize ispan}{0pt} % 項目間の余白
    \setbeamertemplate{itemize items}[default]
    \setlength{\leftmargini}{1.5em}
    % Beamerの内部変数を直接操作して行間を詰める
    \addtobeamertemplate{itemize/enumerate body begin}{}{\setlength{\itemsep}{0pt}\setlength{\parskip}{0pt}}
  }
}

%--------------------------
% パス
%--------------------------
\newcommand{\assetpath}{/Volumes/NBPlan/TTC/授業資料/2025年度/}
\graphicspath{{images/}{\assetpath/1020201.アルゴリズム2/12/images/}{../project_assets/images/}{../project_assets/emoji/emoji_pngs/}}

%--------------------------
% フッター
%--------------------------
\newcommand{\myfootertext}{1020201.アルゴリズム2/12}
\setbeamertemplate{footline}{%
  \leavevmode
  \hbox to \paperwidth{%
    \hspace*{0.2cm}
    \scriptsize\color{gray!50} \myfootertext
    \hfill
    \scriptsize\color{gray} \insertframenumber{} / \inserttotalframenumber
    \hspace*{0.4cm}
  }%
  \vspace{1pt}
}

%--------------------------
% TeacherFrame(外部)
%--------------------------
\usepackage{../teacherframe}

%--------------------------
% フレームタイトル:番号. タイトル
% ※ ここで出すだけ。insertframetitle を再定義しない(安全)
%--------------------------
\setbeamertemplate{frametitle}{%
  \vspace{0.6ex}%
  \begin{beamercolorbox}[wd=\paperwidth,sep=0.5ex,leftskip=0.9em,rightskip=0.5em]{frametitle}%
    \usebeamerfont{frametitle}%
    \insertframenumber.\,\insertframetitle%
  \end{beamercolorbox}%
}

%--------------------------
% 表用:列型
%--------------------------
\newcolumntype{C}[1]{>{\centering\arraybackslash}p{#1}}
\newcolumntype{M}[1]{>{\raggedright\arraybackslash}m{#1}}

%--------------------------
% ブロック(必要なら)
%--------------------------
\definecolor{myblue}{HTML}{7488FF}
\definecolor{mylightblue}{HTML}{E3EEFF}
\setbeamertemplate{blocks}[rounded]
\setbeamercolor{block title}{bg=myblue, fg=white}
\setbeamercolor{block body}{bg=mylightblue, fg=black}

%========================================================
% exampleblock(examplebox相当)だけの調整
%  - タイトル文字:白
%  - 背景色:現行のまま(bgは指定しない)
%  - 本文 itemize:文字も●も黒(exampleblock内だけ)
%========================================================

% タイトル文字だけ白(背景は触らない)
\setbeamercolor{block title example}{fg=white}

% 本文の通常文字色は「現行のまま」を基本にする(必要なら黒にしてもよい)
% ここは bg を触らないのが目的なので fg だけ調整可能
\setbeamercolor{block body example}{fg=black}

% exampleblock の中だけ itemize の色(●と文字)を黒に
\AtBeginEnvironment{exampleblock}{%
  \setbeamercolor{itemize item}{fg=black}
  \setbeamercolor{itemize subitem}{fg=black}
  \setbeamercolor{itemize subsubitem}{fg=black}
  \setbeamercolor{item}{fg=black} % 念のため
}

% exampleblock を抜けたらテーマ標準に戻す(色が残る事故防止)
\AtEndEnvironment{exampleblock}{%
  \setbeamercolor{itemize item}{fg=normal text.fg}
  \setbeamercolor{itemize subitem}{fg=normal text.fg}
  \setbeamercolor{itemize subsubitem}{fg=normal text.fg}
  \setbeamercolor{item}{fg=normal text.fg}
}

%--------------------------
% 奇数ページのスライドのを表示する
%(教示用だけでそれ以外はこの処理は動かない)
%--------------------------
% --- 教師用だけ、スライドを奇数開始に強制するトグル ---
\newif\ifoddslideenforce
\oddslideenforcefalse   % デフォルトOFF(pr/hoはOFF)

% --- 再帰防止ガード ---
\newif\ifoddslideguard
\oddslideguardfalse

% --- 偶数ページなら空白スライドを1枚入れて奇数に戻す ---
\newcommand{\ensureoddslide}{%
  \ifoddslideguard\relax\else
    \oddslideguardtrue
    \ifodd\value{page}\relax
      % 何もしない(次が奇数)
    \else
      \begin{frame}[plain,noframenumbering]
        \note{}% notes出力時に2枚消費させる保険
      \end{frame}
    \fi
    \oddslideguardfalse
  \fi
}

% --- frameが始まる直前に自動挿入(教師用だけ)---
\BeforeBeginEnvironment{frame}{%
  \ifoddslideenforce
    \ensureoddslide
  \fi
}

%--------------------------
% note / noteT の「常時安全化」
%  - tech 以外:\noteT は無視(エラーにならない)
%  - tech:notesmode_tech で上書き定義
%--------------------------
\providecommand{\notetitletext}{}      % 既にあっても衝突しない
\providecommand{\noteT}[2]{}           % デフォルトは何もしない

% frame開始ごとにタイトル変数をクリア(前の noteT が残らないように)
\AtBeginEnvironment{frame}{\gdef\notetitletext{}}

%--------------------------
% 切替(Pythonから差し込み)
%--------------------------
\mypausemodetrue
\teachermodefalse

%-------------

\providecommand{\noteT}[2]{} % noteT を無視


%--------------------------
% 方眼紙(グリッド)をスライドに重ね
%--------------------------
\input{grid_debug}

%-------------------------------------------------------------------------
% 「ラインマーカー」そのものです。線の太さ・色・透明度を自由にできます。
%
% #1 色(省略可)
% #2 下端 yshift
% #3 上端 yshift
% #4 文字
% \marklineA{0.35ex}{1.55ex}{通常サイズ}
% {\Large \marklineA{0.45ex}{2.10ex}{大きい文字}}
%-------------------------------------------------------------------------
\newcommand{\marklineA}[4][yellow]{%
  \tikz[baseline=(X.base)]{%
    \node[inner sep=0pt,outer sep=0pt] (X) {#4};
    \begin{scope}[on background layer]
      \fill[#1, opacity=0.35, rounded corners=0.8pt]
        ([xshift=-0.15em,yshift=#2]X.south west) rectangle
        ([xshift= 0.15em,yshift=#3]X.south east);
    \end{scope}
  }%
}

%----------------------------------------------------------------------------------------
% タイトル
%----------------------------------------------------------------------------------------
\title{ 12 単回帰(回帰の意味と散布図) }
\date{}
\newcommand{\codedir}{\assetpath/1020201.アルゴリズム2/12}

\begin{document}

\begin{frame}[plain,noframenumbering]
  \titlepage
  \bigskip
  \begin{center}
    \ifteachermode 教師用 \fi
  \end{center}
\end{frame}

% セクションページ(必要なら)
\setbeamertemplate{section page}{
  \begin{centering}
    \vfill
    \rule{\linewidth}{2pt}\par
    \vspace{1ex}
    {\usebeamerfont{section title}\Huge\bfseries \insertsection}\par
    \vspace{1ex}
    \rule{\linewidth}{2pt}\par
    \vfill
  \end{centering}
}
\setbeamerfont{section title}{size=\LARGE,series=\bfseries}

\AtBeginSection[]{
  \begin{frame}[plain,noframenumbering]
    \sectionpage
  \end{frame}
}

% 本編開始でフレーム番号を0から(必要なら)
\setcounter{framenumber}{0}

\input{emoji_macros}

% @@@--(metropolis)--@@@
% ===== 第11回:導入・前回回収(単回帰:回帰の意味と散布図)=====
% ※「アルゴリズム2 Beamer共通仕様 v2025」準拠(frame群のみ)
% ※teacherframe は使わない(noteTのみ)

%----------------------------------------------------------------------------------------
% Slide 01: 今日のテーマ
%----------------------------------------------------------------------------------------

%@@PAGEBAND@@
% ----------------------------------------------------------------------------------------
%   page 01
% ----------------------------------------------------------------------------------------
\begin{frame}{回帰で「将来予測」を作る(判断ではなく予測)}
\small

\textbf{今日やること(結論)}
    ハンバーガーショップのデータを使って、\\
    \textbf{「ポテトの注文数を増やすには、何が関係していそうか」}を見つけ、\\
    \textbf{「このくらい売れそう」}という\textbf{予測}を作ります。

\vspace{0.6em}
\begin{itemize}
  \item 今日のキーワード:\textbf{散布図}・\textbf{回帰直線}・\textbf{予測式($\hat{y}=ax+b$)}
  \item 回帰は、\textbf{正しい/間違いを決める}話ではなく、\textbf{未来を予測する}話
  \item ゴールは計算ではなく、\textbf{式の意味を言葉で説明できる}こと
\end{itemize}

\vspace{0.6em}
    \textbf{今日の到達目標}
      \begin{itemize}
        \item 散布図から「増える/減る/関係なし」を説明できる
        \item 回帰直線と式を表示し、\textbf{傾き $a$ の意味}を言える
        \item 回帰式で「もし $x$ がこうなら、$y$ はこのくらい」を\textbf{予測}できる
      \end{itemize}
% \begin{tcolorbox}[title=今日の到達目標]
% \begin{itemize}
%   \item 散布図から「増える/減る/関係なし」を説明できる
%   \item 回帰直線と式を表示し、\textbf{傾き $a$ の意味}を言える
%   \item 回帰式で「もし $x$ がこうなら、$y$ はこのくらい」を\textbf{予測}できる
% \end{itemize}
% \end{tcolorbox}

\noteT{Slide01:第11回の立ち位置(運用)}{
【目的】検定(判断)から回帰(予測)へ、目的が変わることを先に固定する。\\
【口頭補足】回帰は「当てる魔法」ではなく「だいたいの予測モデル」。点は散らばるのが普通。\\
【実習仕様】前半は clean データ(外れ値なし)で基本を成功させる:散布図→直線→式→R\textsuperscript{2}→予測。\\
後半は outlier データ(外れ値あり)で「現実はズレる」を確認:メモ(品切れ/キャンペーン)と結びつける。\\
【運用】開始直後に一言確認:「今日は“判断”と“予測”のどっち?」→「予測」。}
\end{frame}

%----------------------------------------------------------------------------------------
% Slide 02: ストーリー(店長の次の目標)
%----------------------------------------------------------------------------------------

%@@PAGEBAND@@
% ----------------------------------------------------------------------------------------
%   page 02
% ----------------------------------------------------------------------------------------
\begin{frame}{ストーリー:店長の次の目標は「ポテト売上UP」}
\small

前回まで店長は、\textbf{ポテトの重さのばらつき}をデータで見て改善しました。\\
(「本当に減ったと言ってよいか?」を\textbf{判断}できるようになった)

\vspace{0.6em}
\begin{tcolorbox}[title=次の目標(現場の目標)]
ポテトは\textbf{利益率が高い}。\\
だから店長は \textbf{ポテトの売上を伸ばす}ことを目標にしました。
\end{tcolorbox}

\vspace{0.6em}
\textbf{そこで店長は考えます:}
\begin{itemize}
  \item 「お客さんが\textbf{ハンバーガーと一緒に}ポテトを注文する数を増やしたい」
  \item 「そのために、\textbf{何が増えるとポテトも増える}のだろう?」
\end{itemize}

\vspace{0.4em}
\emjpin 今日はこの問いに対して、\textbf{データから予測の式を作る}。

\noteT{Slide02:ストーリーの狙い(例を変えずに目的だけ変える)}{
【目的】ポテト例を継続し、認知負荷を上げずに「回帰=予測」へ移る。\\
【口頭補足】前回まで:品質(ばらつき)を整える話。今回は:売上を伸ばす話。\\
【実習仕様】データは「日別」:ハンバーガー注文数($x$)とポテト注文数($y$)。まず clean で関係をつかむ。\\
【運用】学生に短く発言させる:「ポテトを増やすには何が大事?」→「客数/セット注文」など。}
\end{frame}

%----------------------------------------------------------------------------------------
% Slide 03: 前回回収(混同防止)
%----------------------------------------------------------------------------------------

%@@PAGEBAND@@
% ----------------------------------------------------------------------------------------
%   page 03
% ----------------------------------------------------------------------------------------
\begin{frame}{前回回収:検定は「判断」/回帰は「予測」(混同しない)}
\small

前回までやってきたのは、次のような\textbf{判断}です。\\
\textbf{「平均との差がある」}だけではなく、\textbf{ばらつきの中で珍しいか}で決めました。

\vspace{0.6em}
\begin{tcolorbox}[title=検定(前回)=判断]
\textbf{珍しさ(p値)}を使って、\\
「\textbf{偶然と言ってよいか/言いにくいか}」を決める。
\end{tcolorbox}

\vspace{0.6em}
一方、今日は次の目的です。

\begin{tcolorbox}[title=回帰(今日)=予測]
2つの数値の\textbf{関係}を使って、\\
「\textbf{もし $x$ がこうなら、$y$ はこのくらい}」を予測する。
\end{tcolorbox}

\vspace{0.4em}
\begin{itemize}
  \item 検定は「\textbf{棄却/棄却できない}」で結論を出す
  \item 回帰は「\textbf{予測の式}」を作り、使う(結論の形が違う)
\end{itemize}

\noteT{Slide03:混同防止(今日のモード切替)}{
【目的】検定(判断)と回帰(予測)を混ぜない。結論の形が違うことを固定する。\\
【口頭補足】今日もデータはばらつく。回帰直線は「全部当たる線」ではなく「だいたいの中心」。\\
【実習仕様】clean では回帰の基本(式とR\textsuperscript{2}、予測)を成功させる。outlier では1点の影響とメモ(品切れ/キャンペーン)で理由づけする。\\
【運用】ミニ確認:検定の結論は?→「棄却/棄却できない」。回帰の成果物は?→「予測の式」。}
\end{frame}

%----------------------------------------------------------------------------------------
% Slide 04: 問いの定式化(x→y)
%----------------------------------------------------------------------------------------

%@@PAGEBAND@@
% ----------------------------------------------------------------------------------------
%   page 04
% ----------------------------------------------------------------------------------------
\begin{frame}{問いの定式化:ハンバーガーが増えるとポテトは増える?($x\rightarrow y$)}
\small

店長の目標は「ポテト売上UP」でした。\\
そこで、\textbf{ポテトの注文数}に関係しそうなものを考えます。

\vspace{0.6em}
\begin{tcolorbox}[title=今日の問い(回帰の問い)]
\textbf{ハンバーガー注文数が増える日ほど、ポテト注文数も増えるのか?}
\end{tcolorbox}

\vspace{0.6em}
回帰では、2つの数値の役割を\textbf{先に決めます}。

\vspace{0.3em}
\begin{itemize}
  \item \textbf{$x$(説明変数)}:ハンバーガー注文数(原因\underline{かもしれない}側)
  \item \textbf{$y$(目的変数)}:ポテト注文数(増やしたい結果の側)
\end{itemize}

\vspace{0.4em}
\emjpin 今日やることは、\textbf{$x$を見たら$y$を予測できるか}を確かめること。

\noteT{Slide04:問いを数学の形に落とす(混乱防止)}{
【目的】「何を予測したいか」を1行で固定し、$x$と$y$の役割を先に決める。\\
【口頭補足】回帰は「$x$が増えると$y$が増える\textbf{かもしれない}」をデータで見る。ここでは因果は断定しない。\\
【実習仕様】clean/outlier ともに、$x$=ハンバーガー注文数、$y$=ポテト注文数で統一する(途中で変えない)。\\
【運用】学生に確認:「今日の$y$は何?」→「ポテト注文数」。}
\end{frame}

%----------------------------------------------------------------------------------------
% Slide 05: 散布図の定義
%----------------------------------------------------------------------------------------

%@@PAGEBAND@@
% ----------------------------------------------------------------------------------------
%   page 05
% ----------------------------------------------------------------------------------------
\begin{frame}{散布図とは:2つの数値の関係を「点」で見る図(定義)}
\small

回帰を始める前に、まず\textbf{散布図}で関係を\textbf{目で見ます}。

\vspace{0.6em}
\begin{tcolorbox}[title=散布図(定義)]
\textbf{1つのデータ(1日)を1つの点}として、\\
横軸に $x$、縦軸に $y$ をとって並べた図。
\end{tcolorbox}

\vspace{0.6em}
今回の例では、点の意味はこうです。
\begin{itemize}
  \item ある1日:\textbf{(ハンバーガー注文数 $x$ , ポテト注文数 $y$)}
  \item 30日分なら、\textbf{点が30個}できる
\end{itemize}

\vspace{0.4em}
\emjpin 散布図を見ると、\textbf{増える/減る/関係なし}が直感的に分かる。

\noteT{Slide05:散布図を先に固定する理由}{
【目的】回帰直線の前に「点の雲」を理解させる(線の意味が分かる土台)。\\
【口頭補足】散布図は「同時に観測した2つの数値」の関係を見る図。1点=1日の記録。\\
【実習仕様】まず散布図だけ作らせる(近似直線は次段階)。軸ラベルと単位も入れる。\\
【運用】ここで小確認:「1点は何?」→「1日のデータ」。}
\end{frame}

%----------------------------------------------------------------------------------------
% Slide 06: 散布図の読み方①(3パターン模式図)
%----------------------------------------------------------------------------------------

%@@PAGEBAND@@
% ----------------------------------------------------------------------------------------
%   page 06
% ----------------------------------------------------------------------------------------
\begin{frame}{散布図の読み方①:右上がり/右下がり/関係なし(3パターン)}
\small

散布図は、点の並び方で「関係の方向」を読みます。\\
(細かな計算の前に、まず目で判断できるようにする)

\vspace{0.4em}
\begin{columns}
  %-------------------------------
  % 左:右上がり
  %-------------------------------
  \begin{column}{0.32\linewidth}
  \centering
  \textbf{右上がり}\\
  \scriptsize($x$が増えると$y$も増えやすい)
  \vspace{0.2em}

  \begin{tikzpicture}[x=1.0cm,y=1.0cm]
    \draw[->] (0,0) -- (2.2,0);
    \draw[->] (0,0) -- (0,2.0);
    \node[below] at (1.1,0) {\scriptsize $x$};
    \node[left]  at (0,1.0) {\scriptsize $y$};

    % 点(右上がりっぽく)
    \foreach \p in {(0.4,0.5),(0.6,0.7),(0.9,0.8),(1.1,1.1),(1.3,1.2),(1.6,1.6),(1.8,1.7)}{
      \fill \p circle (1.7pt);
    }
  \end{tikzpicture}
  \end{column}

  %-------------------------------
  % 中:右下がり
  %-------------------------------
  \begin{column}{0.32\linewidth}
  \centering
  \textbf{右下がり}\\
  \scriptsize($x$が増えると$y$は減りやすい)
  \vspace{0.2em}

  \begin{tikzpicture}[x=1.0cm,y=1.0cm]
    \draw[->] (0,0) -- (2.2,0);
    \draw[->] (0,0) -- (0,2.0);
    \node[below] at (1.1,0) {\scriptsize $x$};
    \node[left]  at (0,1.0) {\scriptsize $y$};

    % 点(右下がりっぽく)
    \foreach \p in {(0.4,1.6),(0.6,1.4),(0.9,1.2),(1.1,1.0),(1.3,0.9),(1.6,0.6),(1.8,0.5)}{
      \fill \p circle (1.7pt);
    }
  \end{tikzpicture}
  \end{column}

  %-------------------------------
  % 右:関係なし
  %-------------------------------
  \begin{column}{0.32\linewidth}
  \centering
  \textbf{関係なし}\\
  \scriptsize($x$が変わっても$y$が読めない)
  \vspace{0.2em}

  \begin{tikzpicture}[x=1.0cm,y=1.0cm]
    \draw[->] (0,0) -- (2.2,0);
    \draw[->] (0,0) -- (0,2.0);
    \node[below] at (1.1,0) {\scriptsize $x$};
    \node[left]  at (0,1.0) {\scriptsize $y$};

    % 点(ばらばら)
    \foreach \p in {(0.4,1.0),(0.6,0.6),(0.8,1.7),(1.1,1.2),(1.2,0.4),(1.5,1.5),(1.8,0.9)}{
      \fill \p circle (1.7pt);
    }
  \end{tikzpicture}
  \end{column}
\end{columns}

\vspace{0.4em}
\emjpin 今日のデータ(ハンバーガー$\rightarrow$ポテト)は、\textbf{右上がりになりそう}かをまず確認する。

\noteT{Slide06:模式図を入れる理由(視覚で固定)}{
【目的】留学生でも「方向」を一目で判断できるように、まず3パターンを固定する。\\
【口頭補足】この段階では「強さ」はまだ言わない。「増える/減る/読めない」の3分類だけ。\\
【実習仕様】散布図を作ったら、まずこの3つのどれに近いかを言語化させる(1行でOK)。\\
【運用】学生に挙手で答えさせる:「今日の関係はどれっぽい?」}
\end{frame}

%----------------------------------------------------------------------------------------
% Slide 07: 散布図の読み方②(ばらつき・外れ値)
%----------------------------------------------------------------------------------------

%@@PAGEBAND@@
% ----------------------------------------------------------------------------------------
%   page 07
% ----------------------------------------------------------------------------------------
\begin{frame}{散布図の読み方②:ばらつき・外れ値(現実データの特徴)}
\small

現実のデータは、きれいに一直線には並びません。\\
同じ $x$ でも $y$ が少し違うのが普通です。

\vspace{0.6em}
\begin{tcolorbox}[title=重要:データには「ばらつき」がある]
\textbf{ばらつき}があるから、予測は\textbf{100\%当たる}わけではない。\\
でも、\textbf{中心の傾向}をつかめば、だいたいの予測はできる。
\end{tcolorbox}

\vspace{0.6em}
さらに、\textbf{外れ値}が混ざることもあります。

\vspace{0.2em}
\begin{itemize}
  \item 例:品切れでポテトが売れなかった日(普段より\textbf{極端に少ない})
  \item 例:キャンペーンでポテトが急に売れた日(普段より\textbf{極端に多い})
\end{itemize}

\vspace{0.4em}
\emjpin 後半の実習では、外れ値があると\textbf{回帰直線がどう変わるか}を確かめる。

\noteT{Slide07:ばらつきと外れ値を先に言う(後半実習の伏線)}{
【目的】「回帰直線=全部当たる線」という誤解をここで止める。\\
【口頭補足】ばらつきは悪いものではない(現実そのもの)。外れ値は「理由」があることが多い。\\
【実習仕様】前半cleanで基本を成功(外れ値なし)。後半outlierで外れ値の影響を見る。メモ列で理由づけする(品切れ/キャンペーン)。\\
【運用】外れ値を見つけたら「なぜ?」をメモで確認させる(説明できればOK)。}
\end{frame}

%----------------------------------------------------------------------------------------
% Slide 08: 回帰の目的(理由)
%----------------------------------------------------------------------------------------

%@@PAGEBAND@@
% ----------------------------------------------------------------------------------------
%   page 08
% ----------------------------------------------------------------------------------------
\begin{frame}{回帰の目的:点の雲を「一本の線」で要約し、予測に使う}
\small

散布図を作ると、点は\textbf{雲(ばらついた集まり)}になります。\\
このままだと、「次はどれくらい?」が言いにくい。

\vspace{0.6em}
\begin{tcolorbox}[title=回帰の目的(理由)]
点の雲を\textbf{一本の線}で要約して、\\
\textbf{予測に使える形}(式)にする。
\end{tcolorbox}

\vspace{0.6em}
回帰でやりたいことは、次の2つです。

\begin{itemize}
  \item \textbf{関係を要約する}:$x$が増えると$y$はどう変わりやすいか
  \item \textbf{予測する}:もし$x$がこの値なら、$y$はこのくらい($\hat{y}$)
\end{itemize}

\vspace{0.5em}
\emjpin ここでの「線」は、\textbf{全部当てる線}ではなく、\textbf{中心の傾向}を表す線。

\noteT{Slide08:回帰が必要な理由(散布図→線への納得)}{
【目的】「なぜ線を引くのか」を言葉で納得させる(式の前に理由)。\\
【口頭補足】点がバラバラでも、中心の流れが見えるなら予測はできる。回帰はその中心を“一本化”する。\\
【実習仕様】cleanで「線を引くと予測できる」成功体験を作る。outlierで「線がぶれる」も体験させる。\\
【運用】学生に一言で言わせる:「回帰の目的は?」→「点を線で要約して予測する」。}
\end{frame}

%----------------------------------------------------------------------------------------
% Slide 09: 回帰直線のイメージ(模式図)
%----------------------------------------------------------------------------------------

%@@PAGEBAND@@
% ----------------------------------------------------------------------------------------
%   page 09
% ----------------------------------------------------------------------------------------
\begin{frame}{回帰直線(イメージ):各$x$に対して“中心”を通る予測の線}
\small

回帰では、点の雲の\textbf{中心を通るような線}を考えます。\\
この線を\textbf{回帰直線}と呼びます。

\vspace{0.6em}
\begin{columns}
  %-------------------------------
  % 左:言葉
  %-------------------------------
  \begin{column}{0.54\linewidth}
  \begin{tcolorbox}[title=回帰直線(イメージ)]
\textbf{同じ$x$のときの$y$の“中心”}を結んだような線。\\
この線上の値が、$y$の\textbf{予測値}($\hat{y}$)。
  \end{tcolorbox}

  \vspace{0.4em}
  \begin{itemize}
    \item 点は線の周りに\textbf{散らばる}(ばらつき)
    \item それでも線があると、\textbf{だいたいの予測}ができる
  \end{itemize}

  \vspace{0.4em}
  \emjpin \textbf{点の中心を表す線}だと思えばよい(まずはこれでOK)。
  \end{column}

  %-------------------------------
  % 右:模式図
  %-------------------------------
  \begin{column}{0.42\linewidth}
  \centering
  \begin{tikzpicture}[x=1.1cm,y=1.0cm]
    % axes
    \draw[->] (0,0) -- (3.0,0);
    \draw[->] (0,0) -- (0,2.6);
    \node[below] at (1.5,0) {\scriptsize $x$(ハンバーガー)};
    \node[left]  at (0,1.3) {\scriptsize $y$(ポテト)};

    % points (cloud around a line)
    \foreach \p in {(0.5,0.8),(0.7,0.7),(0.9,1.1),(1.1,1.0),(1.3,1.2),
                   (1.5,1.4),(1.7,1.3),(1.9,1.7),(2.1,1.6),(2.3,1.9)}{
      \fill \p circle (1.7pt);
    }

    % regression line
    \draw[thick] (0.4,0.6) -- (2.6,2.1);
    \node[align=left] at (2.15,2.35) {\scriptsize 回帰直線\\\scriptsize(中心の傾向)};

    % a vertical guide for prediction
    \draw[dashed] (2.0,0) -- (2.0,1.65);
    \fill (2.0,1.65) circle (1.8pt);
    \node[right] at (2.0,1.65) {\scriptsize $\hat{y}$};
  \end{tikzpicture}

  \vspace{0.2em}
  \scriptsize
  点が散らばっていても、線があれば\\
  「この$x$なら$\hat{y}$はこのくらい」が言える
  \end{column}
\end{columns}

\noteT{Slide09:回帰直線の直感(数式前に図で固定)}{
【目的】「回帰直線=中心の傾向」「$\hat{y}$=線上の予測値」を図で固定する。\\
【口頭補足】同じ$x$でも$y$は毎回違う。その中心を線で表す。点が線からずれる分が“誤差/残差”。\\
【実習仕様】Excelでは散布図→近似直線を表示し、線上の式が出ることを確認させる。\\
【運用】学生に指さし確認:「$\hat{y}$はどこ?」→「線の上」。}
\end{frame}

%----------------------------------------------------------------------------------------
% Slide 10: 回帰式の説明(文章)
%----------------------------------------------------------------------------------------

%@@PAGEBAND@@
% ----------------------------------------------------------------------------------------
%   page 10
% ----------------------------------------------------------------------------------------
\begin{frame}{回帰式:$\hat{y}=ax+b$($\hat{y}$=予測値)を文章で理解する}
\small

回帰直線は、\textbf{式}で表せると便利です。\\
(Excelが計算して表示してくれる式を、\textbf{言葉で説明できる}のが今日の目標)

\vspace{0.6em}
\begin{center}
{\Large $\hat{y}=ax+b$}
\end{center}

\vspace{0.4em}
\begin{tcolorbox}[title=この式が言っていること(文章)]
\textbf{$x$(ハンバーガー注文数)が分かれば、}\\
\textbf{ポテト注文数の予測値 $\hat{y}$ を計算できる。}
\end{tcolorbox}

\vspace{0.5em}
\begin{itemize}
  \item $\hat{y}$(ハット付き)は\textbf{予測値}(実測の$y$とは違う)
  \item 点はばらつくので、\textbf{実測$y$は$\hat{y}$と一致しないことが多い}
\end{itemize}

\vspace{0.4em}
\emjpin 今日の実習は、\textbf{Excelで式を出し、その意味を言葉で説明する}こと。

\noteT{Slide10:式を“暗記”させないための言い換え}{
【目的】$\hat{y}=ax+b$を数式としてではなく「予測の計算ルール」として理解させる。\\
【口頭補足】$\hat{y}$は“当たった値”ではなく“予測した中心”。ハットが付くのは「予測だから」。\\
【実習仕様】近似直線の式を表示し、$x$に具体的な数(例:150)を入れて$\hat{y}$を計算させる。\\
【運用】短い確認:「$y$と$\hat{y}$の違いは?」→「$y$は実測、$\hat{y}$は予測」。}
\end{frame}

%----------------------------------------------------------------------------------------
% Slide 11: パラメータの意味(a と b)
%----------------------------------------------------------------------------------------

%@@PAGEBAND@@
% ----------------------------------------------------------------------------------------
%   page 11
% ----------------------------------------------------------------------------------------
\begin{frame}{パラメータの意味:傾き$a$=増え方/切片$b$=$x=0$の予測}
\small

回帰式 $\hat{y}=ax+b$ の\textbf{意味}は、$a$ と $b$ を見れば説明できます。

\vspace{0.6em}
\begin{tcolorbox}[title=傾き $a$(いちばん大事)]
\textbf{$x$が1増えると、$\hat{y}$が平均でどれくらい増えるか}。\\
例:$a=0.6$ なら、バーガーが10増えるとポテトは\textbf{約6増える}。
\end{tcolorbox}

\vspace{0.6em}
\begin{tcolorbox}[title=切片 $b$(注意して使う)]
\textbf{$x=0$ のときの予測値}。\\
ただし、現実に $x=0$(バーガー0)が\textbf{起きない範囲}なら、\\
$b$の現実的な意味は\textbf{薄い}ことがある。
\end{tcolorbox}

\vspace{0.4em}
\emjpin 実習では、まず\textbf{$a$(増え方)を言葉で説明できる}ことを優先する。

\noteT{Slide11:aを主役、bは最小限(混乱防止)}{
【目的】初心者がつまずきやすい「切片の意味」を最小限にし、傾きの解釈を主役にする。\\
【口頭補足】ビジネス上の意味はたいてい傾き(増え方)。切片は“計算の都合”として出ることもある。\\
【実習仕様】cleanデータの式で、傾きaを「10増えたら何増える?」に言い換えさせる。\\
【運用】学生に口頭で言わせる:「aは何?」→「増え方(xが1増えたときのyの増え方)」。}
\end{frame}

%----------------------------------------------------------------------------------------
% Slide 12: R^2 とは(正解率ではない)
%----------------------------------------------------------------------------------------

%@@PAGEBAND@@
% ----------------------------------------------------------------------------------------
%   page 12
% ----------------------------------------------------------------------------------------
\begin{frame}{R\textsuperscript{2}とは:線で説明できる割合の目安(正解率ではない)}
\small

回帰直線を引いたら、次に気になるのはこうです。\\
「この線は、どれくらい\textbf{データに合っている}の?」

\vspace{0.6em}
\begin{tcolorbox}[title=R\textsuperscript{2}(決定係数)]
回帰直線で、$y$のばらつきのうち\\
\textbf{どれくらいが$x$で説明できたか}を表す\textbf{目安}。
\end{tcolorbox}

\vspace{0.5em}
\begin{itemize}
  \item R\textsuperscript{2} は \textbf{0〜1} の間の数(例:0.72)
  \item \textbf{1に近い}ほど、点が線に\textbf{近く}、説明できる割合が高い
  \item \textbf{0に近い}ほど、点が線から\textbf{散らばり}、説明できる割合が低い
\end{itemize}

\vspace{0.6em}
\begin{tcolorbox}[title=ここが重要(誤解防止)]
R\textsuperscript{2} は \textbf{正解率ではない}。\\
「未来が必ず当たる」ことは保証しない。
\end{tcolorbox}

\noteT{Slide12:R\textsuperscript{2}の言い方を固定する(留学生対応)}{
【目的】R\textsuperscript{2}を「当てた回数」や「正解率」と誤解させない。意味を1文で固定する。\\
【口頭補足】R\textsuperscript{2}は「線で説明できる割合の目安」。未来が当たる保証ではない。\\
【実習仕様】Excelの近似直線オプションで「R\textsuperscript{2}値を表示」をONにし、cleanとoutlierで値がどう変わるか比較させる。\\
【運用】学生に一言で言わせる:「R\textsuperscript{2}って何?」→「線で説明できる割合の目安」。}
\end{frame}

%----------------------------------------------------------------------------------------
% Slide 13: 注意(相関≠因果/当てる保証ではない)
%----------------------------------------------------------------------------------------

%@@PAGEBAND@@
% ----------------------------------------------------------------------------------------
%   page 13
% ----------------------------------------------------------------------------------------
\begin{frame}{注意:相関≠因果/回帰は「当てる保証」ではない}
\small

回帰ができると、つい次のように言いたくなります。\\
「バーガーが増えた\textbf{から}、ポテトが増えた!」

\vspace{0.6em}
\begin{tcolorbox}[title=でも注意]
散布図や回帰で分かるのは、基本的に\textbf{関係(相関)}であって、\\
\textbf{原因(因果)}を確定することではない。
\end{tcolorbox}

\vspace{0.5em}
\textbf{例:} バーガーとポテトが同時に増える本当の理由は、別にあるかもしれません。
\begin{itemize}
  \item その日は\textbf{来客数}が多かった(両方増える)
  \item \textbf{キャンペーン}をやっていた(セット注文が増える)
  \item \textbf{品切れ}があった(片方だけ減る)
\end{itemize}

\vspace{-0.6em}
\begin{tcolorbox}[title=今日の結論の形]
回帰で言えるのは、\\
\textbf{「この範囲では、増える傾向がある/予測に使えるかもしれない」}まで。
\end{tcolorbox}

\noteT{Slide13:相関と因果の線引き(やりすぎない)}{
【目的】回帰=因果の証明、と誤解させない。特に留学生は断定表現に引っ張られやすい。\\
【口頭補足】回帰は「予測の道具」。原因は別調査(実験・追加データ)が必要になることが多い。\\
【実習仕様】outlierの回で、メモ列(品切れ/キャンペーン)を使い「別要因がある」ことを体験させる。\\
【運用】ここでは深掘りしない。「相関≠因果」を合言葉として1回言って終える。}
\end{frame}

%----------------------------------------------------------------------------------------
% Slide 14: 実習で使うデータ(clean / outlier)
%----------------------------------------------------------------------------------------

%@@PAGEBAND@@
% ----------------------------------------------------------------------------------------
%   page 14
% ----------------------------------------------------------------------------------------
\begin{frame}{実習で使うデータ:clean / outlier の2つ(目的が違う)}
\small

今日は同じテーマ(ポテト売上UP)でも、目的の違う2種類のデータで学びます。\\
\textbf{順番が大事}です(まず成功→次に現実)。

\vspace{0.6em}
\begin{tcolorbox}[title=データA:clean(外れ値なし)]
\textbf{回帰の基本を「成功」させるためのデータ}。\\
散布図が読みやすく、回帰直線がきれいに出る。
\end{tcolorbox}

\vspace{-0.4em}
\begin{itemize}
  \item 目的:散布図→直線→式→R\textsuperscript{2}→予測、をスムーズに体験
  \item 「傾き$a$の意味」を言葉で説明できるようにする
\end{itemize}

\vspace{-0.8em}
\begin{tcolorbox}[title=データB:outlier(外れ値あり)]
\textbf{現実のデータはズレる}ことを体験するためのデータ。\\
品切れ/キャンペーンなどで点が極端に外れる日が混ざる。
\end{tcolorbox}

\vspace{-0.4em}
\begin{itemize}
  \item 目的:外れ値が回帰直線・R\textsuperscript{2}・予測に与える影響を知る
  \item 「なぜズレたか」を\textbf{メモ列}から説明する
\end{itemize}

\noteT{Slide14:2データ構成の狙い(疲れない授業設計)}{
【目的】回帰の基本を“成功体験”で固めてから、現実(外れ値)に進む流れを宣言する。\\
【口頭補足】同じ回帰でも、学ぶ目的が違う。cleanは「型を覚える」、outlierは「現実の注意点」。\\
【実習仕様】Excelファイル内でシートを分ける(例:clean / outlier)。列構成は同じにして比較しやすくする。\\
【運用】学生に宣言:「まずAで回帰を完成させる。次にBで“なぜズレる?”を考える」。}
\end{frame}

%----------------------------------------------------------------------------------------
% Slide 15: 実習のゴール(手順の全体像)
%----------------------------------------------------------------------------------------

%@@PAGEBAND@@
% ----------------------------------------------------------------------------------------
%   page 15
% ----------------------------------------------------------------------------------------
\begin{frame}{実習のゴール:散布図→直線→式→R\textsuperscript{2}→予測→(外れ値でズレを考える)}
\small

実習では、回帰を「作って使う」までを\textbf{一気に}やります。\\
最後に、外れ値で\textbf{現実のズレ}も確認します。

\vspace{0.6em}
\begin{tcolorbox}[title=実習の流れ(この順番で固定)]
\begin{enumerate}
  \item \textbf{散布図}を作る($x$と$y$の関係を点で見る)
  \item \textbf{回帰直線}を引く(点の雲を一本の線で要約)
  \item \textbf{回帰式} $\hat{y}=ax+b$ を表示する($\hat{y}$=予測値)
  \item \textbf{R\textsuperscript{2}} を表示する(線で説明できる割合の目安)
  \item 具体的な$x$を入れて \textbf{予測}する($\hat{y}$を計算)
  \item (後半)outlierで \textbf{ズレの原因}を考える(メモ列と結びつける)
\end{enumerate}
\end{tcolorbox}

\vspace{0.4em}
\emjpin 「傾き$a$は何を意味する?」/「ズレた日はなぜ?」

\noteT{Slide15:成果物の形(評価ポイントの明確化)}{
【目的】実習のゴールを「操作」ではなく「説明できる成果物」に置く。\\
【口頭補足】計算はExcelがする。大事なのは“意味を言える”こと($a$、$\hat{y}$、R\textsuperscript{2}、外れ値)。\\
【実習仕様】確認物:①散布図(軸ラベルあり)②式とR\textsuperscript{2}表示 ③予測例($x$の値と$\hat{y}$)④外れ値の説明(メモ参照)。\\
【運用】時間が厳しい場合:cleanで⑤まで必達、outlierは⑥を口頭共有だけでも可。}
\end{frame}

% ===== 第11回:実習①(clean:基本を成功させる)(16〜18)=====
% ※「アルゴリズム2 Beamer共通仕様 v2025」準拠(frame群のみ)
% ※teacherframe は使わない(noteTのみ)
% ※提出なし:授業内での確認・共有を前提
%----------------------------------------------------------------------------------------
% Slide 16: 実習①-1 散布図を作る
%----------------------------------------------------------------------------------------

%@@PAGEBAND@@
% ----------------------------------------------------------------------------------------
%   page 16
% ----------------------------------------------------------------------------------------
\begin{frame}{実習①-1(clean):散布図を作る($x$=バーガー数、$y$=ポテト数)}
\small

\textbf{目的:} 回帰はまず\textbf{散布図}から始める。点の並びで「関係の方向」を読む。

\vspace{0.6em}
\begin{tcolorbox}[title=やること(操作)]
\begin{enumerate}
  \item \textbf{clean} シートを開く(外れ値なし)
  \item $x$(横軸)に \textbf{バーガー注文数}、$y$(縦軸)に \textbf{ポテト注文数} を選択
  \item \textbf{散布図(点)} を作る
\end{enumerate}
\end{tcolorbox}

\vspace{0.4em}
\begin{tcolorbox}[title=必ずやる(見た目のルール)]
\begin{itemize}
  \item グラフタイトル:\textbf{「バーガー数とポテト数(clean)」}
  \item 横軸ラベル:\textbf{バーガー注文数(件)}
  \item 縦軸ラベル:\textbf{ポテト注文数(件)}
\end{itemize}
\end{tcolorbox}

\vspace{0.4em}
\emjpin まずは\textbf{右上がり/右下がり/関係なし}のどれに近いかを言えるようにする。

\noteT{Slide16:実習仕様(授業内での確認ポイント)}{
【目的】回帰の出発点は散布図。軸の意味を固定してから線に進む。\\
【口頭補足】点が右上がりなら「増える傾向」。この段階では“強さ”は言わない。\\
【実習仕様】提出なし。教師は教室を回り、(1)散布図になっているか (2)軸ラベルが正しいか を確認。\\
【運用】早い学生に質問:「この散布図は3パターンのどれ?」→一言で答えさせる。}
\end{frame}

%----------------------------------------------------------------------------------------
% Slide 17: 実習①-2 回帰直線・式・R^2を表示
%----------------------------------------------------------------------------------------

%@@PAGEBAND@@
% ----------------------------------------------------------------------------------------
%   page 17
% ----------------------------------------------------------------------------------------
\begin{frame}{実習①-2(clean):回帰直線・式・R\textsuperscript{2}を表示(線形近似)}
\small

\textbf{目的:} 点の雲を\textbf{一本の線}で要約し、\textbf{式}と\textbf{R\textsuperscript{2}}を表示する。

\vspace{0.6em}
\begin{tcolorbox}[title=やること(操作)]
\begin{enumerate}
  \item 散布図上の点をクリック
  \item \textbf{近似曲線(トレンドライン)}を追加(\textbf{線形}を選ぶ)
  \item 「\textbf{グラフに数式を表示する}」をON
  \item 「\textbf{グラフにR\textsuperscript{2}値を表示する}」をON
\end{enumerate}
\end{tcolorbox}

\vspace{0.4em}
\begin{tcolorbox}[title=表示場所の指示(読みやすくする)]
\begin{itemize}
  \item 式とR\textsuperscript{2}は、点に重ならない場所へ\textbf{ドラッグして移動}
  \item 点が見えにくい場合は、グラフを少し拡大してOK
\end{itemize}
\end{tcolorbox}

\vspace{0.4em}
\emjpin ここで得るものは2つ:\textbf{回帰式}(予測ルール)と \textbf{R\textsuperscript{2}}(当てはまりの目安)。

\noteT{Slide17:Excel操作で迷わせない(最短手順)}{
【目的】操作で詰まると回帰の意味が飛ぶため、手順を短く固定する。\\
【口頭補足】線は“中心の傾向”。式は“予測の計算ルール”。R\textsuperscript{2}は“説明できる割合の目安”。\\
【実習仕様】提出なし。教師は(1)線形近似になっているか (2)式とR\textsuperscript{2}が表示されているか を確認。\\
【運用】うまく出ない学生への定番支援:「点を選択→近似曲線→線形→式表示→R\textsuperscript{2}表示」。}
\end{frame}

%----------------------------------------------------------------------------------------
% Slide 18: 実習①-3 読み取りと予測(傾きの解釈+指定xでŷ計算)
%----------------------------------------------------------------------------------------

%@@PAGEBAND@@
% ----------------------------------------------------------------------------------------
%   page 18
% ----------------------------------------------------------------------------------------
\begin{frame}{実習①-3(clean):読み取りと予測(傾きの解釈+指定$x$で$\hat{y}$計算)}
\small

\textbf{目的:} 出てきた式を「読める」「使える」にする。\\
(計算よりも\textbf{意味を言葉で説明}できることがゴール)

\vspace{0.6em}
\begin{tcolorbox}[title=Step1:傾き$a$を言葉で説明する]
回帰式 $\hat{y}=ax+b$ の \textbf{$a$} は、\\
\textbf{バーガーが1増えたとき、ポテトが平均でどれくらい増えるか}。
\end{tcolorbox}

\vspace{0.4em}
\begin{itemize}
  \item 例:$a=0.6$ なら「バーガーが10増えると、ポテトは\textbf{約6増える}」
\end{itemize}

\vspace{0.7em}
\begin{tcolorbox}[title=Step2:指定$x$で予測値$\hat{y}$を計算する]
教師が指定する $x$(例:$x=150$)を式に代入し、$\hat{y}$を計算する。\\
\textbf{線の上の値}=予測値(ハット付き)。
\end{tcolorbox}

\vspace{0.4em}
\emjpin 口頭チェック:\textbf{$y$は実測、$\hat{y}$は予測}(同じではない)。

\noteT{Slide18:授業内での確認(提出なし運用)}{
【目的】回帰の本質(傾きの解釈+予測)を“言葉”で言えるようにする。\\
【口頭補足】b(切片)は今回は深入りしない。主役はa(増え方)。$\hat{y}$は中心の予測で、当たりを保証しない。\\
【実習仕様】提出なし。教師指定の$x$を黒板/スライドで提示し、学生は各自の式で$\hat{y}$を計算。\\
【運用】数名を指名して発表:①aの意味を日本語で言う ②$x=150$のときの$\hat{y}$を言う(値は多少ズレてもOK、式の違いが原因と説明)。}
\end{frame}

% ===== 第11回:実習②(outlier:現実はズレる)(19〜20)=====
% ※「アルゴリズム2 Beamer共通仕様 v2025」準拠(frame群のみ)
% ※teacherframe は使わない(noteTのみ)
% ※提出なし:授業内での確認・共有を前提
%----------------------------------------------------------------------------------------
% Slide 19: 実習② 外れ値入りで再実行(比較+理由づけ)
%----------------------------------------------------------------------------------------

%@@PAGEBAND@@
% ----------------------------------------------------------------------------------------
%   page 19
% ----------------------------------------------------------------------------------------
\begin{frame}{実習②(outlier):外れ値入りで再実行(線・式・R\textsuperscript{2}の変化を見る)}
\small

\textbf{目的:} 現実のデータは「きれい」ではない。\\
外れ値が入ると、回帰直線・式・R\textsuperscript{2}・予測が\textbf{どう変わるか}を体験する。

\vspace{0.6em}
\begin{tcolorbox}[title=やること(操作は実習①と同じ)]
%\begin{tcolorbox}[title={やること 操作は実習1と同じ}]
\begin{enumerate}
  \item \textbf{outlier} シートを開く
  \item 散布図を作る(軸ラベルも同じ)
%  \item 線形の回帰直線を追加し、\textbf{式とR\textsuperscript{2}}を表示する
  \item 線形の回帰直線を追加し、式とR\textsuperscript{2}を表示する
\end{enumerate}
\end{tcolorbox}

\vspace{0.5em}
\begin{tcolorbox}[title=比較するポイント(ここが学び)]
\begin{itemize}
  \item \textbf{線の向き・位置}は変わったか?
  \item \textbf{傾き$a$}はどれくらい変わったか?
  \item \textbf{R\textsuperscript{2}}は上がった?下がった?
  \item 同じ$x$(例:$x=150$)で\textbf{$\hat{y}$はどれくらい変わる}?
\end{itemize}
\end{tcolorbox}

\vspace{0.6em}
\begin{tcolorbox}[title=理由づけ:メモ列で説明する]
外れている点を見つけたら、\textbf{メモ}を見て理由を言葉にする。\\
例:\textbf{品切れ}(本当は売れるのに売れない)/\textbf{キャンペーン}(急に増える)
\end{tcolorbox}

\noteT{Slide19:外れ値を“発見→説明”させる運用}{
【目的】外れ値の影響(線・式・R\textsuperscript{2}・予測のズレ)を、計算ではなく比較で理解させる。\\
【口頭補足】外れ値は「ミス」ではなく「現実の事情」を含むことがある。メモはその手がかり。\\
【実習仕様】提出なし。教師指定で「外れている点を1つ見つけて、メモで理由を言う」を必達にする。\\
【運用】共有のしかた:2〜3名に発表させる(①どの点が外れ値?②メモは?③それで線やR\textsuperscript{2}はどう変わった?)。}
\end{frame}

%----------------------------------------------------------------------------------------
% Slide 20: まとめ+次回への橋(残差→最小二乗法)
%----------------------------------------------------------------------------------------

%@@PAGEBAND@@
% ----------------------------------------------------------------------------------------
%   page 20
% ----------------------------------------------------------------------------------------
\begin{frame}{まとめ}
\small

\begin{tcolorbox}[title=今日の到達点(ここだけ)]
\begin{itemize}
  \item 散布図で「関係」を見て、回帰直線で\textbf{一本の線}に要約した
  \item 回帰式 $\hat{y}=ax+b$ を使って、\textbf{予測}を計算できるようになった
  \item R\textsuperscript{2}は「当てはまりの目安」で、\textbf{正解率ではない}と分かった
\end{itemize}
\end{tcolorbox}

\vspace{-0.6em}
\begin{tcolorbox}[title=現実のポイント:ズレ(残差)が必ずある]
実測 $y$ と予測 $\hat{y}$ は一致しないことが多い。\\
この\textbf{ズレ}を \textbf{残差}(誤差)と呼ぶ。
\end{tcolorbox}

\vspace{-0.5em}
\begin{itemize}
  \item clean:ズレは小さめ → 回帰の基本が見える
  \item outlier:ズレが大きくなる → 予測がぶれる/R\textsuperscript{2}も変わる
\end{itemize}

\vspace{-0.6em}
\begin{tcolorbox}[title=次回への橋:では、線はどうやって決めている?]
Excelは勝手に線を引いているように見えるが、\\
実は「ズレ(残差)を小さくする」ルールで線を決めている。\\
次回はその決め方=\textbf{最小二乗法}を学ぶ。
\end{tcolorbox}

\noteT{Slide20:次回(最小二乗法)への接続を自然にする}{
【目的】回帰の“結果”から“決め方”へ関心を移す。最小二乗法は「ズレを小さくするルール」として導入。\\
【口頭補足】残差=点から線までの縦の距離(まずはこの直感でOK)。なぜ二乗?は次回に回す。\\
【実習仕様】最後に口頭確認:①回帰は何を作る?→「予測モデル」②ズレの名前は?→「残差」。\\
【運用】次回の導入に使える一言問い:「線はどうやって決めたの?」→「ズレを最小にする(最小二乗法)」。}
\end{frame}
\end{document}
