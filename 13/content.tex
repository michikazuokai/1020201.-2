% @@@--(metropolis)--@@@
% ===== 第12回:単回帰(最小二乗法)A-1〜B-5(箱減らし版)=====
% ※「アルゴリズム2 Beamer共通仕様 v2025」準拠(frame群のみ)
% ※tcolorbox は「ここぞ」のみに限定(目安:2〜3枚に1回)
% ※teacherframeは使わない(noteTのみ)

%----------------------------------------------------------------------------------------
% A-1: 今日のテーマ(tcolorboxは1つだけ)
%----------------------------------------------------------------------------------------
\begin{frame}{回帰直線は「どう決める?」(最小二乗法)}
\small

\begin{tcolorbox}[title=今日の結論(先に言う)]
回帰直線は、\textbf{ズレ(残差)の二乗の合計}が\\
\textbf{いちばん小さくなる}ように決められています。
\end{tcolorbox}

\vspace{0.4em}
\begin{itemize}
  \item 前回:散布図から回帰直線を引き、式と $R^2$ を見て予測した
  \item 今日は:その「直線の決め方」を理解する(\textbf{アルゴリズム})
  \item ゴール:Excelの結果をブラックボックスにせず、\textbf{なぜその線か}を説明できる
\end{itemize}

\vspace{0.3em}
\textbf{到達目標}
\begin{itemize}
  \item 残差($y-\hat{y}$)を「点と線のズレ」として説明できる
  \item 残差平方和(SSE)を作り、\textbf{小さいほど良い}と説明できる
  \item 「最小二乗法=SSE最小の線」と言葉で言える
\end{itemize}

\noteT{A-1:第12回の立ち位置(運用)}{
【目的】「今日は何を理解する回か」を最初に固定する(最小二乗法=線の決め方)。\\
【口頭補足】第11回は「使う(予測する)」回。第12回は「作り方(決め方)」の回。\\
【実習仕様】cleanデータで、\hat{y}→残差→残差の二乗→SSE(合計)をExcelで作る(提出なし)。\\
【運用】冒頭で確認:回帰は“判断”ではなく“予測”。(検定と混同させない)}
\end{frame}

%----------------------------------------------------------------------------------------
% A-2: ストーリー回収(箱なし)
%----------------------------------------------------------------------------------------
\begin{frame}{ストーリー:次の目標は「ポテト売上UP」}
\small

店長は、ポテトのばらつきを改善したあと、次の目標を立てました。

\vspace{0.3em}
\textbf{店長の目標:}
\begin{itemize}
  \item ポテトは利益率が高い
  \item だから「\textbf{ポテトの売上を伸ばす}」ことを目標にする
\end{itemize}

\vspace{0.4em}
そのために知りたいこと:
\begin{itemize}
  \item ハンバーガーの注文が増えると、ポテトの注文も増えるのか?
  \item 関係があるなら、\textbf{来週の注文数}を\textbf{予測}できるか?
\end{itemize}

\vspace{0.4em}
\emjpin 今日の視点:\textbf{関係を式(直線)にして予測に使う。}\\
その直線を\textbf{どうやって決めるか}が今日のテーマ。

\noteT{A-2:例を継続する理由(口頭補足)}{
【目的】例(ハンバーガーショップ)を固定し、内容理解に集中させる。\\
【口頭補足】例が変わると「状況理解」からやり直しになる。店長ストーリーで一貫させる。\\
【実習仕様】x=バーガー数、y=ポテト数(clean / outlier の2データで扱う)。\\
【運用】ここで「今日は予測の話」と言い切る(検定の“判断”と混同しやすい)。}
\end{frame}

%----------------------------------------------------------------------------------------
% A-3: 前回回収(箱なし:比較を明瞭に)
%----------------------------------------------------------------------------------------
\begin{frame}{前回回収:検定は「判断」/回帰は「予測」(混同しない)}
\small

\textbf{前回(検定)と今回(回帰)は目的が違います。}

\vspace{0.4em}
\textbf{前回:仮説検定(判断)}
\begin{itemize}
  \item 「平均が基準と違うと言ってよいか?」を\textbf{珍しさ(p値)}で判断する
  \item 結論は「棄却/棄却できない」で書く
\end{itemize}

\vspace{0.4em}
\textbf{今回:回帰(予測)}
\begin{itemize}
  \item 「xが増えるとyはどう変わるか?」を\textbf{式(回帰直線)}で表す
  \item 式を使って\textbf{未来のy}を予測する
\end{itemize}

\vspace{0.4em}
\emjpin 今日は p値は使わない。代わりに「\textbf{ズレが小さい線}」を考える。

\noteT{A-3:混同防止(短く強く)}{
【目的】検定と回帰の混同を最初に止める(留学生ほど混ざりやすい)。\\
【口頭補足】検定=結論を“判断”。回帰=値を“予測”。今日はp値は使わない。\\
【実習仕様】回帰では「ズレ(残差)」を作り、その二乗の合計(SSE)で線の良さを比べる。\\
【運用】確認質問:検定は何?→判断。回帰は何?→予測。今日のキーワードは?→残差・二乗・最小。}
\end{frame}

%----------------------------------------------------------------------------------------
% B-4: 問い(ここぞでtcolorbox 1つ)
%----------------------------------------------------------------------------------------
\begin{frame}{問い:同じ散布図に「線」は何本でも引ける}
\small

散布図を見ると「だいたい右上がり」に見えてきます。\\
でも、\textbf{直線は1本に決まっていません}。

\vspace{0.6em}
\begin{tcolorbox}[title=問い]
この散布図に引く直線として、\\
\textbf{どの線が「良い線」でしょうか?}
\end{tcolorbox}

\vspace{0.4em}
\begin{itemize}
  \item 傾きが少し違う線/上下にずれた線……どれも「それっぽく」見える
  \item だから、見た目ではなく\textbf{評価基準(ルール)}が必要
\end{itemize}

% ---- 図の提案 ---------------------------------------------------------------
% 同じ散布図に候補の直線を3本重ねる(傾き違い or 切片違い)
% → 次スライドの「ズレを数で決める」へ接続しやすい
% ---------------------------------------------------------------------------

\noteT{B-4:問いの狙い(運用)}{
【目的】「回帰直線は1本に決まる」という前提を崩し、評価基準が必要だと気づかせる。\\
【口頭補足】学生に質問:「線A・B・Cどれが良い?」→理由も言わせる(“近い”“ずれが小さい”を引き出す)。\\
【実習仕様】後半で、Excelで「ズレ」を数値化して比べる(SSEまで作る)。\\
【運用】ここではまだ数式を出さない。まず“線の良さを決めたい”という欲求を作る。}
\end{frame}

%----------------------------------------------------------------------------------------
% B-5: 評価の方針(箱なし)
%----------------------------------------------------------------------------------------
\begin{frame}{評価の方針:良い線=点に近い(ズレが小さい)を数で決める}
\small

\textbf{良い線}を決めるために、次の方針を使います。

\vspace{0.4em}
\emjpin \textbf{良い線}とは、点と線の\textbf{ズレ}が\textbf{全体として小さい}線。

\vspace{0.5em}
ズレを「数」で決めるために、次を行います。
\begin{itemize}
  \item 各点について「ズレ」を測る
  \item 全部の点のズレを\textbf{まとめて}1つの数にする
  \item その数が\textbf{いちばん小さい線}を選ぶ
\end{itemize}

\vspace{0.5em}
次に出てくる言葉:
\begin{itemize}
  \item 点と線のズレを \textbf{残差(ざんさ)} と呼ぶ
  \item 次スライドで、残差を\textbf{図と言葉と式}で定義する
\end{itemize}

\noteT{B-5:評価基準を先に固定(口頭補足)}{
【目的】回帰の本体は「線を決めるルール」だと理解させる。\\
【口頭補足】良い線=“当たる保証”ではなく、“ズレが小さい線”。(未来の保証ではない)\\
【実習仕様】残差→残差の二乗→合計(SSE)をExcelで作り、線の良さを比較する。\\
【運用】次スライドにつなぐ一言:「ズレ=残差。まず“残差って何?”を定義する」。}
\end{frame}

%----------------------------------------------------------------------------------------
% C-6: 残差の定義(ズレを定義する)
%----------------------------------------------------------------------------------------
\begin{frame}{残差の定義:残差=実測 $y$ − 予測 $\hat{y}$(縦のズレ)}
\small

回帰直線(予測の線)を引くと、各点には「ズレ」が生まれます。\\
このズレを\textbf{残差(ざんさ)}と呼びます。

\vspace{0.6em}
\begin{center}
{\Large
\[
\text{残差 } e = y - \hat{y}
\]
}
\end{center}

\vspace{0.3em}
\begin{itemize}
  \item $y$:実測値(実際に観測された値)
  \item $\hat{y}$:予測値(直線が予測する値)
  \item 残差 $e$ は、\textbf{点と線の「縦方向の距離」}(上下のズレ)
\end{itemize}

\vspace{0.6em}
\emjpin 回帰直線は「全部を完全に当てる線」ではない。\\
だから、残差(ズレ)は必ず出る。

% ---- 図の提案 ---------------------------------------------------------------
% 図案(おすすめ):1つの点から回帰直線へ「縦の矢印(残差)」を描く
%  ・点の座標: (x, y)
%  ・線上の予測値: (x, yhat)
%  ・縦の距離に e = y - yhat とラベル
% ---------------------------------------------------------------------------

\noteT{C-6:残差を「距離」として固定(運用)}{
【目的】残差を「式の引き算」ではなく「点と線の縦のズレ」として理解させる。\\
【口頭補足】横方向ではなく縦方向のズレを見る(xは説明変数として固定して考える)。\\
【実習仕様】Excelで各行に $\hat{y}$(予測)列を作り、残差列 $e=y-\hat{y}$ を作る。\\
【運用】一言確認:「残差って何?」→「実測−予測(縦のズレ)」。}
\end{frame}

%----------------------------------------------------------------------------------------
% C-7: 残差の符号(相殺が起きる)
%----------------------------------------------------------------------------------------
\begin{frame}{残差の符号:線より上は+/下は−(そのまま足すと相殺する)}
\small

残差には\textbf{符号(プラス/マイナス)}があります。

\vspace{0.4em}
\begin{itemize}
  \item 点が直線より\textbf{上}にある:$y > \hat{y}$ なので、\textbf{残差は+}
  \item 点が直線より\textbf{下}にある:$y < \hat{y}$ なので、\textbf{残差は−}
\end{itemize}

\vspace{0.6em}
ここで大事な問題が起きます。

\vspace{0.4em}
\emjpin 残差を\textbf{そのまま足す}と、\\
+と−が\textbf{打ち消し合って}(相殺して)小さく見えてしまう。

\vspace{0.6em}
\textbf{例(イメージ)}
\begin{itemize}
  \item $+5$ と $-5$ を足すと $0$(ズレがないように見える)
  \item でも実際には、上下に\textbf{大きくズレている}
\end{itemize}

% ---- 次への橋渡し(ここではtcolorboxを使わない) ---------------------------
% 次スライドで「相殺を防ぐために二乗する(残差^2)」へ自然につなぐ
% ---------------------------------------------------------------------------

\noteT{C-7:なぜ“そのまま足せない”のか(運用)}{
【目的】「相殺」という問題を先に理解させ、次の“二乗”が必要な理由を作る。\\
【口頭補足】ズレは“方向”ではなく“大きさ”を評価したい(だから符号は邪魔になる)。\\
【実習仕様】Excelで残差列を作った後、残差の二乗を作る(相殺を防ぐ準備)。\\
【運用】次につなぐ問い:「相殺を防ぐにはどうする?」→「二乗する」。}
\end{frame}

%----------------------------------------------------------------------------------------
% D-8: 問題:残差の合計は基準にならない
%----------------------------------------------------------------------------------------
\begin{frame}{問題:残差の合計は「+と−」で打ち消し合い、基準にならない}
\small

前スライドで見た通り、残差 $e=y-\hat{y}$ には\textbf{+と−}があります。\\
そのため、残差をそのまま足すと「ズレの大きさ」を表せません。

\vspace{0.6em}
\textbf{なぜダメか(相殺)}
\begin{itemize}
  \item 例:$+5$ と $-5$ を足すと $0$
  \item 合計が $0$ でも、実際には上下に\textbf{大きくズレている}
\end{itemize}

\vspace{0.6em}
\emjpin だから、回帰では「符号つきの合計」ではなく、\\
\textbf{ズレの大きさ}を足し合わせる方法が必要。

\noteT{D-8:二乗への導入(運用)}{
【目的】「合計では評価できない」という問題意識を確定し、二乗の必然性を作る。\\
【口頭補足】ズレは“方向”ではなく“大きさ”を評価したい、という言い方に統一する。\\
【実習仕様】この直後に「残差の二乗」を作らせ、相殺が消えることを体感させる。\\
【運用】ここで問いかけ:「相殺しない方法は?」→ 次スライドで二乗を提示。}
\end{frame}

%----------------------------------------------------------------------------------------
% D-9: 二乗する理由(ここぞ:tcolorbox 1つ)
%----------------------------------------------------------------------------------------
\begin{frame}{二乗する理由:なぜ $(y-\hat{y})^2$ なのか}
\small

相殺を防ぐために、残差を\textbf{二乗}してから足し合わせます。

\vspace{0.6em}
\begin{tcolorbox}[title=二乗する3つの理由(ここだけ覚える)]
\begin{enumerate}
  \item \textbf{マイナスが消える}(相殺しない)
  \item \textbf{大きいズレを強く罰する}(大外れを重く扱う)
  \item \textbf{計算として扱いやすい}(後で「最小」を作りやすい)
\end{enumerate}
\end{tcolorbox}

\vspace{0.4em}
\emjpin 回帰は「全部当てる」ではなく、\\
\textbf{ズレをできるだけ小さくする}ことが目的。

\noteT{D-9:二乗の納得(口頭補足)}{
【目的】二乗を“暗記”ではなく“必要性”として納得させる。\\
【口頭補足】③は深掘りしない。「二乗だと数学的に最小を作りやすい」程度で止める。\\
【実習仕様】残差列と残差の二乗列を並べ、相殺が消えることを確認させる。\\
【運用】学生に一言で言わせる:「なぜ二乗?」→「相殺しない/大ズレを重くする」。}
\end{frame}

%----------------------------------------------------------------------------------------
% D-10: 残差平方和(SSE)の定義
%----------------------------------------------------------------------------------------
\begin{frame}{残差平方和(SSE):ズレの大きさの合計を1つの数にする}
\small

残差を二乗して、\textbf{全部足した}数を作ります。\\
これが「線の良さ」を表す指標になります。

\vspace{0.6em}
\begin{center}
{\Large
\[
\mathrm{SSE}=\sum (y-\hat{y})^2
\]
}
\end{center}

\vspace{0.4em}
\begin{itemize}
  \item 各点のズレ:$(y-\hat{y})$
  \item そのズレの大きさ:$(y-\hat{y})^2$
  \item それを全部足す:$\sum (y-\hat{y})^2$
\end{itemize}

\vspace{0.6em}
\emjpin \textbf{SSEが小さいほど、点に近い(ズレが小さい)線}。

\noteT{D-10:SSEを「1つの数」にする意味(運用)}{
【目的】多数の点のズレを「比較できる1つの数」にまとめる発想を固定する。\\
【口頭補足】SSEは「ズレの大きさの合計」。単位は $y$ の二乗になるが、ここでは深掘りしない。\\
【実習仕様】Excelで残差の二乗を作り、SUMでSSEを作成(線ごとにSSE比較も可能)。\\
【運用】次スライドにつなぐ問い:「良い線は?」→「SSEが最小の線」。}
\end{frame}

%----------------------------------------------------------------------------------------
% D-11: 最小二乗法(今日の結論の回収)
%----------------------------------------------------------------------------------------
\begin{frame}{最小二乗法:SSEが最小になる直線が「回帰直線」}
\small

同じ散布図に引ける直線は何本もあります。\\
その中から\textbf{1本}を選ぶルールが、最小二乗法です。

\vspace{0.6em}
\textbf{最小二乗法(言葉での定義)}
\begin{itemize}
  \item 直線ごとに SSE(残差平方和)を計算する
  \item \textbf{SSEがいちばん小さい直線}を選ぶ
\end{itemize}

\vspace{0.6em}
\emjpin つまり、回帰直線は\\
\textbf{「ズレの二乗の合計が最小」}になるように決まる。

\vspace{0.6em}
\textbf{注意(ここで止める)}
\begin{itemize}
  \item 今日は「最小にする考え方」を理解する回
  \item 次の塊で「実際にどうやって最小を作るか(Excelで体験)」へ進む
\end{itemize}

\noteT{D-11:結論の回収(運用)}{
【目的】A-1で先に出した結論を、SSEの定義を経て“納得”として回収する。\\
【口頭補足】ここで微分や公式導出は不要。最小二乗法=「SSEが最小」と言えれば十分。\\
【実習仕様】ExcelでSSEを実際に計算し、線を変えるとSSEが変わることを体感させる。\\
【運用】ここで一言確認:「回帰直線はどう決まる?」→「SSEが最小の線」。}
\end{frame}

%----------------------------------------------------------------------------------------
% E-12: 模式図(最小のイメージ)
%----------------------------------------------------------------------------------------
\begin{frame}{直感:線を少し動かすとSSEが増える/減る(“最小”のイメージ)}
\small

回帰直線は「当てずっぽう」で決めているのではなく、\\
\textbf{SSE(ズレの二乗の合計)を最小にする}というルールで決まります。

\vspace{0.4em}
\begin{columns}
  %-------------------------------
  % 左:散布図+線を動かすイメージ
  %-------------------------------
  \begin{column}{0.54\linewidth}
  \centering
  \begin{tikzpicture}[x=0.85cm,y=0.85cm]
    % axes
    \draw[->] (0,0) -- (4.4,0) node[below] {\scriptsize $x$};
    \draw[->] (0,0) -- (0,3.6) node[left] {\scriptsize $y$};

    % points
    \fill (1,1.6) circle (2pt);
    \fill (2,1.9) circle (2pt);
    \fill (3,2.9) circle (2pt);
    \fill (3.6,2.2) circle (2pt);

    % three candidate lines (same slope, shifted)
    \draw[thick] (0.3,0.8) -- (4.2,3.3);   % middle (good)
    \draw[dashed] (0.3,1.2) -- (4.2,3.7); % shifted up
    \draw[dashed] (0.3,0.4) -- (4.2,2.9); % shifted down

    \node at (3.95,2.85) {\scriptsize 良さげ};
    \node at (3.95,3.65) {\scriptsize 上にずらす};
    \node at (3.95,1.85) {\scriptsize 下にずらす};
  \end{tikzpicture}

  \vspace{0.2em}
  \scriptsize
  直線を上下に動かすだけでも、\\
  点とのズレが変わる
  \end{column}

  %-------------------------------
  % 右:SSEが最小になる「谷」のイメージ
  %-------------------------------
  \begin{column}{0.44\linewidth}
  \centering
  \begin{tikzpicture}[x=1.0cm,y=1.0cm]
    % axes
    \draw[->] (0,0) -- (4.2,0) node[below] {\scriptsize 線の位置(例:切片)};
    \draw[->] (0,0) -- (0,3.2) node[left] {\scriptsize SSE};

    % parabola (valley)
    \draw[thick, domain=0.4:3.8, samples=60]
      plot (\x, {0.55*(\x-2.2)*(\x-2.2)+0.6});

    % minimum point
    \fill (2.2,0.6) circle (2.4pt);
    \draw[dashed] (2.2,0) -- (2.2,0.6);
    \node[below] at (2.2,0) {\scriptsize 最小};

    \node[align=left] at (2.55,2.55) {\scriptsize SSEは\\\scriptsize “谷”になる};
  \end{tikzpicture}

  \vspace{0.2em}
  \scriptsize
  SSEがいちばん小さい場所が1つ決まる
  \end{column}
\end{columns}

\noteT{E-12:最小の直感を作る(運用)}{
【目的】「最小二乗法=SSE最小」を“谷のイメージ”で直感的に理解させる。\\
【口頭補足】右図は「SSEが最小になる場所がある」というイメージ図(厳密な導出ではない)。\\
【実習仕様】後でExcelで線(トレンドライン)を出す=SSE最小の線が自動で選ばれる、と接続する。\\
【運用】ここで確認:「回帰直線はどう決まる?」→「SSEが最小の線」。}
\end{frame}

%----------------------------------------------------------------------------------------
% E-13: ミニ例(3点)SSE比較
%----------------------------------------------------------------------------------------
\begin{frame}{ミニ例(3点):線Aと線BでSSEを比べる(計算は最小限)}
\small

3つの点があるとします(散布図の一部だと思ってOK):

\vspace{0.2em}
\begin{center}
\begin{tabular}{c|ccc}
\hline
点 & $(x,y)$ & 線Aの$\hat{y}$ & 線Bの$\hat{y}$\\
\hline
1 & $(1,2)$ & $2$ & $1$ \\
2 & $(2,2)$ & $3$ & $2$ \\
3 & $(3,4)$ & $4$ & $3$ \\
\hline
\end{tabular}
\end{center}

\vspace{-1.0em}
ここで2本の候補:
\vspace{-0.5em}
\begin{itemize}
  \item 線A:$\hat{y}=x+1$
  \item 線B:$\hat{y}=x$
\end{itemize}

\vspace{-0.4em}
\textbf{残差} $e=y-\hat{y}$ と \textbf{SSE}(残差の二乗の合計)を比べます。

\vspace{-0.2em}
\begin{center}
\begin{tabular}{c|cc|cc}
\hline
点 & $e_A$ & $e_A^2$ & $e_B$ & $e_B^2$\\
\hline
1 & $2-2=0$   & $0$ & $2-1=1$   & $1$ \\
2 & $2-3=-1$  & $1$ & $2-2=0$   & $0$ \\
3 & $4-4=0$   & $0$ & $4-3=1$   & $1$ \\
\hline
合計(SSE) & \multicolumn{2}{c|}{\textbf{1}} & \multicolumn{2}{c}{\textbf{2}} \\
\hline
\end{tabular}
\end{center}

\vspace{-0.2em}
\emjpin \textbf{SSEが小さい線Aの方が「良い線」}(この例ではAが勝ち)

\noteT{E-13:SSE比較を“手で一回だけ”やる理由(運用)}{
【目的】SSEが「点のズレの二乗の合計」で、比較に使えることを具体例で固定する。\\
【口頭補足】ここでは導出はしない。「SSEで比べると、良い線が決まる」を体感させる。\\
【実習仕様】Excelで同じことを大量の点で自動的にやっている(手作業はこの例だけ)。\\
【運用】計算は最小限:残差→二乗→合計、の流れだけを覚えさせる。}
\end{frame}

%----------------------------------------------------------------------------------------
% E-14: Excelとつながる(トレンドライン=SSE最小)
%----------------------------------------------------------------------------------------
\begin{frame}{Excelとつながる:トレンドラインの式=SSE最小の結果}
\small

Excelで散布図に「近似曲線(トレンドライン)」を追加すると、\\
直線の式 $\hat{y}=ax+b$ が表示できます。

\vspace{-0.6em}
\begin{tcolorbox}[title=ここがポイント]
Excelのトレンドラインは、裏側で \textbf{SSEが最小になる}ように\\
$a$(傾き)と $b$(切片)を自動で決めている。
\end{tcolorbox}

\vspace{-0.1em}
\textbf{つまり:}
\vspace{-0.2em}
\begin{itemize}
  \item 私たちがやってきた「SSEで良い線を決める」ルールを
  \item Excelが\textbf{自動で実行}して、式を出している
\end{itemize}

\vspace{-0.5em}
\textbf{実習でやる操作(次の塊につながる)}
\begin{itemize}
  \item 散布図を作る(x=バーガー数、y=ポテト数)
  \item 近似曲線(線形)を追加し、「\textbf{数式を表示}」「\textbf{$R^2$を表示}」
  \item その式で $\hat{y}$ を作り、残差とSSEを作る
\end{itemize}

\noteT{E-14:Excelのブラックボックスを壊す(運用)}{
【目的】「トレンドラインの式=勝手に出てくる数字」ではなく「SSE最小の結果」と結びつける。\\
【口頭補足】Excelは“魔法”ではない。残差の二乗を全部足して最小になる線を選んでいるだけ。\\
【実習仕様】散布図→トレンドライン→式/Rの二乗表示→$\hat{y}$列→残差列→残差の二乗列→SSE(SUM)。\\
【運用】次の実習へ自然につなぐ:「じゃあExcelで同じことをやってみよう」。}
\end{frame}

%----------------------------------------------------------------------------------------
% F-15: 実習① 全体手順(SSEを自分で作る)
%----------------------------------------------------------------------------------------
\begin{frame}{実習①(clean):SSEを自分で作って「良い線」を理解する(全体手順)}
\small

Excelのトレンドラインは「SSEが最小の直線」を自動で出しています。\\
実習①では、その中身を\textbf{自分の手で}再現します。

\vspace{0.6em}
\begin{tcolorbox}[title=実習①のゴール(ここだけ)]
\textbf{予測値 $\hat{y}$ → 残差 → 残差$^2$ → 合計(SSE)} を作り、\\
\textbf{SSEが小さいほど良い線}だと確かめる
\end{tcolorbox}

\vspace{0.5em}
\begin{enumerate}
  \item 散布図から回帰式を表示し、傾き $a$ と切片 $b$ を取り出す
  \item 各行で $\hat{y}=ax+b$ を計算して「予測値」列を作る
  \item 残差 $e=y-\hat{y}$ 列を作る(縦のズレ)
  \item 残差$^2$ 列を作る(マイナスを消す+大ズレを強く罰する)
  \item 残差$^2$ を合計して \textbf{SSE} を出す(SUM)
\end{enumerate}

\vspace{0.3em}
\emjpin \textbf{最後に $a$ や $b$ を少し変えて、SSEが増えることを確認する}

\noteT{F-15:実習①の位置づけ(運用)}{
【目的】「SSE最小」がExcelの裏側のルールだと体験で納得させる。\\
【口頭補足】ここでは“導出”しない。SSEを作って比較するだけで十分。\\
【実習仕様】データはclean版を使用(外れ値なしで成功体験を作る)。\\
【運用】このスライドを見せたら、すぐExcel操作へ移る(説明疲れ対策)。}
\end{frame}

%----------------------------------------------------------------------------------------
% F-16: 実習①-1(clean)予測値 y-hat 列を作る
%----------------------------------------------------------------------------------------
\begin{frame}{実習①-1(clean):$\hat{y}$(予測値)列を作る(回帰式を使う)}
\small

まず、散布図にトレンドライン(線形)を追加し、\\
\textbf{数式 $\hat{y}=ax+b$ を表示}して $a$ と $b$ をメモします。

\vspace{0.6em}
\begin{itemize}
  \item $x$:バーガー注文数(説明変数)
  \item $y$:ポテト注文数(目的変数)
  \item $\hat{y}$:回帰式が出す「予測値」
\end{itemize}

\vspace{0.5em}
\begin{tcolorbox}[title=Excelで作る列(例)]
\textbf{(例)} $a$ と $b$ をセルに置く:\\
\quad $a$ を \texttt{H1}、$b$ を \texttt{H2} に入力\\
\textbf{予測値列}(例:\texttt{D2})\\
\quad \texttt{=\$H\$1*A2 + \$H\$2}
\end{tcolorbox}

\vspace{0.3em}
\emjpin \textbf{ポイント:} \$ を付けて $a,b$ のセルを固定すると、下までコピーできる

\noteT{F-16:操作の落とし穴(運用)}{
【目的】$\hat{y}$列(予測値)を全員が確実に作れるようにする。\\
【口頭補足】$\hat{y}$は「点の中心を通る線が出す予測」。実測$y$とは一致しない。\\
【実習仕様】(1)散布図作成→(2)線形近似→(3)数式表示→(4)$a,b$をセルへ→(5)$\hat{y}$列作成。\\
【運用】$a,b$を式に直書きさせるとミスが増えるので、セル参照(\$固定)を推奨。}
\end{frame}

%----------------------------------------------------------------------------------------
% F-17: 実習①-2 残差・残差^2・SSE
%----------------------------------------------------------------------------------------
\begin{frame}{実習①-2(clean):残差・残差$^2$を作り、SSEを計算する}
\small

\textbf{残差}は「実測」と「予測」のズレです。\\
回帰直線がどれだけ点に近いかを、残差で数にします。

\vspace{0.5em}
\begin{tcolorbox}[title=作る列(定義)]
残差:\quad $e=y-\hat{y}$(縦のズレ)\\
残差$^2$:\quad $e^2$(マイナスを消して足せるようにする)\\
SSE:\quad $\sum e^2$(残差$^2$の合計)
\end{tcolorbox}

\vspace{0.4em}
\textbf{Excel例(列が違ってもOK)}
\begin{itemize}
  \item 残差列(例:\texttt{E2}):\quad \texttt{=B2 - D2} \hfill($y-\hat{y}$)
  \item 残差$^2$列(例:\texttt{F2}):\quad \texttt{=E2$^2$} \ \textbf{または}\ \texttt{=POWER(E2,2)}
  \item SSE(例:\texttt{F1}):\quad \texttt{=SUM(F2:F31)}
\end{itemize}

\vspace{0.3em}
\emjpin \textbf{SSEが小さいほど}、点に近い(=良い線)

\noteT{F-17:ここで固定したい言葉(運用)}{
【目的】残差→二乗→合計(SSE)の意味を“列”として定着させる。\\
【口頭補足】残差の合計は相殺するのでダメ。二乗してから足す。\\
【実習仕様】全員がSSEを1つのセルで出せたら次へ。数値が合わない学生は残差列を確認させる。\\
【運用】$^(上付き)$はエラーになりやすいので、必要ならPOWERを推奨。}
\end{frame}

%----------------------------------------------------------------------------------------
% F-18: 実習①-3 a,bを少し変えてSSE増加を確認
%----------------------------------------------------------------------------------------
\begin{frame}{実習①-3(clean):$a$や$b$を少し変えて、SSEが増えることを確認する}
\small

ここが「体験で納得」パートです。\\
今の回帰式は、Excelが\textbf{SSEが最小}になるように選んだ直線でした。

\vspace{0.6em}
\begin{tcolorbox}[title=やること(超シンプル)]
$a$ または $b$ を\textbf{少しだけ}動かして、SSEがどう変わるかを見る\\
(例)$a$を +0.1 / $b$を +1 など
\end{tcolorbox}

\vspace{0.4em}
\textbf{確認のしかた}
\begin{itemize}
  \item $a,b$ を変える \ \ $\Rightarrow$ \ \ $\hat{y}$列が変わる
  \item $\hat{y}$が変わる \ \ $\Rightarrow$ \ \ 残差・残差$^2$が変わる
  \item その結果、\textbf{SSEが増える}(多くの場合)
\end{itemize}

\vspace{0.4em}
\emjpin \textbf{観察の結論:} 元の$a,b$が「SSEを最小にする」直線だった

\noteT{F-18:実習①の締め(運用)}{
【目的】「最小」の意味を“数値の増減”で納得させる(説明ではなく体験)。\\
【口頭補足】少し動かすだけでSSEが増える=“谷の底”にいた、ということ。\\
【実習仕様】(1)元のSSEをメモ→(2)$a$を少し変える→SSE再確認→(3)$b$も少し変える。\\
【運用】変え幅が大きいと混乱するので「少しだけ」を強調。早い学生には「どちら方向に動かすと増え方が大きい?」を追加。}
\end{frame}

%----------------------------------------------------------------------------------------
% G-19: 実習②(outlier)二乗は外れ値に弱い=線が引っ張られる
%----------------------------------------------------------------------------------------
\begin{frame}{実習②(outlier):二乗は外れ値に弱い(線が引っ張られる)}
\small

実習①(clean)と同じ手順を、\textbf{outlierデータ}でもう一度やります。\\
ここで見たいのは、\textbf{二乗($e^2$)が外れ値に強く反応する}という性質です。

\vspace{-0.6em}
\begin{tcolorbox}[title=今日の観察ポイント(ここだけ)]
外れ値があると、\textbf{1点の大きなズレ}が $e^2$ で\textbf{何倍にも大きく}なり、\\
\textbf{SSEを強く支配する} → 回帰直線が\textbf{外れ値に引っ張られる}
\end{tcolorbox}

\vspace{0.2em}
\textbf{やること(実習①と同型)}
\vspace{-0.4em}
\begin{enumerate}
  \item outlier版で散布図 → トレンドライン → 式($a,b$)を表示
  \item $\hat{y}$列 → 残差 $e$列 → 残差$^2$列 → SSE を作る
  \item clean版と比べて、\textbf{どこが大きく変わったか}を見る
\end{enumerate}

\vspace{-0.4em}
\textbf{見るポイント(比較)}
\vspace{-0.6em}
\begin{itemize}
  \item $a$(傾き)や $b$(切片)はどれくらい変わった?
  \item $R^2$ は上がった?下がった?
  \item 残差$^2$列で、\textbf{異常に大きい行}はどれ?
\end{itemize}


\noteT{G-19:実習②の運用(疲れさせない)}{
【目的】「二乗=外れ値に弱い」を体験で理解させ、現実データの扱いへつなぐ。\\
【口頭補足】外れ値1点の$e^2$が大きすぎて、他の点を押しのけてSSEを支配することがある。\\
【実習仕様】(1)outlierで実習①と同じ列作成→(2)残差$^2$が最大の行を特定→(3)メモ列で理由づけ。\\
【運用】時間が厳しければ「SSE最大の行を探す」だけでもOK(全列を作らせなくてもよい)。}
\end{frame}

%----------------------------------------------------------------------------------------
% H-20: まとめ+次回への橋
%----------------------------------------------------------------------------------------
\begin{frame}{まとめ:回帰直線は「SSE最小」で決まる/今後は“読み取りと応用”へ}
\small

今日のゴールは、回帰直線の裏側にあるルールを、\\
\textbf{式ではなく“列(手順)”で理解すること}でした。

\vspace{0.6em}
\begin{itemize}
  \item \textbf{残差}:$e=y-\hat{y}$(実測と予測のズレ)
  \item \textbf{SSE}:$\sum e^2$(ズレの二乗の合計)
  \item \textbf{回帰直線}:\textbf{SSEが最小}になる直線(最小二乗法)
  \item \textbf{二乗の性質}:大きいズレを強く重視 → 外れ値に引っ張られやすい
\end{itemize}

\vspace{0.6em}
\begin{tcolorbox}[title=今後への橋]
今後は、回帰の結果を\textbf{読む・使う}練習を増やします。\\
\textbf{(例)} 予測値の解釈/残差からの気づき/外れ値の扱い(データの意味づけ)
\end{tcolorbox}

\vspace{0.3em}
\emjpin \textbf{同じ姿勢:} データ → 要約(線) → ズレ(残差) → 解釈(理由)

\noteT{H-20:締めと今後接続(運用)}{
【目的】今日のキーワードを“3点セット”で固定し、次回の「読み取り・応用」に自然につなぐ。\\
【口頭補足】回帰は「線を引いて終わり」ではなく、残差と外れ値から“何が起きたか”を考える道具。\\
【実習仕様】提出なし。最後に全員で「外れ値はなぜ危ない?」を一言で言わせて終了。\\
【運用】次回の冒頭で、clean/outlierの違いを1枚で再確認すると接続が良い。}
\end{frame}