% @@@--(metropolis)--@@@
% ===== 第10回:導入・前回回収(Slide 01-03)=====
% ※「アルゴリズム2 Beamer共通仕様 v2025」準拠(frame群のみ)
% ※teacherframeは使わない(noteTのみ)

%----------------------------------------------------------------------------------------
% Slide 01: 今日のテーマ
%----------------------------------------------------------------------------------------
\begin{frame}{平均の検定(t検定)で「判断」を完成させる}
\small

\begin{block}{今日やること(結論)}
ポテトのデータを使って、\\
\textbf{「平均が基準(135g)と違うと言ってよいか?」}を判断します。
\end{block}

\vspace{0.6em}
\begin{itemize}
  \item 前回作った「判断の型」を、\textbf{平均の検定}として完成させる
  \item 今日も「判断の型」を、\textbf{平均の判断}として完成させる
  \item ゴールは計算ではなく、\textbf{結論文で言える}こと
\end{itemize}

\vspace{0.6em}
\begin{block}{今日の到達目標}
\begin{itemize}
  \item 「位置(ずれ÷ゆれ)」を言葉で説明できる
  \item \textbf{p値と有意水準}を比べて、\textbf{棄却/棄却できない}で判断できる
  \item 結論を\textbf{文章(テンプレ)}で書ける
\end{itemize}
\end{block}

\noteT{Slide01:第10回の立ち位置(運用)}{
【目的】第10回の「判断の型」を前提に、今日のゴールを先に固定する。\\
【口頭補足】分布は増やさない(正規とtだけ)。今日は平均の判断に集中。\\
【実習仕様】このあと「210件の全体→30件標本」でt検定し、結論文まで書かせる。\\
【運用】最初に学生へ一言確認:「判断の型は何?」→「H0→位置→p→有意水準→判断」。}
\end{frame}

%----------------------------------------------------------------------------------------
% Slide 02: ポテトの問い(再提示)
%----------------------------------------------------------------------------------------
\begin{frame}{導入:今日は少ない?たまたま?(ポテトの例)}
\small

ハンバーガーショップのMサイズポテトは、平均\textbf{135g}と公表されています。\\
しかし、毎回まったく同じ重さではなく、\textbf{ばらつき}があります。

\vspace{0.3em}
\begin{block}{今日の問い(平均の判断)}
今日のデータを見ると、平均が135gより\textbf{小さそう}です。\\
これは\textbf{たまたまのばらつき}でしょうか?\\
それとも「\textbf{本当に減った}」と言ってよいでしょうか?
\end{block}

\vspace{-0.5em}
\begin{itemize}
  \item 見た目の印象ではなく、\textbf{データで判断}したい
  \item ただし「平均との差がある」だけでは判断できない(前回)
\end{itemize}

\vspace{0.1em}
\begin{center}
  \includegraphics[scale=0.18]{ポテト比較.png}
\end{center} 

% 図案(必要なら差し替え)
% ・ポテト箱イラスト(通常/今日)を並べ、今日が少ない見た目にする

\noteT{Slide02:導入(ポテト例を継続する理由)}{
【目的】第1回からの例を継続し、認知負荷を上げずに「平均の判断」に入る。\\
【口頭補足】例を変えないのは重要:新しい例にすると「状況理解」からやり直しになる。\\
【実習仕様】210件(30人×7日)のデータを背景として扱い、検定は30件標本で行う。\\
【運用】学生に直感で答えさせる(挙手):たまたま派/減った派 → 次スライドで「判断の型」に回収。}
\end{frame}

%----------------------------------------------------------------------------------------
% Slide 03: 前回回収(判断の型)
%----------------------------------------------------------------------------------------
\begin{frame}{前回の振り返り:判断の型はこれだけ}
\small

前回作った「判断の型」は、今日も\textbf{そのまま}使います。\\
(今日は「位置」を表す数として t を使う。理由はこのあと説明)

\vspace{0.6em}
\begin{block}{判断の型(必ずこの順番)}
\begin{enumerate}
  \item \textbf{いつも通り}を仮定する(帰無仮説 $H_0$)
  \item 結果が分布の上で\textbf{どの位置}かを数値にする(位置)
  \item その位置が\textbf{どれくらい珍しいか}を確率にする(p値)
  \item p値と\textbf{有意水準}を比べて判断する(棄却/棄却できない)
  \item 最後に\textbf{結論文}を書く
\end{enumerate}
\end{block}


\begin{itemize}
  \item p値は「$H_0$が正しい確率」ではなく、\textbf{「$H_0$のもとでの珍しさ」}
  \item 「棄却できない」は、\textbf{「$H_0$を疑う根拠が足りない」}という意味
\end{itemize}

\noteT{Slide03:回収の狙い(今日の追加点を1つに絞る)}{
【目的】第10回の骨格を固定し、今日の新要素を「位置がtになる」に限定する。\\
【口頭補足】今日の中心は「なぜt分布?」:母分散(σ)がわからないときの考え方。\\
【実習仕様】この型のまま、Excelで t→p値→判断→結論文 を再現させる。\\
【運用】ここでミニ確認:p値は何?→「H0のもとでの珍しさ」。判断は?→「p値と有意水準を比べる」。}
\end{frame}


%----------------------------------------------------------------------------------------
% Slide 04: 検定の本質(ずれ ÷ 平均のゆれ)
%----------------------------------------------------------------------------------------
\begin{frame}{検定の本質:珍しさは「ずれ $\div$ 平均のゆれ」}
\small

平均が基準($\mu_0$)と違って見えても、すぐに「本当に違う」とは言えません。\\
\textbf{判断に必要なのは、ずれの大きさだけではなく「どれくらい珍しいか」です。}

\vspace{-0.5em}
\begin{block}{考え方(ここが検定の核)}
\textbf{珍しさ}は、次の比で決まります。\\
\[
\frac{\text{ずれ(平均との差)}}{\text{平均のゆれ(標本平均のばらつき)}}
\]
\end{block}

\vspace{-0.5em}
\begin{columns}
\begin{column}{0.62\linewidth}
\begin{itemize}
  \item \textbf{ずれ}:$\bar{x}-\mu_0$(標本平均と基準平均との差)
  \item \textbf{平均のゆれ}:$\bar{x}$が毎回どれくらい揺れるか(標準誤差)
  \item この比が大きいほど、分布の\textbf{端}に寄る(珍しい)
\end{itemize}

\vspace{0.3em}
\emjpin \textbf{検定は「平均との差」ではなく、}\\
\textbf{「ずれが、ゆれの何倍か」で判断する}
\end{column}

\begin{column}{0.36\linewidth}
\centering
\begin{tikzpicture}[x=1.0cm,y=1.0cm]
  % 軸
  \draw[->] (-2.3,0) -- (2.3,0);
  \node[below right] at (2.25,0) {\scriptsize 位置};

  % ベルカーブ(簡易)
  \draw[smooth, thick, domain=-2.1:2.1, samples=60]
    plot (\x, {1.5*exp(-(\x*\x))});

  % 中心
  \draw[dashed] (0,0) -- (0,1.5);
  \node[below] at (0,0) {\scriptsize 中心};

  % 端(珍しい位置)
  \draw[thick, ->] (1.7,0.15) -- (1.7,0);
  \node[below] at (1.7,0) {\scriptsize 端};

  \node[align=center] at (0,1.8) {\scriptsize 端ほど\\\scriptsize 珍しい};
\end{tikzpicture}

\vspace{0.4em}
\scriptsize
\begin{itemize}
  \item この「位置」を数値にしたものが\\
        \textbf{Z値 / t値}
\end{itemize}
\end{column}
\end{columns}

\noteT{Slide04:ここで固定したい1文(運用)}{
【目的】「検定=平均との差を比べる」誤解を止める。検定の核は「ずれ÷ゆれ」。\\
【口頭補足】同じ2gの差でも、ゆれが大きい店なら珍しくない/ゆれが小さい店なら珍しい。\\
【実習仕様】後で t=(xbar-mu0)/(s/sqrt(n)) を作るが、式暗記ではなく「ずれ÷ゆれ」だと理解させる。\\
【運用】短い問い:『判断に必要なのは差だけ?』→『差とゆれ(何倍か)』。}
\end{frame}

%----------------------------------------------------------------------------------------
% Slide 05: 母分散とは(店全体のばらつき)
%----------------------------------------------------------------------------------------
\begin{frame}{母分散($\sigma^2$)とは:店全体の「ばらつきの大きさ」}
\small

ポテトの重さは毎回同じではありません。\\
この「\textbf{どれくらいバラバラか}」を表すのが\textbf{分散}(ばらつきの大きさ)です。

\vspace{0.6em}
\begin{block}{母分散($\sigma^2$)・母標準偏差($\sigma$)}
\begin{itemize}
  \item \textbf{母集団}:この店のMサイズポテト全体(本当は無限に近い)
  \item \textbf{母分散 $\sigma^2$}:店全体のばらつきの大きさ
  \item \textbf{母標準偏差 $\sigma$}:その平方根(単位がgになるので直感的)
\end{itemize}
\end{block}

\vspace{-0.5em}
\begin{itemize}
  \item $\sigma$ が大きい店:毎回の重さがバラバラ(ゆれが大きい)
  \item $\sigma$ が小さい店:毎回ほぼ同じ(ゆれが小さい)
\end{itemize}

\vspace{-0.4em}
\emjpin \textbf{検定で重要なのは「平均との差」だけでなく、}\\
\textbf{その店の“ゆれ”がどれくらいか($\sigma$)}

% 図提案(必要なら)
% ・同じ中心(135g)で、幅の狭い分布(σ小)と幅の広い分布(σ大)を並べると直感が強い

\noteT{Slide05:留学生がつまずく点(説明の順序)}{
【目的】「母分散=母集団のばらつき」を言葉で固定する。式はまだ出さない。\\
【口頭補足】“母”は「本当の店全体」。私たちが集めるのは一部(標本)。\\
【実習仕様】今日は σ が分からない(未知)世界に入る。その前に「σとは何か」を先に定義する。\\
【運用】例え:『同じ2gの差でも、普段のゆれが大きい店なら珍しくない』をここで言っておく。}
\end{frame}

%----------------------------------------------------------------------------------------
% Slide 06: 母分散既知(σが分かると定規が固定)
%----------------------------------------------------------------------------------------
\begin{frame}{母分散既知の世界:$\sigma$ が分かると「定規」が固定できる}
\small

もし店が長年の大量データを持っていて、\\
「この店のばらつきは $\sigma=\text{一定}$」と\textbf{分かっている}とします。\\
(これが \textbf{母分散既知} の世界です)

\vspace{-0.3em}
\begin{block}{ここが重要:平均のゆれ(標準誤差)が固定できる}
標本数を $n$ とすると、標本平均 $\bar{x}$ のゆれは
\[
\frac{\sigma}{\sqrt{n}}
\]
で決まります。
\end{block}

\vspace{-0.5em}
\begin{itemize}
  \item $\sigma$ が分かる $\Rightarrow$ 分母(定規)が\textbf{固定}される
  \item だから「ずれ $\div$ ゆれ」で作る\textbf{位置}は安定する
  \item その位置を表すのが \textbf{Z値}(正規分布の上の位置)
\end{itemize}

\vspace{-0.4em}
\emjpin \textbf{母分散既知なら:定規が固定 → Z → 正規分布で珍しさ(p値)を読む}

% 図提案(必要なら)
% ・「固定の定規」アイコン(定規の絵)+「σ/√n」ラベル
% ・Z値が正規分布上の位置になることを、前回の図(ベルカーブ)と接続

\noteT{Slide06:ここでの狙い(次スライドへの橋渡し)}{
【目的】「既知=定規が固定」を直感として固定し、次の「未知=定規が揺れる→t」へ自然につなぐ。\\
【口頭補足】現実はたいてい σ は分からない。だから次に「未知」の世界に移る。\\
【実習仕様】前回の式の分母(σ/√n)が固定できない場合、どうするかが今日の核心。\\
【運用】このスライドでは「z検定の厳密さ」を掘らない。目的は“固定の定規”の理解だけ。}
\end{frame}

%----------------------------------------------------------------------------------------
% Slide 07: z値の再掲(前回とのつながり)
%----------------------------------------------------------------------------------------
\begin{frame}{z値の再掲:$Z$ は正規分布の上の「位置」}
\small

前回は、帰無仮説 $H_0:\mu=\mu_0$ のもとで、\\
標本平均 $\bar{x}$ がどれくらい珍しい位置にあるかを、\textbf{Z値}で表しました。

\vspace{0.6em}
\begin{block}{Z値(位置)の形}
\[
Z=\frac{\bar{x}-\mu_{0}}{\sigma/\sqrt{n}}
\]
\end{block}

\vspace{0.4em}
\begin{itemize}
  \item 分子 $\bar{x}-\mu_0$:\textbf{ずれ}(平均との差)
  \item 分母 $\sigma/\sqrt{n}$:\textbf{平均のゆれ}(標本平均の標準誤差)
  \item つまり $Z$ は、\textbf{ずれが「ゆれ」の何倍か}(位置)
\end{itemize}

\vspace{0.5em}
\begin{columns}
\begin{column}{0.55\linewidth}
\begin{itemize}
  \item $Z=0$:中心(珍しくない)
  \item $|Z|$ が大きい:端に寄る(珍しい)
\end{itemize}
\end{column}
\begin{column}{0.43\linewidth}
\centering
\begin{tikzpicture}[x=1.0cm,y=1.0cm]
  \draw[->] (-2.3,0) -- (2.3,0);
  \node[below right] at (2.25,0) {\scriptsize $Z$};
  \draw[smooth, thick, domain=-2.1:2.1, samples=60]
    plot (\x, {1.5*exp(-(\x*\x))});
  \draw[dashed] (0,0) -- (0,1.5);
  \node[below] at (0,0) {\scriptsize 0};

  \draw[thick, ->] (-1.6,0.15) -- (-1.6,0);
  \node[below] at (-1.6,0) {\scriptsize $Z$};
  \node[align=center] at (0,1.75) {\scriptsize 分布の上の\\\scriptsize 位置};
\end{tikzpicture}
\end{column}
\end{columns}

\vspace{0.2em}
\emjpin \textbf{この「位置」→ p値(珍しさ)→ 判断}の流れは、今日も同じ

\noteT{Slide07:再掲の狙い(最小限で回収)}{
【目的】前回のZを「位置(ずれ÷ゆれ)」として回収し、今日のtへつなぐ。\\
【口頭補足】式を覚えるのが目的ではない。「ずれ÷ゆれ=位置」だけ固定すればよい。\\
【実習仕様】今日は分母に $\sigma$ が使えない場合を扱う(次スライドへ)。\\
【運用】問い:Zは何?→「正規分布上の位置」。p値は?→「H0のもとでの珍しさ」。}
\end{frame}

%----------------------------------------------------------------------------------------
% Slide 08: 現実(σは普通わからない=母分散未知)
%----------------------------------------------------------------------------------------
\begin{frame}{現実:$\sigma$ は普通わからない(母分散未知)}
\small

前回のZ値の式には、母標準偏差 $\sigma$ が出てきました。\\
しかし現実には、$\sigma$ は\textbf{最初から分かっていることはほとんどありません}。

\vspace{-0.6em}
\begin{block}{母分散未知とは}
\textbf{母集団(店全体)のばらつき $\sigma$ が分からない}状態
\end{block}

\vspace{-0.5em}
ポテトの例では:
\begin{itemize}
  \item 店が「この店のばらつきは $\sigma=\cdots$」と公表しているわけではない
  \item 私たちが持っているのは、\textbf{集めたデータ(標本)}だけ
\end{itemize}

\vspace{-0.5em}
\begin{block}{ここで問題になる点}
$Z=\dfrac{\bar{x}-\mu_0}{\sigma/\sqrt{n}}$ の分母にある $\sigma$ が分からないと、\\
\textbf{平均のゆれ(標準誤差)=定規が決まらない}
\end{block}

\vspace{0.2em}
\emjpin \textbf{定規が決まらないなら、標本から定規を作る必要がある}

\noteT{Slide08:学生がつまずく言葉の扱い}{
【目的】「母分散未知=σが不明」を、式より先に言葉で固定する。\\
【口頭補足】“未知”は「難しい」ではなく「普通の状態」。現実のデータ分析はほぼ未知から始まる。\\
【実習仕様】このあと標本標準偏差 s を使って「定規(標準誤差)」を作る。\\
【運用】短く言う:『σが分からない=Zの分母が作れない』→ 次へ。\\
\\
学生がつまずく言葉の扱い(なぜσは分からない?)\\
【目的】「母分散未知=σが不明」を、式より先に“現実の理由”で納得させる。\\
【口頭補足(なぜ分からない?)】\\
・σは「店全体(母集団)のばらつき」。本当は店のポテトを\textbf{無限回}量って初めて分かる“真の値”。\\
・私たちが手にできるのは\textbf{一部のデータ(標本)}だけ。店の全てを調べるのは時間・コスト的に不可能。\\
・日によって条件が変わる(油の温度、揚げ時間、担当者、混雑、計量の癖など)ため、\\
 「店のばらつき」を\textbf{一つの確定値として最初から知っている状況はほぼない}。\\
・だから現実は「σ不明」から始まり、標本から\textbf{sで見積もる}。これが“普通”の分析。\\
【実習仕様】このあと STDEV.S で s を作り、標準誤差 s/SQRT(n) を作る(σの代わりの定規)。\\
【運用】短い言い方でまとめる:『σは店全体の情報。私たちは一部しか見ていない。だから分からない』→ 次へ。


}
\end{frame}

%----------------------------------------------------------------------------------------
% Slide 09: 標本からばらつきを見積もる(sを使う)
%----------------------------------------------------------------------------------------
\begin{frame}{標本からばらつきを見積もる:$s$(標本標準偏差)を使う}
\small

$\sigma$ が分からないなら、集めたデータ(標本)から、\\
ばらつきの大きさを\textbf{見積もる}必要があります。そこで使うのが $s$ です。

\vspace{0.6em}
\begin{block}{標本標準偏差 $s$}
\textbf{標本データから計算した「ばらつきの大きさ」}\\
($\sigma$ の代わりとして使う見積もり値)
\end{block}

\vspace{-0.5em}
\begin{itemize}
  \item $\sigma$:店全体の本当のばらつき(普通は分からない)
  \item $s$:標本から作るばらつき(\textbf{見積もり}なので標本ごとに変わる)
\end{itemize}

\vspace{0.6em}
\begin{block}{平均のゆれ(標準誤差)も、標本から作る}
\[
\text{標準誤差}=\frac{s}{\sqrt{n}}
\]
\end{block}

\vspace{0.2em}
\emjpin \textbf{ここがポイント:定規(分母)も標本ごとに少し揺れる}

% 図提案(必要なら)
% ・同じ中心の分布を想定し、標本ごとにsが少し違う → 定規が少し伸び縮みするイメージ図

\noteT{Slide09:次につなぐ(t分布への必然)}{
【目的】σ不明→sで代用、までを「言葉で」理解させる。\\
【口頭補足】sは見積もりなので、標本を取り直すと少し変わる(定規が揺れる)。\\
【実習仕様】Excelでは STDEV.S で s を作り、標準誤差 s/SQRT(n) を計算させる。\\
【運用】次スライドで結論:定規が揺れる世界では、位置の分布が正規ではなく t になる。}
\end{frame}

%----------------------------------------------------------------------------------------
% Slide 10: 不偏標準偏差(STDEV.S と df=n-1 の意味:最小限)
%----------------------------------------------------------------------------------------
\begin{frame}{不偏標準偏差:\texttt{STDEV.S} と $df=n-1$(最小限)}
\small

母集団のばらつき $\sigma$ は分からないので、標本から $s$ を作って代用します。\\
Excelでは、\textbf{\texttt{STDEV.S}} を使います。

\vspace{-0.6em}
\begin{block}{Excelで使う(これでOK)}
\begin{itemize}
  \item 標本標準偏差(不偏):\texttt{STDEV.S(range)}
  \item 自由度:$df=n-1$
\end{itemize}
\end{block}

\begin{block}{今日はここだけ覚える(ルール)}
\begin{itemize}
  \item ばらつきは \texttt{STDEV.S} で計算する(標本からの見積もり)
  \item t分布の自由度は $df=n-1$ を使う
\end{itemize}
\end{block}

\vspace{-0.4em}
\begin{itemize}
  \item $n$ 個のデータがあっても、平均を決めた時点で最後の1個は決まる
  \item その分だけ、ばらつきを少し\textbf{小さく見積もりやすい}ので補正する
\end{itemize}

\vspace{-0.3em}
\emjpin \textbf{今日の目的は計算の暗記ではない:}\\
\textbf{「標本から作るばらつきは $df=n-1$ を使う」}だけ押さえる

\noteT{Slide10:不偏の扱い(深入りしない運用)}{
【目的】「STDEV.Sを使う理由」を一言で納得させ、計算練習にしない。\\
【口頭補足】平均との差の合計が0になる=自由が1つ減る、を短く説明して終える。証明は不要。\\
【実習仕様】Excelは STDEV.S を使わせる(STDEV.Pは使わない)。df は n-1 をセルで明示。\\
【運用】確認質問:dfは?→「n-1」。なぜ?→「平均を使うと自由が1つ減る」。\\
「平均との差を使うと自由が1つ減るので n−1(詳しくは今は扱わない)」}
\end{frame}

%----------------------------------------------------------------------------------------
% Slide 11: 定規が揺れる世界 → t分布(裾が厚い/nが増えると近づく)
%----------------------------------------------------------------------------------------
\begin{frame}{定規が揺れる世界 $\Rightarrow$ t分布(裾が厚い/$n$が増えると近づく)}
\small

$\sigma$ が分からないので $s$ を使うと、分母(定規)も標本ごとに少し変わります。\\
すると「位置」の分布は、正規分布ではなく \textbf{t分布}になります。

\vspace{0.6em}
\begin{block}{結論(ここが「なぜt?」の答え)}
\textbf{$\sigma$ の代わりに $s$ を使う}(定規が揺れる)\\
$\Rightarrow$ \textbf{位置は t分布に従う}
\end{block}

\vspace{0.6em}
\begin{columns}
\begin{column}{0.58\linewidth}
\begin{itemize}
  \item t分布は正規分布より \textbf{裾が厚い}\\
        (\textbf{極端な値が少し起こりやすい}形)
  \item 理由:定規(分母)が揺れるので、位置のブレも大きくなる
  \item ただし $n$ が増えると $s$ は安定し、t分布は \textbf{正規分布に近づく}
\end{itemize}

\vspace{0.4em}
\emjpin \textbf{少ない標本ほど t を使う(安全側)}
\end{column}

\begin{column}{0.40\linewidth}
\centering
% 画像推奨:正規分布とt分布(df小/df大)を重ねた図
% 例:t_vs_normal.png
\includegraphics[width=\linewidth]{t_vs_normal.png}

\vspace{0.3em}
\scriptsize
例:t(df=3) は裾が厚い/dfが大きくなると正規に近づく 
\end{column}
\end{columns}

\noteT{Slide11:図の使い方(ここで納得を作る)}{
【目的】「なぜt分布?」を一枚で納得させる(留学生は図で理解が進む)。\\
【口頭補足】“裾が厚い”=極端が少し出やすい。理由は「σ不明でsを使う=定規が揺れる」から。\\
【実習仕様】この後の計算は t=(xbar-mu0)/(s/sqrt(n))、p値はT.DIST.2Tで出す。\\
【運用】このスライドで一言まとめ:『σが分からないからtを使う。標本が増えると正規に近づく』。}
\end{frame}

%----------------------------------------------------------------------------------------
% Slide 12: t値(位置):言葉で理解
%----------------------------------------------------------------------------------------
%\begin{frame}{t値:$t=\dfrac{\bar{x}-\mu_0}{s/\sqrt{n}}$ は「分布の上の位置」}
\begin{frame}{t値は「分布の上の位置」}
\small

母分散が未知($\sigma$が分からない)なので、標本から作った $s$ を使います。\\
そのときの「位置」を表す数が \textbf{t値}です。

%\vspace{0.6em}
\begin{block}{t値(位置)の形}
\[
t=\frac{\bar{x}-\mu_0}{s/\sqrt{n}}
\]
\end{block}

\vspace{-0.4em}
\begin{itemize}
  \item 分子 $\bar{x}-\mu_0$:\textbf{ずれ}(平均との差)
  \item 分母 $s/\sqrt{n}$:\textbf{平均のゆれ}(標準誤差:標本から作る定規)
  \item つまり $t$ は、\textbf{ずれが「ゆれ」の何倍か}(位置)
\end{itemize}

\vspace{-1.5em}
\begin{columns}
\begin{column}{0.58\linewidth}
\begin{block}{Zとの違い(ここだけ)}
\begin{itemize}
  \item $Z$:分母に $\sigma$(既知)を使う
  \item $t$:分母に $s$(標本の見積もり)を使う
\end{itemize}
\end{block}
\end{column}
\begin{column}{0.40\linewidth}
\centering
\begin{tikzpicture}[x=1.0cm,y=1.0cm]
  \draw[->] (-2.3,0) -- (2.3,0);
  \node[below right] at (2.25,0) {\scriptsize 位置};
  \draw[smooth, thick, domain=-2.1:2.1, samples=60]
    plot (\x, {1.5*exp(-(\x*\x))});
  \draw[dashed] (0,0) -- (0,1.5);
  \node[below] at (0,0) {\scriptsize 0};

  \draw[thick, ->] (1.6,0.15) -- (1.6,0);
  \node[below] at (1.6,0) {\scriptsize $t$};
  \node[align=center] at (0,1.75) {\scriptsize 分布の上の\\\scriptsize 位置};
\end{tikzpicture}
\end{column}
\end{columns}

\vspace{0.2em}
\emjpin \textbf{t値は「計算の暗記」ではなく、}
\textbf{“ずれ÷ゆれ”で作った位置情報}

\noteT{Slide12:ここで固定したい言い方}{
【目的】tの式を“位置(ずれ÷ゆれ)”として言葉で理解させる。\\
【口頭補足】tはZの兄弟。違いは分母がσかsかだけ。\\
【実習仕様】Excelで t=(xbar-mu0)/(s/SQRT(n)) を作らせる(セル参照で)。\\
【運用】ミニ確認:分子は?→「ずれ」。分母は?→「平均のゆれ」。}
\end{frame}

%----------------------------------------------------------------------------------------
% Slide 13: p値の求め方(t分布の外側確率:両側)
%----------------------------------------------------------------------------------------
\begin{frame}{p値の求め方:t分布の「外側の確率」(両側)}
\small

前回と同じく、p値は「$H_0$のもとでの珍しさ」です。\\
ただし今回は \textbf{t分布} を使って珍しさを確率にします。

\vspace{0.6em}
\begin{block}{p値(両側)の意味(文章)}
\textbf{$H_0$が正しいと仮定したとき、}\\
\textbf{$|t|$ 以上に極端な結果が出る確率}
\end{block}

\vspace{0.5em}
\begin{columns}
\begin{column}{0.58\linewidth}
\begin{itemize}
  \item まず $t$ を計算する(位置)
  \item その位置より外側(左右)の面積が p値
  \item 両側検定では左右両方なので \textbf{「外側×2」}になる
\end{itemize}

\vspace{0.4em}
\begin{block}{Excel(例:両側)}
\texttt{=T.DIST.2T(ABS(t), df)}
\end{block}

\scriptsize
※df は $n-1$(前スライドまでの約束)
\end{column}

\begin{column}{0.40\linewidth}
\centering
% 画像提案:t分布に「外側(左右)」を薄く塗った図(両側)
\includegraphics[width=\linewidth]{t_pvalue_twoside.png}

% \begin{tikzpicture}[x=1.0cm,y=1.0cm]
%   \draw[->] (-2.3,0) -- (2.3,0);
%   \node[below right] at (2.25,0) {\scriptsize $t$};

%   % ざっくり山(模式)
%   \draw[smooth, thick, domain=-2.1:2.1, samples=60]
%     plot (\x, {1.45*exp(-(\x*\x)/1.2)});

%   \draw[dashed] (0,0) -- (0,1.45);
%   \node[below] at (0,0) {\scriptsize 0};

%   % t位置
%   \draw[thick, ->] (1.5,0.15) -- (1.5,0);
%   \node[below] at (1.5,0) {\scriptsize $t$};

%   % 外側(模式:左右に薄い領域)
%   \fill[gray, opacity=0.25] (1.5,0) -- (2.2,0) -- (2.2,0.10) -- cycle;
%   \fill[gray, opacity=0.25] (-1.5,0) -- (-2.2,0) -- (-2.2,0.10) -- cycle;

%   \node[align=center] at (0,1.75) {\scriptsize 外側の面積\\\scriptsize $\Rightarrow p$値};
% \end{tikzpicture}

\vspace{-0.3em}
\scriptsize
両側:左右の外側を合計
\end{column}
\end{columns}

\vspace{0.2em}
\emjpin \textbf{前回と同型:位置 $\rightarrow$ 外側確率(p値)}

\noteT{Slide13:前回と同じだと気づかせる}{
【目的】p値の意味を「H0のもとでの外側確率」として固定し、分布がtに変わっただけだと理解させる。\\
【口頭補足】Zのときは正規分布の外側、今回はt分布の外側。型は同じ。\\
【実習仕様】Excelで T.DIST.2T を使い、必ず ABS(t) と df=n-1 を入れさせる。\\
【運用】確認:p値は何?→「H0のもとで |t|以上の外側確率」。}
\end{frame}

%----------------------------------------------------------------------------------------
% Slide 14: 判断のルール再確認(p値と有意水準)
%----------------------------------------------------------------------------------------
\begin{frame}{判断のルール(再確認):p値と有意水準を比べる}
\small

判断は前回と同じで、\textbf{p値}と\textbf{有意水準}を比べるだけです。\\
(分布が正規→tに変わっても、判断ルールは同じ)

\vspace{0.6em}
\begin{block}{判断}
\begin{itemize}
  \item \textbf{p値 $<$ 有意水準}:$H_0$ を\textbf{棄却}する(いつも通りは怪しい)
  \item \textbf{p値 $\ge$ 有意水準}:$H_0$ を\textbf{棄却できない}(怪しいと言えない)
\end{itemize}
\end{block}

\vspace{0.4em}
\begin{itemize}
  \item p値は\textbf{データから出る}(珍しさ)
  \item 有意水準は\textbf{先に決める約束}(線引き)
\end{itemize}

\vspace{0.3em}
\emjpin \textbf{「棄却できない」=「正しいと確定」ではない}

\noteT{Slide14:言葉の統一(結論を崩さない)}{
【目的】結論表現を「棄却/棄却できない」に統一し、正しい/間違いと言わせない。\\
【口頭補足】判断ルールは分布が変わっても不変。変わるのは「p値の出し方」だけ。\\
【実習仕様】有意水準0.05を固定し、p値との大小だけで結論を出させる。\\
【運用】ミニ問題:p=0.12なら?→「棄却できない」。p=0.01なら?→「棄却」。}
\end{frame}

%----------------------------------------------------------------------------------------
% Slide 15: 結論文テンプレ(前回の型を継続:この回用に微調整)
%----------------------------------------------------------------------------------------
\begin{frame}{結論文テンプレ(前回の型を継続:t検定版)}
\small

計算ができても、最後は\textbf{文章で判断}します。\\
前回と同じ型で書けばOKです(丸暗記でよい)。

\vspace{0.6em}
\begin{block}{結論文テンプレ(t検定)}
有意水準を \underline{\hspace{2.0em}} とすると、p値は \underline{\hspace{2.5em}} であり、\\
p値は有意水準より \underline{\hspace{2.5em}}(小さい/大きい)ので、\\
帰無仮説($\mu=\mu_0$)は \underline{\hspace{4.0em}}(棄却する/棄却できない)。
\end{block}

\vspace{0.5em}
\begin{itemize}
  \item 「正しい/間違い」とは書かない
  \item 結論の意味は「\textbf{減ったと言ってよいか?}」の形で言い換えてもよい
\end{itemize}

\noteT{Slide15:提出物の形(授業運用)}{
【目的】結論の書き方を固定し、実習の提出物を揃える。\\
【口頭補足】結論文は「比較(pと0.05)」が中心。言い換えは最後に一言足す程度でよい。\\
【実習仕様】提出物:(1)H0の文章 (2)t値 (3)df (4)p値 (5)有意水準 (6)結論文。\\
【運用】早く終わった学生には「結論を日常語で言い換え」を追加:『減ったと言える/言えない』。}
\end{frame}


%----------------------------------------------------------------------------------------
% Slide 16: 210件データの位置づけ(母集団っぽい全体のばらつきを見る)
%----------------------------------------------------------------------------------------
\begin{frame}{210件データの位置づけ:「母集団っぽい全体」を先に見る}
\small

今回のポテト調査は、\textbf{30人×7日=210件}のデータがあります。\\
これは「店の全て」ではありませんが、授業では \textbf{母集団に近い全体データ}として扱います。

\vspace{0.6em}
\begin{block}{この210件でやりたいこと(推測)}
\begin{itemize}
  \item 平均はだいたい \textbf{何g}か(中心)
  \item ばらつきは \textbf{どれくらい}か(散らばり)
  \item 外れ値や偏りはないか(分布の形)
\end{itemize}
\end{block}

\vspace{0.5em}
\begin{itemize}
  \item \textbf{ここは「判断」ではなく「状況把握(推測)」}の段階
  \item この全体像が、次の「標本で判断する」実習の土台になる
\end{itemize}

\vspace{0.5em}
% 図提案:210件のヒストグラム+平均線(135gの線)を1枚で見せる
% \includegraphics[width=0.85\linewidth]{potato_210_hist.png}

\emjpin \textbf{まず「ばらつきの大きさ」を目で見て納得してから、検定に進む}

\noteT{Slide16:210件を置く理由(推測→判断の接続準備)}{
【目的】いきなり検定に入らず、データの全体像(中心・ばらつき)を先に確認させる。\\
【口頭補足】210件は本当の母集団ではないが、授業では「母集団に近い全体」として使う。\\
【実習仕様】このスライド後にExcelでヒストグラム(210件)と平均・標準偏差を出させてもよい(短時間)。\\
【運用】質問:ばらつきは大きい?小さい?→ 感覚を言わせる。次スライドで「標本で判断」へ。}
\end{frame}

%----------------------------------------------------------------------------------------
% Slide 17: 標本30件で判断する(ランダム抽出→検定へ)
%----------------------------------------------------------------------------------------
\begin{frame}{標本30件で判断する:ランダム抽出 $\rightarrow$ 検定へ}
\small

現実の統計では、いつも全体データを持てるとは限りません。\\
だから \textbf{標本(サンプル)}を取り、そこから判断します。

\vspace{0.6em}
\begin{block}{今回の実習の流れ(推測→判断)}
\begin{enumerate}
  \item 210件全体で「ばらつきの感じ」をつかむ(推測)
  \item そこから \textbf{30件をランダム抽出}して標本を作る
  \item 標本の平均 $\bar{x}$ が 135g と違うと言ってよいかを \textbf{t検定}で判断する
\end{enumerate}
\end{block}

\vspace{0.5em}
\begin{itemize}
  \item ランダム抽出:\textbf{偏りを減らすための約束}
  \item 標本では $\sigma$ が分からないので、\textbf{$s$ を使う}(ここまでの話)
  \item そして \textbf{t値→p値→有意水準}で判断する(第9回と同じ型)
\end{itemize}

\vspace{0.3em}
\emjpin \textbf{今日は「標本から判断する」までを、Excelで再現する}

\noteT{Slide17:実習への切り替え(学生を疲れさせない)}{
【目的】ここから「説明」→「手を動かす」へ切り替える合図にする。\\
【口頭補足】抽出がランダムでないと結論が偏る(例:混雑日だけ等)。ただし深入りしない。\\
【実習仕様】抽出方法:RAND関数で乱数列→並べ替え→上から30件、など(手順は次スライドで)。\\
【運用】ここで「これから実習に入る」と宣言し、作業のゴール(結論文まで)を再提示。}
\end{frame}

%----------------------------------------------------------------------------------------
% Slide 18: 実習①:1標本t検定(30件)全体手順(5ステップ)
%----------------------------------------------------------------------------------------
\begin{frame}{実習①:1標本t検定(30件)全体手順(5ステップ)}
\small
\textbf{目的:} 30件の標本から、平均が135gと違うと言ってよいかを判断する。\\
(ゴールは \textbf{結論文まで})

\vspace{0.6em}
\begin{block}{5ステップ(この順番で作業する)}
\begin{enumerate}
  \item \textbf{標本30件}を作る(ランダム抽出)
  \item 標本から \textbf{$\bar{x}$(平均), $s$(標準偏差), $n$(件数)}を計算する
  \item \textbf{t値}を作る:$t=\dfrac{\bar{x}-\mu_0}{s/\sqrt{n}}$($\mu_0=135$)
  \item \textbf{p値}を出す(両側):\texttt{T.DIST.2T(ABS(t), df)}($df=n-1$)
  \item \textbf{有意水準}と比べて \textbf{結論文}を書く(棄却/棄却できない)
\end{enumerate}
\end{block}

\vspace{0.4em}
\begin{itemize}
  \item 迷ったら「判断の型」に戻る:$H_0 \rightarrow$ 位置 $\rightarrow p \rightarrow$ 比較 $\rightarrow$ 結論文
\end{itemize}

\noteT{Slide18:実習仕様(Excelでの具体)}{
【データ】210件データからランダムに30件を抽出(標本)して別表に並べる。\\
【基準】$\mu_0=135$(公表値)。両側検定(「違うか?」)で行う。\\
【Excel関数の例】\\
・平均:\texttt{=AVERAGE(range)}\\
・標本標準偏差:\texttt{=STDEV.S(range)}\\
・件数:\texttt{=COUNT(range)}\\
・標準誤差:\texttt{=s/SQRT(n)}\\
・t値:\texttt{=(xbar-mu0)/(s/SQRT(n))}\\
・自由度:\texttt{=n-1}\\
・p値(両側):\texttt{=T.DIST.2T(ABS(t),df)}\\
【提出物】(1)$H_0$の文章 (2)$\bar{x}$ (3)$s$ (4)$n$ (5)$t$ (6)$df$ (7)p値 (8)有意水準 (9)結論文。\\
【運用】先に「表の完成形」を見せてから作業させると迷子が減る。}
\end{frame}

%----------------------------------------------------------------------------------------
% Slide 19: 実習②(発展):対応ありt検定の考え方(差を見る)
%----------------------------------------------------------------------------------------
\begin{frame}{実習②(発展):対応ありt検定の考え方(差を見る)}
\small
\textbf{時間があれば}、同じ「判断の型」を別の形で体験します。\\
対応あり(ペア)では、\textbf{2つを別々に比べるのではなく「差」を見る}のがポイントです。

\vspace{0.6em}
\begin{block}{例(対応ありの状況)}
同じ人のポテトを「\textbf{月曜}」と「\textbf{金曜}」で比べる。\\
(人による食べ方・取り分けのクセを打ち消したい)
\end{block}

\vspace{0.5em}
\begin{block}{考え方(ここだけ)}
\begin{enumerate}
  \item 各人について \textbf{差} $d = (\text{金曜})-(\text{月曜})$ を作る
  \item 「差の平均」が 0 と違うかを \textbf{1標本t検定}で判断する
\end{enumerate}
\end{block}

\vspace{0.4em}
\begin{itemize}
  \item \textbf{見るのは差}:元の2列ではなく、差の1列
  \item あとは同じ:$\bar{d}$, $s_d$, $n$, $t$, p値, 有意水準, 結論文
\end{itemize}

\noteT{Slide19:扱い方(時間次第で軽く)}{
【目的】対応あり=「差を作ってから1標本t検定」という最短理解を作る。\\
【口頭補足】対応ありは「同じ人を2回測る」など、ペアが自然に作れるときに使う。\\
【実習仕様(最小)】2列データ → 差の列を作る → 差の列に対して Slide18の手順を適用。\\
【運用】時間が無ければ“説明のみ”でOK。分布は増やさず t の中で完結することを強調。}
\end{frame}

%----------------------------------------------------------------------------------------
% Slide 20: まとめ+次回への橋(同じ型で進める/分布は正規+tで十分)
%----------------------------------------------------------------------------------------
\begin{frame}{まとめ+次回への橋:検定は「同じ型」で進める}
\small
今日のゴールは、t検定を「計算」としてではなく、\textbf{判断の型}として完成させることでした。

\vspace{0.6em}
\begin{block}{今日できるようになったこと}
\begin{itemize}
  \item $\sigma$ が分からない(母分散未知)ときは、$s$ を使って \textbf{t値(位置)}を作る
  \item p値は \textbf{t分布の外側確率}(両側)で出す
  \item \textbf{p値と有意水準}を比べて、棄却/棄却できないで判断する
  \item 最後に \textbf{結論文}で説明できる
\end{itemize}
\end{block}

\vspace{0.5em}
\begin{block}{次回への橋}
検定は、\textbf{分布が変わっても型は同じ}:\\
$H_0 \rightarrow$ 位置 $\rightarrow p \rightarrow$ 比較 $\rightarrow$ 結論文
\end{block}

\vspace{0.3em}
\emjpin \textbf{この授業では分布は「正規+t」で十分。増やしすぎない。}

\noteT{Slide20:締め方(学生の頭を整理する)}{
【目的】情報を“型”に戻して整理し、分布の増加で混乱させない。\\
【口頭補足】Zとtは「位置」の道具。判断のルールは同じ。\\
【実習仕様】提出物を回収し、何人かに結論文を読ませて全体で確認する。\\
【運用】最後に一言でまとめさせる:『なぜt?』→『σが分からないから』。}
\end{frame}
