%----------------------------------------------------------------------------------------
%  metropolis template (refactored)
%----------------------------------------------------------------------------------------
\documentclass[handout,aspectratio=169]{beamer}

% \documentclass の直後で hyperref のオプションを渡す(metropolisでも安全)
\PassOptionsToPackage{unicode=true,colorlinks=true,linkcolor=blue,urlcolor=blue}{hyperref}

\usetheme{metropolis}
\metroset{block=fill, sectionpage=progressbar, progressbar=foot}

% 背景色(tech のとき白など): Pythonで差し込み
\setbeamercolor{background canvas}{bg=white}

%--------------------------
% 日本語
%--------------------------
\usepackage{luatexja}
\usepackage{luatexja-fontspec}
\usepackage{luatexja-ruby}
\setsansjfont{Hiragino Sans}[BoldFont={Hiragino Sans W6}]

%--------------------------
% 基本パッケージ(重複なし)
%--------------------------
\usepackage[table]{xcolor}
\usepackage{graphicx}
\usepackage[abs]{overpic}
\usepackage{tikz}
\usetikzlibrary{positioning, shapes.geometric} % ← shapes.geometric を追加
\usepackage{array}
\usepackage{tabularx}
\usepackage{booktabs}
\usepackage{makecell}
\usepackage{mathtools}
\usepackage{longtable}
\usepackage{pdfpages}
\usepackage{etoolbox} % AtBeginEnvironment 等
\usepackage[normalem]{ulem}
\usetikzlibrary{calc}
\usetikzlibrary{backgrounds}

\usepackage{pgf}

% マーカー(ハイライター)の定義
\newcommand{\markline}[2][yellow]{%
  \tikz[baseline=(X.base)]{%
    \node[inner sep=0pt,outer sep=0pt] (X) {#2};
    \begin{scope}[on background layer]
      \fill[#1, opacity=0.35, rounded corners=0.8pt]
        ([xshift=-0.15em,yshift=0.00ex]X.south west) rectangle
        ([xshift= 0.15em,yshift=2.15ex]X.south east);
    \end{scope}
  }%
}



% minted(※ -shell-escape 必須)
\usepackage{minted}
\setminted{
  frame=single,
  framesep=2mm,
  fontsize=\footnotesize,
  breaklines=true
}

% (必要なときだけ)tcolorbox
\usepackage[most]{tcolorbox}
\tcbuselibrary{skins}        % 高度なデザイン機能
\tcbuselibrary{raster}

% hyperref は最後
\usepackage{hyperref}

% --- 1. カラーパレットの定義 ---
\definecolor{CanvaGreen}{HTML}{2E7D32} % メインの濃い緑(タイトル・枠線)
\definecolor{PaleGreen}{HTML}{F1F8E9}  % 背景の薄い緑
\definecolor{MyDarkGreen}{HTML}{587a7f}
\definecolor{DeepText}{HTML}{1C1C1C}   % 本文の文字色
\definecolor{MyWhiteBlue}{HTML}{F2FAFB} % 追加:あなたが指定した色
\definecolor{BananaColor}{HTML}{FFFD78}


% --- 2. tcolorbox のデフォルト設定(全ボックスに適用) ---
%\tcbset{
%    enhanced,                          % 高度な装飾を有効化
%    colback=white,                 % 本文背景色
%    colframe=MyDarkGreen,               % 枠線の色
%    coltitle=white,                    % タイトル文字色
%    fonttitle=\bfseries\sffamily,      % タイトルを太字・ゴシック
%    boxrule=1pt,                       % 枠線の太さ
%    arc=2mm,                           % 角の丸み
%    left=3mm, right=3mm,               % 左右の余白
%    top=0.5mm, bottom=0.5mm,               % 本文の上下余白
%    toptitle=0.8mm, bottomtitle=0.5mm, % タイトル内の上下余白
%    before skip=0.8em, after skip=0.2em,   % ボックス前後の行間
%    shadow={0mm}{0mm}{0mm}{black!0}    % 影を完全に消してフラットに
%}
% --- 2. tcolorbox のデフォルト設定 ---
\tcbset{
    enhanced,
    colback=MyWhiteBlue,            % 本文背景を F2FAFB に変更
    colframe=MyDarkGreen,
    coltitle=white,                 % タイトル文字は白(背景が濃い色の場合)
    % coltitle=DeepText,            % もしタイトル背景も薄くするならこちら
    fonttitle=\bfseries\sffamily\small, % タイトルを少し小さくしてスリム化
    boxrule=0.5pt,                  % 1pt から 0.5pt へ細分化
    arc=1mm,                        % 2mm から 1mm へ変更
    sharp corners=south,            % 下側を直角に固定
    % --- 余白の調整 ---
    left=3mm, right=3mm,
    top=0.5mm, bottom=0.5mm,
    toptitle=0mm,                   % タイトル上の余白をゼロに
    bottomtitle=0mm,                % タイトル下の余白をゼロに
    % ------------------
    before skip=0.8em, after skip=0.2em,
    shadow={0mm}{0mm}{0mm}{black!0}
}

% --- カスタムボックスの定義 ---
\newtcolorbox{myListbox}[1]{
  enhanced,
  detach title,              % 1. 標準のタイトル位置を解除
  % 2. 本文が始まる直前(before upper)でタイトルを直接描画する
  before upper={{\bfseries\large #1}\par\medskip},
  % タイトルの後に改行を入れる設定
  after title={\par\medskip},
  colbacktitle=white, 
  colframe=gray!50,
  colback=white,
  titlerule=0pt,
  boxrule=1pt,
  fonttitle=\bfseries\color{black},
  title=#1,
  after skip=1.5ex,
  % --- 箇条書きの余白を強制的にゼロにする設定 ---
  before upper={
    \setbeamertemplate{itemize ispan}{0pt} % 項目間の余白
    \setbeamertemplate{itemize items}[default]
    \setlength{\leftmargini}{1.5em}
    % Beamerの内部変数を直接操作して行間を詰める
    \addtobeamertemplate{itemize/enumerate body begin}{}{\setlength{\itemsep}{0pt}\setlength{\parskip}{0pt}}
  }
}

%--------------------------
% パス
%--------------------------
\newcommand{\assetpath}{/Volumes/NBPlan/TTC/授業資料/2025年度/}
\graphicspath{{images/}{\assetpath/1020201.アルゴリズム2/03/images/}{../project_assets/images/}{../project_assets/emoji/emoji_pngs/}}
% PGFがある特定の階層を定義
\newcommand{\pgfpath}{\assetpath/1020201.アルゴリズム2/03/images/}

%--------------------------
% フッター
%--------------------------
\newcommand{\myfootertext}{1020201.アルゴリズム2/03}
\setbeamertemplate{footline}{%
  \leavevmode
  \hbox to \paperwidth{%
    \hspace*{0.2cm}
    \scriptsize\color{gray!50} \myfootertext
    \hfill
    \scriptsize\color{gray} \insertframenumber{} / \inserttotalframenumber
    \hspace*{0.4cm}
  }%
  \vspace{1pt}
}

%--------------------------
% TeacherFrame(外部)
%--------------------------
\usepackage{../teacherframe}

%--------------------------
% フレームタイトル:番号. タイトル
% ※ ここで出すだけ。insertframetitle を再定義しない(安全)
%--------------------------
\setbeamertemplate{frametitle}{%
  \vspace{0.6ex}%
  \begin{beamercolorbox}[wd=\paperwidth,sep=0.5ex,leftskip=0.9em,rightskip=0.5em]{frametitle}%
    \usebeamerfont{frametitle}%
    \insertframenumber.\,\insertframetitle%
  \end{beamercolorbox}%
}

%--------------------------
% 表用:列型
%--------------------------
\newcolumntype{C}[1]{>{\centering\arraybackslash}p{#1}}
\newcolumntype{M}[1]{>{\raggedright\arraybackslash}m{#1}}

%--------------------------
% ブロック(必要なら)
%--------------------------
\definecolor{myblue}{HTML}{7488FF}
\definecolor{mylightblue}{HTML}{E3EEFF}
\setbeamertemplate{blocks}[rounded]
\setbeamercolor{block title}{bg=myblue, fg=white}
\setbeamercolor{block body}{bg=mylightblue, fg=black}

%========================================================
% exampleblock(examplebox相当)だけの調整
%  - タイトル文字:白
%  - 背景色:現行のまま(bgは指定しない)
%  - 本文 itemize:文字も●も黒(exampleblock内だけ)
%========================================================

% タイトル文字だけ白(背景は触らない)
\setbeamercolor{block title example}{fg=white}

% 本文の通常文字色は「現行のまま」を基本にする(必要なら黒にしてもよい)
% ここは bg を触らないのが目的なので fg だけ調整可能
\setbeamercolor{block body example}{fg=black}

% exampleblock の中だけ itemize の色(●と文字)を黒に
\AtBeginEnvironment{exampleblock}{%
  \setbeamercolor{itemize item}{fg=black}
  \setbeamercolor{itemize subitem}{fg=black}
  \setbeamercolor{itemize subsubitem}{fg=black}
  \setbeamercolor{item}{fg=black} % 念のため
}

% exampleblock を抜けたらテーマ標準に戻す(色が残る事故防止)
\AtEndEnvironment{exampleblock}{%
  \setbeamercolor{itemize item}{fg=normal text.fg}
  \setbeamercolor{itemize subitem}{fg=normal text.fg}
  \setbeamercolor{itemize subsubitem}{fg=normal text.fg}
  \setbeamercolor{item}{fg=normal text.fg}
}

%--------------------------
% 奇数ページのスライドのを表示する
%(教示用だけでそれ以外はこの処理は動かない)
%--------------------------
% --- 教師用だけ、スライドを奇数開始に強制するトグル ---
\newif\ifoddslideenforce
\oddslideenforcefalse   % デフォルトOFF(pr/hoはOFF)

% --- 再帰防止ガード ---
\newif\ifoddslideguard
\oddslideguardfalse

% --- 偶数ページなら空白スライドを1枚入れて奇数に戻す ---
\newcommand{\ensureoddslide}{%
  \ifoddslideguard\relax\else
    \oddslideguardtrue
    \ifodd\value{page}\relax
      % 何もしない(次が奇数)
    \else
      \begin{frame}[plain,noframenumbering]
        \note{}% notes出力時に2枚消費させる保険
      \end{frame}
    \fi
    \oddslideguardfalse
  \fi
}

% --- frameが始まる直前に自動挿入(教師用だけ)---
\BeforeBeginEnvironment{frame}{%
  \ifoddslideenforce
    \ensureoddslide
  \fi
}

%--------------------------
% note / noteT の「常時安全化」
%  - tech 以外:\noteT は無視(エラーにならない)
%  - tech:notesmode_tech で上書き定義
%--------------------------
\providecommand{\notetitletext}{}      % 既にあっても衝突しない
\providecommand{\noteT}[2]{}           % デフォルトは何もしない

% frame開始ごとにタイトル変数をクリア(前の noteT が残らないように)
\AtBeginEnvironment{frame}{\gdef\notetitletext{}}

%--------------------------
% 切替(Pythonから差し込み)
%--------------------------
\mypausemodetrue
\teachermodetrue
\setbeameroption{show notes}
%-------------

% --- tech のときだけ noteT を有効化(テンプレートの \providecommand を上書き) ---
\makeatletter
\renewcommand{\noteT}[2]{%
 \gdef\notetitletext{#1}%
 \note{#2}%
}
% タイトル未指定のときのために初期化
\renewcommand{\notetitletext}{}%

\setbeamertemplate{note page}{%
 \begin{minipage}{\linewidth}
 \vspace{1.2ex} % タイトルを少し下げる(必要に応じて調整)
 {\Large\bfseries
 \ifx\notetitletext\@empty
 \insertframetitle
 \else
 \notetitletext
 \fi
 }\par
 \vspace{-1.2ex}
 \rule{\linewidth}{0.8pt}\par
 \vspace{0.8ex}
 {\scriptsize \insertnote}
 \end{minipage}
}
\makeatother

%教師用のPDFは奇数ページからスライドを出力
\oddslideenforcetrue


%--------------------------
% 方眼紙(グリッド)をスライドに重ね
%--------------------------
\input{grid_debug}

%-------------------------------------------------------------------------
% 「ラインマーカー」そのものです。線の太さ・色・透明度を自由にできます。
%
% #1 色(省略可)
% #2 下端 yshift
% #3 上端 yshift
% #4 文字
% \marklineA{0.35ex}{1.55ex}{通常サイズ}
% {\Large \marklineA{0.45ex}{2.10ex}{大きい文字}}
%-------------------------------------------------------------------------
\newcommand{\marklineA}[4][yellow]{%
  \tikz[baseline=(X.base)]{%
    \node[inner sep=0pt,outer sep=0pt] (X) {#4};
    \begin{scope}[on background layer]
      \fill[#1, opacity=0.35, rounded corners=0.8pt]
        ([xshift=-0.15em,yshift=#2]X.south west) rectangle
        ([xshift= 0.15em,yshift=#3]X.south east);
    \end{scope}
  }%
}

%----------------------------------------------------------------------------------------
% タイトル
%----------------------------------------------------------------------------------------
\title{ 03 記述統計(集計・可視化・分布の把握) }
\date{}
\newcommand{\codedir}{\assetpath/1020201.アルゴリズム2/03}

\begin{document}

\begin{frame}[plain,noframenumbering]
  \titlepage
  \bigskip
  \begin{center}
    \ifteachermode 教師用 \fi
  \end{center}
\end{frame}

% セクションページ(必要なら)
\setbeamertemplate{section page}{
  \begin{centering}
    \vfill
    \rule{\linewidth}{2pt}\par
    \vspace{1ex}
    {\usebeamerfont{section title}\Huge\bfseries \insertsection}\par
    \vspace{1ex}
    \rule{\linewidth}{2pt}\par
    \vfill
  \end{centering}
}
\setbeamerfont{section title}{size=\LARGE,series=\bfseries}

\AtBeginSection[]{
  \begin{frame}[plain,noframenumbering]
    \sectionpage
  \end{frame}
}

% 本編開始でフレーム番号を0から(必要なら)
\setcounter{framenumber}{0}

\input{emoji_macros}

% @@@--(metropolis)--@@@

%@@PAGEBAND@@
% ----------------------------------------------------------------------------------------
%   page 01
% ----------------------------------------------------------------------------------------
\begin{frame}{第2回:記述統計 ― データの全体像をつかむ}
前回は、\textbf{1人で測ったポテトの重さ}を使って、
「同じ商品でも重さが毎回違う」ことを確認しました。

\vspace{0.6em}
今回は、
\begin{itemize}
  \item データの数を増やし
  \item 全体をまとめて見て
  \item その特徴を整理する
\end{itemize}
ことで、\textbf{データの全体像を読む}ことを目指します。

\vspace{0.6em}
このように、手元にあるデータを整理・要約する考え方を
\textbf{\ruby{記述統計}{きじゅつとうけい}}と呼びます。

\noteT{導入}{
・第1回との連続性をまず強調する。\\
・「今日は何をする回か」を先に明確にする。
}
\end{frame}
% ----------------------------------------------------------------------------------------
%   Slide 02: 前回の疑問
% ----------------------------------------------------------------------------------------

%@@PAGEBAND@@
% ----------------------------------------------------------------------------------------
%   page 02
% ----------------------------------------------------------------------------------------
\begin{frame}{前回の測定で残った疑問}
前回は、\textbf{「私」1人}がポテトの重さを測りました。

\vspace{0.6em}
その結果、
\begin{itemize}
  \item 平均はだいたい決まっていそう
  \item でも、軽い日・重い日がある
\end{itemize}
ことが分かりました。

\vspace{0.6em}
ここで、1つ疑問が残ります。

\vspace{0.4em}
\textbf{このばらつきは、}
\begin{itemize}
  \item たまたまなのか?
  \item 「私」だけの測定結果なのか?
\end{itemize}

\noteT{疑問提示}{
・「1人データの限界」を学生自身に気づかせる。\\
・次の行動(人数を増やす)への動機づけ。
}
\end{frame}

% ----------------------------------------------------------------------------------------
%   Slide 03: 調査チームを作る
% ----------------------------------------------------------------------------------------

%@@PAGEBAND@@
% ----------------------------------------------------------------------------------------
%   page 03
% ----------------------------------------------------------------------------------------
\begin{frame}{調査チームを作って確かめることにした}
そこで今回は、
\textbf{クラス30人全員}に声をかけて、
調査チームを作ることにしました。

\vspace{0.6em}
\textbf{調査のルール}
\begin{itemize}
  \item 同じ店のポテトを使う
  \item 1週間、それぞれが測定する
  \item 重さを記録し、気づいたことをメモする
\end{itemize}

\vspace{0.6em}
こうして集まったのが、
\textbf{30人分のポテト重量データ}です。

\noteT{ストーリー}{
・「調査チーム」という言葉で参加感を出す。\\
・ルールを簡潔に示し、データの前提条件を共有する。
}
\end{frame}

% ----------------------------------------------------------------------------------------
%   Slide 04: 今日やること
% ----------------------------------------------------------------------------------------

%@@PAGEBAND@@
% ----------------------------------------------------------------------------------------
%   page 04
% ----------------------------------------------------------------------------------------
\begin{frame}{今日やること:集まったデータをどう見るか}
今日は、この30人分のデータを使って、
次のことを順番に行います。

\vspace{0.6em}
\begin{enumerate}
  \item 数でまとめてみる(平均・最小・最大)
  \item グラフで形を見てみる
  \item 「なぜこうなったか」を考える
\end{enumerate}

\vspace{0.6em}
まずは、
\textbf{全体を1つの数で見たらどうなるか}を
確かめてみましょう。

\vspace{0.4em}
\centering
→ 演習①へ

\noteT{演習への接続}{
・分析手法を先に言わず、行動の流れだけを示す。\\
・このまま演習①に自然につなげる。
}
\end{frame}

% ----------------------------------------------------------------------------------------
%   Slide A: 演習① 全体の基本統計量(重さ_g)
% ----------------------------------------------------------------------------------------

%@@PAGEBAND@@
% ----------------------------------------------------------------------------------------
%   page 05
% ----------------------------------------------------------------------------------------
\begin{frame}{演習①:ポテト重量の基本統計量を求める}
クラス30人が1週間調査したポテト重量データ(\texttt{poteto30.xlsx})を用いて、
全体の数値的な特徴を確認します。

\vspace{0.5em}
\textbf{対象とする列}
\begin{itemize}
  \item \textbf{重さ\_g}(ポテトの重量)
\end{itemize}

\vspace{0.5em}
\textbf{作業内容}
\begin{enumerate}
  \item \texttt{AVERAGE} 関数で \textbf{平均値} を求める
  \item \texttt{MEDIAN} 関数で \textbf{中央値} を求める
  \item \texttt{MIN}, \texttt{MAX} 関数で \textbf{最小値・最大値} を求める
\end{enumerate}

\vspace{0.5em}
\textbf{確認するポイント}
\begin{itemize}
  \item 平均値と中央値は近いか、離れているか
  \item 最小値・最大値は、平均からどの程度離れているか
\end{itemize}

\noteT{演習①の狙い}{
・まずは「全体を1つの数で表す」体験をさせる。\\
・ここでは評価や解釈を行わず、数値を出すことに集中させる。\\
・次のスライドで「平均だけで十分か?」という問いにつなげる。
}
\end{frame}

% ----------------------------------------------------------------------------------------
%   Slide B: 【問い】平均値だけで判断できるか?
% ----------------------------------------------------------------------------------------

%@@PAGEBAND@@
% ----------------------------------------------------------------------------------------
%   page 06
% ----------------------------------------------------------------------------------------
\begin{frame}{【問い】平均値だけで、この店のポテトを評価できるか?}
演習①で、ポテト重量の \textbf{平均値}・\textbf{中央値}・\textbf{最小値}・\textbf{最大値} を求めました。

\vspace{0.8em}
ここで、次の問いを考えてみましょう。

\vspace{-0.5em}
\begin{tcolorbox}[title=問い]
平均値が \textbf{約133g} だったとします。\\
この数値だけを見て、
\begin{itemize}
  \item 「この店のポテトは、だいたい133gだ」
  \item 「品質として問題ない」
\end{itemize}
と判断してよいでしょうか?
\end{tcolorbox}

\textbf{ヒント:}
\vspace{-0.5em}
\begin{itemize}
  \item 最小値や最大値は、平均からどのくらい離れていましたか?
  \item 133g から大きく外れたデータは、いくつありましたか?
\end{itemize}

\noteT{問いの狙い}{
・学生が自然に「平均だけでは不安だ」と感じる状態を作る。\\
・まだ「分布」という言葉は使わず、違和感だけを持たせる。\\
・次の演習(ヒストグラム)への動機づけとする。
}
\end{frame}

% ----------------------------------------------------------------------------------------
%   Slide C: 演習② ヒストグラムで全体の形を見る
% ----------------------------------------------------------------------------------------

%@@PAGEBAND@@
% ----------------------------------------------------------------------------------------
%   page 07
% ----------------------------------------------------------------------------------------
\begin{frame}{演習②:ヒストグラムで全体の形を確かめる}
演習①で求めた数値(平均・最小・最大)をふまえ、
ポテト重量データ全体の「集まり方」をグラフで確認します。

\vspace{-0.5em}
\textbf{対象とする列}
\begin{itemize}
  \item \textbf{重さ\_g}(ポテトの重量)
\end{itemize}

\vspace{-0.5em}
\textbf{作業内容}
\begin{enumerate}
  \item 「重さ\_g」の列を選択する
  \item 「挿入」タブから \textbf{ヒストグラム} を作成する
\end{enumerate}

\vspace{-0.5em}
\textbf{観察するポイント}
\begin{itemize}
  \item データは、どのあたりに多く集まっているか
  \item 平均値(約133g)は、山の中央にあるか
  \item 端のほうに、極端に離れた値は見えるか
\end{itemize}

\noteT{演習②の狙い}{
・数値ではなく「形」で全体を見る体験をさせる。\\
・平均値を点としてではなく、分布の中に位置づけさせる。\\
・次のスライドで「分布」という言葉を導入する準備とする。\\
\\
エクセルのヒストグラムでグラフを選択して、右クリックでデータ系列の書式を選ぶ\\
\\
文の値と幅でヒストグラムの図を変える(2ぐらいの幅で110からすると...)

}
\end{frame}

% ----------------------------------------------------------------------------------------
%   Slide D: 分布とは何か
% ----------------------------------------------------------------------------------------

%@@PAGEBAND@@
% ----------------------------------------------------------------------------------------
%   page 08
% ----------------------------------------------------------------------------------------
\begin{frame}{分布(Distribution):データの集まり方を見る}
演習②で作成したヒストグラムは、
ポテト重量データの「\ruby{分布}{ぶんぷ}」を表しています。

\textbf{分布とは}
\vspace{-0.6em}
\begin{itemize}
  \item どの値が
  \item どのくらいの回数(\ruby{頻度}{ひんど})で
  \item どのように集まっているか
\end{itemize}
を表したものです。

\textbf{重要なポイント}
\vspace{-0.6em}
\begin{itemize}
  \item 分布を見ると、「平均のまわりにどう広がっているか」が分かる
  \item 同じ平均値でも、分布の形が違えば意味は大きく変わる
\end{itemize}

\textbf{ここまでで分かったこと}
\vspace{-0.6em}
\begin{itemize}
  \item ポテトの重さは、毎回まったく同じではない
  \item そのばらつきは、ヒストグラムの「形」として現れている
\end{itemize}

\noteT{分布の導入}{
・新しい数式や専門用語は出さず、直前のヒストグラム体験と結びつける。\\
・「分布=形を見ること」という感覚をまず定着させる。\\
・次に、分布を見るときの具体的な視点(中心・広がり)へつなげる。
}
\end{frame}

% ----------------------------------------------------------------------------------------
%   Slide E: 分布を見る視点① 中心
% ----------------------------------------------------------------------------------------

%@@PAGEBAND@@
% ----------------------------------------------------------------------------------------
%   page 09
% ----------------------------------------------------------------------------------------
\begin{frame}{分布を見る視点①:中心(平均・中央値)}
\ruby{分布}{ぶんぷ}を見るとき、まず注目するのが
データの「中心」です。

\textbf{中心を表す代表的な数値}
\vspace{-0.6em}
\begin{itemize}
  \item \textbf{平均値:} すべての値を均したときの中心
  \item \textbf{中央値:} 小さい順に並べたときの真ん中
\end{itemize}

\textbf{ヒストグラムと対応づけて考える}
\vspace{-0.6em}
\begin{itemize}
  \item 平均値(約133g)は、分布のどの位置にあるか
  \item 中央値は、山の中心付近にあるか
\end{itemize}

\textbf{重要な点}
\vspace{-0.6em}
\begin{itemize}
  \item 平均と中央値が近い場合、分布は比較的対称と考えられる
  \item 大きくずれている場合、分布に偏りがある可能性が高い
\end{itemize}

\noteT{中心を見る狙い}{
・演習①で求めた数値を、分布の中に「置き直す」意識を持たせる。\\
・平均と中央値は計算方法ではなく、分布との関係で理解させる。\\
・次に「なぜずれるのか?」という問いを自然に生む。
}
\end{frame}

% ----------------------------------------------------------------------------------------
%   Slide F: 分布を見る視点② 広がり(ばらつき)
% ----------------------------------------------------------------------------------------

%@@PAGEBAND@@
% ----------------------------------------------------------------------------------------
%   page 10
% ----------------------------------------------------------------------------------------
\begin{frame}{分布を見る視点②:広がり(ばらつき)}
分布を見るときは、中心だけでなく
データが「どのくらい散らばっているか」にも注目します。

\textbf{広がりを感じ取るための手がかり}
\vspace{-0.6em}
\begin{itemize}
  \item \textbf{最小値と最大値:}
        データがどこからどこまで広がっているか
  \item \textbf{中心からの距離:}
        平均(約133g)から大きく離れた値があるか
\end{itemize}

\textbf{ヒストグラムで確認する}
\vspace{-0.6em}
\begin{itemize}
  \item 棒が横にどれくらい広がっているか
  \item 中心付近に集中しているか、ばらけているか
\end{itemize}

\textbf{重要な点}
\vspace{-0.6em}
\begin{itemize}
  \item 広がりが小さいほど、重さは安定していると言える
  \item 広がりが大きい場合、同じ平均でも「ばらつきの大きい店」となる
\end{itemize}

\noteT{広がりを見る狙い}{
・数式や専門用語を使わず、まずは視覚的・感覚的にばらつきを捉えさせる。\\
・「安定している/していない」という評価が、分布の広がりから生まれることを示す。\\
・次に、広がりの中で特に目立つ値=外れ値へ話題をつなげる。
}
\end{frame}

%@@PAGEBAND@@
% ----------------------------------------------------------------------------------------
%   page 11
% ----------------------------------------------------------------------------------------
\begin{frame}{平均と分散:グラフで見るデータの「姿」}
    \begin{columns} % [t] で画像とテキストの上端を揃える
        \begin{column}{0.45\textwidth}
            \vspace{0pt} % 画像のベースライン調整
            % 高さをスライドの7割に制限し、はみ出しを防止
%            \includegraphics[width=\textwidth, height=0.7\textheight, keepaspectratio]{分散グラフ比較.png}
            \includegraphics[scale=0.4]{分散グラフ比較.png}
        \end{column}
        
        \begin{column}{0.55\textwidth}
            \vspace{0pt} % テキストのベースライン調整
            \footnotesize
%            \begin{itemize}[leftmargin=6.5em, labelsep=1ex, itemsep=3ex]
            \begin{description}
                \item[\textbf{平均の限界:}] 
                    平均値と中央値だけを見ると、2つのクラスはそっくりに見えます。 \par
                    しかし、グラフの形(山の高さ)は全く違います。

                \item[\textbf{分散の姿:}] 
                    「分散」は山の「太り具合」を表します。 \par
                    \textbf{分散大:} 山が低く横に広がり、点数がバラバラな状態です。 \par
                    \textbf{分散小:} 山が細く高く、平均付近に集中しています。

                \item[\textbf{結  論:}] 
                    データの「正確な姿」を知るためには、平均だけでなく \par
                    \textbf{「分散」という物差し}を確認することが不可欠です。

            \end{description}
        \end{column}
    \end{columns}

    \noteT{グラフと統計値の対比}{
        平均ライン(中央の破線)はほぼ同じ位置にあることに注目させます。\par
        その上で、山の高さ(分散の大小)が「クラスの雰囲気」をどう変えるかを口頭で補足してください。
    }
\end{frame}

% ----------------------------------------------------------------------------------------
%   Slide H: 演習③ 外れ値になっている行を確認する
% ----------------------------------------------------------------------------------------

%@@PAGEBAND@@
% ----------------------------------------------------------------------------------------
%   page 12
% ----------------------------------------------------------------------------------------
\begin{frame}{演習③:外れ値になっているデータ行を確認する}
ヒストグラムで確認した分布の中から、
中心から大きく離れている値(外れ値になりそうなデータ)を
実際の行として確認します。

\textbf{作業内容}
\vspace{-0.6em}
\begin{enumerate}
  \item \textbf{重さ\_g} の列を並べ替える(昇順・降順)
  \item 特に小さい値・大きい値の行をいくつか選ぶ
  \item その行の \textbf{曜日}・\textbf{学生ID}・\textbf{メモ} を確認する
\end{enumerate}

\textbf{考えてみるポイント}
\vspace{-0.6em}
\begin{itemize}
  \item なぜこの値は、他と比べて大きく(または小さく)なったのか
  \item 測定ミスの可能性はあるか
  \item 特別な状況(忙しさ、盛り付けのクセなど)が考えられるか
\end{itemize}

\noteT{演習③の狙い}{
・「外れ値」という言葉を、実際のデータ行と結びつける。\\
・数値だけでなく、曜日やメモといった文脈情報を見る意識を持たせる。\\
・次に、データを分けて見る(曜日別・人別)必要性につなげる。
}
\end{frame}

% ----------------------------------------------------------------------------------------
%   Slide I: 演習④ 曜日別に分けて見る
% ----------------------------------------------------------------------------------------

%@@PAGEBAND@@
% ----------------------------------------------------------------------------------------
%   page 13
% ----------------------------------------------------------------------------------------
\begin{frame}{演習④:曜日別に分けて、平均とばらつきを比べる}
これまでの演習では、全データをまとめて見てきました。
ここでは、データを \textbf{曜日ごと} に分けて見てみます。

\textbf{作業内容}
\vspace{-0.6em}
\begin{enumerate}
  \item ピボットテーブルを作成する
  \item 行に \textbf{曜日} を配置する
  \item 値に \textbf{重さ\_g(平均)} と \textbf{重さ\_g(件数)} を配置する
\end{enumerate}

\textbf{確認するポイント}
\vspace{-0.6em}
\begin{itemize}
  \item 曜日によって、平均値は同じか、違うか
  \item 件数はどの曜日も同じか
  \item 特定の曜日だけ、重め・軽めになっていないか
\end{itemize}

\textbf{ここで考えること}
\vspace{-0.6em}
\begin{itemize}
  \item もし曜日で傾向が違うなら、全体の分布は何が混ざったものか
\end{itemize}

\noteT{演習④の狙い}{
・「まとめて見る」と「分けて見る」で、見えるものが変わることを体験させる。\\
・曜日という文脈情報が、分布の形に影響する可能性を意識させる。\\
・次に、人ごとの違い(測定者のクセ)へ自然につなげる。\\

ピボットは重さと曜日の列を選択\\
挿入タブのピボット作成で作成する\\
値のところからiマークをクリックして対象の平均、個数をセットする
}
\end{frame}

% ----------------------------------------------------------------------------------------
%   Slide J: 演習⑤ 学生ID別に分けて見る
% ----------------------------------------------------------------------------------------

%@@PAGEBAND@@
% ----------------------------------------------------------------------------------------
%   page 14
% ----------------------------------------------------------------------------------------
\begin{frame}{演習⑤:学生ID別に分けて、平均を比べる}
次に、同じポテトでも
\textbf{測定する人(学生)} によって違いがあるかを確認します。

\textbf{作業内容}
\vspace{-0.6em}
\begin{enumerate}
  \item ピボットテーブルを作成する
  \item 行に \textbf{学生ID} を配置する
  \item 値に \textbf{重さ\_g(平均)} を配置する
\end{enumerate}

\textbf{確認するポイント}
\vspace{-0.6em}
\begin{itemize}
  \item 学生ごとに、平均値は同じか、違うか
  \item 明らかに「重め」「軽め」になっている人はいないか
\end{itemize}

\textbf{ここで考えること}
\vspace{-0.6em}
\begin{itemize}
  \item 全体の分布は、どのような分布が混ざった結果だろうか
\end{itemize}

\noteT{演習⑤の狙い}{
・「人による違い」が、分布の形を作る一因であることを体験させる。\\
・全体の分布=1つの原因ではなく、複数の要因の重なりであると気づかせる。\\
・次に、分布を「読む」ことの意味を言語化するまとめへ進む。
}
\end{frame}

% ----------------------------------------------------------------------------------------
%   Slide K: 分布を読むとは何をすることか
% ----------------------------------------------------------------------------------------

%@@PAGEBAND@@
% ----------------------------------------------------------------------------------------
%   page 15
% ----------------------------------------------------------------------------------------
\begin{frame}{分布を「読む」とは何をすることか}
ここまでの演習を通して、私たちは
ポテト重量データをさまざまな角度から見てきました。

\textbf{分布を読むときの3つの視点}
\vspace{-0.6em}
\begin{itemize}
  \item \textbf{中心:}
        どのあたりの値が基準になっているか(平均・中央値)
  \item \textbf{広がり:}
        どの程度ばらついているか(最小〜最大、散らばり方)
  \item \textbf{混ざり:}
        曜日や人ごとの違いが重なっていないか
\end{itemize}

\textbf{重要なポイント}
\vspace{-0.6em}
\begin{itemize}
  \item 分布を見ることで、数値1つでは分からない実態が見えてくる
  \item 全体の分布は、複数の要因が重なった結果であることが多い
\end{itemize}

\noteT{分布を読むまとめ}{
・ここまでの演習(①〜⑤)を、3つの視点に整理して回収する。\\
・新しい知識を足さず、「やったことを言葉にする」役割に徹する。\\
・次のまとめスライドで、授業全体の目的に戻す準備とする。
}
\end{frame}

% ----------------------------------------------------------------------------------------
%   Slide L: 本日のまとめ
% ----------------------------------------------------------------------------------------

%@@PAGEBAND@@
% ----------------------------------------------------------------------------------------
%   page 16
% ----------------------------------------------------------------------------------------
\begin{frame}{本日のまとめ:記述統計でできるようになったこと}
本日の授業では、30人×1週間のポテト重量データを使って、
「記述統計」によるデータの読み取りを行いました。

\textbf{今日できるようになったこと}
\vspace{-0.6em}
\begin{itemize}
  \item データ全体を \textbf{代表値}(平均・中央値)で要約する
  \item ヒストグラムを使って \textbf{分布の形} を確認する
  \item 外れ値になりそうな値を見つけ、その \textbf{背景} を考える
  \item 曜日や人ごとに \textbf{分けて見る} ことで、違いを捉える
\end{itemize}

\textbf{最も大切なポイント}
\vspace{-0.6em}
\begin{itemize}
  \item 数値1つだけでは、データの実態は分からない
  \item 「分布を読む」ことで、現場で起きていることが見えてくる
\end{itemize}

\noteT{本日のまとめ}{
・第2回の目的と、実際に行った演習を対応づけて整理する。\\
・「平均だけでは足りない」という軸に必ず立ち戻る。\\
・学生自身の体験として言語化させる。
}
\end{frame}

% ----------------------------------------------------------------------------------------
%   Slide M: 次回予告
% ----------------------------------------------------------------------------------------

%@@PAGEBAND@@
% ----------------------------------------------------------------------------------------
%   page 17
% ----------------------------------------------------------------------------------------
\begin{frame}{次回予告:分布の背後にある「起こりやすさ」}
本日は、データの「形」を読み取るところまで進みました。

次回は、
\begin{itemize}
  \item なぜ、このような分布の形になるのか
  \item どの重さが、どのくらいの確率で起こるのか
\end{itemize}
を考えるために、\textbf{確率} を扱います。

\textbf{次回のテーマ}
\vspace{-0.6em}
\begin{itemize}
  \item 分布を「感覚」ではなく「数」で表す
  \item 「起こりやすさ」を確率として表現する
\end{itemize}

本日のポテトの分布が、そのまま次回の出発点になります。

\noteT{次回への接続}{
・今日扱った分布が、そのまま次回の教材になることを明示する。\\
・「確率=別の話」ではなく、連続した学びであると印象づける。\\
・第3回への心理的ハードルを下げる。
}
\end{frame}
\end{document}
