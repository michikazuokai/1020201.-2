%----------------------------------------------------------------------------------------
%  metropolis template (refactored)
%----------------------------------------------------------------------------------------
\documentclass[handout,aspectratio=169]{beamer}

% \documentclass の直後で hyperref のオプションを渡す(metropolisでも安全)
\PassOptionsToPackage{unicode=true,colorlinks=true,linkcolor=blue,urlcolor=blue}{hyperref}

\usetheme{metropolis}
\metroset{block=fill, sectionpage=progressbar, progressbar=foot}

% 背景色(tech のとき白など): Pythonで差し込み
\setbeamercolor{background canvas}{bg=white}

%--------------------------
% 日本語
%--------------------------
\usepackage{luatexja}
\usepackage{luatexja-fontspec}
\usepackage{luatexja-ruby}
\setsansjfont{Hiragino Sans}[BoldFont={Hiragino Sans W6}]

%--------------------------
% 基本パッケージ(重複なし)
%--------------------------
\usepackage[table]{xcolor}
\usepackage{graphicx}
\usepackage[abs]{overpic}
\usepackage{tikz}
\usetikzlibrary{positioning, shapes.geometric} % ← shapes.geometric を追加
\usepackage{array}
\usepackage{tabularx}
\usepackage{booktabs}
\usepackage{makecell}
\usepackage{mathtools}
\usepackage{longtable}
\usepackage{pdfpages}
\usepackage{etoolbox} % AtBeginEnvironment 等
\usepackage[normalem]{ulem}
\usetikzlibrary{calc}
\usetikzlibrary{backgrounds}

\usepackage{pgf}

% マーカー(ハイライター)の定義
\newcommand{\markline}[2][yellow]{%
  \tikz[baseline=(X.base)]{%
    \node[inner sep=0pt,outer sep=0pt] (X) {#2};
    \begin{scope}[on background layer]
      \fill[#1, opacity=0.35, rounded corners=0.8pt]
        ([xshift=-0.15em,yshift=0.00ex]X.south west) rectangle
        ([xshift= 0.15em,yshift=2.15ex]X.south east);
    \end{scope}
  }%
}



% minted(※ -shell-escape 必須)
\usepackage{minted}
\setminted{
  frame=single,
  framesep=2mm,
  fontsize=\footnotesize,
  breaklines=true
}

% (必要なときだけ)tcolorbox
\usepackage[most]{tcolorbox}
\tcbuselibrary{skins}        % 高度なデザイン機能
\tcbuselibrary{raster}

% hyperref は最後
\usepackage{hyperref}

% --- 1. カラーパレットの定義 ---
\definecolor{CanvaGreen}{HTML}{2E7D32} % メインの濃い緑(タイトル・枠線)
\definecolor{PaleGreen}{HTML}{F1F8E9}  % 背景の薄い緑
\definecolor{MyDarkGreen}{HTML}{587a7f}
\definecolor{DeepText}{HTML}{1C1C1C}   % 本文の文字色
\definecolor{MyWhiteBlue}{HTML}{F2FAFB} % 追加:あなたが指定した色
\definecolor{BananaColor}{HTML}{FFFD78}


% --- 2. tcolorbox のデフォルト設定(全ボックスに適用) ---
%\tcbset{
%    enhanced,                          % 高度な装飾を有効化
%    colback=white,                 % 本文背景色
%    colframe=MyDarkGreen,               % 枠線の色
%    coltitle=white,                    % タイトル文字色
%    fonttitle=\bfseries\sffamily,      % タイトルを太字・ゴシック
%    boxrule=1pt,                       % 枠線の太さ
%    arc=2mm,                           % 角の丸み
%    left=3mm, right=3mm,               % 左右の余白
%    top=0.5mm, bottom=0.5mm,               % 本文の上下余白
%    toptitle=0.8mm, bottomtitle=0.5mm, % タイトル内の上下余白
%    before skip=0.8em, after skip=0.2em,   % ボックス前後の行間
%    shadow={0mm}{0mm}{0mm}{black!0}    % 影を完全に消してフラットに
%}
% --- 2. tcolorbox のデフォルト設定 ---
\tcbset{
    enhanced,
    colback=MyWhiteBlue,            % 本文背景を F2FAFB に変更
    colframe=MyDarkGreen,
    coltitle=white,                 % タイトル文字は白(背景が濃い色の場合)
    % coltitle=DeepText,            % もしタイトル背景も薄くするならこちら
    fonttitle=\bfseries\sffamily\small, % タイトルを少し小さくしてスリム化
    boxrule=0.5pt,                  % 1pt から 0.5pt へ細分化
    arc=1mm,                        % 2mm から 1mm へ変更
    sharp corners=south,            % 下側を直角に固定
    % --- 余白の調整 ---
    left=3mm, right=3mm,
    top=0.5mm, bottom=0.5mm,
    toptitle=0mm,                   % タイトル上の余白をゼロに
    bottomtitle=0mm,                % タイトル下の余白をゼロに
    % ------------------
    before skip=0.8em, after skip=0.2em,
    shadow={0mm}{0mm}{0mm}{black!0}
}

% --- カスタムボックスの定義 ---
\newtcolorbox{myListbox}[1]{
  enhanced,
  detach title,              % 1. 標準のタイトル位置を解除
  % 2. 本文が始まる直前(before upper)でタイトルを直接描画する
  before upper={{\bfseries\large #1}\par\medskip},
  % タイトルの後に改行を入れる設定
  after title={\par\medskip},
  colbacktitle=white, 
  colframe=gray!50,
  colback=white,
  titlerule=0pt,
  boxrule=1pt,
  fonttitle=\bfseries\color{black},
  title=#1,
  after skip=1.5ex,
  % --- 箇条書きの余白を強制的にゼロにする設定 ---
  before upper={
    \setbeamertemplate{itemize ispan}{0pt} % 項目間の余白
    \setbeamertemplate{itemize items}[default]
    \setlength{\leftmargini}{1.5em}
    % Beamerの内部変数を直接操作して行間を詰める
    \addtobeamertemplate{itemize/enumerate body begin}{}{\setlength{\itemsep}{0pt}\setlength{\parskip}{0pt}}
  }
}

%--------------------------
% パス
%--------------------------
\newcommand{\assetpath}{/Volumes/NBPlan/TTC/授業資料/2025年度/}
\graphicspath{{images/}{\assetpath/1020201.アルゴリズム2/09/images/}{../project_assets/images/}{../project_assets/emoji/emoji_pngs/}}
% PGFがある特定の階層を定義
\newcommand{\pgfpath}{\assetpath/1020201.アルゴリズム2/09/images/}

%--------------------------
% フッター
%--------------------------
\newcommand{\myfootertext}{1020201.アルゴリズム2/09}
\setbeamertemplate{footline}{%
  \leavevmode
  \hbox to \paperwidth{%
    \hspace*{0.2cm}
    \scriptsize\color{gray!50} \myfootertext
    \hfill
    \scriptsize\color{gray} \insertframenumber{} / \inserttotalframenumber
    \hspace*{0.4cm}
  }%
  \vspace{1pt}
}

%--------------------------
% TeacherFrame(外部)
%--------------------------
\usepackage{../teacherframe}

%--------------------------
% フレームタイトル:番号. タイトル
% ※ ここで出すだけ。insertframetitle を再定義しない(安全)
%--------------------------
\setbeamertemplate{frametitle}{%
  \vspace{0.6ex}%
  \begin{beamercolorbox}[wd=\paperwidth,sep=0.5ex,leftskip=0.9em,rightskip=0.5em]{frametitle}%
    \usebeamerfont{frametitle}%
    \insertframenumber.\,\insertframetitle%
  \end{beamercolorbox}%
}

%--------------------------
% 表用:列型
%--------------------------
\newcolumntype{C}[1]{>{\centering\arraybackslash}p{#1}}
\newcolumntype{M}[1]{>{\raggedright\arraybackslash}m{#1}}

%--------------------------
% ブロック(必要なら)
%--------------------------
\definecolor{myblue}{HTML}{7488FF}
\definecolor{mylightblue}{HTML}{E3EEFF}
\setbeamertemplate{blocks}[rounded]
\setbeamercolor{block title}{bg=myblue, fg=white}
\setbeamercolor{block body}{bg=mylightblue, fg=black}

%========================================================
% exampleblock(examplebox相当)だけの調整
%  - タイトル文字:白
%  - 背景色:現行のまま(bgは指定しない)
%  - 本文 itemize:文字も●も黒(exampleblock内だけ)
%========================================================

% タイトル文字だけ白(背景は触らない)
\setbeamercolor{block title example}{fg=white}

% 本文の通常文字色は「現行のまま」を基本にする(必要なら黒にしてもよい)
% ここは bg を触らないのが目的なので fg だけ調整可能
\setbeamercolor{block body example}{fg=black}

% exampleblock の中だけ itemize の色(●と文字)を黒に
\AtBeginEnvironment{exampleblock}{%
  \setbeamercolor{itemize item}{fg=black}
  \setbeamercolor{itemize subitem}{fg=black}
  \setbeamercolor{itemize subsubitem}{fg=black}
  \setbeamercolor{item}{fg=black} % 念のため
}

% exampleblock を抜けたらテーマ標準に戻す(色が残る事故防止)
\AtEndEnvironment{exampleblock}{%
  \setbeamercolor{itemize item}{fg=normal text.fg}
  \setbeamercolor{itemize subitem}{fg=normal text.fg}
  \setbeamercolor{itemize subsubitem}{fg=normal text.fg}
  \setbeamercolor{item}{fg=normal text.fg}
}

%--------------------------
% 奇数ページのスライドのを表示する
%(教示用だけでそれ以外はこの処理は動かない)
%--------------------------
% --- 教師用だけ、スライドを奇数開始に強制するトグル ---
\newif\ifoddslideenforce
\oddslideenforcefalse   % デフォルトOFF(pr/hoはOFF)

% --- 再帰防止ガード ---
\newif\ifoddslideguard
\oddslideguardfalse

% --- 偶数ページなら空白スライドを1枚入れて奇数に戻す ---
\newcommand{\ensureoddslide}{%
  \ifoddslideguard\relax\else
    \oddslideguardtrue
    \ifodd\value{page}\relax
      % 何もしない(次が奇数)
    \else
      \begin{frame}[plain,noframenumbering]
        \note{}% notes出力時に2枚消費させる保険
      \end{frame}
    \fi
    \oddslideguardfalse
  \fi
}

% --- frameが始まる直前に自動挿入(教師用だけ)---
\BeforeBeginEnvironment{frame}{%
  \ifoddslideenforce
    \ensureoddslide
  \fi
}

%--------------------------
% note / noteT の「常時安全化」
%  - tech 以外:\noteT は無視(エラーにならない)
%  - tech:notesmode_tech で上書き定義
%--------------------------
\providecommand{\notetitletext}{}      % 既にあっても衝突しない
\providecommand{\noteT}[2]{}           % デフォルトは何もしない

% frame開始ごとにタイトル変数をクリア(前の noteT が残らないように)
\AtBeginEnvironment{frame}{\gdef\notetitletext{}}

%--------------------------
% 切替(Pythonから差し込み)
%--------------------------
\mypausemodetrue
\teachermodetrue
\setbeameroption{show notes}
%-------------

% --- tech のときだけ noteT を有効化(テンプレートの \providecommand を上書き) ---
\makeatletter
\renewcommand{\noteT}[2]{%
 \gdef\notetitletext{#1}%
 \note{#2}%
}
% タイトル未指定のときのために初期化
\renewcommand{\notetitletext}{}%

\setbeamertemplate{note page}{%
 \begin{minipage}{\linewidth}
 \vspace{1.2ex} % タイトルを少し下げる(必要に応じて調整)
 {\Large\bfseries
 \ifx\notetitletext\@empty
 \insertframetitle
 \else
 \notetitletext
 \fi
 }\par
 \vspace{-1.2ex}
 \rule{\linewidth}{0.8pt}\par
 \vspace{0.8ex}
 {\scriptsize \insertnote}
 \end{minipage}
}
\makeatother

%教師用のPDFは奇数ページからスライドを出力
\oddslideenforcetrue


%--------------------------
% 方眼紙(グリッド)をスライドに重ね
%--------------------------
\input{grid_debug}

%-------------------------------------------------------------------------
% 「ラインマーカー」そのものです。線の太さ・色・透明度を自由にできます。
%
% #1 色(省略可)
% #2 下端 yshift
% #3 上端 yshift
% #4 文字
% \marklineA{0.35ex}{1.55ex}{通常サイズ}
% {\Large \marklineA{0.45ex}{2.10ex}{大きい文字}}
%-------------------------------------------------------------------------
\newcommand{\marklineA}[4][yellow]{%
  \tikz[baseline=(X.base)]{%
    \node[inner sep=0pt,outer sep=0pt] (X) {#4};
    \begin{scope}[on background layer]
      \fill[#1, opacity=0.35, rounded corners=0.8pt]
        ([xshift=-0.15em,yshift=#2]X.south west) rectangle
        ([xshift= 0.15em,yshift=#3]X.south east);
    \end{scope}
  }%
}

%----------------------------------------------------------------------------------------
% タイトル
%----------------------------------------------------------------------------------------
\title{ 09 推測統計の考え方(標本と母集団) }
\date{}
\newcommand{\codedir}{\assetpath/1020201.アルゴリズム2/09}

\begin{document}

\begin{frame}[plain,noframenumbering]
  \titlepage
  \bigskip
  \begin{center}
    \ifteachermode 教師用 \fi
  \end{center}
\end{frame}

% セクションページ(必要なら)
\setbeamertemplate{section page}{
  \begin{centering}
    \vfill
    \rule{\linewidth}{2pt}\par
    \vspace{1ex}
    {\usebeamerfont{section title}\Huge\bfseries \insertsection}\par
    \vspace{1ex}
    \rule{\linewidth}{2pt}\par
    \vfill
  \end{centering}
}
\setbeamerfont{section title}{size=\LARGE,series=\bfseries}

\AtBeginSection[]{
  \begin{frame}[plain,noframenumbering]
    \sectionpage
  \end{frame}
}

% 本編開始でフレーム番号を0から(必要なら)
\setcounter{framenumber}{0}

\input{emoji_macros}

% @@@--(metropolis)--@@@
% ------------------------------------------------------------
% page
% ------------------------------------------------------------

%@@PAGEBAND@@
% ----------------------------------------------------------------------------------------
%   page 01
% ----------------------------------------------------------------------------------------
\begin{frame}{本当に知りたいのは、どの重さか}
ハンバーガーショップのポテトについて、

\vspace{0.5em}
\begin{itemize}
  \item 今日の30人分の重さ
  \item 明日の30人分の重さ
\end{itemize}
は、それぞれ違います。

\vspace{0.6em}
では、私たちが本当に知りたいのは、
\begin{center}
\textbf{どの「重さ」でしょうか?}
\end{center}

\noteT{導入の狙い}{
「1回のデータ」ではなく「全体」を知りたい、
という動機をはっきりさせる。
}
\end{frame}
% ------------------------------------------------------------
% page
% ------------------------------------------------------------

%@@PAGEBAND@@
% ----------------------------------------------------------------------------------------
%   page 02
% ----------------------------------------------------------------------------------------
\begin{frame}{前回の復習:平均は揺れる}
前回の授業では、

\vspace{0.5em}
\begin{itemize}
  \item 標本を取り直すと
  \item 平均が少しずつ変わる
\end{itemize}
ことを体験しました。

\vspace{0.6em}
この「揺れ」は、
\begin{center}
\textbf{間違いではなく、避けられない性質}
\end{center}
でした。

\noteT{接続の確認}{
揺れを否定せず、今日の話題の前提として再確認する。
}
\end{frame}
% ------------------------------------------------------------
% page
% ------------------------------------------------------------

%@@PAGEBAND@@
% ----------------------------------------------------------------------------------------
%   page 03
% ----------------------------------------------------------------------------------------
\begin{frame}{今日のゴール}
今日の授業では、

\vspace{0.6em}
\begin{itemize}
  \item 標本から母集団を考えるとはどういうことか
  \item 1つの数(点)で推測する方法
  \item 幅をもたせて推測する方法
\end{itemize}
を学びます。

\vspace{0.7em}
\begin{center}
\textbf{「ズレ」を前提にして判断する}
\end{center}
ことが、今日のテーマです。

\noteT{導入の締め}{
「計算が目的ではない」ことをここで一度強調。
}
\end{frame}

% ------------------------------------------------------------
% slide7-1
% ------------------------------------------------------------

%@@PAGEBAND@@
% ----------------------------------------------------------------------------------------
%   page 04
% ----------------------------------------------------------------------------------------
\begin{frame}{標本から1つの数を作る}
母集団の平均は、直接調べることができません。

\vspace{0.5em}
私たちが手にできるのは、
\begin{itemize}
  \item 母集団の一部として取り出した
  \item 限られた数のデータ(標本)
\end{itemize}

\vspace{0.6em}
そこでまず、
\begin{center}
\textbf{標本を「1つの代表的な数」にまとめる}
\end{center}
ことを考えます。

\noteT{ここでの位置づけ}{
この直後に実習①(点推定に向けた体験)を入れる。\\
操作手順はスライドに書かず口頭で指示する。\\
【Excel実習・簡単仕様】\\
・全データからランダムに30件を抽出\\
・その30件の平均を計算\\
・「代表値が1つ得られる」ことを体験させる\\
}
\end{frame}

% ------------------------------------------------------------
% slide7-2
% ------------------------------------------------------------

%@@PAGEBAND@@
% ----------------------------------------------------------------------------------------
%   page 05
% ----------------------------------------------------------------------------------------
\begin{frame}{その数は毎回同じか?}
今作った代表値は、
\begin{itemize}
  \item ある標本(例:30件)から
  \item 1回だけ計算した結果
\end{itemize}

\vspace{0.6em}
ここで重要なのは、
\begin{center}
\textbf{標本が変われば、代表値も変わりうる}
\end{center}
という点です。

\vspace{0.5em}
つまり、代表値は「固定」ではなく、\textbf{揺れを持つ}可能性があります。

\noteT{実習①の回収}{
実習①で得た平均値を例に、\\
「別のランダム30件に変えると値が変わる」ことを短く口頭確認する。\\
ここでは点推定という用語はまだ出さない。\\
}
\end{frame}

% ------------------------------------------------------------
% slide7-3
% ------------------------------------------------------------

%@@PAGEBAND@@
% ----------------------------------------------------------------------------------------
%   page 06
% ----------------------------------------------------------------------------------------
\begin{frame}{なぜ「平均」を使うのか}
標本の中には、
\begin{itemize}
  \item 軽いデータ
  \item 重いデータ
\end{itemize}
が混ざっています。

\vspace{0.6em}
平均は、
\begin{itemize}
  \item すべての値を使って
  \item 全体の中心を表す
\end{itemize}
\textbf{代表的な数}です。

\vspace{0.6em}
また、平均には
\begin{center}
\textbf{1つの値に極端に引っぱられにくい}
\end{center}
という利点があります。

\noteT{補足(短く)}{
「平均との差が正負で打ち消し合う」という直感を一言で触れる程度に留める。\\
理論(CLT)は後で回収する。\\
}
\end{frame}

% ------------------------------------------------------------
% slide7-4
% ------------------------------------------------------------

%@@PAGEBAND@@
% ----------------------------------------------------------------------------------------
%   page 07
% ----------------------------------------------------------------------------------------
\begin{frame}{この平均は何をしている数か}
ここまでの操作を整理します。

\vspace{0.6em}
私たちは、
\begin{itemize}
  \item 標本から代表値(平均)を作り
  \item それを使って
  \item 母集団の平均を考えようとしています
\end{itemize}

\vspace{0.3ex}
このように、
\vspace{-1ex}
\begin{center}
\textbf{標本から母集団の値を1つの数で推測する}
\end{center}
ことを、
\vspace{-1ex}

\begin{center}
	\begin{tcolorbox}[width=8cm,  colback=BananaColor!70,sharp corners, arc=0mm, frame hidden]
		\centering
		\textbf{点推定}
	\end{tcolorbox}
\end{center}
\vspace{-0.6ex}
といいます。
\vspace{0.4em}
実習①で計算した平均は、\textbf{母平均の点推定値}です。

\noteT{ここで初めて用語を確定}{
「さっきの平均は点推定だった」と後づけで意味づけするのがポイント。\\
先に体験→あとで命名(第7回の作り方と同じ)。\\
}
\end{frame}

% ------------------------------------------------------------
% slide7-5
% ------------------------------------------------------------

%@@PAGEBAND@@
% ----------------------------------------------------------------------------------------
%   page 08
% ----------------------------------------------------------------------------------------
\begin{frame}{点推定の限界}
点推定は、
\begin{itemize}
  \item 手軽で
  \item わかりやすい
\end{itemize}
方法です。

\vspace{-0.6ex}
しかし、点推定だけでは
\vspace{-0.6ex}
\begin{center}
\textbf{どれくらいズレるか(誤差の大きさ)が分かりません}
\vspace{-0.6ex}
\end{center}
という限界があります。

\vspace{0.2em}
次は、
\begin{center}
\textbf{ズレの大きさも一緒に示す方法}
\end{center}
\vspace{-0.6ex}
\begin{center}
	\begin{tcolorbox}[width=10cm,  colback=BananaColor!70,sharp corners, arc=0mm, frame hidden]
		\centering
		\textbf{区間推定}
	\end{tcolorbox}
\end{center}
\vspace{-0.6ex}
を考えます。

\noteT{次の実習②への橋}{
この直後に実習②(区間推定)を入れる。\\
【Excel実習・簡単仕様】\\
・実習①の標本を使い標準偏差を計算\\
・母平均の信頼区間を算出\\
操作手順は口頭で指示する。\\
}
\end{frame}

% ------------------------------------------------------------
% slide8-1
% ------------------------------------------------------------

%@@PAGEBAND@@
% ----------------------------------------------------------------------------------------
%   page 09
% ----------------------------------------------------------------------------------------
\begin{frame}{点推定だけでは足りない}
点推定では、標本から \textbf{1つの数}を作りました。

\vspace{0.6em}
しかし、その数には次の弱点があります。

\begin{itemize}
  \item 標本が変われば値も変わる(揺れる)
  \item それでも「どれくらい揺れるか」が分からない
\end{itemize}

\vspace{0.7em}
つまり、
\begin{center}
\textbf{推定値の「不確かさ」を表せていない}
\end{center}
のです。

\noteT{位置づけ}{
点推定の限界を「ズレの大きさが不明」として再確認するスライド。
ここではまだ「信頼区間」などの用語を出さず、
「不確かさを表したい」という目的だけを明確にする。
}
\end{frame}

% ------------------------------------------------------------
% slide8-2
% ------------------------------------------------------------

%@@PAGEBAND@@
% ----------------------------------------------------------------------------------------
%   page 10
% ----------------------------------------------------------------------------------------
\begin{frame}{不確かさは「幅」で表す}
推定値が揺れるなら、
\textbf{1点だけで言い切る}のではなく、

\vspace{0.6em}
\begin{center}
\textbf{ある幅(範囲)で示す}
\end{center}
ほうが自然です。

\vspace{0.7em}
例えば、次のような言い方です。

\begin{itemize}
  \item 「母平均は150g \textbf{くらい}」ではなく
  \item 「母平均は \textbf{148g〜152g} のあいだにありそう」
\end{itemize}

\noteT{狙い}{
「区間推定=範囲で言う」という直感を先に作る。
ここではまだ計算方法を出さない。
}
\end{frame}

% ------------------------------------------------------------
% slide8-3
% ------------------------------------------------------------

%@@PAGEBAND@@
% ----------------------------------------------------------------------------------------
%   page 11
% ----------------------------------------------------------------------------------------
\begin{frame}{区間推定とは何か}
標本から母集団の値を
\textbf{1つの数}で推測するのが点推定でした。

\vspace{0.6em}
それに対して区間推定は、
\begin{center}
\textbf{母集団の値を「範囲」で推測する}
\end{center}
方法です。

\vspace{0.2em}
この範囲を
\vspace{-0.6ex}
\begin{center}
	\begin{tcolorbox}[width=10cm,  colback=BananaColor!70,sharp corners, arc=0mm, frame hidden]
		\centering
		\textbf{推定区間(信頼区間)}
	\end{tcolorbox}
\end{center}
\vspace{-0.6ex}
と呼びます。
\noteT{用語導入}{
ここで初めて「区間推定」「推定区間(信頼区間)」の用語を出す。
「点→区間」の対比を短い文章で固定する。
}
\end{frame}

% ------------------------------------------------------------
% slide8-4
% ------------------------------------------------------------

%@@PAGEBAND@@
% ----------------------------------------------------------------------------------------
%   page 12
% ----------------------------------------------------------------------------------------
\begin{frame}{なぜ「範囲」が作れるのか(根拠)}
区間推定ができるのは、
\textbf{標本平均の揺れ方に規則がある}からです。

\vspace{0.6em}
\begin{itemize}
  \item 標本平均は真ん中付近に集まりやすい
  \item 極端に外れた平均は出にくい
\end{itemize}

\vspace{0.7em}
この規則性(第7回で体験した性質)が、
\begin{center}
\textbf{「どの範囲なら普通か」を決める土台}
\end{center}
になります。

\noteT{橋渡し}{
CLT/LLNの詳しい説明は別スライド(後のグループ)で回収する前提。
ここでは「規則があるから範囲が作れる」までを短く言い切る。
}
\end{frame}

% ------------------------------------------------------------
% slide8-5
% ------------------------------------------------------------

%@@PAGEBAND@@
% ----------------------------------------------------------------------------------------
%   page 13
% ----------------------------------------------------------------------------------------
\begin{frame}{区間推定で言えること}
区間推定をすると、推定結果は
\begin{center}
\textbf{1つの数+その周りの幅}
\end{center}
になります。

\vspace{0.6em}
これにより、
\begin{itemize}
  \item 点推定だけでは分からなかった「不確かさ」
  \item 推定値がどれくらい信用できそうか
\end{itemize}
を、文章で説明できるようになります。

\noteT{狙い}{
区間推定を「計算の手順」ではなく
「言える内容が増える」ものとして位置づける。
この後すぐ実習②で計算へ入る。
}
\end{frame}

% ------------------------------------------------------------
% slide8-6(差し替え)
% ------------------------------------------------------------

%@@PAGEBAND@@
% ----------------------------------------------------------------------------------------
%   page 14
% ----------------------------------------------------------------------------------------
\begin{frame}{区間推定で「標準偏差」が必要な理由}
区間推定では、点推定(標本平均)に
\textbf{幅}を付けて「ありそうな範囲」を作ります。

\vspace{0.6em}
この幅を決める材料は、主に2つです。

\begin{itemize}
  \item \textbf{ばらつきの大きさ}(標本標準偏差 $s$)
  \item \textbf{標本の大きさ}(件数 $n$)
\end{itemize}

\vspace{0.7em}
標準偏差が大きいほど、平均は \textbf{揺れやすい}ので
\begin{center}
\textbf{区間は広くなる}
\end{center}
逆に標準偏差が小さいほど
\begin{center}
\textbf{区間は狭くなる}
\end{center}
のです。

\noteT{狙い}{
区間推定の「幅」は偶然ではなく、
データのばらつき(標準偏差)と件数で決まることを言語化する。
式は次スライドで最小限だけ出す(または口頭で示す)。
}
\end{frame}

% ------------------------------------------------------------
% slide8-6.5(追加)
% ------------------------------------------------------------

%@@PAGEBAND@@
% ----------------------------------------------------------------------------------------
%   page 15
% ----------------------------------------------------------------------------------------
\begin{frame}{幅は「標準誤差」で決まる}
標本平均の揺れやすさは、次の量で表せます。

\vspace{0.6em}
\begin{center}
\textbf{標準誤差} $\;\displaystyle SE = \frac{s}{\sqrt{n}}$
\end{center}

\vspace{0.7em}
\begin{itemize}
  \item $s$ が大きい(ばらつき大) → $SE$ も大 → 区間が広い
  \item $n$ が大きい(件数多) → $\sqrt{n}$ が大 → $SE$ 小 → 区間が狭い
\end{itemize}

\vspace{0.7em}
区間推定は、ざっくり言うと
\vspace{-0.5em}
\begin{center}
	\begin{tcolorbox}[width=10cm,  colback=BananaColor!70,sharp corners, arc=0mm, frame hidden]
		\centering
		\textbf{平均 $\pm$(何個か分の $SE$)}
	\end{tcolorbox}
\end{center}
\vspace{-0.5em}
という形で作られます。

\noteT{補足(式の扱い)}{
ここでは「覚える公式」にしない。
幅=ばらつき÷件数の効果(平方根)という意味だけを理解させる。
95\%などの係数(t値)は、次の実習でツールに任せてもよい。
}
\end{frame}

% ------------------------------------------------------------
% slide8-7
% ------------------------------------------------------------

%@@PAGEBAND@@
% ----------------------------------------------------------------------------------------
%   page 16
% ----------------------------------------------------------------------------------------
\begin{frame}{実習②:区間推定を計算する}
ここからは、\textbf{区間推定}を実際に計算します。

\vspace{0.6em}
今日ここで行うことは、次の2つです。

\begin{itemize}
  \item 標本平均を確認する(点推定)
  \item その平均の「ズレの大きさ」を考え、範囲を作る
\end{itemize}

\vspace{0.7em}
\begin{center}
\textbf{母平均を「1つの数」ではなく
「幅をもった範囲」で表す}
\end{center}

\noteT{実習②(区間推定)の仕様}{
【目的】
点推定に「不確かさ」を加え、
母平均を範囲として表現できるようにする。

【使用データ】
・実習①で固定したランダム30件の標本

【Excel操作(詳細は口頭)】
・標本平均 $\bar{x}$ を計算
・標本標準偏差 $s$ を計算
・件数 $n=30$ を確認
・これらを用いて母平均の信頼区間を算出

【重要な観察ポイント】
・区間の幅は「標準偏差」によって決まる
・同じ n=30 でも、ばらつきが大きい標本ほど区間は広くなる

【ミニ確認(時間が許せば)】
・別のランダム30件を取り直し、
  平均・標準偏差・区間幅の変化を比較する

【時間目安】
約15分
}
\end{frame}

% ------------------------------------------------------------
% slide8-8
% ------------------------------------------------------------

%@@PAGEBAND@@
% ----------------------------------------------------------------------------------------
%   page 17
% ----------------------------------------------------------------------------------------
\begin{frame}{区間推定の読み方}
推定区間が
\[
148\ \text{g} \sim 152\ \text{g}
\]
のように得られたとします。

\vspace{0.6em}
このとき、私たちが言いたいのは
\begin{center}
\textbf{母平均はこの範囲にありそうだ}
\end{center}
ということです。

\vspace{0.6em}
点推定(150g)だけのときよりも、
\textbf{どれくらいズレそうか}が
一緒に分かるようになっています。

\noteT{口頭補足(注意点)}{
・「必ずこの中に入る」とは言わない
・信頼水準(95\%など)の厳密な意味は
  次の塊や次回で回収してよい
・ここでは「幅をもって主張する」ことの価値を強調
}
\end{frame}

% ------------------------------------------------------------
% slide8-9
% ------------------------------------------------------------

%@@PAGEBAND@@
% ----------------------------------------------------------------------------------------
%   page 18
% ----------------------------------------------------------------------------------------
\begin{frame}{点推定と区間推定の整理}
今日ここまでで、
母平均の推測には2つの形があると分かりました。

\vspace{0.6em}
\begin{itemize}
  \item \textbf{点推定}:
        母平均を1つの数で表す
  \item \textbf{区間推定}:
        母平均を幅のある範囲で表す
\end{itemize}

\vspace{0.7em}
区間推定によって、
\begin{center}
\textbf{推測の「不確かさ」も含めて
判断できる}
\end{center}
ようになります。

\noteT{次回への接続}{
次回は、
この「ズレの大きさ」を基準にして
「どれくらい外れたら外れているのか?」を判断する
仮説検定に進む。
}
\end{frame}

% ------------------------------------------------------------
% slide8-10
% ------------------------------------------------------------

%@@PAGEBAND@@
% ----------------------------------------------------------------------------------------
%   page 19
% ----------------------------------------------------------------------------------------
\begin{frame}{ズレの大きさに「基準」を与える}
区間推定では、
\begin{center}
\textbf{どれくらいのズレまでを「普通」と考えるか}
\end{center}
を、あらかじめ決めます。

\vspace{0.6em}
よく使われる基準の1つが、
	\begin{tcolorbox}[on line,hbox, colback=BananaColor!70,sharp corners, arc=0mm, frame hidden]
		\textbf{95\%}
	\end{tcolorbox}
です。

\vspace{0.6em}
これは、
\begin{itemize}
  \item 標本の揺れ方を考えたとき
  \item \textbf{その範囲に入ることが多い}
\end{itemize}
という意味で使われます。

\noteT{口頭補足(信頼水準)}{
・95\%は「絶対」ではなく、判断のための約束
・100\%にすると区間は広くなりすぎる
・90\%や99\%など、目的に応じて選ばれる
・ここでは「ズレの基準を数で決めている」ことを理解させる
}
\end{frame}

% ------------------------------------------------------------
% slide8-11
% ------------------------------------------------------------

%@@PAGEBAND@@
% ----------------------------------------------------------------------------------------
%   page 20
% ----------------------------------------------------------------------------------------
\begin{frame}{この回の到達点}
今日の授業で、次のことができるようになりました。

\vspace{0.6em}
\begin{itemize}
  \item 標本から母平均を \textbf{点} で推測できる
  \item ズレの大きさを考え、\textbf{区間} で推測できる
  \item そのズレに \textbf{基準(95\%)} を与えられる
\end{itemize}

\vspace{0.7em}
これにより、
\begin{center}
\textbf{「どれくらい外れたら外れているか」を考える準備}
\end{center}
が整いました。

\noteT{次回への接続}{
次回は、
この区間やズレの大きさを根拠にして、
「この結果は偶然か?それとも違いがあるのか?」
を判断する仮説検定に進む。
}
\end{frame}

% ------------------------------------------------------------
% slide8-12(最終ページ)
% ------------------------------------------------------------

%@@PAGEBAND@@
% ----------------------------------------------------------------------------------------
%   page 21
% ----------------------------------------------------------------------------------------
\begin{frame}{まとめ:なぜ推測できるのか}
母集団は、直接見ることができません。

\vspace{0.6em}
それでも私たちは、
\begin{itemize}
  \item 標本を取り
  \item その揺れ方を理解し
  \item 数で基準を決める
\end{itemize}
ことで、母集団について考えてきました。

\vspace{0.7em}
\begin{center}
\textbf{推測統計とは、
標本の「揺れ」を前提にして判断する方法}
\end{center}

\noteT{授業の閉じ(口頭)}{
・「当てる」ではなく「根拠をもって判断する」
・この考え方が次回の仮説検定につながる
・今日は計算よりも「考え方」が主役だったことを強調
}
\end{frame}
\end{document}
