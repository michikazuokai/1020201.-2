% @@@--(metropolis)--@@@
% ----------------------------------------------------------------------------------------
%   Slide 01: タイトル
% ----------------------------------------------------------------------------------------
% @@@--(metropolis)--@@@

% ----------------------------------------------------------------------------------------
%   Slide 01: 本日のテーマ(第6回)
% ----------------------------------------------------------------------------------------
\begin{frame}{第6回:確率分布②(正規分布)}
\begin{itemize}
  \item 今日の主役:\textbf{正規分布(ベルカーブ)}
  \item キーワード:\textbf{平均(中心)} と \textbf{ばらつき}
  \item 重要ルール:\textbf{確率は「高さ」ではなく「面積」}
\end{itemize}

\vspace{0.8em}
\textbf{題材:ハンバーガーショップのポテト}\\
\textbf{基準値:150g}(盛り付けは毎回ズレる)

\noteT{狙い}{
ポテトを“誤差が自然に出る連続データ”として導入し、\\
平均だけでは判断できない問題を提示する。\\
ばらつきをどう扱うかは、この後で新しい道具として導入する。
}
\end{frame}

% ----------------------------------------------------------------------------------------
%   Slide 2-0: ハンバーガーショップの話を思い出そう(導入の振り返り)
% ----------------------------------------------------------------------------------------
\begin{frame}{ハンバーガーショップの話を思い出そう}
\textbf{状況:} ポテトは「基準150g」で盛り付けているつもりでも、毎回少しずつ重さのズレが生じる(誤差が出る)

\vspace{0.6em}
\textbf{そこでやったこと:}
\begin{itemize}
  \item 30人分(30回分)のポテトの重さを調査してデータ化した
  \item 平均・中央値・最小/最大など、\textbf{基礎統計量}をExcelで計算した
\end{itemize}

\vspace{0.6em}
\textbf{でも、疑問が残った:}
\begin{itemize}
  \item 「平均との差が○g」は \textbf{大きいの?普通なの?}
  \item 「この重さ」は \textbf{よくある?珍しい?}(判断の基準が欲しい)
  \item どこからが \textbf{重すぎ/軽すぎ} と言えるのか?
\end{itemize}

\noteT{狙い}{
基礎統計量は出せたが「位置づけ(判断)」ができない、という課題を言語化する。
この疑問を解決する道具として正規分布(山の形)へつなぐ。
}
\end{frame}

% ----------------------------------------------------------------------------------------
%   Slide 02: 前回の復習 → 今日への橋渡し
% ----------------------------------------------------------------------------------------
\begin{frame}{前回(離散)から今日(連続)へ}
\begin{itemize}
  \item 前回:コインを何回投げたか(\textbf{0回〜10回})\ \ \textbf{離散(とびとび)}
  \item 今日:ポテトの重さ(\textbf{149.8g, 150.2g ...})\ \ \textbf{連続(つながる)}
\end{itemize}

\vspace{0.8em}
\textbf{違いの一言:}\\
\textbf{離散}=「数える」/ \textbf{連続}=「測る」

\vspace{0.8em}
% ---- 図の挿入位置 ----
\begin{center}
    \includegraphics[scale=0.70]{グラフ比較図.png}
\end{center}
\noteT{接続}{
“回数”は整数しか取らない。一方“重さ”は小数でいくらでも細かくなる。
ここで「正規分布は連続データの代表」という位置づけに入る。 
}
\end{frame}


% ----------------------------------------------------------------------------------------
%   Slide 03: 今日のデータ(ポテト30人分)
% ----------------------------------------------------------------------------------------
\begin{frame}{今日使うデータ:ポテト30人分(重さの誤差)}
\textbf{状況:} 同じお店、同じ「基準150g」のポテトでも、毎回ズレが出る。

\vspace{0.6em}
\begin{itemize}
  \item データ数:\textbf{30個}
  \item 見たいこと:
    \begin{itemize}
      \item 150g 付近に集まる?(中心)
      \item どれくらい散らばる?(ばらつき)
      \item 「めったにない重さ」を判断できる?(確率)
    \end{itemize}
\end{itemize}

\noteT{狙い}{
ここで“30人の調査データを使う”宣言をして、以降はExcel実習に自然に移れるようにする。
}
\end{frame}

% ----------------------------------------------------------------------------------------
%   Slide 04: 正規分布の超要点(形と意味)
% ----------------------------------------------------------------------------------------
\begin{frame}{正規分布:一番よく出てくる「山の形」} 
\begin{itemize}
  \item 中央が高く、左右対称(ベルカーブ)
  \item \textbf{平均に近い値が出やすい}
  \item 平均から離れるほど出にくい
\end{itemize}

\vspace{0.6em}
\textbf{今日の読み替え:}\\
\textbf{150g に近い重さが一番多い}/\textbf{極端に重い・軽いは珍しい}

\vspace{0.8em}
% ---- 図の挿入位置 ----
\begin{center}
    \includegraphics[scale=0.25]{正規分布の基本グラフ.png}
\end{center}

\noteT{ポイント}{
「正規分布=普通に起きる誤差の集まり」であることを、ポテトに結びつけて言い切る。
}
\end{frame}


% % ----------------------------------------------------------------------------------------
% %   Slide 05-A: 平均 μ が変わる
% % ----------------------------------------------------------------------------------------
% \begin{frame}{平均 $\mu$ が変わると、山はどう動くか}
% \begin{itemize}
%   \item 平均($\mu$)は \textbf{山の中心位置} を決める
%   \item $\mu$ が変わると、\textbf{山全体が左右にスライド}する
%   \item \textbf{形(ばらつき)は変わらない}
% \end{itemize}

% \vspace{0.6em}
% \textbf{ポテトの読み替え:}\\
% 基準は同じでも、盛り付けの「中心」がずれると、
% 「だいたい150g」の位置そのものが変わる

% \vspace{0.8em}
% % ---- 図の挿入位置 ----
% \begin{center}
%     \includegraphics[scale=0.4]{正規分布平均.png}
% \end{center}

% \noteT{狙い}{
% 平均は「どこを基準にしているか」を表す数字。
% ここでは \textbf{σには触れず}、「位置だけが変わる」ことを視覚で固定する。
% }
% \end{frame}

% % ----------------------------------------------------------------------------------------
% %   Slide 05-B: 標準偏差 σ が変わる
% % ----------------------------------------------------------------------------------------
% \begin{frame}{標準偏差 $\sigma$ が変わると、山はどう変わるか}
% \begin{itemize}
%   \item 標準偏差($\sigma$)は \textbf{ばらつきの大きさ} を表す
%   \item $\sigma$ が小さい → \textbf{細く高い山(安定)}
%   \item $\sigma$ が大きい → \textbf{太く低い山(ばらつく)}
% \end{itemize}
 
% \vspace{0.6em}
% \textbf{ポテトの読み替え:}\\
% 毎回ほぼ同じ量なら $\sigma$ は小さい/
% 日によって多かったり少なかったりすると $\sigma$ は大きい

% \begin{center}
%     \includegraphics[scale=0.4]{正規分布偏差.png}
% \end{center}


% \noteT{狙い}{
% 平均は同じでも「安定しているかどうか」は別問題。
% 次のスライドで「どれくらいズレたら珍しい?」へ接続する準備。
% }
% \end{frame}

% ----------------------------------------------------------------------------------------
%   Slide 05: ヒストグラムで見えたこと(再確認)
% ----------------------------------------------------------------------------------------
\begin{frame}{ヒストグラムで見えたこと}
\textbf{これまでの授業で行ったこと}

\vspace{0.6em}
\begin{columns}[T,onlytextwidth]
    \begin{column}{0.6\textwidth}
        \begin{itemize}
            \item 30人分のポテトの重さデータを使って
            \item ヒストグラムを作成した
        \end{itemize}
        \vspace{0.6em}
        \textbf{そこから分かったこと}
        \begin{itemize}
            \item 真ん中あたり(150g付近)にデータが多い
            \item 左右に少しずつ広がっている
            \item 全体として「山のような形」になっている
        \end{itemize}
    \end{column}
    \begin{column}{0.4\textwidth}
        \begin{center}
            \includegraphics[scale=0.4]{potatohist.png}
        \end{center}
    \end{column}
\end{columns}



\noteT{狙い}{
学生に「知っている内容」「やったこと」を思い出させるスライド。
ここでは評価や不足は言わず、事実の確認だけに徹する。
}
\end{frame}

% ----------------------------------------------------------------------------------------
%   Slide 06: ヒストグラムだけで判断できる?
% ----------------------------------------------------------------------------------------
\begin{frame}{でも、ヒストグラムだけで判断できる?}
\textbf{ここで疑問}

\vspace{0.6em}
\begin{itemize}
  \item 148g のポテトは、\textbf{どれくらい珍しい}?
  \item 「重すぎ/軽すぎ」と言える \textbf{基準}は?
  \item そのような重さは \textbf{何%くらい} 起きる?
\end{itemize}

\vspace{0.8em}
\textbf{ヒストグラムの役割}

\begin{itemize}
  \item \textbf{分布の可視化:} データの散らばりや偏りを「山の形」で捉える
  \item \textbf{異常の発見:} 全体から外れた「極端な値」をひと目で特定する
  \item \textbf{背景の推測:} 山の数や位置から「データの裏にある事実」を探る
\end{itemize}

\vspace{0.8em}
\textbf{結論}

\begin{center}
\textbf{観察はできるが、判断はできない}
\end{center}

\noteT{狙い}{
「ヒストグラムがダメ」ではなく、
「役割が違う」という整理をする。
ここで初めて「次の道具」が必要だと学生に感じさせる。
}
\end{frame}

% % ----------------------------------------------------------------------------------------
% %   Slide A: ヒストグラムでわかること
% % ----------------------------------------------------------------------------------------
% \begin{frame}{ヒストグラムで見えたこと}
% ポテト30人分の重さをヒストグラムで見てみると──

% \vspace{0.6em}
% \begin{itemize}
%   \item 150g 付近に多く集まっていそう
%   \item ある程度の広がり(ばらつき)がありそう
%   \item 極端に重い/軽いものは少なそう
% \end{itemize}

% \vspace{0.8em}
% \textbf{ここまでで言えること:}\\
% 重さの \textbf{傾向や雰囲気} は、ヒストグラムから読み取れる

% \vspace{-0.8em}
% % ---- 図の挿入位置 ----
% \begin{center}
%   \includegraphics[scale=0.3]{potatohist.png}
% \end{center}

% \noteT{狙い}{
% 「ヒストグラム=形を見る道具」として、正しく評価する。
% ここでは判断や結論を急がない。
% }
% \end{frame}

% % ----------------------------------------------------------------------------------------
% %   Slide B: ヒストグラムの限界
% % ----------------------------------------------------------------------------------------
% \begin{frame}{しかし、ヒストグラムだけでは判断できない}
% ヒストグラムを見ると「形」は分かりました。  
% しかし、次の問いには答えられません。

% \vspace{0.6em}
% \begin{itemize}
%   \item この重さは \textbf{ハンバーガーショップの説明と比べて普通か?}
%   \item ショップが想定している重さの範囲は \textbf{どれくらいか?}
%   \item どれくらい外れたら \textbf{外れている} と言えるのか?
% \end{itemize}

% \vspace{0.8em}
% \textbf{重要な指摘:}\\
% ヒストグラムでは  
% \textbf{「どれくらい外れているか」を判断する基準は持てない}

% \noteT{狙い}{
% 学生の中に「見えているのに判断できない」という
% 健全な違和感を生む。
% }
% \end{frame}

% ----------------------------------------------------------------------------------------
%   Slide C: 次に必要な考え方
% ----------------------------------------------------------------------------------------
\begin{frame}{次に必要な考え方は何か}
私たちが知りたいのは、単なる形ではありません。

\vspace{0.6em}
\begin{itemize}
  \item ハンバーガーショップの説明と \textbf{合っているか}
  \item その重さが \textbf{どれくらい起こりやすいか}
  \item 「普通」と「外れている」を \textbf{同じ基準で} 判断したい
\end{itemize}

\vspace{0.8em}
\textbf{そのために必要なのは:}\\
\textbf{データの形を、判断に使える「共通の考え方」として扱うこと}

\vspace{0.6em}
このあと、その代表的な考え方として  
\textbf{正規分布} を導入します。

\noteT{狙い}{
正規分布を「突然の用語」ではなく、
ヒストグラムの限界を埋める必然的な道具として登場させる。
}
\end{frame}

% ----------------------------------------------------------------------------------------
%   Slide D: この形は自然界でよく現れる
% ----------------------------------------------------------------------------------------
\begin{frame}{この形は、実はよく現れる}
ヒストグラムで見えた「山の形」は、  
ポテトだけに特有のものではありません。

\vspace{0.6em}
\begin{itemize}
  \item この形は自然界に多く見られる
  \item 人の身長・体重
  \item テストの点数、作業時間や誤差
  \item 機械で作られる製品の重さや精度
\end{itemize}

\vspace{0.8em}
\textbf{共通点:}
\begin{itemize}
  \item 真ん中あたりが一番多い
  \item 左右に少しずつ広がる
  \item 極端な値はあまり起きない
\end{itemize}

\vspace{0.8em}
このような「よくある山の形」には、  
\textbf{正規分布} という名前がついています。

\noteT{狙い}{
正規分布を「突然の数学用語」ではなく、
「よく見かける形の名前」として受け止めさせる。
}
\end{frame}
% ----------------------------------------------------------------------------------------
%   Slide 
% ----------------------------------------------------------------------------------------
\begin{frame}{正規分布の事例}
  \begin{center}
      \includegraphics[scale=0.6]{正規分布事例.png}
  \end{center}
\end{frame}
% ----------------------------------------------------------------------------------------
%   Slide E: なぜこの形が「判断」に使えるのか
% ----------------------------------------------------------------------------------------
\begin{frame}{なぜこの形が「判断」に使えるのか}
ヒストグラムは  
\textbf{「今回集めたデータの様子」} を見せてくれました。

\vspace{0.6em}
しかし私たちが本当に知りたいのは──

\vspace{0.4em}
\begin{itemize}
  \item この重さは、\textbf{説明どおりと言えそうか}
  \item どの範囲までを \textbf{普通} と考えるか
  \item どれくらい外れたら \textbf{外れている} と言えるか
\end{itemize}

\vspace{0.8em}
\textbf{ポイント:}

\begin{itemize}
  \item ヒストグラム:\textbf{今回のデータを見る道具}
  \item 正規分布:\textbf{多くの場合に成り立つ共通の考え方}
\end{itemize}

\vspace{0.8em}
だから正規分布は  
\textbf{「この店の説明は妥当か?」を考えるための基準}  
として使える。

\noteT{狙い}{
「正規分布=判断のための共通ルール」という位置づけを明確にする。
ここではまだ計算には入らない。
}
\end{frame}
% ----------------------------------------------------------------------------------------
%   Slide F(新規・必須):正規分布でできること(今日のゴール)
% ----------------------------------------------------------------------------------------
\begin{frame}{正規分布を使うと、何ができるのか}
ここまでで分かったことを整理します。

\vspace{0.6em}
\begin{itemize}
  \item ヒストグラムは「今回のデータ」を見る道具
  \item 正規分布は「判断の基準」を与えてくれる考え方
\end{itemize}

\vspace{0.8em}
\textbf{正規分布を使うと:}
\begin{itemize}
  \item どの範囲を「普通」と考えるか
  \item どれくらい外れたら「外れている」と言えるか
  \item そのような重さが、どれくらい起こりにくいか
\end{itemize}

\vspace{0.8em}
今日はまず、  
\textbf{「正規分布で判断するとはどういうことか」}  
を考えます。
\end{frame}

% ----------------------------------------------------------------------------------------
%   Slide G: 山のどこを見るかが重要になる
% ----------------------------------------------------------------------------------------
\begin{frame}{判断するとき、山のどこを見るか}
ヒストグラムでは、これまで  
\textbf{「どこが高いか」} を見てきました。

\vspace{0.6em}
しかし、正規分布で判断するときは  
\textbf{見る場所が変わります。}

\vspace{0.8em}
\begin{itemize}
  \item 1つの重さ(1点)を見るのではない
  \item 山の \textbf{一部の広がり} に注目する
\end{itemize}

\vspace{0.8em}
\begin{center}
\textbf{判断のポイントは「高さ」ではなく「範囲」}
\end{center}

\noteT{狙い}{
ヒストグラム的な「棒の高さ」思考から、
正規分布的な「範囲」思考へ視点を切り替える。
}
\end{frame}

% ----------------------------------------------------------------------------------------
%   Slide H: 重さを範囲で考える
% ----------------------------------------------------------------------------------------
\begin{frame}{重さを「範囲」で考えてみる}
次の2つを比べてみましょう。

\vspace{0.6em}
\begin{itemize}
  \item 148.7g という \textbf{1つの値}
  \item 145g〜155g という \textbf{重さの範囲}
\end{itemize}

\vspace{0.8em}
私たちが本当に知りたいのは──

\vspace{0.4em}
\begin{itemize}
  \item このくらいの重さのポテトは多いのか?
  \item それともあまり出ないのか?
\end{itemize}

\vspace{0.8em}
\begin{center}
\textbf{正規分布では「範囲」で考える}
\end{center}

\noteT{狙い}{
1点の判断が難しいことを納得させ、
「範囲」という考え方を自然に受け入れさせる。
}
\end{frame}

% ----------------------------------------------------------------------------------------
%   Slide I: 山の量を考える
% ----------------------------------------------------------------------------------------
\begin{frame}{山の「量」を考えるという発想}
正規分布の山は、  
\textbf{全体で「全部」} を表しています。

\vspace{0.6em}
その山の一部を切り取ると──

\vspace{0.4em}
\begin{itemize}
  \item その範囲に \textbf{どれくらい含まれるか}
  \item 全体の中で \textbf{どのくらいの割合か}
\end{itemize}

\vspace{0.8em}
\begin{center}
\textbf{正規分布では、  
「山の量」が意味を持つ}
\end{center}

\noteT{狙い}{
次に出てくる「面積=割合(確率)」を
言葉なしで理解できる土台を作る。
}
\end{frame}

% ----------------------------------------------------------------------------------------
%   Slide J: 面積=割合(確率)
% ----------------------------------------------------------------------------------------
\begin{frame}{正規分布では「面積」が割合を表す}
正規分布の山は、全体で「全部(100\%)」でした。

\vspace{0.7em}
\textbf{次の一歩:}
\begin{itemize}
  \item 山の \textbf{一部の面積} = 全体の中での \textbf{割合}
  \item 割合を「確率」と呼ぶ
\end{itemize}

\vspace{0.8em}
\begin{center}
\Large \textbf{高さではなく、面積が意味を持つ}
\end{center}

\vspace{0.8em}
% ---- 図の挿入位置(面積を塗った正規分布) ----
\begin{center}
  \includegraphics[scale=0.55]{正規分布幅.png}
\end{center}

\noteT{狙い}{
「確率=面積」をここで初めて言い切る。
前スライド I で「量」の発想が入っているので、違和感なく接続できる。
}
\end{frame}

% ----------------------------------------------------------------------------------------
%   Slide K: 点ではなく範囲で考える(連続の特徴)
% ----------------------------------------------------------------------------------------
\begin{frame}{連続データでは「点」より「範囲」で考える}
ポテトの重さは \textbf{連続}(小数までいくらでも細かい)です。

\vspace{0.7em}
\textbf{ここが重要:}
\begin{itemize}
  \item \textbf{148.0gちょうど} が出る確率は「ほぼ0」
  \item だから \textbf{148g付近} のように「範囲」で考える
\end{itemize}

\vspace{-0.5em}
\begin{center}
  \textbf{連続データの確率 = 範囲の面積}
\end{center}

\vspace{-0.3em}
% ---- 図の挿入位置(点の細線と、幅のある範囲の塗り) ----
\begin{center}
  \includegraphics[scale=0.5]{point_vs_range.png} 
\end{center}

\noteT{範囲}{
確率は本来「全体に対する割合」。\\
コイントスでは回数を数えて割合を出したが、ポテトの重さは連続データなので同じ方法は使えない。\\
そこで正規分布では、山全体を100%と約束し、その中で範囲が占める割合を見る。\\
この「割合」を、確率と呼ぶ。\\
意味が変わったのではなく、割合の測り方が変わっただけ。\\
}
\end{frame}

% ----------------------------------------------------------------------------------------
%   Slide L: 150±5g を「範囲」で表現する
% ----------------------------------------------------------------------------------------
\begin{frame}{「普通の範囲」をどう表す?(例:150$\pm$5g)}
ハンバーガーショップの説明を、範囲で書き直します。

\vspace{0.6em}
\textbf{例:普通を「150$\pm$5g」と考えるなら}
\begin{itemize}
  \item 普通の範囲:\textbf{145g 〜 155g}
  \item 知りたいのは:
    \begin{itemize}
      \item その範囲に \textbf{どれくらい入るか(割合)}
      \item 範囲の外は \textbf{どれくらい珍しいか}
    \end{itemize}
\end{itemize}


\noteT{狙い}{
「普通=範囲」という言葉の形を確定させる。
ここではまだ計算しない。次で“面積をどう求める?”に繋ぐ。
}
\end{frame}

% ----------------------------------------------------------------------------------------
%   Slide N: 面積(割合)を数値にするには?
% ----------------------------------------------------------------------------------------
\begin{frame}{では、その面積(割合)をどうやって数値にする?}
145g〜155g の \textbf{面積の割合} を知りたい。  
でも、山の面積は \textbf{定規では測れない} ので──

\vspace{0.6em}
\textbf{次に必要になる道具:}
\begin{itemize}
  \item \textbf{山の位置を「共通の物差し」に変換する}(標準化)
  \item その結果から \textbf{面積(割合)を読み取る方法}(表・関数)
\end{itemize}

\vspace{0.8em}
\begin{center}
\textbf{次のステップ:}\\
145g と 155g を、\\
「山の上での位置」として表す
\end{center}

\noteT{狙い}{
ここで「標準偏差」や「z値」を突然出さない。
まず「位置を共通化する必要がある」という必然を作る。
次スライドで「平均との差」「ばらつき(尺度)」の導入へ自然につなげる。
}
\end{frame}

% ----------------------------------------------------------------------------------------
%   Slide O: 割合に名前をつける
% ----------------------------------------------------------------------------------------
\begin{frame}{「割合」を、何と呼ぶか}
ここまで私たちは、ずっと \textbf{割合} という言葉を使ってきました。

\vspace{0.6em}
\begin{itemize}
  \item ある範囲に \textbf{どれくらい含まれるか}
  \item 全体の中で \textbf{どれくらいの割合か}
\end{itemize}

\vspace{0.8em}
この「割合」に、統計では名前があります。

\vspace{0.6em}
\begin{center}
\Large
\textbf{この割合を「確率」と呼ぶ}
\end{center}

\noteT{狙い}{
ここで初めて「確率」という言葉を正式に導入する。
新しい考え方ではなく、これまで使ってきた割合に
名前をつけただけだと強調する。
}
\end{frame}

% ----------------------------------------------------------------------------------------
%   Slide P: なぜ確率と呼べるのか
% ----------------------------------------------------------------------------------------
\begin{frame}{なぜ、この割合を「確率」と呼べるのか}
理由はシンプルです。

\vspace{0.6em}
\begin{itemize}
  \item 同じ条件で何度も測ると
  \item その範囲に入る割合は
  \item \textbf{だいたい同じ値に落ち着く}
\end{itemize}

\vspace{0.8em}
つまり──

\vspace{0.4em}
\begin{center}
\textbf{長い目で見たときの割合}
\end{center}

\vspace{0.4em}
これが、確率の意味です。

\noteT{狙い}{
確率=未来予言ではないことを明確にする。
「繰り返したときの割合」という
統計的な意味づけに限定する。
}
\end{frame}
% ----------------------------------------------------------------------------------------
%   Slide Q: なぜ点ではなく範囲なのか
% ----------------------------------------------------------------------------------------
\begin{frame}{なぜ「点」ではなく「範囲」で考えるのか}
ここで重要な事実があります。

\vspace{0.6em}
\begin{itemize}
  \item ポテトの重さは \textbf{連続データ}
  \item 150.0g ちょうど、という値は
  \item 無限に細かい数の中の \textbf{1点}
\end{itemize}

\vspace{0.8em}
その結果──

\vspace{0.4em}
\begin{center}
\textbf{1点だけの確率は 0}
\end{center}

\vspace{0.6em}
だから私たちは必ず  
\textbf{「範囲」に含まれる確率} を考えます。

\noteT{狙い}{
「確率が0」という表現で混乱させない。
連続量では点に意味がない、という
考え方だけを伝える。
}
\end{frame}

% ----------------------------------------------------------------------------------------
%   Slide R: 正規分布でやっていること
% ----------------------------------------------------------------------------------------
\begin{frame}{正規分布で、何をしているのか}
ここまでの話を整理します。

\vspace{0.6em}
\begin{itemize}
  \item 山全体:\textbf{すべて(100\%)}
  \item 山の一部:\textbf{その範囲に入る割合}
  \item その割合を:\textbf{確率と呼ぶ}
\end{itemize}

\vspace{0.8em}
つまり正規分布では──

\vspace{0.4em}
\begin{center}
\textbf{「どの範囲に、どれくらい含まれるか」}
\end{center}

\vspace{0.4em}
を考えているだけです。

\noteT{狙い}{
正規分布を「難しい数式」ではなく、
考え方の整理として理解させる。
}
\end{frame}

% ----------------------------------------------------------------------------------------
%   Slide S: なぜこの形が基準になるのか
% ----------------------------------------------------------------------------------------
\begin{frame}{なぜ、この形が判断の基準になるのか}
この「山の形」は、特別なものではありません。

\vspace{0.6em}
\begin{itemize}
  \item 誤差が少しずつ重なる
  \item 小さなズレが積み重なる
  \item プラスとマイナスが混ざる
\end{itemize}

\vspace{0.8em}
その結果、自然に  
\textbf{この形に近づく} ことが多い。

\vspace{0.6em}
だから正規分布は  
\textbf{判断の基準として使われる}。

\noteT{狙い}{
「仮定している」のではなく
「よく現れるから使う」という
現実的な理由を与える。
}
\end{frame}

% ----------------------------------------------------------------------------------------
%   Slide T: 本日のまとめ
% ----------------------------------------------------------------------------------------
\begin{frame}{本日のまとめ}
今日学んだことを整理します。

\vspace{0.6em}
\begin{itemize}
  \item ヒストグラムは「観察」の道具
  \item 正規分布は「判断」の基準
  \item 連続データでは「範囲」で考える
  \item 山の面積は「割合」=「確率」
\end{itemize}

\vspace{0.8em}
次回は──

\vspace{0.4em}
\begin{center}
\textbf{この山を、どうやって数値で扱うか}
\end{center}

\noteT{狙い}{
第7回・第8回(σ、標準化、中心極限定理)への
自然な橋渡し。
}
\end{frame}
